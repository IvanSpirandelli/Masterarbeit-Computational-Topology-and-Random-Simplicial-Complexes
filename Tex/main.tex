%\documentclass[a4paper, 11pt,openany,oneside]{scrbook}
\documentclass[a4paper, 11pt]{report}
\usepackage[english,ngerman]{babel} 
\usepackage[utf8]{inputenc}

\usepackage{float}
\usepackage{tikz}
\usepackage{xinttools}
\usepackage{subcaption}
\usepackage{xcolor}

\usetikzlibrary{positioning}
\usepackage{pgfplots}
\pgfplotsset{compat=1.16}

\usepackage{tikz}
\usepackage{tkz-berge}

\usetikzlibrary{calc,
intersections,
arrows.meta,
decorations.pathreplacing,
petri,
topaths,
matrix,
backgrounds,
3d,
shapes}

\newlength{\mycircle}
\setlength{\mycircle}{6pt}
  
\newlength{\myeps}
\setlength{\myeps}{5pt}
  
\newlength{\myopac}
\setlength{\myopac}{.6pt}
  
\newlength{\myup}
\setlength{\myup}{3.5pt}

\definecolor{myblue}{rgb}{0.12156862745098039, 0.4666666666666667, 0.7058823529411765}
\definecolor{myorange}{rgb}{1.0, 0.4980392156862745, 0.054901960784313725}
\definecolor{mygreen}{rgb}{0.17254901960784313, 0.6274509803921569, 0.17254901960784313}
\definecolor{myred}{rgb}{0.8392156862745098, 0.15294117647058825, 0.1568627450980392}
\definecolor{mypurple}{rgb}{0.5803921568627451, 0.403921568627451, 0.7411764705882353}
\definecolor{mycyan}{rgb}{0.09019607843137255, 0.7450980392156863, 0.8117647058823529}
\usepackage{tikz}
\usepackage{xinttools}
\usetikzlibrary{calc,%
%    arrows.meta,%
    intersections,%
    decorations.pathreplacing,%
    decorations.markings,
    arrows,%
    petri,%
    topaths}
    
\colorlet{customer_col}{myblue}
\colorlet{facility_col}{myred}
\colorlet{supplier_col}{mygreen}

\colorlet{cost_col}{olive}
\colorlet{assignment_col}{black}

\colorlet{d1_col}{white!40!black}
\colorlet{d2_col}{white!60!black}

\def\n{23} % number of random points
\def\maxxy{3} % random points are in [-\maxxy,\maxxy]x[-\maxxy,\maxxy]

%\definecolor{customer_col}{RGB}{26,0,192}
%\definecolor{facility_col}{RGB}{180,0,0}
%\definecolor{cost_col}{RGB}{255,105,66}
%\definecolor{assignment_col}{RGB}{255,255,255}

\tikzset{
customer/.style={draw, circle, fill = customer_col, text height = 0pt, text width = 0pt, text depth = 0pt, inner sep = 0pt, minimum size = 5pt},
supplier/.style={draw, circle, fill = supplier_col, text height = 0pt, text width = 0pt, text depth = 0pt, inner sep = 0pt, minimum size = 5pt},
facility/.style={draw, rectangle, fill = facility_col, text height = 0pt, text width = 0pt, text depth = 0pt, inner sep = 0pt, minimum size = 5pt},
cost/.style={thick,cost_col},
assignment/.style={ultra thick, assignment_col},
d1/.style={ultra thick, d1_col},
d2/.style={ultra thick, d2_col}
}

\def\biglen{8cm} % playing role of infinity (should be < .25\maxdimen)
% define the "half plane" to be clipped (#1 = half the distance between cells)

\tikzset{
  half plane/.style={ to path={
       ($(\tikztostart)!.5!(\tikztotarget)!#1!(\tikztotarget)!\biglen!90:(\tikztotarget)$)
    -- ($(\tikztostart)!.5!(\tikztotarget)!#1!(\tikztotarget)!\biglen!-90:(\tikztotarget)$)
    -- ([turn]0,2*\biglen) -- ([turn]0,2*\biglen) -- cycle}},
  half plane/.default={0.5pt}
}
vechain vs iota
     
\tikzset{node1/.style={circle, draw, fill=white, inner sep=0pt, text width=0pt, text height=0pt, text depth=0pt, minimum size = 5pt,}, edge1/.style={draw, thick},
}
\tikzset{node2/.style={shape=circle, draw, fill=white, inner sep=0pt, text width=0pt, text height=0pt, text depth=0pt, minimum size = 3pt}, edge2/.style={},
}


\tikzset{circle/.style={shape=circle, draw, fill=white, line width = 1pt,inner sep=0pt, text width=0pt, text height=0pt, text depth=0pt, minimum size = 4pt}, edge2/.style={},
}
\tikzset{b_circle/.style={shape=circle, draw, fill=black, inner sep=0pt, text width=0pt, text height=0pt, text depth=0pt, minimum size = 4pt}, edge2/.style={},
}
\tikzset{g_circle/.style={shape=circle, fill=gray, inner sep=0pt, text width=0pt, text height=0pt, text depth=0pt, minimum size = 6pt}, edge2/.style={},
}

\tikzset{->-/.style={decoration={
  markings,
  mark=at position .5 with {\arrow{latex}}},postaction={decorate}}}

\tikzset{->/.style={decoration={
  markings,
  mark=at position 1 with {\arrow{latex}}},postaction={decorate}}}
  
\usepackage{tkz-berge}



\newcommand{\PD}[1]{\textcolor{red}{\textbf{PD}\noindent\textbf{[#1]}}}

\usepackage{titlesec}
\titleformat{\chapter}[display]   
{\normalfont\huge\bfseries}{\chaptertitlename\ \thechapter}{20pt}{\Huge}   
\titlespacing*{\chapter}{0pt}{-50pt}{40pt}

\usepackage[linesnumbered,lined,boxed,commentsnumbered]{algorithm2e}
\usepackage{enumerate}
\usepackage{blkarray}

\usepackage{amsmath}
\usepackage{amsthm}
\usepackage{amssymb}
\usepackage{mathtools}
\usepackage{dsfont}
\usepackage{bm}
\usepackage{csquotes}
\RequirePackage{doi}
\usepackage{hyperref}



\DeclareMathOperator{\low}{low}
\newcommand{\norm}[1]{\left\lVert#1\right\rVert}

\theoremstyle{plain}
\newtheorem{thm}{Theorem}[chapter]
\newtheorem{lemma}[thm]{Lemma}
\newtheorem{prop}[thm]{Proposition}
\newtheorem{cor}[thm]{Corollary}
\newtheorem{corl}[thm]{Corollary}

\theoremstyle{definition}
\newtheorem{defi}[thm]{Definition}
\newtheorem{obs}[thm]{Observation}

\title{Computational Topology And Random Simplicial Complexes}
\author{Ivan Spirandelli}
\date{March 2020}

\begin{document}

\begin{titlepage}

%linker Rand 
\setlength{\hoffset}{.5cm}

 \begin{center}
 
    \large
    
    \renewcommand{\baselinestretch}{2}\normalsize
    
    {\huge \scshape Persistent Homology Of Randomized Simplicial Complexes}
    
    \renewcommand{\baselinestretch}{1}\normalsize
    
    \vspace*{.7cm}
    
    \huge{Masterarbeit}
    
    \vspace*{.7cm}
    
    \Large
    vorgelegt von\\
    \vspace*{.5cm}
     \huge{{\scshape Ivan Lothar Arbo Spirandelli}} \\
     \vspace*{.5cm}
    \Large
    zur Erlangung des akademischen Grades \\
  \vspace*{.5cm}
    \huge{Master of Science\\Mathematik}	
    
   
    
    \vspace*{.5cm}
    \large    
    \vspace*{1cm}
   \includegraphics[height=3cm]{TUBerlin_Logo_rot}
    \vspace*{1cm}
    
    Betreuer: Dr. Frank Lutz \\[1ex]
    Zweitgutachter: - \\[3ex]
    
   Technische Universit\"at Berlin\\
   
   Fakult\"at f\"ur Mathematik und Naturwissenschaften\\
   
   Institut f\"ur Mathematik\\[7ex]
   
       Berlin, den --.--.----
   
  \end{center}
\end{titlepage}
\newpage
\thispagestyle{empty}
\qquad

%%%%%%%%%% eidesstattliche Erklaerung %%%%%%%%%%%%%%
\cleardoublepage
\thispagestyle{empty}

\setlength{\hoffset}{.5cm}

\vspace*{4cm}
%\renewcommand{\baselinestretch}{3}\normalsize

 Hiermit erkl\"are ich, dass ich die vorliegende Arbeit selbstst\"andig und eigenh\"andig sowie ohne unerlaubte fremde Hilfe und ausschlie\ss lich unter Verwendung der aufgef\"uhrten Quellen und Hilfsmittel angefertigt habe.

\vspace{2.5cm}

Berlin, den --.--.----

% \renewcommand{\baselinestretch}{1}\normalsize

\hspace*{\fill}
\begin{tabular}{@{}l@{}}\hline
\makebox[5cm]{\small{Ivan Lothar Arbo Spirandelli}}
\end{tabular}

\cleardoublepage

\newpage
\thispagestyle{empty}
\qquad
\selectlanguage{english}
\begin{abstract}
In this thesis we will give a brief introduction on simplicial persistent homology and discrete Morse theory. We will discuss the mathematical background and the most popular algorithms
for persistence computation in the Chapters 1--3. In Chapter 4 we will extend some recent discussions of canonically defined persistence pairs, which are closely related to discrete Morse functions. We will formulate and prove a theorem linking them to the number of column additions in the standard algorithm used to compute persistent homology. Furthermore we will introduce a new construction of examples that achieves the worst case bound for this algorithm and compare it to a previously known construction. 

In the final Chapter 5 we will do computational experiments on random simplicial complexes with respect to the concepts established in Chapter 4. Here random simplicial complex is meant in the sense of analogues of the Erdös--Rényi random graph model and of alpha complexes constructed on random point clouds. 
\end{abstract}
\selectlanguage{ngerman}
\begin{abstract}
In dieser Arbeit werden in den ersten Kapiteln die Grundlagen von simplizialer persistenter Homologie und diskreter Morse-Theorie eingeführt. Es werden sowohl die mathematischen Strukturen als auch die Standardalgorithmen der Themengebiete diskutiert. Im vierten Kapitel greifen wir jüngere Erkenntnisse zu kanonisch definierten Persistenzpaaren auf, die in engem Zusammenhang mit diskreter Morse-Theorie stehen, und erweitern diese. Wir formulieren und beweisen ein Theorem, dass sie mit der Laufzeit des Standardalgorithmus zur Berechnung von persistenter Homologie verknüpft. Des weiteren führen wir eine neue Konstruktion von Beispielen ein, die die schlechtest mögliche Laufzeit in Bezug auf den Standardalgorithmus haben. Wir vergleichen die Konstruktion mit einer bereits bekannten Konstruktion dieser Art. 

Im letzten Kapitel untersuchen wir zufällige Simplizialkomplexe in Bezug auf die Erkenntnisse aus dem vierten Kapitel. Hierbei meint zufällig sowohl im Sinne von Analoga zu dem zufälligen Graphen Model von Erdös-Rényi, als auch im Sinne der Konstruktion von Alpha-Komplexen auf zufälligen Punktwolken. 
\end{abstract}
\selectlanguage{english}
\newpage
\renewcommand{\abstractname}{Acknowledgements}
\begin{abstract}
 I'd like to thank Frank Lutz and $\text{Pawe\l}$ $\text{D\l otko}$ for their support and guidance during the times of me working on this thesis. 
\end{abstract}
\thispagestyle{empty}

\pagenumbering{roman}
\tableofcontents
\newpage
\pagenumbering{arabic}
\setcounter{page}{1}
\thispagestyle{empty}

\chapter{Simplicial Complexes}
The fundamental concept, that all other concepts discussed in this work will rely upon, is that of simplicial complexes. Simplicial complexes are topological spaces we can describe in a discrete manner and they are of interest in a variety of different topics. In computer graphics, every triangular mesh is a simplicial complex. Meshes are also used to solve physical problems via finite element methods. Simplicial complexes play a role in network theory, in material sciences and in the topological analysis of point clouds. The latter will be one of the central topics in this work. 

\section{Basic Definitions} 
\label{sec:SimplicialComplexes}

We assume that the reader is familiar with affine and convex combinations and hulls. For a thorough introduction in discrete geometry we refer you to \cite{Polyhedral+and+Algebraic+Methods} or \cite{Computational+Topology}, which also are the basis for this section. We begin by defining simplices.

\begin{defi}
A \textbf{geometric $\bm{k}$-simplex} is the convex hull of $k+1$ affinely independent points in $\mathbb{R}^n$. We call $k$ the \textbf{dimension} of the simplex. The empty set is a geometric simplex of dimension $-1$.
\end{defi}

This means a geometric 0-simplex is a point, a geometric 1-simplex is a line, a geometric 2-simplex is a triangle and so on and so forth. We will often omit the number $k$ and just talk about geometric simplices. See the following figure for an example of a geometric 2-simplex and a geometric 3-simplex. 

\begin{figure}[H]
%\centering%
\begin{subfigure}[c]{0.49\textwidth}
\begin{center}
\begin{tikzpicture}

\node[b_circle, name path=1, label=above:{a}] (1) at  (0.75,1) {};
\node[b_circle, name path=2, label=left:{b}] (2) at  (0,-0.5) {};
\node[b_circle, name path=3, label=right:{c}] (3) at  (1.5,-0.5) {};

\fill [opacity = 0.2] (0,-0.5)--(1.5,-0.5)--(0.75,1)--(0,-0.5);

\draw [-,very thick] (2)--(3);
\draw [-,very thick] (3)--(1);
\draw [-,very thick] (1)--(2);

\end{tikzpicture}
\end{center}
\end{subfigure}
\begin{subfigure}[c]{0.49\textwidth}
\begin{center}
\begin{tikzpicture}



\node[b_circle, name path=1, label=right:{c}] (1) at  (2,0,0.25) {};
\node[b_circle, name path=2, label=left:{a}] (2) at  (0.25,0,1) {};
\node[b_circle, name path=3, label=below:{b}] (3) at (2,0,2) {};
\node[b_circle, name path=4, label=above:{d}] (4) at (1,2,2) {};

\fill [opacity = 0.2] (0.25,0,1)--(2,0,2)--(1,2,2)--(0.25,0,1);
\fill [opacity = 0.2] (2,0,0.25)--(2,0,2)--(1,2,2)--(2,0,0.25);

\draw [dashed, thin] (1) -- (2);
\draw [-,very thick] (3) -- (2);
\draw [-,very thick] (3) -- (1);
\draw [-,very thick] (3) -- (4);
\draw [-,very thick] (1) -- (4);
\draw [-,very thick] (2) -- (4);

\end{tikzpicture}
\end{center}
\end{subfigure}
\caption{A geometric 2-simplex and a geometric 3-simplex}
\label{fig:simplices}
\end{figure}


\begin{defi} \cite[III.1]{Computational+Topology} 
    Consider some geometric simplex $\sigma$ of dimension $k$, which is the convex hull of affinely independent points $S = \{s_0, \dots , s_{k}\}$. Let $\gamma$ be the convex hull of a non-empty subset $U$ of $S$. Then $\gamma$ is called a \textbf{face} of $\sigma$ and $\sigma$ is called \textbf{coface} of $\gamma$.
A face is called \textbf{proper}, if $|U| \leq k$. Finally if $\operatorname{dim}(\gamma) = k-1$, we say that $\gamma$ is a \textbf{facet} of $\sigma$ and $\sigma$ is a \textbf{cofacet} of $\gamma$. 
\end{defi}

A popular notation, that we will also use, is to write $\tau \leq \sigma$ if $\tau$ is a face of $\sigma$ and $\tau < \sigma$ if it is proper.

We will now introduce a particular kind of collection of simplices in which faces and intersections of simplices are also contained in the collection.

\begin{defi}
    Let $K$ be a finite collection of geometric simplices. $K$ is a \textbf{geometric simplicial complex} if the following two conditions hold:
    \begin{enumerate}
        \item If $\sigma \in K \operatorname{and} \tau < \sigma$, then $\tau \in K$
        \item If $\sigma_0, \sigma_1 \in K$ then $\sigma_0 \cap \sigma_1 \in K$
    \end{enumerate}
The \textbf{dimension} of $K$ is the maximum dimension of any of its simplices.
    A subset $L \subset K$ is called a \textbf{geometric subcomplex} of $K$ if it is a geometric simplicial complex itself.
    The \textbf{$\bm{k}$-skeleton} of $K$ is the set of all simplices of dimension k or less.
\end{defi}

We will refer to the elements of a simplicial complex as its simplices or its \textbf{cells}.

The following Figure \ref{fig:simplicial_complexes} gives (a) an example of a geometric simplicial complex in two dimensions and (b) of a set of geometric simplices, that is not a geometric simplicial complex.

\begin{figure}[H]
%\centering%
\begin{subfigure}[c]{0.49\textwidth}
\begin{center}
\begin{tikzpicture}

\node[b_circle, name path=1, label=below:{a}] (1) at  (0,0) {};
\node[b_circle, name path=2, label=below:{b}] (2) at  (2,0) {};
\node[b_circle, name path=3, label=left:{c}] (3) at (0,2) {};
\node[b_circle, name path=4, label=below right:{d}] (4) at (2,2) {};
\node[b_circle, name path=5, label=right:{e}] (5) at (4,2) {};
\node[b_circle, name path=6, label=right:{f}] (6) at (2,4) {};


\fill [opacity = 0.2] (0,0)--(2,0)--(2,2)--(0,2)--(0,0);

\draw [-,very thick] (1)--(2);
\draw [-,very thick] (2)--(3);
\draw [-,very thick] (3)--(1);
\draw [-,very thick] (3)--(4);
\draw [-,very thick] (4)--(5);
\draw [-,very thick] (4)--(2);
\draw [-,very thick] (4)--(6);

\end{tikzpicture}
\subcaption{A geometric simplicial complex}
\end{center}
\end{subfigure}
\begin{subfigure}[c]{0.49\textwidth}
\begin{center}
\begin{tikzpicture}

\node[b_circle, name path=1, label=below:{a}] (1) at  (0,0) {};
\node[b_circle, name path=2, label=left:{b}] (2) at  (0,2) {};
\node[b_circle, name path=3, label=left:{c}] (3) at (2,4) {};
\node[b_circle, name path=4, label=below:{d}] (4) at (3,0) {};
\node[b_circle, name path=5, label=right:{e}] (5) at (3,2) {};
\node[b_circle, name path=6, label=right:{f}] (6) at (1,4) {};


\fill [opacity = 0.2] (0,0)--(0,2)--(2,4)--(0,0);
\fill [opacity = 0.2] (3,0)--(3,2)--(1,4)--(3,0);

\draw [-,very thick] (1)--(2);
\draw [-,very thick] (3)--(2);
\draw [-,very thick] (1)--(3);
\draw [-,very thick] (4)--(5);
\draw [-,very thick] (4)--(6);
\draw [-,very thick] (5)--(6);

\foreach \Point in {(0,0),(0,2),(2,4),(3,0),(3,2),(1,4)}{
\node[black] at \Point {\textbullet};
}
\end{tikzpicture}
\subcaption{Not a geometric simplicial complex}
\end{center}
\end{subfigure}
\caption{Two different sets of simplices, where the first collection fulfills the conditions of a geometric simplicial complex and the second does not.}
\label{fig:simplicial_complexes}
\end{figure}


Now when we look at the sets of vertices in each simplex of Figure \ref{fig:simplicial_complexes} (a) and list them, we get:
\begin{center}
    \begin{tabular}{lc}
        Empty set: & $\emptyset$, \\
        Vertices: & $\{a\}, \{b\}, \{c\}, \{d\}, \{e\}, \{f\}$, \\
        Edges: & $\{a,b\},\{a,c\},\{b,c\},\{b,d\},\{c,d\},\{d,e\},\{d,f\}$, \\
        Triangles: & $\{a,b,c\}, \{b,c,d\}$.
    \end{tabular}
\end{center}

Forgetting the explicit coordinates of each vertex and only considering the set of labels we can still see that there is some vertex $\{a\}$ that has two adjacent edges $\{a,b\}$ and $\{a,c\}$ while some other vertices $\{e\}$ and $\{f\}$ only have one adjacent edge. 
This type of combinatorial data is captured in what we call an abstract simplicial complex.

\begin{defi}
    Consider a finite set $V = \{v_1, v_2, \dots ,v_n\}$, which we will call a \textbf{vertex set}.
    Let $A$ be a set of subsets of $V$. 
    We call $A$ an \textbf{abstract simplicial complex} over $V$ if the following condition holds:
    \begin{itemize}
    	\item If $\alpha \in A$ and $\beta \subset \alpha$ then $\beta \in A$.
    \end{itemize}
    Note that $V = \bigcup_{\sigma \in A} \sigma$.
\end{defi}

We call the sets in $A$ \textbf{abstract simplices,} and the \textbf{dimension} of an abstract simplex $\sigma$ is its cardinality minus one, i.e.,\[ \operatorname{dim}(\sigma) = |\sigma| - 1.\]
The notions of faces and cofaces are analogous to those of geometric simplicial complexes.
The dimension of the abstract simplicial complex $A$ is the maximal dimension of one of its simplices. A subset $B \subseteq A$ is called an (abstract) subcomplex of $A$ if it is an abstract simplicial complex itself.

When comparing the definition of an abstract simplicial complex to the definition of a geometric simplicial complex, we can see that the second
condition for geometric simplicial complexes is automatically fulfilled for abstract simplicial complexes.


Now consider a vertex set $V = \{a,b,c,d,e,f\}$. The following set

\[
    A = \{\emptyset, \{a\}, \{b\}, \{c\}, \{d\}, \{e\}, \{f\}, \{a,b\}, \{a,c\}, \{b,c\},
\]
\[
 \{b,d\},\{c,d\},\{d,e\},\{d,f\},\{a,b,c\},\{b,c,d\}\}
\]
is an abstract simplicial complex over $V$. 

If we interpret the sets with two elements in $A$ as edges and those with three elements as triangles, this corresponds to the tabular above, in which we listed the simplices of the geometric simplicial complex in Figure \ref{fig:simplicial_complexes}.

The simplicial complex in Figure \ref{fig:simplicial_complexes} is realizing the abstract simplicial complex $A$. 

More formally, let \[g:V \rightarrow \mathbb{R}^m\] be a map that assigns linearly independent coordinates in $\mathbb{R}^m$ to the vertices of an abstract simplicial complex A. The map on the vertices induces a map on each individual abstract simplex $\sigma = \{v_1, \dots ,v_i\}$ by sending $\sigma$ to $\operatorname{conv}(\{g(v_0),\dots,g(v_i)\})$. We abuse notation slightly and also refer to this induced map as $g$. 

In case $g$ is injective and $g(\sigma)\cap g(\tau) = g(\sigma \cap \tau)$ for all pairs $\sigma, \tau \in A$, then $g$ is called a \textbf{geometric realization} of $A$. 

Given a geometric simplicial complex $K$, we obtain an abstract simplicial complex $A$, by taking the set of vertex labels for each geometric simplex in $K$. Then $A$ is called a \textbf{vertex scheme} of $K$. Hence, every geometric simplicial complex yields an abstract simplicial complex by simply omitting the coordinate information for the vertices of $K$. It also holds that every abstract simplicial complex has a geometric realization.  

One way to define a geometric realization of an abstract simplicial complex $A$ over vertex set $V = \{v_1,\dots,v_m\}$ is to define $g:V \rightarrow \mathbb{R}^m$ as \begin{itemize}
    \item $g(v_i) = e_i$,  for $i = 1, \dots, m$ 
    \item $g(\sigma) = \operatorname{conv}(\{g(v) \mid v \in \sigma \})$, for $\sigma \in A$. 
\end{itemize}
Then $K \coloneqq g(A)$ is a subcomplex of an $(m-1)$-simplex embedded in $\mathbb{R}^{m}$. 
We will conclude our discussion of the relation between abstract and geometric simplicial complexes by stating the following theorem, which allows a lower dimensional realization. See \cite[p.64]{Computational+Topology}.

\begin{thm}[Geometric Realization Theorem]
    Every $d$-dimensional abstract simplicial complex has a geometric realization in $\mathbb{R}^{2d+1}$
\end{thm}
% \begin{proof}
% This proof follows the proof given in \cite[p. 64]{Computational+Topology} closely. 
% Consider an abstract simplicial complex $A$ and let $V$ be the set of its vertices, i.e. its zero dimensional elements. We define an injection $f:V \rightarrow \mathbb{R}^{2d+1}$ such that the vertices are mapped to a set of points in general position. This means, that any subset of mapped vertices of cardinality $2d+2$ or less is affinely independent. 
% We need to show that the intersection of two simplices of the realization $\sigma_0 = \operatorname{conv}(f(\alpha_0))$ and $\sigma_1 = \operatorname{conv}(f(\alpha_1))$ is either empty or equals $\operatorname{conv}(f(\alpha_0 \cap \alpha_1))$.\\
% To this end consider two abstract simplices $\alpha_0$ and $\alpha_1$ of dimension $k_0$ and $k_1$ respectively. It holds that \[
% |\alpha_0 \cup \alpha_1| = |\alpha_0| + |\alpha_1| - |\alpha_0 \cap \alpha_1| \leq k_0 + k_1 + 2 \leq 2d+2
% \]
% which implies that by construction of $f$ the points in $\{f(a)\mid a \in \alpha_0 \cup \alpha_1\}$ are affinely independent. Therefor we know that every convex combination $x$ of points in $\{f(a) \mid a \in \alpha_0 \cup \alpha_1\}$ is unique. Hence $x \in \operatorname{conv}(f(\alpha_0))$ and $x \in \operatorname{conv}(f(\alpha_1))$ if and only if $x$ is a convex combination of $\{f(a)\mid a \in \alpha_0 \cap \alpha_1\}$. Therefor we get that the intersection of $\sigma_0$ and $\sigma_1$ is indeed empty or equals $\operatorname{conv}(f(\alpha_0 \cap \alpha_1))$.
% \end{proof}

We have seen how readily the concepts of abstract and geometric simplicial complex can be translated into one another. Therefore from now on, we omit the keywords abstract and geometric, and just talk about simplicial complexes, simplices, etc., unless we explicitly need to distinguish the two concepts. We are usually referring to abstract simplicial complexes. Often however we will see examples in which these abstract simplicial complexes are realized geometrically without discussing explicitly, that the realization is but one possible geometric realization of the abstract simplicial complex.

For brevity, we will introduce the following notation of writing abstract and geometric simplices in square brackets like so: \[
[a,b,c]
\] which stands for the convex hull $\operatorname{conv}(\{a,b,c\})$, i.e., a geometric simplex and/or the set $\{a,b,c\}$, i.e., an abstract simplex.


We have already talked about the notion of subcomplexes. Now we will consider sequences of subcomplexes, that are increasing with respect to inclusion. 

\begin{defi}
\label{def:filtration}
Let $K$ be a simplicial complex. A \textbf{filtration} of $K$ is a sequence of subcomplexes $F = \{K_0, \dots, K_n = K\}$ such that $K_i \subseteq K_{i+1}$ for $i = 0, \dots, n-1$.
\end{defi}

Note that two consecutive elements of the filtration might be equal, or might differ substantially. An intuitively useful extension of the last definition is the following. 
\begin{defi}
Let $F = \{K_0, \dots ,K_n\}$ be a filtration. If $K_0 = \emptyset$ and $|K_{i+1} \setminus K_i| = 1$ for all $i \in 0, \dots ,n-1$ we call the filtration \textbf{simplexwise} and denote it by $F_*$.
\end{defi}
A notation we will use in this work is to write simplexwise filtrations as \[
    F_* = (\sigma_1, \dots , \sigma_m),
\] where we mean that $F_*$ is a filtration of $K = \{\sigma_1, \dots, \sigma_m\}$ and subcomplex $K_i$ is the set of the first $i$ simplices, i.e. $K_i = \{\sigma_1, \dots ,\sigma_i\}$.

One way to define a filtration is by taking a function $f: K \rightarrow \mathbb{R}$ that is monotonously increasing with respect to inclusion, i.e, for $\sigma, \tau \in K$, $\sigma < \tau$ we have $f(\sigma) \leq f(\tau)$.

Given such a function we can define subcomplexes of $K$ by looking at the sub level sets of $f$. Let $a \in \mathbb{R}$, then $f^{-1}((-\infty,a]) \subseteq K$ is a subcomplex of $K$. Now we can take a sequence of elements in $\mathbb{R}$, $a_0 < a_1 < \dots < a_n$ and take $f^{-1}(a_0), \dots, f^{-1}(a_n)$ to define a filtration. 

On the other hand, given a filtration $F = \{K_0, \dots, K_n\}$ we can define such a function by setting $f(\sigma) = 0$ for $\sigma \in K_0$, and $f(\sigma) = i$ for $\sigma \in K_i \setminus K_{i-1}$ for $i = 1, \dots, n$.

Furthermore for a filtration $F = \{K_0, \dots, K_n = K\}$ and a function $f$ defined in that manner we call $F_* = (\sigma_1,\dots,\sigma_m)$ with $f(\sigma_i) \leq f(\sigma_{i+1})$ and $i < j$ if $\sigma_i < \sigma_j$ a \textbf{simplexwise refinement} of $F$. 
 
Sometimes we want to consider an ordering of that type of the simplices of a filtration $F$. In this case we will just consider a simplexwise refinement $F_*$ without further discussion. 

The following figure gives a simple example of a simplexwise filtration.  

\begin{figure}[H]
%\centering%
\begin{subfigure}[b]{0.3\textwidth}
\begin{center}
\input{SimplicialComplexes/Figures/simplicial_complex_1}
\subcaption{$K_1$}
\end{center}
\end{subfigure}
\begin{subfigure}[b]{0.3\textwidth}
\begin{center}
\input{SimplicialComplexes/Figures/simplicial_complex_2}
\subcaption{$K_2$}
\end{center}
\end{subfigure}
\begin{subfigure}[b]{0.3\textwidth}
\begin{center}
\begin{tikzpicture}

\node[b_circle, name path=0, label=left:{$v_1$}] (0) at (0,0) {};

\node[b_circle, name path=1, label=right:{$v_2$}] (1) at (2,0) {};

\node[b_circle, name path=2, label=above:{$v_3$}] (2) at  (1,2) {};


\end{tikzpicture}

\subcaption{$K_3$}
\end{center}
\end{subfigure}\\
\begin{subfigure}[b]{0.3\textwidth}

\begin{center}
\input{SimplicialComplexes/Figures/simplicial_complex_4}
\subcaption{$K_4$}
\end{center}
\end{subfigure}
\begin{subfigure}[b]{0.3\textwidth}

\begin{center}
\input{SimplicialComplexes/Figures/simplicial_complex_5}
\subcaption{$K_5$}
\end{center}
\end{subfigure}
\begin{subfigure}[b]{0.3\textwidth}

\begin{center}
\input{SimplicialComplexes/Figures/simplicial_complex_6}
\subcaption{$K_6$}
\end{center}
\end{subfigure}\\
\begin{subfigure}[b]{0.99\textwidth}
\begin{center}
\begin{tikzpicture}


\node[b_circle, name path=0, label=left:{$v_1$}] (0) at (0,0) {};

\node[b_circle, name path=1, label=right:{$v_2$}] (1) at (2,0) {};

\node[b_circle, name path=2, label=above:{$v_3$}] (2) at  (1,2) {};



\fill[opacity = 0.2] (0,0) -- (2,0) -- (1,2);

\draw[-,thick] (2) to (1);
\draw[-,thick] (1) to (0);
\draw[-,thick] (2) to (0);


\end{tikzpicture}

\subcaption{$K = K_7$}
\end{center}
\end{subfigure}
\caption{A simplexwise filtration of a triangle}
\label{fig:filtration}
\end{figure}


\section{Filtrations from Point Clouds}
\label{sec:filtrations_from_point_clouds}
As we will see later on, we can use filtrations to analyze how the topology of a simplicial complex changes 'over time', i.e. with respect to the sequence of the filtration. A major practical use for this is to analyze the topology of point clouds. To this end, however, we first need to have a natural way to get from a point cloud to some simplicial complex and a filtration. More detailed information on this can be found in \cite{Computational+Topology} and \cite{elementary_topology}. 

The first construction we will discuss is the so called Vietoris-Rips complex.

\begin{defi}
Let $S \subset \mathbb{R}^n$ be a finite set of points and consider some $\epsilon > 0$. The set \[
   \operatorname{VR}_\epsilon(S) \coloneqq \{\sigma \subseteq S \mid \operatorname{dist}(s_i,s_j)\leq \epsilon \text{, }s_i,s_j \in \sigma\}
\]
is called \textbf{Vietoris-Rips complex} of $S$.
\end{defi}

In other words, $\operatorname{VR}_\epsilon(S)$ is the set of all simplices defined by all sets of finitely many points in $S$ that have a pairwise distance smaller than $\epsilon$.

It is easy to see, that this yields an abstract simplicial complex over the vertex set $S$. Since all subsets and intersections of sets in $\operatorname{VR}_\epsilon(S)$ are again in $\operatorname{VR}_\epsilon(S)$.

While the Vietoris-Rips complex is straightforward to compute, it will result in simplicial complexes, that obscure lower dimensional topological structure in our data set. This happens since, whenever there is a set of edges, that is the edge set of some high dimensional simplex, this simplex is also included.

Another construction, that is similar to the Vietoris-Rips complex, but a little more aware of lower dimensional structure is the following.
\begin{defi}
Let $S \subset \mathbb{R}^n$ be finite and let $\epsilon > 0$. The set \[
    \text{\v{C}}_\epsilon(S) \coloneqq \{ \sigma \subseteq S \mid \bigcap\limits_{s \in \sigma} B_\epsilon(s) \neq \emptyset \},
\]
where $B_\epsilon(s)$ is the closed $n$-dimensional ball of radius $\epsilon$ with center $s$, is called \textbf{\v{C}ech complex} of $S$.
\end{defi}

In other words, the \v{C}ech complex consists of the simplices, whose vertices have intersecting balls of radius epsilon centered on them. The following Figure \ref{fig:chech_vr_complex} gives a small example of a  \v{C}ech and a Vietoris-Rips complex.

\begin{figure}[H]
%\centering%
\begin{subfigure}[b]{0.49\textwidth}
\begin{center}
\begin{tikzpicture}
\node[b_circle, name path=1, label=left:{}] (1) at (-1.5,1) {};

\node[b_circle, name path=2, label=above:{}] (2) at  (0,2) {};

\node[b_circle, name path=3, label=right:{}] (3) at  (1.5,1) {};
\node[b_circle, name path=3, label=right:{}] (31) at  (2,1) {};
\node[b_circle, name path=3, label=right:{}] (32) at  (1.75,0.5) {};


\node[b_circle, name path=4, label=below:{}] (4) at (-0.25,-0.25) {};

\node[b_circle, name path=5, label=below left:{}] (5) at  (-2,-0.6) {};

\end{tikzpicture}

\subcaption{Set of points $S$.}
\end{center}
\end{subfigure}
\begin{subfigure}[b]{0.49\textwidth}
\begin{center}
\input{SimplicialComplexes/Figures/chech2}
\subcaption{Balls of radius $\epsilon$ around each point.}
\end{center}
\end{subfigure}
\begin{subfigure}[c]{0.49\textwidth}
\begin{center}
\input{SimplicialComplexes/Figures/chech3}
\subcaption{The resulting \v{C}ech complex $\text{\v{C}}_{\epsilon}(S)$.}
\end{center}
\end{subfigure}
\begin{subfigure}[c]{0.49\textwidth}
\begin{center}
\input{SimplicialComplexes/Figures/vietoris1}
\subcaption{The Vietoris-Rips complex $\operatorname{VR}_{\epsilon}(S)$.}
\end{center}
\end{subfigure}
\caption{Example of a \v{C}ech and a VR complex.}
\label{fig:chech_vr_complex}
\end{figure}

The above figure illustrates how a \v{C}ech and a Vietoris-Rips complex differ for the same value $\epsilon$. It generally holds, that $\text{\v{C}}_\epsilon(S) \subseteq \operatorname{VR}_\epsilon(S) \subseteq \text{\v{C}}_{\sqrt{2}\epsilon}(S)$ which is proven in \cite{Computational+Topology}[Chapter III]. A notable property of both is that they not necessarily have a geometric realization in the dimension of the underlying point set. For example, let us consider the points in Figure \ref{fig:chech_vr_complex}(b). If we choose our radius $\epsilon$ large enough, the resulting simplicial complex will be a $6$-simplex on seven vertices with all its faces, which has a geometric realization in spaces of dimension six or higher, while the underlying point set is in dimension two. 


We will introduce a final construction, that has a geometric realization in the ambient space of the underlying point set. This is the so called Alpha Complex, which in a sense mixes the construction of \v{C}ech complexes with Voronoi diagrams.

\begin{defi}
Let $S \subset \mathbb{R}^n$ be finite set of points. The \textbf{Voronoi cell} of a point $q \in S$ is the set of points that is not closer to any other point in $S$ than $q$, i.e. \[
    V_q \coloneqq \{x \in \mathbb{R}^n \mid \norm{x-q} \leq \norm{x-s}, s \in S\setminus\{q\}\}.
\]

The \textbf{Voronoi diagram} of $S$ is the set of all its Voronoi cells. 
\end{defi}

In other words, a Voronoi diagram is a partitioning of $\mathbb{R}^n$ into areas that are closest to a particular pvechain vs iotaoint in the set $S$. The intersections of the Voronoi cells, are areas, in which several points of $S$ are equally close. 

\begin{figure}[H]
%\centering%
\begin{center}
\begin{tikzpicture}

    \def\pts{(-1.5,1),(0,2),(1.5,1),(2,1),(1.75,0.5),(-0.25,-0.25),(-2,-0.6)}
    % draw the points and their cells
    \xintForpair #1#2 in \pts \do{
      \edef\pta{#1,#2}
      \begin{scope}
        \xintForpair \#3#4 in \pts \do{
          \edef\ptb{#3,#4}
          \ifx\pta\ptb\relax % check if (#1,#2) == (#3,#4) ?
            \tikzstyle{myclip}=[];
          \else
            \tikzstyle{myclip}=[clip];
          \fi;
          \path[myclip] (#3,#4) to[half plane] (#1,#2);
        }
        \clip (-\maxxy,-\maxxy+1) rectangle (\maxxy,\maxxy); % last clip
        %\definecolor{randcolor}{hsb}{\randhue,.5,1}
        \fill[white] (#1,#2) circle (4*\biglen); % fill the cell with random color
        \fill[opacity = 0.25] (#1,#2) circle (4*\biglen); % fill the cell with random color
      \end{scope}
    }
    \pgfresetboundingbox
    \draw[white, very thick] (-\maxxy,-\maxxy+1) rectangle (\maxxy,\maxxy);
    
    \begin{pgfonlayer}{background}
    
    \fill[opacity = 1] (-\maxxy,-\maxxy+1) rectangle (\maxxy,\maxxy);
    \end{pgfonlayer}
    
    \node[b_circle, name path=1, label=left:{}] (1) at (-1.5,1) {};

    \node[b_circle, name path=2, label=above:{}] (2) at  (0,2) {};

    \node[b_circle, name path=3, label=right:{}] (3) at  (1.5,1) {};
    \node[b_circle, name path=3, label=right:{}] (31) at  (2,1) {};
    \node[b_circle, name path=3, label=right:{}] (32) at  (1.75,0.5) {};


    \node[b_circle, name path=4, label=below:{}] (4) at (-0.25,-0.25) {};

    \node[b_circle, name path=5, label=below left:{}] (5) at  (-2,-0.6) {};
    
  \end{tikzpicture}
\end{center}
\caption{Voronoi diagram of the point set from Figure \ref{fig:chech_vr_complex} (a).}
\label{fig:voronoi}
\end{figure}

\begin{defi}
Let $S \in \mathbb{R}^n$ be a finite set of points. Let $V(S)$ be the Voronoi diagram of $S$. Then
\[
D(S) \coloneqq \{\sigma \subseteq S \mid  \bigcap\limits_{s \in \sigma}V_s \neq \emptyset \}
\] 
is the \textbf{Delaunay complex} of $S$.
\end{defi}

If the points of a finite point set $S \subset \mathbb{R}^n$ are in general position, in the sense that no $n+2$ points lie on a common $n$-sphere, then the Delaunay complex is a triangulation of the convex hull of $S$ and is called the \textbf{Delaunay triangulation}. 

The following Figure \ref{fig:delaunay} (a) shows the Delaunay triangulation corresponding to the Voronoi diagram from Figure \ref{fig:voronoi}. Figure \ref{fig:delaunay} (b) shows a Voronoi diagram in two dimensions on  a set of four points that lie on a circle. The corresponding Delaunay complex has a three dimensional cell, i.e., is not a triangulation of the convex hull of the point set.

\begin{figure}[H]
%\centering%
\begin{subfigure}[b]{0.49\textwidth}
\begin{center}
\input{SimplicialComplexes/Figures/delaunay_tri}
\subcaption{Delaunay triangulation corresponding to Figure \ref{fig:voronoi}.}
\end{center}
\end{subfigure}
\begin{subfigure}[b]{0.49\textwidth}
\begin{center}
\begin{tikzpicture}

\node[b_circle, name path=0, label=below:{}] (0) at (0,0) {};

\node[b_circle, name path=1, label=below:{}] (1) at (3,0) {};

\node[b_circle, name path=2, label=above:{}] (2) at  (0,3) {};

\node[b_circle, name path=2, label=above:{}] (3) at  (3,3) {};


\fill[opacity = 0.2] (-1,-1) -- (4,-1) -- (4,4) -- (-1,4);

\draw[-,thick] (1.5,1.5) to (1.5,4);
\draw[-,thick] (1.5,1.5) to (4,1.5);
\draw[-,thick] (1.5,1.5) to (1.5,-1);
\draw[-,thick] (1.5,1.5) to (-1,1.5);

\end{tikzpicture}
\subcaption{Voronoi diagram on points not in general position.}
\end{center}
\end{subfigure}
\caption{Illustrating Delaunay triangulations and complexes.}
\label{fig:delaunay}
\end{figure}

We will now combine Voronoi diagrams with the constructing idea of \v{C}ech complexes. 

\begin{defi}
We define the \textbf{alpha complex} of a finite set of points $S \subset \mathbb{R}^n$ as \[
\operatorname{Alpha(\epsilon)} \coloneqq \{\sigma \subseteq S \mid \bigcap\limits_{s \in \sigma}(B_\epsilon(s) \cap V_s) \neq \emptyset\},
\]
with $B_\epsilon(s)$ being the ball of radius $\epsilon$ centered at $s$ and $V_s$ being the Voronoi cell of $s$.
\end{defi}

It is apparent from the definition, that if we choose $\epsilon$ large enough, so that each ball $B_\epsilon(s)$ covers all points in $S$, all simplices will be defined by the incidences in the Voronoi diagram. Hence, for $S$ in general position we can take the convex hulls of the abstract simplices defined by our Alpha complex and will get a geometric realization in the ambient space of the set of points. See the following example, where the resulting alpha complex is equal to the \v{C}ech complex from Figure \ref{fig:voronoi}. 

\begin{figure}[H]
%\centering%
\begin{subfigure}[b]{0.99\textwidth}
\begin{center}
\begin{tikzpicture}

\draw[-, thick, name path = c1] (-1.5,1) circle (0.97);
\draw[-, thick, name path = c2] (0,2) circle (0.97);
\draw[-, thick, name path = c3] (1.5,1) circle (0.97);
\draw[-, thick, name path = c31] (2,1)  circle (0.97);
\draw[-, thick, name path = c32] (1.75,0.5) circle (0.97);
\draw[-, thick, name path = c4] (-0.25,-0.25)  circle (0.97);
\draw[-, thick, name path = c5] (-2,-0.6) circle (0.97);

\node[] (1) at (-1.5,1) {};
\node[] (2) at  (0,2) {};
\node[] (3) at  (1.5,1) {};
\node[] (31) at  (2,1) {};
\node[] (32) at  (1.75,0.5) {};
\node[] (4) at (-0.25,-0.25) {};
\node[] (5) at  (-2,-0.6) {};

\fill[radius=0.96,
    white]
      (1) circle[]
      (2) circle[]
      (4) circle[]
      (5) circle[]
      (31) circle[]
      (32) circle[]
      (3) circle[];
      
\fill[radius=0.96,
    opacity = 0.25]
      (1) circle[]
      (2) circle[]
      (4) circle[]
      (5) circle[]
      (31) circle[]
      (32) circle[]
      (3) circle[];
      
\node[b_circle, name path=1, label=left:{}] at (1){};
\node[b_circle, name path=2, label=above:{}] at (2){};
\node[b_circle, name path=3, label=right:{}] at (3){};
\node[b_circle, name path=3, label=right:{}] at (31){};
\node[b_circle, name path=3, label=right:{}] at (32){};
\node[b_circle, name path=4, label=below:{}] at (4){};
\node[b_circle, name path=5, label=below left:{}] at (5){};

\path[name intersections={of= c1 and c2,by={i1,i2}}];
\draw[-, thick] (i1) to (i2);

\path[name intersections={of= c2 and c3,by={i3,i4}}];
\draw[-, thick] (i3) to (i4);

\path[name intersections={of= c1 and c4,by={i5,i6}}];
\draw[-, thick] (i5) to (i6);

\path[name intersections={of= c1 and c5,by={i7,i8}}];
\draw[-, thick] (i7) to (i8);

\path[name intersections={of= c4 and c5,by={i9,i10}}];
\draw[-, thick] (i9) to (i10);

\node[inner sep = 0] (i) at (1.75,0.75){};

\path[name intersections={of= c3 and c31,by={i11,i12}}];
\draw[-, thick] (1.75,0.75) to (i11);

\path[name intersections={of= c31 and c32,by={i13,i14}}];
\draw[-, thick] (1.75,0.75) to (i14);

\path[name intersections={of= c3 and c32,by={i15,i16}}];
\draw[-, thick] (1.75,0.75) to (i16);

\end{tikzpicture}
\end{center}
\end{subfigure}

\caption{The intersection of balls depicted in figure \ref{fig:chech_vr_complex} (b) and the Voronoi cells from figure \ref{fig:voronoi}.}
\label{fig:alpha_complex}
\end{figure}

Now for any of these constructions we can use different parameters $\epsilon_0,\dots,\epsilon_n$ to obtain a sequence of nested simplicial complexes, i.e., a filtration. 

If the number of simplices at some $\epsilon_i$ actually increases, which only happens finitely many times, we can assign each new simplex the value $\epsilon_i$ to get a function $f:K\rightarrow \mathbb{R}$, which describes the given filtration, via the sub level sets as discussed before. 


\section{Simplicial Collapses}
\label{sec:simplicial_collapses}
We conclude this chapter on simplicial complexes by introducing the notion of simplicial collapses. An early reference to these deformations can be found in \cite{whitehead}.

\begin{defi}
Let $K$ be a simplicial complex and let $\sigma,\tau \in K$. Let $\sigma$ be a facet of $\tau$. We call $\sigma$ a \textbf{free face} if the following conditions are fulfilled:
\begin{itemize}
    \item $\sigma$ has no cofacet except $\tau$
    \item $\tau$ has no cofacet
\end{itemize}
\end{defi}

\begin{defi}
Let $K$ be a simplicial complex and let $\sigma$ be a free face with coface $\tau$. 

Define $C_{\sigma} \coloneqq \{\gamma \in K \mid \sigma \leq \gamma \leq \tau\}$. Constructing $K' \coloneqq K \setminus C_{\sigma}$ by removing the simplices in $C_\sigma$ from $K$ is called a \textbf{simplicial collapse}. If $\operatorname{dim}(\sigma) = \operatorname{dim}(\tau)-1$, we call the collapse \textbf{elementary}. We denote an (elementary) simplicial collapse by $K \searrow_{\sigma} K'$.
\end{defi}

Note that $K'$ is a subcomplex of $K$. In the following example, Figure \ref{fig:collapses}, we start with a triangle as simplicial complex $K$ and successively collapse free faces until we have reached a single point.

\begin{figure}[H]
%\centering%
\begin{subfigure}[c]{0.45\textwidth}
\begin{center}
\begin{tikzpicture}

\node[b_circle, name path=0, label=below:{}] (0) at (0,0) {};

\node[b_circle, name path=1, label=below:{}] (1) at (2,0) {};

\node[b_circle, name path=2, label=above:{}] (2) at  (1,2) {};



\path[-, draw] (0) to[] node[anchor = center, below] {$\sigma_3$} (1);
\path[-, draw] (1) to[] node[anchor = center, right] {$\sigma_2$} (2);
\path[-, draw] (0) to[] node[anchor = center, left] {$\sigma_1$} (2);

\fill[opacity = 0.2] (0,0) -- (2,0) -- (1,2);
\end{tikzpicture}

\subcaption{Simplicial complex $K$ with free faces $\sigma_1,\sigma_2$ and $\sigma_3$.}
\end{center}
\end{subfigure}
\begin{subfigure}[c]{0.45\textwidth}
\begin{center}
\begin{tikzpicture}

\node[b_circle, name path=0, label=below:{$v_1$}] (0) at (0,0) {};

\node[b_circle, name path=1, label=below:{$v_2$}] (1) at (2,0) {};

\node[b_circle, name path=2, label=above:{}] (2) at  (1,2) {};


\path[-, draw] (1) to[] node[anchor = center, right] {$\sigma_2$} (2);
\path[-, draw] (0) to[] node[anchor = center, left] {$\sigma_1$} (2);

\end{tikzpicture}

\subcaption{Result of $K \searrow_{\sigma_3} K'$. Now $v_1$ and $v_2$ are free faces, with cofacets $\sigma_1$ and $\sigma_2$ respectively.}
\end{center}
\end{subfigure}
\begin{subfigure}[c]{0.45\textwidth}
\begin{center}
\begin{tikzpicture}

\node[b_circle, name path=0, label=below:{$v_1$}] (0) at (0,0) {};


\node[b_circle, name path=2, label=above:{$v_0$}] (2) at  (1,2) {};

\path[-, draw] (0) to[] node[anchor = center, left] {$\sigma_1$} (2);

\end{tikzpicture}

\subcaption{Result of $K' \searrow_{v_2} K''$. Now $v_0$ and $v_1$ are free faces.}
\end{center}
\end{subfigure}
\begin{subfigure}[c]{0.45\textwidth}
\begin{center}
\begin{tikzpicture}

\node[] (0) at (0,0) {};

\node[b_circle, name path=2, label=above:{$v_0$}] (2) at  (1,2) {};

\end{tikzpicture}

\subcaption{Result of $K'' \searrow_{v_1} K'''$. }
\end{center}
\end{subfigure}
\caption{A sequence of elementary collapses.}
\label{fig:collapses}
\end{figure}

If there is a sequence of collapses, such that some simplicial complex $K$ collapses to a point by following that sequence, we say that $K$ is a \textbf{collapsible}.

In topology, a space is \textbf{contractible}, if it is homotopy-equivalent to a point. Every collapsible complex is contractible. However contractible does not imply collapsible as the famous example of the \enquote{Dunce hat}, see Figure \ref{fig:dunce}, illustrates. 
More on this topic can be found in \cite{cohenhom}.

\begin{figure}[H]
%\centering%
\begin{subfigure}[c]{0.95\textwidth}
\begin{center}
\begin{tikzpicture}
\node[b_circle, name path=1, label=above:{$0$}] (0t) at (3,5.2) {};
\node[b_circle, name path=1, label=below left:{$0$}] (0l) at (0,0) {};
\node[b_circle, name path=1, label=below right:{$0$}] (0r) at (6,0) {};

\node[b_circle, name path=1, label=below:{$1$}] (1b) at (2,0) {};
\node[b_circle, name path=1, label=left:{$1$}] (1l) at ($(0l)!0.33!(0t)$) {};
\node[b_circle, name path=1, label=right:{$1$}] (1r) at ($(0r)!0.33!(0t)$) {};

\node[b_circle, name path=1, label=below:{$2$}] (2b) at (4,0) {};
\node[b_circle, name path=1, label=left:{$2$}] (2l) at ($(0l)!0.66!(0t)$) {};
\node[b_circle, name path=1, label=right:{$2$}] (2r) at ($(0r)!0.66!(0t)$) {};

\node[b_circle, name path=1, label=above:{}] (3) at ($(1l)!0.33!(1b)$) {};
\node[b_circle, name path=1, label=above right:{}] (4) at ($(1l)!0.66!(1b)$) {};

\node[b_circle, name path=1, label=above right:{}] (5) at ($(2l)!0.33!(2r)$) {};
\node[b_circle, name path=1, label=below: {}] (6) at ($(2l)!0.66!(2r)$) {};

\node[b_circle, name path=1, label=above:{}] (7) at ($(2b)!0.5!(1r)$) {};

\path[-,draw] (0t) to (2l) to (1l) to (0l) to (1b) to (2b) to (0r) to (1r) to (2r) to (0t);
\path[-,draw] (1l) to (3) to (4) to (1b);
\path[-,draw] (2b) to (7) to (1r);
\path[-,draw] (2l) to (5) to (6) to (2r);
\path[-,draw] (1r) to (6);
\path[-,draw] (0t) to (5);
\path[-,draw] (0t) to (6);
\path[-,draw] (0l) to (3);
\path[-,draw] (0l) to (4);
\path[-,draw] (1l) to (5);
\path[-,draw] (3) to (5);
\path[-,draw] (4) to (5);
\path[-,draw] (4) to (7);
\path[-,draw] (2b) to (4);
\path[-,draw] (7) to (5);
\path[-,draw] (7) to (6);
\path[-,draw] (7) to (0r);

\fill[opacity = 0.2] (0t.center)--(0l.center)--(0r.center)--(0t.center);
\end{tikzpicture}
\end{center}
\end{subfigure}
\caption{The dunce hat.}
\label{fig:dunce}
\end{figure}

In Figure \ref{fig:dunce} vertices with the same label indicate that the respective vertices are identified. In this example the sides of the triangle we see here are therefore glued together.

It is easy to verify that the dunce hat has no free face and therefore is not collapsible. However it is contractible, c.f. \cite{ZEEMAN1963341}.

\newpage
%\newpage\null\thispagestyle{empty}\newpage
\chapter{Homology}
\label{ch:homology}
\section{Homology Groups}
This section is mainly based on \cite{Computational+Topology}[Chapter IV]. Homology groups are topological invariants that describe holes in topological spaces. The topological spaces we will consider are simplicial complexes. Restricting the theory in this way is also referred to as simplicial homology. We will begin by introducing chains of simplices. A handy notation we will use in the following definition is to write $\sigma^{(k)} \in K$ for a $k$-dimensional simplex in some simplicial complex $K$. All concepts we will discuss work just as well when we include the empty face but since nothing that is of interest to us happens, when including it, we will omit it for brevity. Strictly speaking in all our examples and concepts we will discuss simplicial complexes without the empty face. Note that including the empty face yields reduced homology theory which is different on an algebraic level. 

\begin{defi}
Let $K$ be a simplicial complex. Let $K^{(k)} \coloneqq \{\sigma^{(k)} \in K\}$ be the set of $k$-simplices in $K$. A formal sum \[
c = \sum_{i=1}^{|K^{(k)}|} a_i \sigma_i
\]
with $a_i \in \mathcal{R}$, where $\mathcal{R}$ is a ring, and $\sigma_i \in K^{(k)}$, $i \in 1,\dots,|K^{(k)} |$, is called a \textbf{$\bm{k}$-chain}. We denote the set of all $k$-chains of $K$ by $C_k(K)$.
\end{defi}

In general, we will omit the bounds of the sum and just write $\sum a_i \sigma_i$. Furthermore we will omit the reference to the simplicial conmplex $K$ and just write $C_k$.

\begin{defi}
We define an addition on the set of k-chains $C_k$: \[
+: C_k \times C_k \rightarrow C_k, \]\[
\sum a_i \sigma_i + \sum b_i \sigma_i \coloneqq \sum (a_i+b_i) \sigma_i.
\] 
\label{def:addition_of_kchains}
\end{defi}

\begin{lemma}
The k-chains together with the above defined addition form an abelian group that we denote as $(C_k,+)$
\end{lemma}
\begin{proof}
Associativity follows from the associativity of the coefficients from the ring $\mathcal{R}$. The neutral element is $0 \coloneqq \sum 0\sigma^{(k)}_i$ and each chain $c = \sum a_i\sigma^{(k)}_i$ has an inverse $-c \coloneqq \sum (-a_i)\sigma^{(k)}_i$. The group is abelian, since addition in $\mathcal{R}$ is abelian.
\end{proof}

As mentioned above, we want to describe holes in a simplicial complex. We do this by considering their boundary. The following definition introduces a boundary operator for simplices that naturally extends to k-chains. 

\begin{defi}
Let $\sigma^{(k)} = [s_0,\dots,s_k]$ be some simplex. We define the \textbf{boundary operator}
\[
\partial_k \sigma \coloneqq \sum_{j=0}^{k}(-1)^j[s_0,\dots,\hat{s_j},\dots,s_k]
\]
where $[s_0,\dots,\hat{s_j},\dots,s_k]$ is the facet of $\sigma$ that does not contain $s_j$. For a $k$-chain $c = \sum a_i \sigma_i^{(k)}$ the boundary operator is defined as
\[
\partial_k c = \sum a_i \partial_k \sigma_i.
\]
\end{defi}
The boundary operator maps $k$-chains to $(k-1)$-chains, and since 
\[\partial_k (c_1 + c_2) = \partial_k c_1 + \partial_k c_2\]
the boundary operator 
\[
    \partial_k: C_k \rightarrow C_{k-1}
\]
is a group homomorphism between the groups of $k$-chains and $(k-1)$-chains. We can now introduce $k$-cycles and $k$-boundaries and their respective groups. 

\begin{defi}
Let $c$ be a $\bm{k}$\textbf{-chain}. If $\partial_kc = 0$, then $c$ is called a $\bm{k}$\textbf{-cycle}. Consider a $(k+1)$-chain $r$. If $c = \partial_{k+1}r$, then $c$ is called a $\bm{k}$\textbf{-boundary}. 
\end{defi}

By definition, $\partial_k$ commutes with addition and hence we get the following groups.
The group of $\bm{k}$\textbf{-cylces} denoted as $Z_k = Z_k(K)$ and the group of $\bm{k}$\textbf{-boundaries} $B_k = B_k(K)$. Both $Z_k$ and $B_k$ are subgroups of $C_k$. Another perspective on these groups is to interpret them as kernel and image of $\partial$, i.e. $Z_k = \operatorname{ker}(\partial_k)$ and $B_k = \operatorname{im}(\partial_{k+1})$.

Since $C_k$ is abelian, so are $Z_k$ and $B_k$. The following fundamental lemma further specifies the relation between $C_k$ and $B_k$.

\begin{lemma}[Fundamental Lemma of Homology]
$\partial_k \partial_{k+1} c = 0$ for $k \in \mathbb{N}_0$ and $(k+1)$-chain $c$. 
\end{lemma}
\begin{proof}
It suffices to show that $\partial_k \partial_{k+1} \tau = 0$ for every $(k+1)$-simplex $\tau$ since a $(k+1)$-chain is just a sum of $(k+1)$-simplices. 

Let $\tau = [s_0,\dots,s_{k+1}]$. Consider the boundary $\partial_{k+1}\tau$. It consists of all $k$-faces of $\tau$.

Let $\gamma = [s_0,\dots,\hat{s_i},\dots,\hat{s_j},\dots,s_{k+1}]$, i.e., a $(k-1)$-face. It is adjacent to the two faces $k$-faces \[\sigma_i = [s_0,\dots,\hat{s_i},\dots,s_{k+1}] \text{ and }\sigma_j = [s_0,\dots,\hat{s_j},\dots,s_{k+1}]\]
Considering the different possibilities for $i$ and $j$ to be even or odd, a straightforward calculation yields that $\gamma$ cancels out in the sum corresponding to $\partial_k \partial_{k+1} \tau$. Hence $\partial_k \partial_{k+1} \tau = 0$.
\end{proof}

From the lemma we immediately get that $B_k$ is a subgroup of $Z_k$. There is a popular way of depicting these group relations, found in most literature concerned with homology. The following Figure \ref{fig:group_relations} is our version of it.

\begin{figure}[H]
%\centering%
\begin{subfigure}[c]{0.49\textwidth}
\begin{center}
\begin{tikzpicture}

\draw (-4,0) ellipse (1 and 2);
\fill[opacity = 0.1]  (-4,0) ellipse (1 and 2);
\draw (0,0) ellipse (1 and 2);
\fill[opacity = 0.1] (0,0) ellipse (1 and 2);
\draw (4,0) ellipse (1 and 2);
\fill[opacity = 0.1]  (4,0) ellipse (1 and 2);

\draw (-4,-0.6) ellipse (0.7 and 1.4);
\fill[opacity = 0.1]  (-4,-0.6) ellipse (0.7 and 1.4);
\draw (0,-0.6) ellipse (0.7 and 1.4);
\fill[opacity = 0.1]  (0,-0.6) ellipse (0.7 and 1.4);
\draw (4,-0.6) ellipse (0.7 and 1.4);
\fill[opacity = 0.1] (4,-0.6) ellipse (0.7 and 1.4);

\draw (-4,-1.2) ellipse (0.5 and 0.8);
\fill[opacity = 0.1] (-4,-1.2) ellipse (0.5 and 0.8);
\draw (0,-1.2) ellipse (0.5 and 0.8);
\fill[opacity = 0.1] (0,-1.2) ellipse (0.5 and 0.8);
\draw (4,-1.2) ellipse (0.5 and 0.8);
\fill[opacity = 0.1] (4,-1.2) ellipse (0.5 and 0.8);

\node[circle, name path=1, label=below:{0}] (1) at  (-4,-2) {};
\node[circle, name path=1, label=below:{0}] (2) at  (0,-2) {};
\node[circle, name path=1, label=below:{0}] (3) at  (4,-2) {};

\node[label=below:{$C_{p+1}$}] at (-4,2) {};
\node[label=below:{$C_{p}$}] at (0,2) {};
\node[label=below:{$C_{p-1}$}] at (4,2) {};

\node[label=below:{$Z_{p+1}$}] at (-4,0.7) {};
\node[label=below:{$Z_{p}$}]  at (0,0.7) {};
\node[label=below:{$Z_{p-1}$}] at (4,0.7) {};

\node[label=below:{$B_{p+1}$}] at (-4,-0.5) {};
\node[label=below:{$B_{p}$}] at (0,-0.5) {};
\node[label=below:{$B_{p-1}$}] at (4,-0.5) {};

\node[label=below:{$\partial_{p+2}$}] at (-5,-2) {};
\node[label=below:{$\partial_{p+1}$}] at (-2,-2) {};
\node[label=below:{$\partial_{p}$}] at (2,-2) {};
\node[label=below:{$\partial_{p-1}$}] at (5,-2) {};

\draw [-,thick] (-3.8,0.75) -- (2);

\draw [-,thick] (0.2,0.75) -- (3);

\draw [-,thick] (-3.75,1.95) -- (0.1,-0.4);

\draw [-,thick] (0.25,1.95) -- (4.1,-0.4);


\draw [->-,very thick] (1) to (2);
\draw [->-,very thick] (2) to (3);

\draw [->-,very thick] (-6,-2) to (1);
\draw [->-,very thick] (3) to (6,-2);

\end{tikzpicture}
\end{center}
\end{subfigure}
\caption{Connections between cycle and boundary groups, established by the group homomorphisms $\partial$.}
\label{fig:group_relations}
\end{figure}


Since $B_k$ is a subgroup of $Z_k$, we are allowed to take the group quotient $Z_k/B_k$. This leads us to the definition of Homology groups and Betti numbers.
\begin{defi}
We define the  $\bm{k}$\textbf{-th}  \textbf{Homology group} as
\[
    H_k \coloneqq Z_k/B_k
\]
and the $\bm{k}$\textbf{-th} \textbf{Betti number} as \[ \beta_k \coloneqq \operatorname{rank}(H_k).\]
\end{defi}

What we will see is that, informally speaking, $\beta_k$ counts the number of $k$-dimensional holes in the underlying simplicial complex, except for $\beta_0$, which is counting the connected components.

The $k$-th homology group can be interpreted as the partitioning of the $k$-th cycle group $Z_k$ into classes of cycles that differ by boundaries of the $k$-th boundary group $B_k$. From algebra we know, that $H_k = \{zB_k \mid z \in Z_k\}$ forms a group with \[+: H_k \times H_k \rightarrow H_k, z_1B_K+z_2B_k = (z_1+z_2)B_k\] where the $+$ on the right hand side is the addition of $k$-chains defined in \ref{def:addition_of_kchains}. 

From here on out we will restrict our coefficients to be from the field $\mathbb{Z}_2$, since this suffices for all further concepts discussed in this work and it simplifies some definitions and calculations. For example the coefficients of $k$-chains now just indicate if some simplex is contained in the chain or not. Furthermore we can drop the $(-1)^j$ in the definition of the boundary operator, since in $\mathbb{Z}_2$ we have $-1 = 1$. As we will discuss later on it also simplifies the standard algorithm to compute homology groups.

In the following example we will calculate the Betti number $\beta_1$ of the simplicial complex depicted in the following Figure \ref{fig:simplicial_homology_example} by computing the factorization of $Z_1$ by $B_1$. The classes resulting from this factorization are called \textbf{homology classes} and any two elements within one class are called homologous. Some of the simplifications arising from choosing coefficients from $\mathbb{Z}_2$ will become apparent.

\begin{figure}[H]
%\centering%
\begin{subfigure}[c]{0.99\textwidth}
\begin{center}
\begin{tikzpicture}

\node[b_circle, name path=0, label=left:{$v_0$}] (0) at (0,1) {};
\node[b_circle, name path=1, label=below:{$v_1$}] (1) at (2,0) {};
\node[b_circle, name path=2, label=above:{$v_2$}] (2) at  (2,2) {};
\node[b_circle, name path=2, label=right:{$v_3$}] (3) at  (4,1) {};

%\node[inner sep = 0pt,label=below left:{$e_0$}] at  ($(0)!0.5!(1)$) {};
%\node[inner sep = 0pt,label=above left:{$e_1$}] at  ($(0)!0.5!(2)$) {};
%\node[inner sep = 0pt,label=left:{$e_3$}] at  ($(1)!0.5!(2)$) {};
%\node[inner sep = 0pt,label=below right:{$e_4$}] at  ($(3)!0.5!(1)$) {};
%\node[inner sep = 0pt, label=above right:{$e_5$}] at  ($(3)!0.5!(2)$) {};

\draw[-, thick] (0) -- (1);
\draw[-, thick] (0) -- (2);
\draw[-, thick] (1) -- (2);
\draw[-, thick] (2) -- (3);
\draw[-, thick] (3) -- (1);

\fill[opacity = 0.2] (0,1) -- (2,0) -- (2,2);

\end{tikzpicture}
\end{center}
\end{subfigure}
\caption{A simplicial complex.}
\label{fig:simplicial_homology_example}
\end{figure}

To calculate $H_1$ we need to know what $B_1$ and $Z_1$ are. Let $t_1$ be the triangle with vertices $v_0, v_1$ and $v_2$. Then $\partial_2(t_1) = [v_0,v_1] + [v_0,v_2] + [v_1,v_2]$. Since $t_1$ is the only 2-simplex we get
\[
B_1 = \{0, [v_0,v_1]+[v_0,v_2]+[v_1,v_2]\}.
\]
To compute the cycle group, we need to find all 1-chains that are mapped to $0$ by $\partial_1$, i.e. we have to compute the kernel. We can do this by solving:
\begin{equation*}
    \begin{split}
         0 &= a_0\partial_1[v_0,v_1] + a_1\partial_1[v_0,v_2] + a_2\partial_1[v_1,v_2] + a_3\partial_1[v_1,v_3] +a_4\partial_1[v_2,v_3]\\
    &= a_0(v_0+v_1) + a_1(v_0+v_2) + a_2(v_1+v_2) + a_3(v_1+v_3) + a_4(v_2+v_3) \\
    &= v_0(a_0+a_1) + v_1(a_0+a_2+a_3) + v_2(a_1+a_2+a_4) + v_3(a_3+a_4)
    \end{split}
\end{equation*}
for coefficients $a_i \in \mathbb{Z}_2$. So we get the set of equations: 
\begin{center}
    \begin{tabular}{ccl}
        $(1)$ & $0$ & $= a_0+a_1$\\
        $(2)$ & $0$ & $= a_0+a_2+a_3$\\
        $(3)$ & $0$ & $= a_1+a_2+a_4$\\
        $(4)$ & $0$ & $= a_3+a_4$
\end{tabular}
\end{center}

From equation $(1)$ we get that $a_0 = a_1$ and from $(4)$ we get that $a_3 = a_4$. Therefore equations $(2)$ and $(3)$ tell us if $a_2 = 0$ then $a_0 = a_3$ and $a_1 = a_4$. This implies two possible cycles. Namely the $0$-cycle and \[z_1 = [v_0,v_1]+[v_1,v_3]+[v_2,v_3]+[v_0,v_2].\]
On the other hand. If $a_2 = 1$, we get that either $a_0 = a_1 = 1$ or $a_3 = a_4 = 1$, which implies the cycles 
\[
    z_2 = [v_0,v_1]+[v_0,v_2]+[v_1,v_2]
\]
and 
\[
   z_3 = [v_1,v_2]+[v_1,v_3]+[v_2,v_3].
\]
These are all possible cycles, so overall we get \[
Z_1 = \{0, z_1, z_2, z_3\}.
\]
By adding all elements of $B_1$ to each element of $Z_1$ we get
\begin{center}
\begin{tabular}{rclcl}
    $0+B_1$ &=& $\{0,z_2\}$ &&\\
    $z_1+B_1$&=& $\{z_1 + 0, z_1+z_2\}$&=&$\{z_1,z_3\}$\\
    $z_2+B_1$&=& $\{z_2 + 0, z_2+z_2\}$&=&$\{z_2,0\}$\\
    $z_3+B_1$&=& $\{z_3 + 0, z_3+z_2\}$&=&$\{z_2,z_3\}$
\end{tabular}
\end{center}
This means, that $Z_1$ is partitioned into two classes by factorisation with $B_1$. Namely $\{\{0,z_2\},\{z_1,z_3\}\}$. Now we choose elements representing both classes in our example. We will pick $[0]$ and $[z_3]$ and hence \[
H_1 = \{[0],[z_3]\}.
\]
Therefore $\beta_1 = \operatorname{rank}(H_1) = 1$, which corresponds to the one-dimensional hole bounded by the edges between vertices $v_1,v_2$ and $v_3$. 

\section{Boundary Matrices and Smith Normal Form}
\label{sec:boundary_matrices_and_smith_normal_form}
At the end of the last section we did the calculations by hand to illustrate the discussed concepts. In practice, however, we are often only concerned with the Betti numbers and not with the different homology groups. Therefore we only need to compute the ranks of $Z_k$ and $B_k$. As we will see, we can encode $\partial$ as a matrix over $\mathbb{Z}_2$. Then we can use standard linear algebra to extract the desired information. For more details we refer the reader to \cite{Computational+Topology}[Chapter IV.2], which this section is based on. 

\begin{defi}
Let $K$ be a simplicial complex and assume an arbitrary but fixed ordering of the simplices. Let $n_{k-1}$ be the number of $(k-1)$-simplices and $n_k$ be the number of $k$-simplices. Let $i \in \{1,\dots,n_{k-1}\}$ and $j \in \{1,\dots,n_k\}$, then we define a matrix $\partial_k$ = $[a_{ij}]$ with $a_{ij} = 1$, if the $i$-th $(k-1)$-simplex is a face of the $j$-th $k$-simplex. Otherwise $a_{ij} = 0$. This matrix is called the \textbf{$\bm{k}$-th boundary matrix}. 
\end{defi}

The 1-st boundary matrix of the example in Figure \ref{fig:simplicial_homology_example} is the following \[
\begin{blockarray}{cccccc}
\label{mat:small}
& [v_0,v_1] & [v_0,v_2] & [v_1,v_2] & [v_1,v_3] & [v_2,v_3]\\
\begin{block}{c[ccccc]}
  \left[v_0\right]  & 1 & 1 & 0 & 0 & 0\\*
  \left[v_1\right]  & 1 & 0 & 1 & 1 & 0\\*
  \left[v_2\right]  & 0 & 1 & 1 & 0 & 1\\*
  \left[v_3\right]  & 0 & 0 & 0 & 1 & 1\\*
\end{block}
\end{blockarray}
\]

The rows of the $k$-th boundary matrix, form a basis of $C_{k-1}$ and the columns form a basis of $C_k$. From linear algebra we know that we can exchange and add rows and columns, without changing the rank of the matrix. Furthermore, as is illustrated in \cite{Computational+Topology}[Chapter IV] these alterations always yield rows and columns that represent a basis of the $k$ and $k-1$-chains.

The standard algorithm to extract the ranks of the chain groups is to reduce the boundary matrix to the so called \textbf{Smith normal form} (SNF). 

In the general case, the computation of the SNF allows to read off the Betti numbers. In case that the chosen ring $\mathcal{R}$ is a field, the groups $Z_k = \operatorname{ker}\partial_k$ and $B_k = \operatorname{im}\partial_{k+1}$ can be obtained by standard Gaussian elimination. In particular, this applies to the choice $\mathcal{R} = \mathbb{Z}_2$.
%In the general case, where the coefficients of our chains are not from $\mathbb{Z}_2$ but some ring, this is more complicated and we will not discuss it here. \\ If we were interested in the so called torsion coefficients of the homology groups however, we would need to consider coefficients from $\mathbb{Z}$ or some other ring. \\
%In our case the Smith normal form is just a matrix for which an initial segment of the diagonal is one and all other entries are zero. We can use Gaussian Elimination to get an upper triangular matrix and eliminate the remaining ones, that are not on the diagonal, by row operations afterwards.\\

Now consider some $k$-th boundary matrix $M_k$. And let $\hat{M}_k$ be the matrix in Smith normal form, we get by reducing $M_k$ via row and column operations. As stated, the number of columns $n_k$ is the rank of $C_k$. Now let \[n_k = b_{k-1} + z_k,\]
where $b_{k-1}$ is equal to the number of columns containing a one in $\hat{M}_k$ and $z_k$ the number of columns which have only zeroes for entries. The leftmost $b_{k-1}$ columns of $\hat{M}_k$ represent $k$-chains that form $(k-1)$-boundaries and the rightmost $z_k$ columns of $\hat{M}_k$ represent $k$-cycles that generate $Z_k$. This means the rank of $Z_k$ is $z_k$ and the rank of $B_{k-1}$ is $b_{k-1}$. Therefore, if we have all the boundary matrices of some simplicial complex $K$ and reduce them to Smith normal form, we get the Betti numbers $\beta_k = \operatorname{rank}(Z_k) - \operatorname{rank}(B_k)$. 

\section{Persistent Homology}

Persistent homology is one of the central concepts in Topological Data Analysis. It enables us to extract and describe the topological structure of high-dimensional spaces. 
%A notable practical example was published in \cite{Nicolau7265}, where persistent homology was used to identify a subgroup of breast cancer tumors, undetectable by classical clustering approaches. \\

In the last section, we established a way to talk about holes in simplicial complexes. We already established the notion of a filtration in Definition \ref{def:filtration}. What we will now discuss is a theory that allows us to talk about how holes behave with respect to the sequence of complexes we get from a filtration. 
Recall that the sequence encoded in a filtration is increasing with respect to inclusion. This section is based on \cite{persistence_and_simplification}[Chapter VII] and \cite{pershom}.

Consider some filtration $F = \{K_0,\dots,K_n\}$, we denote the set of their chain groups by $C_*^i$, for $i \in 0,\dots,n$. Similarly we denote the corresponding cycle and boundary groups by $Z_*^i$ and $B_*^i$ respectively. 

For $i,j \in 0, \dots, n$, $i \leq j$, we have an inclusion map from $K_i$ to $K_j$, and therefore for every dimension $k$ we have an induced group homomorphism 
\[h_k^{i,j}: H_k(K_i) \rightarrow H_k(K_j).\]

This means that the sequence of simplicial complexes gives us a sequence of homology groups connected by group homomorphisms, 
\[
0 = H_k(K_0) \rightarrow \dots \rightarrow H_k(K_n) =  H_k(K).
\]

\begin{defi}
For $0 \leq i \leq j \leq n$, we refer to the factor groups \[
    H_k^{i,j} = \frac{Z_k^i}{Z_k^i \cap B_k^{j}}
\] as the \textbf{$\bm{k}$-th persistent homology groups}. The ranks of these groups are the \textbf{$\bm{k}$-th persistent Betti numbers}.
\end{defi}

These groups are well-defined, since the denominator is an intersection of two subgroups of $C_k^{i+p}$ and hence a subgroup of the numerator.
We will refer to the set of chain groups $C_*^i$ as the $\bm{i}$\textbf{-th frame}.

\begin{defi}
Let $\gamma \in H_k(K_i)$. We say $\gamma$ is \textbf{born} in frame $i$, if $\gamma \notin H_k^{i-1,i}$. We say $\gamma$ \textbf{dies} in frame $j$, if 
\[h_k^{i,j-1}(\gamma) \notin H_k^{i-1,j-1} \text{  but  } h_k^{i,j}(\gamma) \in H_k^{i-1,j}.\] 
We call the difference $j-i$ the \textbf{index persistence}. If some simplex never dies, its index persistence equals infinity.
\end{defi}

When considering some function $f: K \rightarrow \mathbb{R}$, that defines a filtration via sublevel sets of $a_0,\dots,a_n$, we call the difference $a_j - a_i$ the \textbf{persistence} of the classes born at $a_i$ and dying at $a_j$.

We now have a notion of how topological features of our space persist through different steps of a filtration of some simplicial complex $K$. In the following section we will introduce the standard algorithm with which persistent homology can be computed. 

\section{Matrix Reduction}
\label{sec:matrix_reduction}
The standard algorithm to compute persistent homology is a variant of the matrix reduction, used to compute homology via the Smith normal form. 

% In the following we will talk about simplexwise filtrations since it simplifies some formulations, but we would like to point out, that any filtration induces a simplexwise filtration in the following sense. Let $F = \{K_0,\dots,K_n\}$ be a filtration. Let $\{\sigma_0^0, \dots ,\sigma_{l_0}^0\}$ be the elements of $K_0$ in lexicographical order. Let $\{\sigma_0^i, \dots ,\sigma_{l_i}^i\} = K_i \setminus K_{i-1}$ for $i \in 1, \dots,n$ in lexicographical order. Then 
% \[F_* = (\sigma_0^0, \dots ,\sigma_{l_0}^0, \dots, \sigma_0^n, \dots ,\sigma_{l_n}^n)\]
% is a simplexwise filtration. \\

Consider a simplexwise filtration $F_*$ of some simplicial complex $K$. Let $\sigma_0,\dots,\sigma_m$ be the ordering of all simplices in $K$, induced by $F_*$. Due to $F_*$ being a filtration, we know that $\sigma_i < \sigma_j$ implies $i<j$. Note that every subsequence of simplices starting at $\sigma_0$ is a simplicial complex and an element of $F_*$. We define the boundary matrix \[
\partial[i,j] = \begin{cases} 
      1 & \text{if $\sigma_i$ is a facet of $\sigma_j$,} \\
      0 & \text{else.} 
   \end{cases}
\]

This matrix contains the same information as all of the $k$-th boundary matrices combined. Each column and row correspond to one simplex. Consider the column corresponding to $\sigma_j$, i.e., column $j$. Every row that has a one as entry in column $j$ corresponds to a facet of $\sigma_j$.
Conversely, the ones in some row $i$ correspond to cofacets of $\sigma_i$.

\begin{defi}
Consider some matrix $M$ with $m+1$ rows and elements in $\mathbb{Z}_2$.
For each column $j$ we define \[
\operatorname{low}(j) = \begin{cases} 

      \operatorname{max}\{i \in 0,\dots,m \mid M[i,j] = 1\} & \text{if column $j$ is non-zero,} \\
      -1 & \text{else.} 
      
   \end{cases}
\]
\end{defi}

The operator $\operatorname{low}()$ gives us the index of the \textbf{lowest $\bm{1}$} of a column of a matrix, if it exists. What the following algorithm will do, is reduce the boundary matrix $\partial$ to a matrix, such that for any two columns $i$ and $j$, it holds that $\operatorname{low}(i) \neq \operatorname{low}(j)$, if $i \neq j$ and if the respective lowest entries exist. 

We will refer to this matrix as the reduced matrix and denote it as $R$. Simultaneously, the algorithm will keep track of the column additions we need to reduce the boundary matrix. We can store this information in a matrix which we will call $A$ for addition matrix. We will write $R[*,j]$, when referring to column $j$ of matrix $R$. Furthermore we write $\operatorname{low}_R()$, to indicate which lowest elements we are interested in. 

\begin{algorithm}[H]
\SetKwData{Left}{left}\SetKwData{This}{this}\SetKwData{Up}{up}\SetKwFunction{Union}{Union}\SetKwFunction{FindCompress}{FindCompress}\SetKwInOut{Input}{Input}\SetKwInOut{Output}{Output}
\Input{boundary matrix $\partial$}
\Output{reduced matrix $R$ and addition matrix $A$}

$R = \partial$\\
$A = I_m$\\
\For{j = 1,\dots,m}
{
    \While{$\operatorname{low}_R(i) \neq -1$ \text{\textbf{and}}\\ there exists $i<j$, with $\operatorname{low}_R(i) = \operatorname{low}_R(j)$ }
    {
        $R[*,j] =  R[*,j] + R[*,i]$\\
        $A[i,j] = 1$ 
    }
}
\textbf{return} $R,A$
\caption{Column reduction algorithm}
\label{algo:column_reduction_algorithm}
\end{algorithm} 
\vspace{0.5cm}

In lines 1 and 2 we just initialize our matrices. Line 3 starts a for-loop over all columns, which is in $\mathcal{O}(m)$. The while loop from line 4 to 8 is also in $\mathcal{O}(m)$. After the while loop, the column $R[*,j]$ is either all zero, or has a lowest entry differing from all lowest entries of preceding columns. To achieve this, we add columns $R[*,i]$ and $R[*,j]$, if they have the same lowest entry. This also runs in $\mathcal{O}(m)$ since each column has $m$ entries. Overall we get a cubic running time in the worst case. 

After adding two columns we are guaranteed to have differing lowest entries for them. Also the index of the lowest entry of a column decreases with each step or the column becomes all zero. Hence the while loop terminates. The matrices we get returned in line 10 fulfill: \[
    R = \partial A.
\]

\begin{defi}
\label{def:col_red_steps}
For a boundary matrix $\partial$ and a column $j$ representing simplex $\tau$ we denote the number of iterations of the while loop starting in line $4$ of Algorithm \ref{algo:column_reduction_algorithm} by $\operatorname{red}(\tau)$. We refer to this number as the number of \textbf{column reduction steps of} $\bm{\tau}$. We call the sum of all column reduction steps over all columns the \textbf{column reduction steps of} $\bm{B}$ and denote it by $\operatorname{red}(B)$.
\end{defi}

The column reduction steps do not count the total number of addition operations needed during the reduction of a column or boundary matrix. It only counts how often columns get added. In each individual column addition there can be $\mathcal{O}(n)$ addition operations needed to add the columns. We discuss an example later on in the section. \\

Similarly to what we discussed in Section \ref{sec:boundary_matrices_and_smith_normal_form}, we can extract the Betti numbers of the simplicial complex $K$ via the ranks of the cycle and boundary groups of $K$. The rank of $Z_k$ is the number of zero columns corresponding to $k$-simplices in the reduced boundary matrix. We will denote these by $\#Zero_k$. The rank of $B_k$ equals the number of rows corresponding to $k$-simplices that contain a lowest $1$. We denote it by $\#Low_k$. Therefore for the $k$th Betti number we get \[
\beta_k = \#Zero_k - \#Low_k.
\]

This however is not the primary goal of our algorithm. We are interested in the appearance and disappearance of cycles in our persistent homology groups. The lowest entries in our reduced matrix define a pairing between simplices. Namely, if $\operatorname{low}_R(j)= i$, the simplex corresponding to row $i$ is paired with the simplex corresponding to column $j$. We will explain what these pairings mean, after formulating a lemma, which shows that this pairing depends on the boundary matrix $\partial$ and not on $R$, which is characterized by being reduced and calculated by left to right column operations. To this end we introduce the following notations.

The lower left sub matrix of some matrix $M$ with top right corner element $M[i,j]$ is denoted as $M_i^j$, i.e., $M_i^j$ consists of the first $j$ columns of $M$ and the last $m-i$ rows. Furthermore, we define \[
r_\partial(i,j) \coloneqq \operatorname{rank}(\partial_i^j) - \operatorname{rank}(\partial_{i+1}^{j}) + \operatorname{rank}(\partial_{i+1}^{j-1}) - \operatorname{rank}(\partial_{i}^{j-1}).
\]
Note that the left to right column operations we do in our algorithm do not change the rank of a matrix, or a submatrix for that matter. 

Therefore we get that $\operatorname{rank}(\partial_i^j) = \operatorname{rank}(R_i^j) $. Hence $r_\partial(i,j) = r_R(i,j)$ for all $i$ and $j$. The following lemma and proof follow \cite{Computational+Topology}[VII.1] closely.

\begin{lemma}[Pairing lemma]

It holds that $i = \operatorname{low}_R(j)$ if and only if $r_\partial(i,j) = 1$. This means the pairings defined by the lowest $1$s of $R$ do not depend on $R$.

\end{lemma}

\begin{proof}
% As stated above $r_\partial(i,j) = r_R(i,j) = \operatorname{rank}(R_i^j) - \operatorname{rank}(R_{i+1}^{j}) + \operatorname{rank}(R_{i+1}^{j-1}) - \operatorname{rank}(R_{i}^{j-1})$.\\
Any linear combination of non zero columns in $R_i^j$ is again non zero, since the lowest $1$s are unique. Hence we get that the rank of $R_i^j$ is equal to its number of non zero columns. We will now consider several cases.
In the first case $R[i,j]$, the top right corner element of $R_i^j$ is the lowest $1$ of $R[*,j]$.
It follows that $\operatorname{rank}(R_{i+1}^{j}) = \operatorname{rank}(R_{i+1}^{j-1}) = \operatorname{rank}(R_{i}^{j-1}) = \operatorname{rank}(R_i^j)-1$. This holds, since in all of these sub matrices, we either cut away row $i$ or column $j$. So column $j$ either becomes a zero column or is cut away. Let us set $x = \operatorname{rank}(R_{i+1}^{j})$, then: \[r_\partial(i,j) = r_R(i,j) = x+1 - x + x -x =1.\]
If $R[i,j]$ is not a lowest $1$ there are two subcases to consider. 

Firstly, let us assume, that none of the columns $1,\dots,j-1$ has a lowest $1$ in row $i$. Then $R_i^{j}$ and $R_{i+1}^{j}$ have the same rank, because cutting away row $i$ does not produce a new zero column. The same holds for $R_i^{j-1}$ and $R_{i+1}^{j-1}$, and since in both cases column $j$ is either a zero or a non zero column, their ranks are also the same. In this case we get $r_\partial(i,j) = r_R(i,j) = 0$.

Secondly, if some column $1,\dots,j-1$ has its lowest $1$ in row $i$ it follows that removing row $i$ yields a zero column in any resulting sub matrix. Therefore $R_{i+1}^{j-1}$ has one more zero column than $R_{i}^{j-1}$ and $R_{i+1}^{j}$ has one more zero column than $R_{i}^{j}$. This again results in $r_\partial(i,j) = r_R(i,j) = 0$. With this the proof is concluded.
\end{proof}

We will now discuss what these pairings mean. Either column $j$ of $R$ is zero. In this case the appearance of $\sigma_j$ in our simplexwise filtration creates a new cycle and therefore gives birth to a new homology class. This is why we call these simplices positive. 

Or column $j$ of $R$ has a lowest $1$ in row $i$. Column $j$ of $R$ stores the boundary of the chain represented by column $j$ of matrix $A$. This means, that the addition of the simplex corresponding to this column causes a homology class to die. Furthermore this class is born in frame $i$, since one of its cycles just died in column $j$ and all other cycles that died with it, have lowest $1$s below row $i$. If they had a lowest $1$ in row $i$ we could have reduced the matrix further and would have obtained $\operatorname{low}(j)<i$, which is a contradiction to our algorithm.

This means, that the pair $(i,j)$ gives us the index persistence $j-i$ of the respective homology class. Furthermore if there is some positive simplex $\sigma_i$ that never gets paired with a negative one, the homology class appearing at $i$ has an index persistence of infinity.

As an example consider the simplicial complex from Figure \ref{fig:simplicial_homology_example}. And let \[
F_* = ([v_0],[v_1],[v_2],[v_3],[v_0,v_1],[v_0,v_2],[v_1,v_2],[v_1,v_3],[v_2,v_3],[v_0,v_1,v_2]),
\]
be a simplexwise filtration. The following matrix $B$ is the reduced boundary matrix of $F_*$ in which we cut away all columns corresponding to vertices. See Matrix \ref{mat:small} for the relevant part of the initial boundary matrix.
\[
\begin{blockarray}{ccccccc}
\label{tab:reduction}
& [v_0,v_1] & [v_0,v_2] & [v_1,v_2] & [v_1,v_3] & [v_2,v_3] & [v_0,v_1,v_2] \\
\begin{block}{c[cccccc]}
  \left[v_0\right]  & 1 & 1 & 0 & 0 & 0 & 0\\*
  \left[v_1\right]  & 1 & 0 & 0 & 1 & 0 & 0\\*
  \left[v_2\right]  & 0 & 1 & 0 & 0 & 0 & 0\\*
  \left[v_3\right]  & 0 & 0 & 0 & 1 & 0 & 0\\*
  \left[v_0,v_1\right]  & 0 & 0 & 0 & 0 & 0 & 1\\*
  \left[v_0,v_2\right]  & 0 & 0 & 0 & 0 & 0 & 1\\*
  \left[v_1,v_2\right]  & 0 & 0 & 0 & 0 & 0 & 1\\*
  \left[v_1,v_3\right]  & 0 & 0 & 0 & 0 & 0 & 0\\*
  \left[v_2,v_3\right]  & 0 & 0 & 0 & 0 & 0 & 0\\*
\end{block}
\end{blockarray}
\]

Column $[v_1,v_2]$ is a zero column since when the simplex $[v_1,v_2]$ is added in the filtration a hole appears that has $[v_0,v_1]$, $[v_0,v_2]$ and $[v_1,v_2]$ as its boundary. The hole is closed when simplex $[v_0,v_1,v_2]$ is added. This means that $[v_1,v_2]$ and $[v_0,v_1,v_2]$ are a persistence pair, which is also indicated by the lowest $1$ in column $[v_0,v_1,v_2]$. Another cycle appears when $[v_2,v_3]$ is added in the filtration. This hole never gets closed. This also implies that there is no column corresponding to a triangle which has a lowest $1$ in row $[v_2,v_3]$. Note that the reduction of both columns requires additions with other columns. Column $[v_1,v_2]$ is reduced by adding columns $[v_0,v_2]$ and $[v_0,v_1]$. Column $[v_2,v_3]$ is reduced by adding columns $[v_1,v_3]$, $[v_0,v_2]$ and $[v_0,v_1]$. This means the number of column reduction steps $\operatorname{red}(B)$ equals \[\operatorname{red}([v_1,v_2]) + \operatorname{red}([v_2,v_3]) = 2 + 3 = 5.\] 

\section{Reduction with a Twist}
\label{sec:twist}
As we have discussed in the last section, the lower ones are not dependent on our reduction strategy. A strategy developed in \cite{with_a_twist} is to do the reduction steps from higher to lower dimensions. Consider some simplicial complex $K$ and filtration $F$. We first reduce the highest dimensional simplices of $K$ in order of appearance in $F$ and then work our way down to dimension zero.

We will refer to the algorithm using this strategy as the \textbf{twisted reduction algorithm}.
The central observation that causes this strategy to yield an improvement in running time is the following. 

When reducing a column $j$, such that $\operatorname{low}(j) = i$ with $i<j$ we know that column $j$ kills some class born at $i$, so column $i$ has to be a zero column. This means we can just set column $i$ to zero and have to do no further computational steps for the reduction of this particular column. This column operation is usually referred to as \textbf{clearing}. 

Considering the example from the previous section we can see that if we start the reduction by processing the triangles, we consider column $[v_0,v_1,v_2]$ first. It is reduced from the beginning. Since it has a lowest one in row $[v_1,v_2]$ we now that column $[v_1,v_2]$ has to be a zero column and we can set it to zero without doing any further computations. Therefore we save some reduction steps compared to the standard reduction scheme.

The twisted algorithm yields the same asymptotic bound, as the standard algorithm, however it often performs much better. We will further discuss this in Chapter 5. 



\newpage
%\newpage\null\thispagestyle{empty}\newpage

\chapter{Discrete Morse Theory}
Discrete Morse theory is a toolkit that enables us to study and analyze simplicial complexes. It is a combinatorial adaptation of Morse Theory, which is a powerful tool to study manifolds. In general, discrete Morse theory can applied to finite regular CW complexes but we will restrict ourselves to simplicial complexes. The theory is used in a variety of applications like denoising, mesh compression, topological data analysis, and homology computations. Especially the latter will be of interest later on. This chapter is based on an overview by Robin Foreman who also developed the theory. See \cite{Morse+Users+Guide}. 

\section{Discrete Morse Functions}

\begin{defi}
\label{def:discrete_morse_function}
Let $K$ be a simplicial complex. A function $f: K \rightarrow \mathbb{R}$ is called a discrete Morse function, if for every $\sigma^{(k)} \in K$ the following two conditions are fulfilled: 
\begin{enumerate}
    \item $|\{\tau^{(k+1)}>\sigma^{(k)} \mid f(\tau) \leq f(\sigma)\}| \leq 1$,
    \item $|\{\gamma^{(k-1)}<\sigma^{(k)} \mid f(\gamma) \geq f(\sigma)\}| \leq 1$.	
\end{enumerate}{}
\end{defi}

This means that a Morse function assigns values to simplices in a simplicial complex, such that simplices of higher dimensions get larger values, with at most one exception for each simplex and its facets and cofacets.

The following Figure \ref{fig:morse_function} illustrates two functions that assign values to the faces of a triangle. The numbers represent the assigned function values. 

\begin{figure}[H]
%\centering%

\begin{subfigure}[c]{0.49\textwidth}
\begin{center}
\input{DiscreteMorseTheory/Figures/no_morse_function_example}
\subcaption{Not a Morse function}
\end{center}
\end{subfigure}
\begin{subfigure}[c]{0.49\textwidth}
\begin{center}
\input{DiscreteMorseTheory/Figures/morse_function_example}
\subcaption{A Morse function}
\end{center}

\end{subfigure}
%\captionsetup{width=0.9\textwidth}
\caption{Two different functions that assign real values to the simplices of the 1-skeleton of a simplicial complex defined by a  triangle.}
\label{fig:morse_function}
\end{figure}

In subfigure (a) the given function is not a Morse function. The vertex at the top gets assigned value 4, while its two adjacent edges get assigned smaller values. I.e., the first condition of Definition \ref{def:discrete_morse_function} is not fulfilled. Furthermore the edge on the right, which gets assigned a value of $2$ has higher values assigned to its two vertices. This violates the second condition of the definition of a discrete Morse function.

The function depicted in Figure \ref{fig:morse_function} (b) however is a Morse function. Note that the vertices with assigned values 3 and 4 have an adjacent edge with a smaller value, but in both cases it is just one. Also the triangle has only one edge with higher function value. 

We now describe faces with exceptional values.

\begin{defi}
Let $f$ be a discrete Morse function on a simplicial complex $K$. A simplex $\sigma^{(k)} \in K$ is called \textbf{critical} w.r.t $f$, if both of the following conditions hold:
\begin{enumerate}
    \item $|\{\tau^{(k+1)}>\sigma \mid f(\tau) \leq f(\sigma)\}| = 0,$
    \item $|\{\gamma^{(k-1)}<\sigma \mid f(\gamma) \geq f(\sigma)\}| = 0.$
\end{enumerate}
\end{defi}

In the example of Figure \ref{fig:morse_function} (b) the vertex $0$ is the only critical face.

One of the central theorems of discrete Morse theory are the so called Morse inequalities which link discrete Morse theory to homology. We will merely state and briefly discuss these results, since the proofs require further theory for which we refer the reader to the aforementioned 
\enquote{A Users Guide To Discrete Morse Theory} by Robin Forman  \cite{Morse+Users+Guide}. 

Let $K$ be a simplicial complex of dimension $n$ with a discrete Morse function $f$. Let $m_k$ denote the number of critical cells of dimension $k$ with respect to $f$. Finally, recall the definition of the Betti numbers, \[
\beta_k = \operatorname{rank}(H_k(K)),
\] from the last chapter. 

\begin{thm} Let $K$, $n$, $f$, $m_k$ and $\beta_k$ be defined as above. Then the following hold. 
\textbf{Weak Morse Inequalities}: 
\begin{enumerate}
    \item for each $k = 0,1,\dots,n$, \[
        m_k \geq \beta_k.
    \]
    \item $m_0 - m_1 + m_2 - \dots + (-1)^n m_n = \beta_0 - \beta_1 + \beta_2 - \dots + (-1)^n \beta_n.$
\end{enumerate}
\textbf{Strong Morse Inequalities}: \\
For each $k = 0,1,2,\dots,n,n+1$, 
\[
    m_k - m_{k-1} + \dots + (-1)^k m_0 \geq \beta_k - \beta_{k-1} + \dots + (-1)^k \beta_0.
\]
\label{thm:morse_ineq}
\end{thm}

The weak Morse inequalities tell us that the number of critical cells of some dimension with respect to a given discrete Morse function on $K$ is bounded from below by the Betti number of the same dimension. Furthermore the alternating sums of the Betti numbers and the numbers of critical cells are equal. 

The strong Morse inequalities even show, that the alternating sum of critical cells up to any dimension is an upper bound on the alternating sum of the Betti numbers up to the same dimension.

\section{Gradient Vector Fields}
In this section we discuss a concept that encodes the information of a discrete Morse function we are usually interested in, without the need to consider the actual function values. 

We will begin by defining discrete vector fields on simplicial complexes.

\begin{defi}
Let $K$ be a simplicial complex. Consider the set of pairs
\[
V = \{(\sigma^{(k)}, \tau^{(k+1)}) \in K \times K \mid \sigma < \tau\}.
\]
We call $V$ a \textbf{discrete vector field} of $K$ if each simplex in $K$ is part of at most one pair in $V$.
\label{def:discrete_vector_field}
\end{defi}

% Figure \ref{def:discrete_vector_field} gives an example.

% \begin{figure}[H]
% %\centering%

% \begin{subfigure}[c]{0.99\textwidth}
% \begin{center}
% 
\begin{tikzpicture}

\node[b_circle, name path=0, label=below left:{}] (0) at (0,0) {};

\node[b_circle, name path=1, label=below right:{}] (1) at (2,0) {};

\node[b_circle, name path=2, label=above:{}] (2) at  (1,2) {};

\node[inner sep = 2pt, label=below:{}] (c) at (1,1) {};


\path[-, draw] (0) to[] node[anchor = center, inner sep = 6pt,below] {} (1);
\path[-, draw] (1) to[] node[anchor = center, inner sep = 6pt,right] {} (2);
\path[-, draw] (0) to[] node[anchor = center, inner sep = 6pt,left] {} (2);

\draw[->, ultra thick] (0) to ($(0)!0.5!(1)$);
\draw[->, ultra thick] (1) to ($(1)!0.5!(2)$);
\draw[->, ultra thick] (2) to ($(0)!0.5!(2)$);

\fill[opacity = 0.2] (0,0) -- (2,0) -- (1,2);

\end{tikzpicture}

% \end{center}
% \end{subfigure}

% \caption{A discrete vector field on the triangle complex.}
% \label{fig:discrete_vector_field}
% \end{figure}

As we will see, Morse functions can be encoded as a special type of discrete vector fields, which we will call \textbf{gradient vector fields} or \textbf{discrete gradients}. In the following we illustrate how the exceptions of a Morse function in the sense of Definition \ref{def:discrete_morse_function} form the pairs of a discrete vector field.

Let $f$ be the Morse function in Figure \ref{fig:morse_function} (b).
Consider the preimage $f^{-1}((-\infty,2])$. It is the set consisting of the top vertex and both its adjacent edges. This is not a simplicial complex. But it can be extended to one by including the missing vertices on both ends of the edges.

\begin{defi}
Let $K$ be a simplicial complex and $f$ a discrete Morse function on $K$. Let $c \in \mathbb{R}$. We define the \textbf{level subcomplex} as 
\[
K(c) = \bigcup_{f(\sigma)\leq c} \, \bigcup_{\gamma\leq\sigma} \gamma
\]
\end{defi}

\begin{figure}[H]
%\centering%
\begin{subfigure}[b]{0.49\textwidth}
\begin{center}
\begin{tikzpicture}

\node[circle, gray, name path=0, label=below left:{\color{gray}$3$}] (0) at (0,0) {};

\node[circle, gray, name path=1, label=below right:{\color{gray}$4$}] (1) at (2,0) {};

\node[b_circle, name path=2, label=above:{$0$}] (2) at  (1,2) {};

\node[inner sep = 2pt, label=below:{\color{gray}$5$}] (c) at (1,1) {};


\path[dashed, draw, gray] (0) to[] node[anchor = center, inner sep = 6pt,below] {\color{gray}$6$} (1);
\path[dashed, draw, gray] (1) to[] node[anchor = center, inner sep = 6pt,right] {\color{gray}$2$} (2);
\path[dashed, draw, gray] (0) to[] node[anchor = center, inner sep = 6pt,left] {\color{gray}$1$} (2);

\fill[opacity = 0.1] (0,0) -- (2,0) -- (1,2);

\end{tikzpicture}
\subcaption{Level subcomplex $K(0)$}
\end{center}
\end{subfigure}
\begin{subfigure}[b]{0.49\textwidth}
\begin{center}
\input{DiscreteMorseTheory/Figures/sub_lvl_1}
\subcaption{$K(1)$}
\end{center}
\end{subfigure}
\begin{subfigure}[b]{0.49\textwidth}
\begin{center}
\begin{tikzpicture}

\node[b_circle, name path=0, label=below left:{$3$}] (0) at (0,0) {};

\node[b_circle, name path=1, label=below right:{$4$}] (1) at (2,0) {};

\node[b_circle, name path=2, label=above:{$0$}] (2) at  (1,2) {};

\node[inner sep = 2pt, label=below:{\color{gray}$5$}] (c) at (1,1) {};


\path[dashed, draw, gray] (0) to[] node[anchor = center, inner sep = 6pt,below] {\color{gray}$6$} (1);
\path[-, draw] (1) to[] node[anchor = center, inner sep = 6pt,right] {$2$} (2);
\path[-, draw] (0) to[] node[anchor = center, inner sep = 6pt,left] {$1$} (2);

\fill[opacity = 0.1] (0,0) -- (2,0) -- (1,2);

\end{tikzpicture}
\subcaption{$K(2) = K(3) = K(4)$}
\end{center}
\end{subfigure}
\begin{subfigure}[b]{0.49\textwidth}
\begin{center}
\input{DiscreteMorseTheory/Figures/sub_lvl_3}
\subcaption{$K(5) = K(6) = K$}
\end{center}
\end{subfigure}
%\captionsetup{width=0.9\textwidth}
\caption{Level subcomplexes of the previous example.}
\label{fig:level_sub_complex}
\end{figure}
Note that in Figure \ref{fig:level_sub_complex} the level subcomplex $K(1)$ contains the vertex with assigned value $3$, while being distinct form the level subcomplex $K(3)$.

Now look at simplicial complex $K$ and discrete Morse function $f$ as depicted in Figures \ref{fig:morse_function}(b) and \ref{fig:level_sub_complex}(d). The only critical simplex is the top vertex with assigned value of $0$. When considering the sequence of level subcomplexes, we realize that the non-critical simplices appear in pairs. Depending on the function values several pairs might appear in a single step or the subcomplexes of several levels might be equal. Nonetheless, if simplices appear, they appear in pairs.

In our example, the edge $f^{-1}(1)$ is not critical because it has vertex $f^{-1}(3)$ as a facet. At the same time vertex $f^{-1}(3)$ is not critical because it has the edge $f^{-1}(1)$ as a cofacet. This observation leads us to the following definition.

\begin{defi}
Let $K$ be a simplicial complex and $f$ a discrete Morse function on $K$.
We call the set of pairs \[
    V = \{(\sigma^{(k)}, \tau^{(k+1)}) \in K \times K  \mid  \sigma < \tau, f(\sigma) \geq f(\tau)\}
\]
the \textbf{gradient vector field} of $f$ on $K$. 
\end{defi}

Notice, that if there were two cofacets of some $\sigma^{(k)}$ with smaller function values, then $f$ would not be a discrete Morse function. The same holds true if there were a face $\tau^{(k+1)}$ with two facets with higher function values.

We will visualize the pairs as arrows pointing from the lower dimensional simplex to the higher dimensional one. This visualization also motivates that we sometimes call the lower dimensional element of a pair in a discrete gradient its \textbf{tail} and the higher dimensional element its \textbf{head}. We will also refer to these pairs as the \textbf{gradient pairs} of $f$.

The following figure illustrates this in the case of our previous example.

\begin{figure}[H]
%\centering%

\begin{subfigure}[c]{0.99\textwidth}
\begin{center}
\input{DiscreteMorseTheory/Figures/discrete_gradient_simple}
\end{center}
\end{subfigure}

\caption{The discrete gradient of $f$ and $K$.}
\label{fig:discrete_gradient}
\end{figure}

This might remind us of the simplicial collapses discussed in Section \ref{sec:simplicial_collapses} and indeed a sequence of collapses defines a discrete gradient. 

A natural interest regarding vector fields is the analysis of the flows they induce. In our case this motivates the following definition.

\begin{defi}
Let $V$ be a discrete vector field on some simplicial complex~$K$. Then a sequence of simplices \[
P = \sigma_0^{(k)}, \tau_0^{(k+1)}, \sigma_1^{(k)}, \tau_1^{(k+1)}, \dots , \sigma_m^{(k)}, \tau_m^{(k+1)}, \sigma_{m+1}^{(k)}
\]
is called \textbf{$\bm{V}$-path}, if $(\sigma_i, \tau_i) \in V$ and $\tau_i > \sigma_{i+1} \neq \sigma_i$ for $i = 0,\dots,m$. 
\end{defi}

The \textbf{length} of the $V$-path is $m+1$ and is denoted as $\operatorname{length}(P)$. We call the $(k+1)$-dimensional elements the \textbf{higher dimensional elements} of the path and the $k$-dimensional elements its \textbf{lower dimensional elements}. Furthermore, as the following theorem
shows, a discrete gradient of a Morse function $f$ is decreasing along its $V$-paths.

\begin{thm}
\label{thm:vpath_decrease}
Let $f$ be a discrete Morse function and $V$ its gradient vector field. A sequence of simplices \[
\sigma_0^{(k)}, \tau_0^{(k+1)}, \sigma_1^{(k)}, \tau_1^{(k+1)}, \dots , \sigma_m^{(k)}, \tau_m^{(k+1)}, \sigma_{m+1}^{(k)}
\] is a $V$-path if and only if $\sigma_i < \tau_i > \sigma_{i+1}$ for each $i = 0,\dots,m$ and 
\[
f(\sigma_0) \geq f(\tau_0) > f(\sigma_1) \geq \dots \geq f(\tau_m) > f(\sigma_{m+1}).
\]
\end{thm}
\begin{proof}
The definitions of the discrete gradient and $V$-path imply that $f(\sigma_i) \geq f(\tau_i)$ for each $i = 0,\dots,m$. Furthermore, the definition of $V$-paths also implies that $\tau_i > \sigma_{i+1}$. Now, for the sake of contradiction assume that $f(\tau_i) \leq f(\sigma_{i+1})$. This means $\tau_i$ has two facets with function values that are greater or equal to that of $\tau$. This contradicts the assumption that $f$ is a discrete Morse function. Hence $f(\tau_i) < f(\sigma_{i+1})$.
\end{proof}
This theorem implies that if $f$ is a Morse function, then there are no nontrivial closed $V$-paths. The following theorem states that also the converse is true. This gives us a characterization of discrete vector fields, that are gradient vector fields of a Morse function.

\begin{thm}
\label{thm:closed_v_paths}
A discrete vector field $V$ is the gradient vector field of a discrete Morse function if and only if there are no non-trivial closed $V$-paths.
\end{thm}

We will get back to this theorem in the next section, since we will establish some further theory that simplifies the proof. 

To conclude and summarize this section we state: Every gradient vector field is a discrete vector field and every discrete vector field without non-trivial closed $V$-paths is a gradient vector field. We will sometimes refer to $V$-paths of gradient vector fields as gradient paths. Note that a gradient vector field corresponds to a whole family of discrete Morse functions.

\section{Morse Matchings}
The combinatorial point of view, we will discuss now, was originally explored in \cite{CHARI2000101} and then also included and extended in \cite{Morse+Users+Guide}[Chapter 6]. 

Consider a simplicial complex $K$. The \textbf{Hasse diagram} $H_K = (W_K, E_K)$ of $K$ is a directed graph with the elements of $K$ being the vertices $W_K$ and $E_K \coloneqq \{(\tau^{(p+1)},\sigma^{(p)}) \mid \sigma, \tau \in K, \sigma < \tau)\}$ the directed edges from a simplex to its facets. 

We deviate from the popular notation of graphs, where vertex sets are denoted by $V$, since we stick to Forman's notation of gradient vector fields being denoted by $V$. 

In the following Figure \ref{fig:hasse_diagram_example} we see the same simplicial complex as in Figure \ref{fig:simplicial_homology_example} and its corresponding Hasse diagram.

\begin{figure}[H]
%\centering%
\begin{subfigure}[c]{0.99\textwidth}
\begin{center}
\input{DiscreteMorseTheory/Figures/example_complex}
\end{center}
\end{subfigure}
\begin{subfigure}[c]{0.99\textwidth}
\begin{center}
\begin{tikzpicture}

\node[circle, name path=0, label=below:{$[v_0]$}] (0) at (-1.5,0) {};
\node[circle, name path=1, label=below:{$[v_1]$}] (1) at (0,0) {};
\node[circle, name path=2, label=below:{$[v_2]$}] (2) at  (1.5,0) {};

\node[circle, name path=2, label=below:{$[v_3]$}] (3) at  (3,0) {};

\node[circle, name path=0, label=right:{$[v_0,v_1]$}] (01) at (-1.6,1.5) {};
\node[circle, name path=1, label=right:{$[v_0,v_2]$}] (02) at (0,1.5) {};
\node[circle, name path=2, label=right:{$[v_1,v_2]$}] (12) at  (1.6,1.5) {};

\node[circle, name path=2, label=above:{$[v_1,v_3]$}] (13) at  (3.5,1.5) {};
\node[circle, name path=2, label=above:{$[v_2,v_3]$}] (23) at  (5,1.5) {};

\node[circle, name path=1, label=above:{$[v_0,v_1,v_2]$}] (012) at (0,3) {};

\draw[->, thick] (01) to (0);
\draw[->, thick] (02) to (0);
\draw[->, thick] (01) to (1);
\draw[->, thick] (12) to (1);
\draw[->, thick] (02) to (2);
\draw[->, thick] (12) to (2);
\draw[->, thick] (13) to (1);
\draw[->, thick] (13) to (3);
\draw[->, thick] (23) to (2);
\draw[->, thick] (23) to (3);
\draw[->, thick] (012) to (01);
\draw[->, thick] (012) to (02);
\draw[->, thick] (012) to (12);

\end{tikzpicture}
\end{center}
\end{subfigure}
\caption{Simplical complex $K$ and the corresponding Hasse diagram $H_K$.}
\label{fig:hasse_diagram_example}
\end{figure}

Note that usually Hasse diagrams also include the empty set, which is excluded in our case. We will now modify the Hasse diagram with respect to a discrete vector field as specified in the following definition.

\begin{defi}
Let $H_K = (W_K, E_K)$ be the Hasse diagram of a simplicial complex $K$ and let $V$ be a discrete vector field on $K$. Define $E_K^V$ as the set of edges where for each edge $e = (\tau^{(p+1)}, \sigma^{(p)}) \in E_K$ we include $e$ in $E_K^V$ if the pair $(\sigma^{(p)}, \tau^{(p+1)})$ is not in $V$ and otherwise include $e^{-1}$. We call $H_K^V = (W_K, E_K^V)$ the \textbf{modified Hasse diagram} of $K$ and $V$.
\end{defi}

What this means is, that we just flip all edges in the Hasse diagram that correspond to pairs in the discrete vector field. Note, that a pair $(\sigma, \tau) \in V$ corresponds to an reversed edge from $\sigma$ to $\tau$ in the modified Hasse diagram $H_K^V$. In the following figure we depict a gradient vector field on the simplicial complex from Figure \ref{fig:hasse_diagram_example} and its modified Hasse diagram. Looking at this example we can see that $V$-paths in our gradient vector field correspond to directed paths in the modified Hasse diagram.

\begin{figure}[H]
%\centering%
\begin{subfigure}[c]{0.99\textwidth}
\begin{center}
\begin{tikzpicture}

\node[b_circle, name path=0, label=left:{$v_0$}] (0) at (0,1) {};
\node[b_circle, name path=1, label=below:{$v_1$}] (1) at (2,0) {};
\node[b_circle, name path=2, label=above:{$v_2$}] (2) at  (2,2) {};
\node[b_circle, name path=2, label=right:{$v_3$}] (3) at  (4,1) {};

\draw[->-, thick] (1) -- (0);
\draw[->-, thick] (2) -- (0);
\draw[-, thick] (1) -- (2);
\draw[-, thick] (2) -- (3);
\draw[->-, thick] (3) -- (1);


\draw[->, thick] (2,1) -- (1.5,1);
\fill[opacity = 0.2] (0,1) -- (2,0) -- (2,2);

\end{tikzpicture}
\end{center}
\end{subfigure}
\begin{subfigure}[c]{0.99\textwidth}
\begin{center}
\input{DiscreteMorseTheory/Figures/modified_hasse_example_complex_with_gradient}
\end{center}
\end{subfigure}
\caption{Simplical complex $K$ and the corresponding Hasse diagram $H_K$.}
\label{fig:modified_hasse_diagram_example}
\end{figure}

\begin{thm}
\label{thm:v_paths_hasse}
For simplicial complex $K$, discrete vector field $V$ and modified Hasse diagram $H_K^V$, it holds that there are no nontrivial closed $V$-paths if and only if there are no closed directed paths in $H_K^V$. 
\end{thm}
\begin{proof}
We will prove the negated equivalence, i.e., there is a nontrivial closed $V$-path in $V$ if and only if there is a closed directed path in $H_K^V$.

\enquote{$\Rightarrow$}: Assume we have a discrete vector field $V$ on $K$ and a nontrivial closed directed path $P = \sigma_0, \tau_0, \dots ,\tau_n, \sigma_0$. By definition of the $V$-path we know that $(\sigma_0,\tau_0) \in V$ hence there is a directed edge from $\sigma_i$ to $\tau_i$ in $H_K^V$. Further we know by definition of the $V$-path, that $\tau_i > \sigma_{i+1}$, and since every simplex can only be part of a single pair in $V$ we have that $(\sigma_{i+1}, \tau_i) \notin V$, hence we have a directed edge in $H_K^V$. In summary this means there is a directed edge in $H_K^V$ between any two consecutive elements of $P$ and therefore a closed directed path. 

\enquote{$\Leftarrow$} Whenever we have a directed edge $e = (\sigma^{(k)}, \tau^{(k+1)})$ pointing upwards with respect to dimension of the simplices corresponding to its vertices, then all outgoing edges of $\tau^{(k+1)}$ point to a vertex corresponding to a $k$-dimensional simplex. Otherwise, $\tau$ would need to be paired with a higher dimensional simplex in $V$, which would mean that $\tau$ is part of two pairs in $V$, which contradicts the definition of discrete vector fields. 

Hence any closed directed path in $H_K^V$ can only consist of vertices that correspond to simplices from two consecutive dimensions. We have already seen the one to one correspondence of these closed directed paths to nontrivial closed $V$-paths when proving the other direction. Therefore the existence of a closed directed path in $H_K^V$ implies the existence of a nontrivial closed $V$-path. 
\end{proof}
A standard result from graph theory states that for a directed graph $G$ there exists a real valued function of the vertices that is strictly decreasing along each directed path if and only if there are no directed cycles in $G$. This together with Theorem \ref{thm:v_paths_hasse} implies Theorem \ref{thm:closed_v_paths}. \\

When only considering the edges in $H_K^V$ that were flipped by some gradient vector field $V$ it follows from the definition that each simplex is only part of one such edge. Furthermore, from Theorem \ref{thm:v_paths_hasse} we get that there are no directed cycles. In a combinatorial sense, $V$ defines a partial matching of $H_K$, which motivates the following definition.

\begin{defi}
For simplicial complex $K$ and directed Hasse diagram $H_K = (W_K, E_K)$ let $E_* \subseteq E_K$ be a subset of edges and let $E_*^{-1} \coloneqq \{e^{-1} \mid e \in E_*\}$. If every $w \in W_K$ is incident to at most one edge in $E_*^{-1}$ and there are no closed directed paths in $H_K^V \coloneqq (W_K, E_*^{-1} \cup (E_K \setminus E_*) )$, we call $E_*$ a \textbf{Morse matching} of $K$.
\end{defi}

As we have already indicated in terms of notation, the edges of a Morse matching $E_*$ correspond to a discrete vector field $V$ and the resulting modified Hasse diagram has no closed directed paths. Therefore $V$ is a gradient vector field. From now on we will not distinguish between a Morse matching and a gradient vector field, since they encapsulate the same information. 

Imagine we have some simplicial complex $K$ and a Morse matching on it. Let us denote the number of critical cells with respect to the Morse matching in dimension $k$ by $m_k$. Then by the Morse inequalities, i.e., Theorem \ref{thm:morse_ineq}, we know that the Betti numbers $\beta_k$ of this complex are bounded from above by $m_k$, i.e., \[
\beta_k \leq m_k, \text{ for all }  k \in \mathbb{N}_0.
\]

\begin{defi}
Let $K$ be a simplicial complex with Morse matching $E_*$. If $m_k = \beta_k$ for all $k$ we say that $E_*$ is a \textbf{perfect} Morse matching. If there is no other Morse matching with fewer critical cells, we say that the Morse matching is \textbf{optimal}.
\end{defi}

Optimal Morse matchings need not be perfect. After stating a slight reformulation of Theorem 6.4 from \cite{Morse+Users+Guide}, we will discuss a particular example.

\begin{thm}
\label{thm:matching_collapse}
Let $K$ be a simplicial complex and let $E_*$ be a Morse matching of $K$, such that the single critical face of $K$ with respect to $E_*$ is a vertex. Then $K$ is collapsible and hence contractible.
\end{thm}

Now recall the example of the Dunce hat from Figure \ref{fig:dunce}. The Dunce hat is contractible, i.e., is homotopic to a point. Therefore all homology groups of dimension one or higher are trivial and the zeroth Betti number equals one \cite{MunkresElements}[Chapter 2], i.e., $\beta_0 = 1$ and $\beta_k = 0$, for all $k\in \mathbb{N}_{\geq 1}$. This means a perfect Morse matching has a single critical vertex. For the sake of contradiction, assume that we have a perfect Morse matching for the Dunce hat. Theorem \ref{thm:matching_collapse} implies that the Dunce hat is collapsible, which is a contradiction. Therefore there can be no perfect Morse matching for the Dunce hat or any other simplicial complex that is contractible but not collapsible. Figure \ref{fig:morse_dunce} shows an optimal Morse matching for the Dunce hat.

\begin{figure}[H]
%\centering%
\begin{subfigure}[c]{0.95\textwidth}
\begin{center}
\begin{tikzpicture}

\node[g_circle, name path=1, label=above:{$0$}] (0t) at (3,5.2) {};
\node[g_circle, name path=1, label=below left:{$0$}] (0l) at (0,0) {};
\node[g_circle, name path=1, label=below right:{$0$}] (0r) at (6,0) {};

\node[b_circle, name path=1, label=below:{$1$}] (1b) at (2,0) {};
\node[b_circle, name path=1, label=left:{$1$}] (1l) at ($(0l)!0.33!(0t)$) {};
\node[b_circle, name path=1, label=right:{$1$}] (1r) at ($(0r)!0.33!(0t)$) {};

\node[b_circle, name path=1, label=below:{$2$}] (2b) at (4,0) {};
\node[b_circle, name path=1, label=left:{$2$}] (2l) at ($(0l)!0.66!(0t)$) {};
\node[b_circle, name path=1, label=right:{$2$}] (2r) at ($(0r)!0.66!(0t)$) {};

\node[b_circle, name path=1, label=above:{}] (3) at ($(1l)!0.33!(1b)$) {};
\node[b_circle, name path=1, label=above right:{}] (4) at ($(1l)!0.66!(1b)$) {};

\node[b_circle, name path=1, label=above right:{}] (5) at ($(2l)!0.33!(2r)$) {};
\node[b_circle, name path=1, label=below: {}] (6) at ($(2l)!0.66!(2r)$) {};

\node[b_circle, name path=1, label=above:{}] (7) at ($(2b)!0.5!(1r)$) {};

\path[gray, line width = 4pt,draw] (2l) to (1l);
\path[gray, line width = 4pt,draw] (2r) to (1r);
\path[gray, line width = 4pt,draw] (2b) to (1b);

\path[-,draw] (1r) to (6);
\path[-,draw] (1l) to (5);
\path[-,draw] (3) to (5);
\path[-,draw] (4) to (5);
\path[-,draw] (4) to (7);
\path[-,draw] (2b) to (4);
\path[-,draw] (7) to (5);
\path[-,draw] (7) to (6);

\path[-,draw] (1l) to (3) to (4) to (1b);
\path[-,draw] (2l) to (5) to (6) to (2r);
\path[-,draw] (2b) to (7) to (1r);


\path[->-, draw] (1l) to (0l);
\path[->-, draw] (3) to (0l);
\path[->-, draw] (4) to (0l);
\path[->-, draw] (1b) to (0l);

\path[->-, draw] (2l) to (0t);
\path[->-, draw] (5) to (0t);
\path[->-, draw] (6) to (0t);
\path[->-, draw] (2r) to (0t);

\path[->-, draw] (2b) to (0r);
\path[->-, draw] (7) to (0r);
\path[->-, draw] (1r) to (0r);

\path[->, draw] ($(2l)!0.5!(5)$) to ($($(2l)!0.5!(5)$)!0.35!(0t)$);
\path[->, draw] ($(6)!0.5!(5)$) to ($($(6)!0.5!(5)$)!0.35!(0t)$);
\path[->, draw] ($(2r)!0.5!(6)$) to ($($(2r)!0.5!(6)$)!0.35!(0t)$);

\path[->, draw] ($(1l)!0.5!(3)$) to ($($(1l)!0.5!(3)$)!0.35!(0l)$);
\path[->, draw] ($(3)!0.5!(4)$) to ($($(3)!0.5!(4)$)!0.35!(0l)$);
\path[->, draw] ($(4)!0.5!(1b)$) to ($($(4)!0.5!(1b)$)!0.35!(0l)$);

\path[->, draw] ($(1r)!0.5!(7)$) to ($($(1r)!0.5!(7)$)!0.35!(0r)$);
\path[->, draw] ($(2b)!0.5!(7)$) to ($($(2b)!0.5!(7)$)!0.35!(0r)$);

\path[->, draw] ($(2b)!0.5!(4)$) to ($($(2b)!0.5!(4)$)!0.35!(1b)$);
\path[->, draw] ($(4)!0.5!(7)$) to ($($(4)!0.5!(7)$)!0.35!(2b)$);


\path[->, draw] ($(5)!0.5!(4)$) to ($($(5)!0.5!(4)$)!0.35!(3)$);
\path[->, draw] ($(5)!0.5!(3)$) to ($($(5)!0.5!(3)$)!0.35!(1l)$);
\path[->, draw] ($(5)!0.5!(1l)$) to ($($(5)!0.5!(1l)$)!0.35!(2l)$);

\path[->, draw] ($(5)!0.5!(7)$) to ($($(5)!0.5!(7)$)!0.35!(6)$);
\path[->, draw] ($(6)!0.5!(7)$) to ($($(6)!0.5!(7)$)!0.35!(1r)$);
\path[->, draw] ($(6)!0.5!(1r)$) to ($($(6)!0.5!(1r)$)!0.35!(2r)$);

\fill[opacity = 0.2] (4.center) -- (7.center) -- (5.center);
\end{tikzpicture}
\end{center}
\end{subfigure}
\caption{The dunce hat with an optimal Morse matching.}
\label{fig:morse_dunce}
\end{figure}

In Figure \ref{fig:morse_dunce}, the critical cells of the depicted discrete gradient are in gray, while all other cells are part of a pair indicated by an arrow. Since we know there can be no perfect Morse matching, the minimal number of critical cells has to be at least two. We also know that the alternating sums of Betti numbers and critical cells must be equal by the discrete Morse inequalities. Hence, if we have a critical edge, there also must be at least an additional critical vertex or a critical triangle. Yielding a total number of critical cells of at least three. This suffices to know that the matching in our example is indeed optimal. 

As Joswig and Pfetsch have shown in \cite{joswig2004computing}, the computation of optimal Morse matchings is NP-hard. In the same article they introduce a linear program that computes optimal Morse matchings, which works well for small instances, but takes too long to solve for large instances, as would be expected. A random approach to compute Morse matchings was developed by Benedetti and Lutz \cite{lutzbenedetti}, in which they try to match and collapse free faces until this is not possible, in which case a random maximum dimensional face is declared critical and deleted. This approach sometimes yields a perfect Morse matching, which is a certificate for the complex being collapsible. A Python version of the random discrete Morse algorithm, which was developed for this thesis, can be found on: \href{https://github.com/IvanSpirandelli/Masterarbeit/blob/master/Algorithms/random_discrete_morse.py}{[GitHub]}, also see \cite{github}. 

\newpage
%\newpage\null\thispagestyle{empty}\newpage
\chapter{Connecting Morse Theory and Persistence Computations}
\label{ch:connection}
This chapter is dedicated to the connection between discrete Morse functions and persistent homology computations, which in general is a popular consideration. See \cite{MorseTheoryForFiltrations} for an example of how Morse theory can be used to simplify persistent homology computations. 

We will consider some direct translations between filtrations and discrete Morse functions and will try to answer the question if and how \enquote{complex} Morse functions translate to filtrations with expensive persistent homology computations. Looking into this was motivated by talks with Frank Lutz and $\text{Pawe\l}$ $\text{D\l otko}$, who suggested that this connection might be interesting to explore.

\section{Discrete Morse Functions and Filtrations}
We will begin by considering some insights Ulrich Bauer has recently discussed in \cite{bauer2019ripser}. The definition of \textbf{apparent pairs}, the first two lemmas regarding this notion and their respective proofs follow \cite{bauer2019ripser} closely. Afterwards we will build on and extend this theory. Algorithm \ref{algo:filtration_to_dmf} is a straight forward approach and Algorithm \ref{algo:dmf_to_filtration} was suggested by $\text{Pawe\l}$ $\text{D\l otko}$. The lemmas concerned with the relation of these algorithms, namely Lemma \ref{lemma:vector_field_apparent_pairs} and Lemma \ref{lemma:inclusion_of_critical_sets}, were developed for this thesis.

\begin{defi}
Let $K$ be a simplicial complex and let $F_*$ be a simplexwise filtration of $K$. We call a pair $(\sigma,\tau)$ of simplices in $K$ an \textbf{apparent pair} if the following holds
\begin{itemize}
    \item $\sigma$ is the youngest facet of $\tau$,
    \item $\tau$ is the oldest cofacet of $\sigma$.
\end{itemize}{}
\end{defi}

Here, \textbf{young} means, coming late in the filtration and \textbf{old} means coming early in the filtration. Consider the simplexwise filtration of the simplicial complex of a triangle with all its faces. In this chapter, we will label vertices by $0$ to number of vertices. With respect to a triangle this means we label the vertices $0$, $1$ and $2$, and therefore get the following filtration for the simplicial complex of a $2$-simplex: \[
    F_* = ([0],[1],[2],[0,1],[0,2],[1,2],[0,1,2]).
\]
The youngest facet of the edge $[0,1]$ is the vertex $[1]$ and the oldest cofacet of vertex $[1]$ is the edge $[0,1]$. Hence $([1],[0,1])$ is an apparent pair. 
Consider the boundary matrix of this example, in which we have highlighted the lowest $1$s in each column by a gray box:
\[
\begin{blockarray}{cccccccc}
& [0] & [1] & [2] & [0,1] & [0,2] & [1,2] & [0,1,2]  \\
\begin{block}{c[ccccccc]}
  [0] & 0 & 0 & 0 & 1 & 1 & 0 & 0\\*
  \left[1\right] & 0 & 0 & 0 & \colorbox{lightgray}{1} & 0 & 1 & 0\\*
  \left[2\right] & 0 & 0 & 0  & 0 & \colorbox{lightgray}{1} &\colorbox{lightgray}{1} & 0\\*
  \left[0,1\right] & 0 & 0 & 0  & 0 & 0 & 0 & 1 \\*
  \left[0,2\right] & 0 & 0 & 0  & 0 & 0 & 0 & 1\\*
  \left[1,2\right] & 0 & 0 & 0  & 0 & 0 & 0 & \colorbox{lightgray}{1}\\*
\end{block}
\end{blockarray}
\]
As we can see in this case, all three apparent pairs are also persistence pairs, since the respective columns are already reduced and have their lowest $1$s in the rows corresponding to the other element of the apparent pair. This generally holds true.
\begin{lemma}
\label{lem:app_is_pers}
An apparent pair of a simplexwise filtration is a persistence pair.
\end{lemma}
\begin{proof}
Let $K$ be a simplicial complex. Let $F_*$ be a simplexwise filtration of $K$ and let $B$ be its boundary matrix. Consider an apparent pair $(\sigma, \tau)$ of~$F_*$. 

Assume $\beta \neq \sigma$ is the row with the lowest $1$ in column $\tau$. Then $\beta$ is a facet of $\tau$ coming later in the filtration than $\sigma$. This contradicts $(\sigma, \tau)$ being an apparent pair. Similarly assume there is another column to the left of $\tau$, which has a non-zero entry in row $\sigma$. This means, that $\tau$ is not the oldest cofacet of $\sigma$. 

Again this contradicts $(\sigma, \tau)$ being an apparent pair.
 
Hence column $\tau$ is already reduced and by the pairing lemma it follows that  $(\sigma, \tau)$ is a persistence pair. 
\end{proof}

Furthermore the set of apparent pairs is a discrete gradient on $K$.

\begin{lemma}
The apparent pairs of a simplexwise filtration $F_* = (\sigma_0, \dots, \sigma_m)$ form a gradient vector field.
\label{lemma:apparent_pairs_form_gradient}
\end{lemma}
\begin{proof}
The proof will be done in two steps. First we prove, that the apparent pairs form a discrete vector field. Then we define a Morse function that has the apparent pairs as gradient pairs. 

Let $(\sigma^{(k)}, \tau^{(k+1)})$ be an apparent pair. Since $\tau$ is uniquely determined by $\sigma$, there can be no other apparent pair $(\sigma^{(k)}, \xi^{(k+1)})$ that contains $\sigma$. We will show that there can also be no apparent pair $(\phi^{(k-1)},\sigma^{(k)})$ containing~$\sigma$. Note that in this case $k \geq 1$. There is another $k$-simplex $\rho \neq \sigma$ that is a facet of $\tau$ and a cofacet of $\phi$. By assumption $\sigma$ is the youngest facet of $\tau$, hence $\rho$ is older than $\sigma$. In particular $\sigma$ is not the oldest cofacet of $\rho$ and hence $(\rho,\sigma)$ is not an apparent pair. 
An analogous argument shows that $\tau$ also can not be in another apparent pair. Therefore the apparent pairs form a discrete vector field. 

We will now show that the apparent pairs are the gradient pairs of a Morse function. Recall $F_* = (\sigma_0, \dots, \sigma_m)$ and define
\[
f(\sigma_j)= \begin{cases} 
      i & \text{if there is an apparent pair$(\sigma_i,\sigma_j)$}, \\
      j & \text{else.} 
   \end{cases}
\]

We will verify that $f$ is a Morse function. It holds that $f(\sigma_l) \leq l$. Furthermore let $\sigma_i$ be a facet of $\sigma_l$, i.e., $i<j$. If $(\sigma_i, \sigma_j)$ is not an apparent pair then $f(\sigma_i)\leq i < j = f(\sigma_j)$. In particular $(\sigma_i, \sigma_j)$ is not a gradient pair of $f$. 

Now assume $(\sigma_i, \sigma_j)$ is an apparent pair, i.e., $\sigma_i$ is the youngest facet of $\sigma_j$. Then we have $h \leq i$ for every $\sigma_h$ that is a facet of $\sigma_j$. Thus $f(\sigma_h) \leq h \leq i = f(\sigma_j)$, where equality holds if and only if $h = i$. Meaning, there is exactly one facet of $\sigma_j$, namely $\sigma_i$, with function value not lower than $f(\sigma_j)$. Hence $f$ is a Morse function and the gradient pairs of $f$ are the apparent pairs of the filtration $F_*$.

\end{proof}

We will also refer to this discrete gradient field as the \textbf{apparent gradient} of $F_*$. As with discrete vector fields we will call the lower dimensional element of an apparent pair its \textbf{tail} and the higher dimensional element its \textbf{head}. Note that from the proof we also get a definition for a Morse function, that has the pairs of the apparent gradient as gradient pairs.

The following example illustrates that the apparent gradient does not necessarily yield an optimal Morse matching. Consider the filtration \[
F_* = ([0],[1],[2],[3],[4],[5],[4,5],[2,3],[0,1],[1,2],[3,5],[0,4]).
\]
Figure \ref{fig:not_opt_example} shows a possible geometric realization and the apparent gradient of $F_*$. There are three critical cells and we know the induced Morse matching is not optimal since we can also match $[2]$ and $[1,2]$ as well as $[4]$ and $[0,4]$ without closing a $V$-path. Indeed adding these pairs yields a perfect Morse matching.


\begin{figure}[H]
\noindent%
\centering%
\begin{tikzpicture}

\node[b_circle, name path=0, label=above:{0}] (0) at (0,4) {};

\node[b_circle, name path=1, label=left:{1}] (1) at (-2,2) {};

\node[b_circle, name path=1, label=right:{4}] (4) at (2,2) {};

\node[b_circle, name path=1, label=left:{2}] (2) at (-2,0) {};

\node[b_circle, name path=1, label=right:{5}] (5) at (2,0) {};

\node[b_circle, name path=1, label=below:{3}] (3) at (0,-2) {};


\draw[thick] (0) to (4);
\draw[thick] (3) to (5);
\draw[thick] (1) to (2);
\draw[->-, very thick] (1) to (0);
\draw[->-, very thick] (3) to (2);
\draw[->-, very thick] (5) to (4);


\end{tikzpicture}

\caption{Geometric realization of the simplicial complex of $F_*$ and its apparent gradient.}
\label{fig:not_opt_example}

\end{figure}


A simple algorithmic approach to construct this gradient vector field is to iterate over all simplices in a simplexwise filtration in order of appearance and for each element search for the youngest facet, then check if the current element is that facets oldest cofacet.\\ The following algorithm does exactly that while keeping track of all critical faces.

\begin{algorithm}[H]
\SetKwData{Left}{left}\SetKwData{This}{this}\SetKwData{Up}{up}\SetKwFunction{Union}{Union}\SetKwFunction{FindCompress}{FindCompress}\SetKwInOut{Input}{Input}\SetKwInOut{Output}{Output}
\Input{Filtration $F_* = (\sigma_1, ..., \sigma_m)$, simplicial complex $K$}
\Output{Discrete gradient $V$, Critical cells $C$}

Initialize $C$ as the set of vertices and $V$ as the empty set\\
\For{$\sigma \in F$ in order of appearance}
{
    $\gamma = $ youngest facet of $\sigma$\\
    \If{$\sigma$ oldest cofacet of $\gamma$}
    {Add $(\sigma,\gamma)$ to $V$, remove $\gamma$ from $C$}
    \Else{Add $\sigma$ to $C$}
}

$\operatorname{return}(V,C)$
\caption{Filtration to discrete gradient field.}

\label{algo:filtration_to_dmf}
\end{algorithm} 
\vspace{0.5cm}

Note that simplices are added to the set of critical cells and removed later on, if they are paired with a cofacet. 

The outer loop starting in Line 2 runs in $\mathcal{O}(m)$, where $m$ is the number of simplices in the filtration. At the time of considering $\sigma$ there are two possibilities.

Either its youngest facet $\gamma$ is not already paired and $\sigma$ is the oldest cofacet of $\gamma$. See Line 4. In this case we pair $\sigma$ and $\gamma$ and remove $\gamma$ from the set of critical simplices. 

Or the condition is not fulfilled. This causes $\sigma$ to be declared critical for now.

Checking the conditions is in $\mathcal{O}(m)$ each, and assuming that adding and removing elements from $V$ and $C$ takes constant time we get overall quadratic running time in the number of simplices of the filtration.

At the end of the algorithm all simplices remaining in the set $C$ are critical and all others are part of an apparent pair.

Applying this algorithm to the filtration \[
   F_* = ([0],[1],[2],[0,1],[0,2],[1,2],[0,1,2]).
\]
yields the discrete vector field depicted in Figure \ref{fig:algo1_example}. 

\begin{figure}[H]
\noindent%
%\centering%

\centering%
\begin{tikzpicture}

\node[b_circle, name path=0, label=below:{0}] (0) at (0,0) {};

\node[b_circle, name path=1, label=below:{1}] (1) at (2,0) {};

\node[b_circle, name path=2, label=above:{2}] (2) at  (1,2) {};


\draw[thick] (2) to (1);
\draw[->-, thick] (1) to (0);

\draw[->-, thick] (2) to (0);
\draw[->, thick] (1.5,1) to (1,0.75);

\fill[opacity = 0.2] (0,0) -- (2,0) -- (1,2);

\end{tikzpicture}

\caption{A simplicial complex with discrete gradient}
\label{fig:algo1_example}

\end{figure}


A natural question is, how to invert this process, i.e., how to get from a discrete gradient to a filtration in which the pairs of the gradient correspond to apparent pairs.

Recall the modified Hasse diagram from the last chapter. The modified Hasse diagram of the simplicial complex and discrete Morse function from Figure \ref{fig:algo1_example} is depicted in the following Figure \ref{fig:modified_hasse}. 

\begin{figure}[H]
\noindent%
%\centering%
\centering%
\begin{tikzpicture}

\node[circle, name path=0, label=below:{[0]}] (0) at (-1.5,0) {};
\node[circle, name path=1, label=below:{[1]}] (1) at (0,0) {};
\node[circle, name path=2, label=below:{[2]}] (2) at  (1.5,0) {};

\node[circle, name path=0, label=left:{[0,1]}] (3) at (-1.5,1.5) {};
\node[circle, name path=1, label=right:{[0,2]}] (4) at (0,1.5) {};
\node[circle, name path=2, label=right:{[1,2]}] (5) at  (1.5,1.5) {};

\node[circle, name path=1, label=above:{[0,1,2]}] (6) at (0,3) {};

\draw[->, thick] (5) to (6);
\draw[->, thick] (6) to (3);
\draw[->, thick] (6) to (4);

\draw[->, thick] (3) to (0);
\draw[->, thick] (4) to (0);

\draw[->, thick] (5) to (1);
\draw[->, thick] (1) to (3);

\draw[->, thick] (2) to (4);
\draw[->, thick] (5) to (2);

\end{tikzpicture}
\caption{The modified Hasse diagram of the previously seen complex and discrete vector field.}
\label{fig:modified_hasse}
\end{figure}


The following algorithm to obtain a simplexwise filtration from a discrete vector field $V$ and a simplicial complex $K$ was suggested by $\text{Pawe\l}$ $\text{D\l otko}$. We assume the reader is familiar with topological sorting. 

In the version of it we are using, the set of all elements is partitioned such that the first set has no dependencies, the elements of the second set depend on the first etc. Within these sets we will order the elements lexicographically. 

Note that the usual convention regarding topological sorting is, that a directed edge from some node $a$ to some node $b$ implies $a \prec b$. In our case it is the other way around. 


\begin{algorithm}[H]
\SetKwData{Left}{left}\SetKwData{This}{this}\SetKwData{Up}{up}\SetKwFunction{Union}{Union}\SetKwFunction{FindCompress}{FindCompress}\SetKwInOut{Input}{Input}\SetKwInOut{Output}{Output}
\Input{Simplicial complex $K$, Discrete gradient field $V$}
\Output{Simplexwise filtration $F_*$}

$H_K^V =$ modified Hasse diagram of $K$ and $V$\\
$Filtration = $ topological sorting of $(H_K^V)$\\
Sort $Filtration$ by dimension in a stable manner.\\
$\operatorname{return}(Filtration)$
\caption{Discrete gradient field to filtration}
\label{algo:dmf_to_filtration}
\end{algorithm} 
\vspace{0.5cm}

By sorting the simplices by dimension in Line 3 we are guaranteed to get a valid simplexwise filtration. Sorting in a stable manner means, that two elements of the same size, i.e. dimension, remain in the same internal order as before. 

This algorithm only generates filtrations that are ordered by dimension which is quite restrictive in a sense. If we are mainly interested in the number of reduction steps needed to reduce the corresponding boundary matrix, however, this is no restriction at all. Any filtration $F$ and a filtration~$F'$ which is equal to $F$ sorted by dimension in a stable manner, yield the same persistence pairs and need their boundary matrices to be reduced by the same steps. 


For an example of the algorithm consider the modified Hasse diagram of Figure \ref{fig:modified_hasse}. The topological ordering we get in line 2 is
\[
     ([0], [0,1], [0,2], [0,1,2], [1], [2], [1,2]).
\]
At this point we do not have a valid filtration. But after the algorithm executes Line 3, in which we sort by dimension, while keeping the order within the dimensions intact, we get 
\begin{center}
$F_* = ([0],[1],[2],[0,1],[0,2],[1,2],[0,1,2])$.
\end{center} 

We will now formally describe in which way the two Algorithms~\ref{algo:filtration_to_dmf}~and~ \ref{algo:dmf_to_filtration} are inverse to one another. We will write $F_* = \mathcal{F}(V)$ for the filtration that is generated by the discrete vector field $V$ in Algorithm \ref{algo:dmf_to_filtration}. For brevity, we will omit a reference to the underlying simplicial complex $K$. Conversly we write $V = \mathcal{V}(F)$ for the gradient vector field generated by Algorithm \ref{algo:filtration_to_dmf}.
The following Lemma proves that Algorithm \ref{algo:dmf_to_filtration} generates a filtration with the desired properties. 

\begin{lemma}
Let $V$ be a gradient vector field on simplicial complex $K$ and let $F_* = \mathcal{F}(V)$ be the generated filtration. Every pair in $V$ is an apparent pair in $F_*$. 
\label{lemma:vector_field_apparent_pairs}
\end{lemma}
\begin{proof}
Let $(\sigma^{(p-1)},\tau^{(p)})$ be a pair in $V$. Recall that, by definition, $\tau$ and $\sigma$ can not be part of any other pair. Now consider $H_K^V$, the modified Hasse diagram of $K$ and $V$. Let $\gamma$ be a facet of $\tau$, with $\gamma \neq \sigma$.

In $H_K^V$, $\tau$ has a directed edge to $\gamma$, since $(\gamma,\tau)$ is not a pair in $V$. But $\sigma$ has a directed edge to its cofacet $\tau$, since $(\sigma,\tau)$ is a pair in $V$, so in $H_K^V$ we get a directed edge from $\sigma$ to $\tau$ and therefore a directed path from $\sigma$ to $\gamma$. Hence, in the topological order $\gamma \prec \sigma$ and therefore in the filtration every such $\gamma$ comes before $\sigma$. It follows that $\sigma$ is the youngest facet of $\tau$. 

Analogously, let $\theta$ be a cofacet of $\sigma$, $\theta \neq \tau$. Now in $H_K^V$ we have a directed edge from $\sigma$ to  $\tau$ and another directed edge from $\theta$ to $\sigma$. Again it follows, that $\tau \prec \theta$, meaning that $\tau$ is the oldest cofacet of $\sigma$. 

The pair $(\sigma,\tau) \in V$ was chosen arbitrarily, hence the claim follows. 
\end{proof}

\begin{lemma}
Let $V_0$ be a discrete vector field and let $C_0$ be the set of critical cells of $V_0$. Now let $V_1 = \mathcal{V}(\mathcal{F}(V_0))$ be the discrete vector field generated by concatenating the respective algorithms. Let $C_1$ be the set of critical cells of~$V_1$. Then $C_1 \subseteq C_0$.
\label{lemma:inclusion_of_critical_sets}
\end{lemma}
\begin{proof}
By Lemma \ref{lemma:vector_field_apparent_pairs}, every pair in $V_1$ is an apparent pair in $\mathcal{F}(V_1)$, which again, is a pair in $ V_2 = \mathcal{V}(\mathcal{F}(V_1))$ by Lemma \ref{lemma:apparent_pairs_form_gradient}. Since every simplex that is not part of a pair, is critical, the claim follows. 
\end{proof}

\begin{corl}
Let $V_0$ be a discrete vector field and define $V_i = \mathcal{V}\mathcal{F}((V_{i-1}))$ for $i\in \mathbb{N}$. Then after finitely many steps $k$, we get that $V_k = V_{k-1}$.
\end{corl}
 
This means that after repeatedly applying the algorithms  all apparent pairs of the filtration are pairs in the apparent gradient and vice versa. For an example of an iteration that actually decreases the number of critical cells, consider the simplicial complex from Figure \ref{fig:not_opt_example} again. The corresponding filtration is  \[
F_*^1 = ([0],[1],[2],[3],[4],[5],[4,5],[2,3],[0,1],[1,2],[3,5],[0,4]).
\]
Using $F_*^1$ as input for Algorithm \ref{algo:filtration_to_dmf} yields the apparent gradient \[V_1 = \{([1],[0,1]),([5],[4,5]),([2],[2,3])\}.\] Using $V_1$ and the underlying simplicial complex of $F_*^1$ as input for Algorithm \ref{algo:dmf_to_filtration} yields the filtration \[
F_*^2 = ([0], [2], [4], [1], [3], [5], [0, 1], [0, 4], [2, 3], [4, 5], [1, 2], [3, 5]),
\]
which has apparent gradient \[V_2 = \{([1],[0,1]),([4],[0,4]),([5],[4,5]),([2],[2,3])\}.\] This means the number of critical cells actually decreased and two critical cells remain.

This stabilization process yields altering filtrations and discrete gradients, but we never loose an apparent pair. This means that the filtrations remain close in the sense, that every pair that is an apparent pair once remains a persistence pair in every iteration of the process. Analogously no apparent gradient ever looses a pairing. 

It remains to be uncovered if this relates to other notions of closeness of persistence computations we do not discuss in this work or if there are bounds on how many iterations we have to do until the process stabilizes. 

\section{V-Paths and Standard Reduction}
In this section we will explore relations between $V$-paths of the apparent gradient of a filtration and the number of conflicts in the standard reduction scheme. All lemmas and the theorem discussed in this section are not based on previously known works but were developed for this thesis with guidance by $\text{Pawe\l}$ $\text{D\l otko}$.

Recall Definition \ref{def:col_red_steps}, i.e., the definition of the number of column reduction steps $\operatorname{red}(B)$ during the reduction of some boundary matrix $B$ corresponding to filtration $F_*$. We will now link this to the length of $V$-paths of the apparent gradient of $F_*$.


The following example will provide an intuition for the connection. Consider the following filtration:
\[
    F_* = ([0],[1],[2],[3],[4],[0,1],[0,2], [1,3], [2,4], [1,2], [3,4]).
\]

A geometric realization of the underlying abstract simplicial complex is depicted in the following figure. It also shows the apparent gradient of $F_*$.

\begin{figure}[H]
\noindent%
%\centering%
\centering%
\begin{tikzpicture}

\node[b_circle, name path=0, label=below:{0}] (0) at (4,1) {};

\node[b_circle, name path=1, label=above:{1}] (1) at (2,2) {};
\node[b_circle, name path=2, label=above:{3}] (3) at (0,2) {};

\node[b_circle, name path=1, label=below:{2}] (2) at (2,0) {};
\node[b_circle, name path=2, label=below:{4}] (4) at (0,0) {};


\draw[->-, thick] (1) to (0);
\draw[->-, thick] (3) to (1);

\draw[->-, thick] (2) to (0);
\draw[->-, thick] (4) to (2);

\draw[-, thick] (3) to (4);
\draw[-, thick] (1) to (2);

\end{tikzpicture}
\caption{An example we will use to illustrate the connection between V-Paths and conflicts in the standard reduction scheme}
\label{fig:paths_and_computations}
\end{figure}

The cropped boundary matrix of $F_*$ looks as follows:
\[
\begin{blockarray}{ccccccc}
& [0,1] & [0,2] & [1,3] & [2,4] & [1,2] & [3,4] \\
\begin{block}{c[cccccc]}
             [0] & 1 & 1 & 0 & 0 & 0 & 0 \\*
  \left[1\right] & \colorbox{lightgray}{1} & 0 & 1 & 0 & 1 & 0 \\*
  \left[2\right] & 0 & \colorbox{lightgray}{1}  & 0 & 1 & \colorbox{lightgray}{1}  & 0 \\*
  \left[3\right] & 0 & 0 & \colorbox{lightgray}{1}  & 0 & 0 & 1 \\*
  \left[4\right] & 0 & 0 & 0 & \colorbox{lightgray}{1}  & 0 & \colorbox{lightgray}{1}   \\*
\end{block}
\end{blockarray}
\]

We have highlighted the lowest $1$s in each column with a small gray box. As we can see every column corresponding to the head of an apparent pair is already reduced, i.e., the lowest $1$ is in a row such that no column to the left of it has its lowest $1$ in the same row.

Now lets try to resolve the conflicts. Column $[1,2]$ is reduced by first adding column $[0,2]$ and then column $[0,1]$. This corresponds to following the $V$-paths originating in the facets of edge $[1,2]$ to the critical vertex $[0]$.

Similarly, reducing column $[3,4]$ requires us to add colum $[2,4]$, $[1,3]$, $[0,2]$ and finally $[0,1]$ which again corresponds to following the $V$-paths originating in the facets of edge the $[3,4]$ and ending in the critical vertex $[0]$. \\

We will now generalize this example. 
For some simplexwise filtration $F_* = (\mu_1,\dots,\mu_m)$ on a simplicial complex $K$, we define $f:K \rightarrow \mathbb{R}$ as $f(\mu_i) = i$ and
call $f$ the \textbf{filtration values} of $F_*$. 

Let $V$ be the apparent gradient of $F_*$. We will refer to simplices that are part of a pair in $V$ as \textbf{paired} and to simplices that are not part of a pair in $V$ as \textbf{unpaired}. Consider some $\tau^{(k+1)} \in K$ and let us denote by $\mathcal{P}_\tau$ the set of all $V$-paths originating in a facet of $\tau$ and ending in an unpaired simplex or a simplex that is the head of an apparent pair. We will call them the \textbf{complete facet paths} of $\tau$. 

In the above example the complete facet paths of edge $[1,2]$ are the paths $P_1 = [1],[0,1],[0]$ and $P_2 = [2],[0,2],[0]$, hence we write $\mathcal{P}_{[1,2]} = \{P_1,P_2\}$.
%When discussing the reduction of some boundary matrix we will abuse notation slightly and refer to the column corresponding to some simplex in the boundary matrix, by the simplex itself, i.e., we will refer to the column coresponding to some simplex $\sigma$ as column $\sigma$. Furthermore for two simplices $\sigma$ and $\alpha$ with $f(\sigma)<f(\alpha)$ we will say that row $\sigma$ is \textbf{above} row $\alpha$, and row $\alpha$ is \textbf{below} row $\sigma$. This stems from the fact that the simplex with lowest filtration value corresponds to the first row of the boundary matrix and the simplex with the highest filtration value corresponds to the last row of the boundary matrix.

\begin{lemma}
\label{lem:lowest_apparent}
With $K,F_*,V$ and $f$ as above. Let $\tau$ be an unpaired simplex, and if there are unpaired elements that are the last elements of a path in $\mathcal{P}_\tau$, let $\sigma^{(k)}$ be the one with highest filtration value. If such a $\sigma$, exists it holds that:
\begin{enumerate}[{(}1{)}]
\item During the reduction of column $\tau$, every lowest $1$ with higher filtration value than $\sigma$ is the tail of an apparent pair and an element of a path in~$\mathcal{P}_\tau$.
\item The reduced column $\tau$ is zero or has a lowest $1$ in row $\sigma$ or a row with lower filtration value.
\end{enumerate}

%\item Every one appearing below row $\sigma$ during reduction is part of an apparent pair and an element in a path in $\mathcal{P}_\tau$.

\end{lemma}
\begin{proof}
Initally all ones in column $\tau$ correspond to facets of $\tau$. Let $\alpha$ be the row corresponding to the lowest $1$. If $\alpha = \sigma$, we are done, so let $\alpha \neq \sigma$.
We will show that $\alpha$ has to be the tail of an apparent pair. 

Assume for the sake of contradiction that $\alpha$ is unpaired. Then the trivial path $P = \alpha$ is in $\mathcal{P}_\tau$. This contradicts the assumption that $\sigma$ is the unpaired simplex with highest filtration value that is the last element of a path in $\mathcal{P}_\tau$. 

Therefore $\alpha$ has to be part of an apparent pair. Assume $\alpha$ is the head of an apparent pair. Then $\alpha$ is part of a persistence pair with some lower dimensional simplex and can not be part of a persistence pair with some higher dimensional simplex. This means, no column can have its lowest $1$ in row $\alpha$. In particular, it can never be the lowest $1$ of column $\tau$ during its reduction. This is a contradiction to the assumption that $\alpha$ corresponds to the lowest $1$ in column $\tau$. Hence, $\alpha$ is the tail of an apparent pair $(\alpha, \beta)$. Furthermore, $\alpha$ is an element of all paths in $\mathcal{P}_\tau$ that begin with the sequence $\alpha,\beta$. In particular $\alpha$ is an element of a path in $\mathcal{P}_\tau$.

To reduce the lowest $1$ in row $\alpha$ we have to add column $\beta$. Since $\beta$ is the head of an apparent pair its column is reduced from the beginning. Therefore the new ones appearing in column $\tau$ correspond to facets of $\beta$ and are themselves lower dimensional elements of a path in $\mathcal{P}_\tau$. 

By the same reasoning as before, every lowest $1$ with higher filtration value than row $\sigma$, corresponds to some tail $\alpha_*$ of an apparent pair and all appearing ones correspond to elements in a path of $\mathcal{P}_\tau$. This proves (1). Furthermore, (2) is a direct consequence of (1), since every lowest $1$ that is the tail of an apparent pair is reduced by adding the column of its head. This continues until the column is zero or the lowest $1$ appears in row $\sigma$. Either $\tau$ and $\sigma$ get paired persistence-wise and the lowest $1$ remains in row $\sigma$, or if $\sigma$ is already paired with another column the reduction continues and after the next reduction step the column is zero or has its lowest $1$ in a row with lower filtration value than $\sigma$.
\end{proof}

Consider the example from Figure \ref{fig:paths_and_computations}. Reducing column $[1,2]$ follows along the previously stated paths in  $\mathcal{P}_{[1,2]}$ and reducing column $[3,4]$ follows along the paths $P_1 = [3],[1,3],[1],[0,1],[0]$ and $P_2 = [4],[2,4],[2],[0,2],[0]$. In the next section we will see an example in which we have a lowest $1$ in a row with lower filtration value than the $\sigma$ from the lemma. 

In the following we will prove one-to-one correspondence between lowest $1$s during the reduction of column $\tau$ and a subset of $\mathcal{P}_\tau$. For some simplex $\mu \in K$, we define $\mathcal{P}_\tau^\mu$, as the set of $V$-paths originating in a facet of $\tau$ and ending in $\mu$, and call them the \textbf{facet paths from} $\bm{\tau}$ \textbf{to} $\bm{\mu}$. We define the number of these paths \[
x_\tau^\mu \coloneqq |\mathcal{P}_\tau^\mu|,
\]
to be the \textbf{$\bm{\tau}$-multiplicity of $\bm{\mu}$}. Note that the elements of $\mathcal{P}_\tau^\mu$ are subsequences of the elements of $\mathcal{P}_\tau$. 

When considering the above example and looking at the multiplicities of the vertices with respect to the different edges, we see that even multiplicity implies that the vertex is never a lowest $1$ in the reduction of a column while odd multiplicity implies the opposite. For example consider the reduction of column $[1,2]$. The vertices $[1]$ and $[2]$ both have odd $[1,2]$-multiplicity and appear as lowest $1$s while vertex $[0]$ has even $[1,2]$-multiplicity and does not appear as a lowest $1$.

\begin{lemma}
\label{lem:oddest_ones}
With the same notations and assumptions as in Lemma \ref{lem:lowest_apparent}, i.e., $\tau^{(k+1)}$ an unpaired simplex and if there are unpaired elements that are the last elements of a path in $\mathcal{P}_\tau$, let $\sigma^{(k)}$ be the one with highest filtration value. Assuming such a $\sigma$ exists, let $\mu^{(k)}\in K$ with $f(\mu)>f(\sigma)$. Then the lowest $1$ of column $\tau$ is in row $\mu$ at some point during the reduction if and only if $x_\tau^\mu$ is odd.
\end{lemma}
\begin{proof}
We split the proof into the two directions of the equivalence. Both directions are proven by inductive arguments.

\enquote{$\Rightarrow$}:
We will prove this direction by an induction over the number of appearing lowest $1$s.

Let the initial lowest $1$ be in row $\mu$. Then $\mathcal{P}_\tau^\mu$ contains the single path $P = \mu$, i.e., $x_\tau^\mu$ is odd. Assume this is true for the first $n$ appearing lowest $1$s. We will now show that it is then also true for the next lowest $1$.
 
Consider $\mu$ with $f(\mu) > f(\sigma)$ to be the row that contains the lowest $1$ after $n$ reduction steps. The entry in row $\mu$ changes its value whenever some column corresponding to a cofacet of $\mu$ gets added to column $\tau$. Let $\{\beta_1, \dots ,\beta_l\}$ be all $l$ columns that are added to column $\tau$ during its reduction that change the value in row $\mu$. Since $f(\mu) > f(\sigma)$, we know by Lemma \ref{lem:lowest_apparent} that each of the $\beta_i$ is part of an apparent pair $(\alpha_i, \beta_i)$ and $\alpha_i$ was a lowest $1$ at some previous point in the reduction. Therefore, by assumption, $x_\tau^{\alpha_i}$ is odd. Furthermore $x_\tau^\mu = \sum_{i = 1}^l x_\tau^{\alpha_i}$. 

We have to consider two cases.

Case one: the entry in row $\mu$ of column $\tau$ initally equals one. Then $l$ has to be even for the entry to be one after the first $n$ reduction steps. If $l>0$, $x_\tau^\mu$ is the sum of an even number of odd values. Hence it is odd. If $l=0$, i.e., $\mu$ is a facet of $\tau$ that is reached only by the trivial path $P$ = $\mu$, then $x_\tau^\mu = 1$. 

Case two: the entry in row $\mu$ of column $\tau$ initally equals zero. Then $l$ has to be odd for the entry to be one after the first $n$ reduction steps. Therefore $x_\tau^\mu$ is the sum of an odd number of odd values. Hence it is odd.

\enquote{$\Leftarrow$}: We will prove the claim by induction over $x_\tau^\mu$ and begin with $x_\tau^\mu = 1$. This means we have some path $P = \alpha_0, \beta_0, \dots, \alpha_r, \beta_r, \mu$ with $\alpha_0$ being a facet of $\tau$. This means the entry in row $\alpha_0$ of column $\tau$ initially equals $1$, and by Lemma \ref{lem:lowest_apparent} (2) we know it has to be set to zero at some point. Assume that the entry in row $\alpha_0$ gets set to zero when some column $\xi$ gets added to column $\tau$. By Lemma \ref{lem:lowest_apparent} (1) we know that $\xi$ is part of an apparent pair and a path in $\mathcal{P}_\tau$. This implies there is another path from a facet of $\tau$ to $\mu$ that contains $P$ as a subsequence. This is a contradiction to the assumption $x_\tau^\mu = 1$. Therefore $\alpha_0$ is the lowest $1$ at some point and reducing it creates a $1$ in row $\alpha_1$. By the same reasoning as with $\alpha_0$ we get that $\alpha_1$ has to be a lowest $1$, and following the $V$-path along to $\mu$ yields that it also has to be a lowest $1$ at some point. 

Assume this holds true for $x_\tau^\mu = n$ with $n$ being odd. We will show it is also true for $x_\tau^\mu = n+2$. 

Let $x_\tau^\mu = n+2$ and let $\gamma$ be the simplex with highest filtration value in which paths in $\mathcal{P}_\tau^\mu$ intersect that are not equal on at least one apparent pair of which the tail has higher filtration value than $\gamma$. This means we are considering paths that might or might not originate in the same facet of $\tau$, are not equal at some point and then intersect (again) in simplex $\gamma$. Note that it is possible that $\gamma = \mu$.

Let $P$ and $Q$ be two of these paths. Assume $Q$ and $P$ are initially the same sequence $\alpha_0,\beta_0, \dots, \alpha_l,\beta_l$. By the assumptions on $\gamma$, there can be no other path in $\mathcal{P}_\tau^\mu$ intersecting this sequence. Therefore by the same reasoning as for the case $x_\tau^\mu = 1$ at some point column  $\beta_l$ gets added to column $\tau$ and the rows corresponding to the first distinct elements of $Q$ and $P$ get set to $1$. Let us denote these sequences by $Q' = \omega_1, \xi_1, \dots, \omega_y, \xi_y, \gamma$ and $P' = \theta_1, \psi_1, \dots, \theta_z, \psi_z, \gamma$. As neither $Q'$ nor $P'$ get intersected due to the assumptions on $\gamma$, we know that at some point of the reduction columns $\xi_y$ and $\psi_z$ get added to column $\tau$. The first addition causes the entry in row $\gamma$ of column $\tau$ to be set to $1$. The second one causes it to be set to $0$. This means that $Q$ and $P$ have no influence on whether there is a lowest $1$ in row $\mu$ at some point. Hence $\mu$ being a lowest $1$ depends on the odd $n$ other paths in $\mathcal{P}_\tau^\mu$, which, by the induction assumption, tells us that $\mu$ indeed is a lowest $1$ at some point.
\end{proof}

With this we also know how lowest $1$s appear with respect to $V$-paths originating in a facet of $\tau$. Again assuming the existence of unpaired last elements of a path in $\mathcal{P}_\tau$ we choose $\sigma$ to be the one with highest filtration value and define: \[
\mathcal{O}_\tau^\sigma \coloneqq \{(\alpha, \beta) \in V \mid f(\alpha) > f(\sigma), \, x_\tau^\alpha \operatorname{mod} 2 = 1 \},
\]
the \textbf{odd pairs} between $\sigma$ and $\tau$. 

\begin{thm}
\label{thm:vpath_delta}
Let $F_*,K,V$, and $f$ be defined as above. Let $\tau^{(k+1)}\in K$ be a simplex that is either not part of an apparent pair or the tail of an apparent pair. If there is an unpaired cell that is the last element of a path in $\mathcal{P}_\tau$, let $\sigma$ be the one with highest filtration value. If $\sigma$ exists, then:
\[
    \operatorname{red}(\tau) \geq |\mathcal{O}_\tau^\sigma|,
\]
with equality holding if $\tau$ and $\sigma$ are a persistence pair or if column $\tau$ gets reduced to zero.
\end{thm}
\begin{proof}
By Lemma \ref{lem:lowest_apparent} (2) we know that every lowest $1$ in column $\tau$ must be reduced until column $\tau$ is zero or has a lowest $1$ in row $\sigma$. By Lemma \ref{lem:oddest_ones} we know that each lowest $1$ corresponds to an element in $\mathcal{O}_\tau^\sigma$. If $\sigma$ is not yet part of a persistence pair or column $\tau$ is reduced to zero, we get that $\operatorname{red}(\tau) = |\mathcal{O}_\tau^\sigma|$. If $\sigma$ is part of a persistence pair, the reduction continues and $\operatorname{red}(\tau) > |\mathcal{O}_\tau^\sigma|$. 
\end{proof}
This theorem naturally extends onto the whole boundary matrix $B$ of the filtration $F_*$.
Consider the set $\{\tau_0^{(k+1)}, \dots, \tau_r^{(k+1)}\}$ of all simplices such that each $\tau_i$ is either unpaired or the tail of an apparent pair and for which an unpaired simplex that is the last element of a path in $\mathcal{P}_{\tau_i}$ exists. Let $\sigma_i^{(k)}$ be the ones with highest filtration value for the respective $\tau_i$. Then \[
\operatorname{red}(B) \geq \sum_{i=0}^r|\mathcal{O}_{\tau_i}^{\sigma_i}|,
\]
where equality holds if all columns $\tau_i$ get reduced to zero or $\tau_i$ gets paired with $\sigma_i$.

Note that in a practical application one might just set the columns of lower dimensional elements of apparent pairs to zero before doing any reduction. In this case we would still get a lower bound on the needed reduction steps for the cleared matrix by only considering the critical simplices with respect to the apparent gradient.

The special case we have seen at the beginning of this section follows from Theorem \ref{thm:vpath_delta} and some further insight. 
It holds that for a connected simplicial complex $K$ and any Morse matching $M$ on $K$ we can construct a Morse matching $M'$ with exactly one critical vertex and at most the number of higher dimensional critical cells. See \cite{joswig2004computing}[Lemma 2.2]. 

This implies, that an optimal Morse matching of a connected simplicial complex always has a single critical vertex.

\begin{cor}
Let $K$ be a finite one-dimensional connected simplicial complex and $F_*$ a simplexwise filtration, such that the apparent gradient is an optimal Morse matching on $K$. Let $C$ be the set of critical edges with respect to $V$ and let $\mu_c$ be the single critical vertex. Then
\[ \operatorname{red}(B) = \sum_{\tau \in C} |\mathcal{O}_\tau^{\mu_c}|.
\]
\end{cor}
\begin{proof}
Every critical edge in $K$ has to be reduced to zero. By Theorem \ref{thm:vpath_delta} the claim follows.
\end{proof}

This corollary extends to higher dimensional simplicial complexes in the sense that any optimal Morse matching $V$ on a simplicial complex $K$ induces an optimal Morse matching $V^1$ on the one-skeleton $K^1$ of $K$. To see this, consider that $V$ matches all except one vertex, which is the best we can achieve. 

Hence, if we have some filtration $F$ for which the apparent gradient $V$ is an optimal Morse matching, we know that the number of column additions in the boundary matrix of $F$ that correspond to edges is equal to $\sum_{\tau \in C_e} |\mathcal{O}_{\tau}^{\mu_c}|$, where $C_e$ is the set of critical edges and edges that are the tail of an apparent pair.

An interesting example for a similar behaviour in dimension two is the Dunce hat from Figure \ref{fig:morse_dunce}. Let us denote the discrete gradient depicted there by $V$ and let $F_* = \mathcal{F}(V)$. Reducing the boundary matrix $B$ of $F_*$ takes a total of $42$ reduction steps. Of those, $34$ happen in columns corresponding to edges. To be more precise, consider one of the $17$ edges that is not the head of an apparent pair or is critical. The corresponding column is reduced to zero by exactly two column additions which correspond to the $V$-paths of length one, originating in the facets of the edge and ending in the critical vertex $[0]$. 

Furthermore there are eight reduction steps needed to reduce the column corresponding to the critical triangle, which correspond to the $V$-paths from the triangles boundary to the critical edge $[1,2]$. 

\section{Worst Case Example for the running time of the standard reduction algorithm}
As we have discussed in Chapter \ref{ch:homology}, the worst case running time for the standard reduction scheme lies in $\mathcal{O}(N^3)$, where $N$ is the number of faces of the simplicial complex.

A popular construction of filtrations that achieve a bound of $\Theta(N^3)$ was introduced by Dimitriy Morozov in \cite{morozov}. We will introduce a new series of examples that is inspired by the series of Dimitriy Morozov but achieves the bound for slightly different reasons. Two advantages of the new series of examples are that it is easy to construct algorithmically and that it grows faster than the previously known examples. \\

Consider the vertices of an $n$-gon $\{[1], \dots, [n] \}$, which we will call the \textbf{outer vertices}. Furthermore we have a \textbf{center vertex} $[0]$ inside the $n$-gon. We will call the edges of the $n$-gon $[i,i+1]$ for $i = 1,\dots,n-1$ and $[1,n]$ the \textbf{outer edges} and we will include edges $[0,i]$ between each bounding vertex and the center which we will call the \textbf{base edges}. 

We add the triangles $[0,i,i+1]$ and call them \textbf{base triangles}. Furthermore we add the triangle $[0,1,n]$ which we call the \textbf{closing triangle}. The following figure illustrates our construction so far for $n=3$, i.e., three bounding vertices.

\begin{figure}[H]
\noindent%
\centering%
\begin{tikzpicture}

\node[b_circle, name path=0, label=below:{0}] (0) at (0,-0.5) {};

\node[circle, name path=1, label=above:{1}] (1) at (0,2) {};

\node[circle, name path=2, label=right:{2}] (2) at  (2,-2) {};

\node[circle, name path=2, label=left:{3}] (3) at  (-2,-2) {};

\draw[dashed, thick] (1) to (2) to (3) to (1);
\draw[dotted, very thick] (0) to (1);
\draw[dotted, very thick] (0) to (2);
\draw[dotted, very thick] (0) to (3);

\fill[opacity = 0.4] (0.center)--(2.center)--(3.center);
\fill[opacity = 0.4] (0.center)--(1.center)--(2.center);
\fill[opacity = 0.2] (0.center)--(1.center)--(3.center);

\node[circle, name path=1, label=above:{1}] (1) at (0,2) {};

\node[circle, name path=2, label=right:{2}] (2) at  (2,-2) {};

\node[circle, name path=2, label=left:{3}] (3) at  (-2,-2) {};

\end{tikzpicture}

\caption{First part of the construction of a series of worst case examples.}
\label{fig:fin_pizza_base}
\end{figure}


In Figure \ref{fig:fin_pizza_base} the outer vertices are white, the center vertex is black, the outer edges are dashed lines, the base edges are dotted lines. The base triangles are dark gray, while the closing triangle is depicted in a light gray. 

We will now extend the construction. We place (in $3$-space) a vertex above each of the base edges. This means, we get another $n$ vertices $\{[n+1], \dots , [n+n]\}$ which we will call the \textbf{fin vertices}. Then we will connect these vertices with the central vertex and the other end of the base edge they are placed above. This means, we include the edges $\{[0,n+1], \dots, [0,n+n]\}$ and $\{[1,n+1], \dots, [n,n+n]\}$. We will refer to them as \textbf{fin edges}. Finally, we add the triangles $\{[0,1,n+1], \dots ,[0,n,n+n]\}$, which we call  \textbf{fin triangles}. We call this construction an $\bm{n}$\textbf{-gon with center and fins}.

Those familiar with the example by Dimitriy Morozov might recognize the similarities with respect to the construction and naming. With this the construction of our space is complete. See the following figure for an illustration of the complete construction for $n = 3$.

\begin{figure}[H]
\noindent%
\centering%
\begin{tikzpicture}

\node[circle, name path=0, label=below:{0}] (0) at (0,0,-0.5) {};

\node[circle, name path=1, label=below:{3}] (3) at (0,0,4) {};

\node[circle, name path=2, label=right:{2}] (2) at  (3,0,-3) {};

\node[circle, name path=2, label=left:{1}] (1) at  (-3,0,-3) {};

\node[b_circle, name path=1, label=left:{6}] (6) at (0,0.75,2) {};

\node[b_circle, name path=1, label=left:{5}] (5) at (0.5,.075,-2) {};


\node[b_circle, name path=1, label=right:{4}] (4) at (-1.5,0.75,-1.5) {};

\draw[dashed, thick] (1) to (2) to (3) to (1);
\draw[dotted, thick] (0) to (1);
\draw[dotted, thick] (0) to (2);
\draw[dotted, thick] (0) to (3);

\draw[thick] (3) to (6);
\draw[thick] (0) to (6);


\draw[thick] (2) to (5);
\draw[thick] (0) to (5);

\draw[thick] (1) to (4);
\draw[thick] (0) to (4);

\fill[opacity = 0.4] (0.center) -- (1.center) -- (4.center);
\fill[opacity = 0.4] (0.center) -- (2.center) -- (5.center);
\fill[opacity = 0.4] (0.center) -- (3.center) -- (6.center);

\end{tikzpicture}

\caption{Adding the fins.}
\label{fig:fin_pizza_complete}
\end{figure}

In Figure \ref{fig:fin_pizza_complete} we can see the fin vertices in black, the fin edges as solid lines, and the fin triangles in dark gray. 

For any $n$, this simplicial complex has $1+2n$ vertices, $4n$ edges and $2n$ triangles. In particular, the number of simplices lies in $\Theta(n)$. The following table lists all simplices for some $n$ in the categories we created:
\begin{center}
    \begin{tabular}{ll}
        Central vertex: & $[0]$, \\
        Outer vertices: & $[1],\dots,[n]$, \\
        Fin vertices: & $[n+1],\dots,[n+n]$, \\
        Bounding edges: & $[1,2],[2,3],\dots,[n-1,n],[1,n]$,\\
        Fin edges: & $[0,1+n],\dots,[0,n+n],[1,1+n],\dots,[n,n+n]$,\\
        Base edges: & $[0,1],[0,2], \dots, [0,n]$, \\
        Base triangles: & $[0,1,2],\dots,[0,n-1,n]$,\\
        Closing triangle: & $[0,1,n]$, \\
        Fin triangles: & $[0,1,n+1],\dots,[0,n,n+n]$.
    \end{tabular}
\end{center}

Consider the filtration $F_*$ defined by concatenating all rows of the above table in order of appearance. Consider the relations between the base edges and the base triangles. Edge $[0,1]$ is not part of an apparent pair, since its oldest cofacet $[0,1,2]$ has edge $[0,2]$ as a youngest facet. But this means $([0,2],[0,1,2])$ is an apparent pair. Similarly $([0,i],[0,i-1,i])$ is an apparent pair for $i = 3,\dots,n$.
So all base edges, except $[0,1]$ are paired with a base triangle. 

We will now prove that the number of additions with respect to the standard reduction scheme for filtrations specified by this construction lie in $\Theta(n^3)$. Note that we consider $n \geq 3$.

To illustrate our reasoning we will follow along an example for $n = 5$. In the following figure we see boundary matrices corresponding to this example. We have cropped them such that only the relations between triangles and edges are visible. The light squares indicate $0$s while the dark squares indicate $1$s.

\begin{figure}[H]
\noindent%
\centering%
\begin{subfigure}[l]{0.49\textwidth}
\begin{center}
\includegraphics[scale=0.6]{ConnectingMorsePersistence/Figures/Reduction/default.png}
\subcaption{Submatrix of edges to triangles before any reduction steps are done.}
\end{center}
\end{subfigure}
\begin{subfigure}[r]{0.49\textwidth}
\begin{center}
\includegraphics[scale=0.6]{ConnectingMorsePersistence/Figures/Reduction/finaltri.png}
\subcaption{Submatrix of edges to triangles after the closing triangle has been reduced.}
\end{center}
\end{subfigure}
\caption{Cropped boundary matrices of the $n$-gon with center and fins for $n=5$.}
\label{fig:redfin}
\end{figure}

Consider Figure \ref{fig:redfin} (a). The base triangle columns have lowest $1$s in the rows of the respective other part of the apparent pair they are an element~of. The first conflict appears in row $[0,5]$ of column $[0,1,5]$. Removing it by adding column $[0,4,5]$ creates a new conflict in row $[0,4]$ which is again resolved by adding the column of the corrseponding apparent pair. After four steps, or in the general case after $n-1$ steps, we arrive at the situation depicted in Figure  \ref{fig:redfin} (b). The closing triangle is paired with the last bounding edge $[1,5]$, or $[1,n]$ in the general case. We now have to reduce the fin triangles. In the example, these are the triangles $[0,1,6],\dots, [0,5,10]$.

The first fin triangle has no conflicts. It is indeed already paired with the base edge $[0,1]$. The second fin triangle has a conflict with the first base triangle and resolving it creates a conflict in row $[0,1]$, which is resolved by adding the column of the first fin triangle. Since it is sufficient for our proof, we will only count the $1$s in the columns we are reducing. Consider the number of additions in these first two reduction steps. When resolving the first conflict there are three $1$s in the column of the second fin triangle. This means we get three additions. See Figure \ref{fig:finfin} (a) for the boundary matrix of our example after this addition. 

\begin{figure}[H]
\noindent%
\centering%
\begin{subfigure}[l]{0.49\textwidth}
\begin{center}
\includegraphics[scale=0.6]{ConnectingMorsePersistence/Figures/Reduction/secondfin.png}
\subcaption{Submatrix of edges to triangles before any reduction steps are done.}
\end{center}
\end{subfigure}
\begin{subfigure}[r]{0.49\textwidth}
\begin{center}
\includegraphics[scale=0.6]{ConnectingMorsePersistence/Figures/Reduction/final.png}
\subcaption{Submatrix of edges to triangles after the final triangle has been reduced.}
\end{center}
\end{subfigure}
\caption{Reducing the fin triangles.}
\label{fig:finfin}
\end{figure} 

Whenever we add a base triangle $[0,i,i+1]$ to some fin triangle to resolve the conflict in row $[0,i+1]$ there is a new lowest $1$ appearing in row $[0,i]$. This procedure continues until we reach a lowest $1$ in row $[0,1]$ which is resolved by adding the column of the first fin triangle. After this step each fin triangle is reduced. Furthermore, whenever we add a base triangle there is a $1$ appearing in the bounding edge that is a facet of the base triangle. See the entry in row $[1,2]$ of column $[0,2,7]$ in Figure \ref{fig:finfin} (a). 

This means there are $1$s amassing during the reduction of a fin triangle. For $k = 2,\dots,n$, the $k$th fin triangle is reduced in $k$ steps. The first $k-1$ correspond to base triangles and the last one to the first fin triangle. Since each addition with a base triangle increases the number of $1$s in the column we are reducing and since we start at three ones for each fin triangle for the reduction of the $k$th fin triangle, we get 
\[
	3 + 4 + \dots + (3 + (k-1)) > 1 + 2 + \dots + (1+k-1) = \sum_{j=1}^k j
\]
additions. Considering our example we can see these amassed $1$s in the top right corner of the matrix in \ref{fig:finfin} (b).

Summing up the additions to reduce all the fin triangles starting at the second one, we get 
\[
\sum_{k=2}^n \sum_{j=1}^k j
\]
additions. Recall that we also had to reduce the closing triangle in more than one step. Hence for the total number of additions $a$ it holds that: 

\begin{equation}
\begin{split}
 a & \geq 1+\sum_{k=2}^n \sum_{j=1}^k j \\
& = \sum_{k=1}^n \sum_{j=1}^k j\\
& = n1 + (n-1)2 + ... + 2(n-1) + 1n\\
& = \sum_{l=0}^n (n-l)(l+1) \\
& \geq \sum_{l=0}^n (n-l)l \\
& = \sum_{l=0}^n nl - \sum_{l=0}^n l^2.
\end{split}
\end{equation}

With $\sum_{l=0}^n l = \frac{n(n+1)}{2}$ and $\sum_{l=0}^n l^2 = \frac{n(n+1)(2n+1)}{6}$ we get
\begin{equation}
\begin{split}
 \sum_{l=0}^n nl - \sum_{l=0}^n l^2  & = n\frac{n(n+1)}{2} -\frac{n(n+1)(2n+1)}{6} \\
& = n(\frac{n(n+1)}{2} -\frac{(n+1)(2n+1)}{6}) \\
& = n\frac{3n(n+1)-(n+1)(2n+1)}{6} \\
& = \frac{n(n+1)(3n-2n-1)}{6} \\
&= \frac{(n-1)n(n+1)}{6} \in \Theta(n^3).
\end{split}
\end{equation}
This means we get cubic running time for the construction of the $n$-gon with center and fins for any $n\geq 3$. 

The following figure compares this construction to the one of Dimitriy Morozov. On the $x$-axis we have the number of faces of the underlying simplicial complex and on the $y$-axis the number of additions in the standard reduction scheme.

\begin{figure}[H]
\noindent%
\centering%
% This file was created by tikzplotlib v0.9.2.
\begin{tikzpicture}
\definecolor{color0}{rgb}{0.0235294117647059,0.603921568627451,0.952941176470588}
\definecolor{color1}{rgb}{0.976470588235294,0.450980392156863,0.0235294117647059}

\begin{axis}[
xlabel = Number of faces in underlying simplical complex,
tick align=outside,
tick pos=left,
x grid style={white!69.0196078431373!black},
xmin=36.4, xmax=665.6,
xtick style={color=black},
ylabel = Number of additions,
y grid style={white!69.0196078431373!black},
ymin=-5447.05, ymax=126620.05,
ytick style={color=black},
yticklabel style={
        /pgf/number format/fixed,
        /pgf/number format/precision=5
},
scaled y ticks=false
]
\addplot [only marks, mark=*, draw=color0, fill=color0, colormap/viridis]
table{%
x                      y
77 569
91 723
105 980
119 1203
133 1439
147 1651
161 1930
175 2273
189 2694
203 2931
217 3408
231 3792
245 4184
259 4544
273 4994
287 5559
301 6417
315 6696
329 7053
343 7798
357 8627
371 9133
385 9713
399 10495
413 11620
427 11901
441 12674
455 13391
469 14095
483 14775
497 15544
511 16727
525 18461
539 18782
553 19196
567 20405
581 22205
595 23053
609 23980
623 25158
637 26965
};
\addplot [only marks, mark=*, draw=color1, fill=color1, colormap/viridis]
table{%
x                      y
65 556
73 707
81 879
89 1073
97 1290
105 1531
113 1797
121 2089
129 2408
137 2755
145 3131
153 3537
161 3974
169 4443
177 4945
185 5481
193 6052
201 6659
209 7303
217 7985
225 8706
233 9467
241 10269
249 11113
257 12000
265 12931
273 13907
281 14929
289 15998
297 17115
305 18281
313 19497
321 20764
329 22083
337 23455
345 24881
353 26362
361 27899
369 29493
377 31145
385 32856
393 34627
401 36459
409 38353
417 40310
425 42331
433 44417
441 46569
449 48788
457 51075
465 53431
473 55857
481 58354
489 60923
497 63565
505 66281
513 69072
521 71939
529 74883
537 77905
545 81006
553 84187
561 87449
569 90793
577 94220
585 97731
593 101327
601 105009
609 108778
617 112635
625 116581
633 120617
};
\end{axis}

\end{tikzpicture}

\caption{Comparision of the two constructions that yield cubical running time. Construction by Dimitriy Morozov in blue and the one propsed in this section in orange.}
\label{fig:compare}
\end{figure}
As we can see the construction propsed here grows much faster. As we have previously discussed the number of vertices, edges and triangles of the $n$-gon with fins depends on $n$. To be precise the $f$-vector of the construction is $(1+2n,4n,2n)$, i.e. all entries depend linearly on $n$, in particular all entries depend linearly on the number of vertices. For the construction of Dimitriy Morozov it also holds that the number of edges and triangles depends linearly on the number of vertices. 

Python code, to generate the filtrations of both constructions can be found on \href{https://github.com/IvanSpirandelli/Masterarbeit/blob/master/Examples/worst_case_examples.py}{[GitHub]}, see also \cite{github}. 

When considering the $n$-gon construction in terms of Theorem \ref{thm:vpath_delta} we see that for the reduction of fin triangles we follow $V$-paths defined by the apparent pairs of base triangles and base edges. Indeed, we do not get equality since the final reduction step always corresponds to an addition of the first fin triangle, which is unpaired with respect to the apparent gradient.

The lower bound we get by Theorem \ref{thm:vpath_delta} equals $\sum_{r=1}^{n-1} r = \frac{(n-1)n}{2}$ which is not the exact number of column additions but at least asymptotically correct. 

However we can generate an almost identical filtration by reversing the order of the base edges which by the same reasoning results in the same asymptotic bound of additions but has a single apparent pair. In this case the bound we get by Theorem \ref{thm:vpath_delta} is as bad as asymptotically possible.

\section{Conclusions}
The initial question that motivated this chapter was how complicated discrete Morse functions correspond to difficult persistent homology computations. While we can not give a complete answer to this we have shed some light on the relation between filtrations and gradient vector fields. We have proven a new theorem (Theorem \ref{thm:vpath_delta}), which states how the apparent gradients give a lower bound on the complexity of persistent homology computations. On the other hand we have also seen that two different filtrations of the same simplicial complex can yield an asymptotically perfect bound or a trivial constant bound that has an arbitrarily large difference to the actual number of column additions. Nevertheless, we hope that the results we discussed here prove fruitful for future research on the topic.


\newpage
\newpage\null\thispagestyle{empty}\newpage
\chapter{Random Simplicial Complexes}
\label{ch:random}
In this chapter we carry out some experiments regarding apparent pairs on random simplicial complexes. We will consider two different notions of random simplicial complexes. Namely, alpha complexes constructed on randomly generated point clouds and two-dimensional analogues of the Erdös--Rényi random graph model as introduced by Meshulam and Linial in \cite{LinialMeshulam} and extended by Meshulam and Wallach in \cite{Meshulam_Wallach_2009}. An extensive discussion of results regarding these was done by Kahle in \cite{kahle2016random}. For some insights and experiments regarding apparent pairs and Vietoris--Rips complexes see \cite{bauer2019ripser}.

\section{Erdös--Rényi Analogues}
Consider the set $G(n)$ of all graphs on vertex set $\{1,\dots,n\}$. The Erdös--Rényi random graph model $G(n,p)$ is the probability distribution on $G(n)$, \enquote{where every edge is included with probability $p$ jointly independently.} \cite{kahle2016random}[page 1].

A general analogue in terms of simplicial complexes is the random $k$-complex $Y_k(n,p)$ introduced in \cite{Meshulam_Wallach_2009}. $Y_k(n,p)$ \enquote{contains the complete $(k-1)$-skeleton of a simplex on $n$ vertices, and every $k$-dimensional face appears independently with probability $p$.} \cite{kahle2016random}[page 4].

In our case $Y_k(n,p)$ will have $n+1$ vertices since we want to keep it consistent with our previous notations for filtrations.

We will construct filtrations by ordering simplices by dimension first and lexicographically within the respective dimensions. Then we take the full $(k-1)$-skeleton in that order. Finally we include each $k$-face with probability $p$. 
The following Figure \ref{fig:randomk} gives an example of a random $2$-complex on four points, i.e., for $n=3$.

\begin{figure}[H]
%\centering%
\begin{subfigure}[b]{0.99\textwidth}
\begin{center}
\begin{tikzpicture}

\node[b_circle, name path=0, label=below:{0}] (0) at (0,0) {};

\node[b_circle, name path=1, label=below:{1}] (1) at (4,0) {};

\node[b_circle, name path=2, label=above:{2}] (2) at  (2,4) {};

\node[b_circle, name path=2, label=above left:{3}] (3) at  (2,1.5) {};


\draw[thick] (0) to (1);
\draw[thick] (0) to (2);
\draw[thick] (0) to (3);

\draw[thick] (1) to (2);
\draw[thick] (1) to (3);
\draw[thick] (2) to (3);

\fill[opacity = 0.2] (0.center) -- (1.center) -- (3.center);
\fill[opacity = 0.2] (2.center) -- (1.center) -- (3.center);

\end{tikzpicture}
\end{center}
\end{subfigure}
\caption{Random $2$-complex on four points.}
\label{fig:randomk}
\end{figure}

We have not explicitly stated a value for $p$ with respect to Figure \ref{fig:randomk}, since the complex might result from any $p > 0$ with varying probability. The corresponding filtration is \[
F_* = ([0],[1],[2],[3],[0,1],[0,2],[0,3],[1,2],[1,3],[2,3],[0,1,3],[1,2,3]).
\]


\section{Apparent Pairs on Random 2-Complexes}
In this section we will analyze apparent pairs in the context of random $2$-complexes $Y_2(n,p)$ for fixed $n$ and varying $p$. We consider filtrations as specified in the previous section.

An interesting observation is that for $n\in \mathbb{N}$, $p \in \{0,1\}$, and simplices of $Y_2(n,p)$ in lexicographical order, the apparent gradient always yields a perfect Morse matching. 

For $p = 0$ and $n$ arbitrary but fixed we get the following filtration: 
\[
F_* = ([0],[1],\dots,[n],[0,1],\dots,[0,n],[1,2],\dots,[n-1,n]).
\]
Looking at the apparent pairs we see that $[0]$ is not paired, i.e., is a critical cell. Vertex $[1]$ gets paired with $[0,1]$, $[2]$ gets paired with $[0,2]$, and so on until $[n]$ gets paired with $[0,n]$. All other edges remain unpaired, and there are no triangles or higher-dimensional faces. Furthermore, each edge that does not get paired, corresponds to a one-dimensional hole in the simplicial complex. Therefore, we get that the number of critical cells is equal to the Betti number in the respective dimensions. 

If $p = 1$, every edge that is not paired with a vertex gets paired with a triangle. Edge $[1,2]$ is the youngest facet of triangle $[0,1,2]$ which is the oldest cofacet of $[1,2]$. The same holds for $[1,3]$ and $[0,1,3]$ and so on until we reach $[1,n]$ and $[0,1,n]$. Then $[2,3]$ to $[2,n]$ get paired with $[0,2,3]$ to $[0,2,n]$. This pattern continues until $[n-1,n]$ gets paired with $[0,n-1,n]$. This means all vertices get paired with the edge between this vertex and $[0]$, while all edges that do not contain vertex $[0]$ get paired with the triangle containing this edge and vertex $[0]$.

For the following figure we sample $Y_2(10,p)$ for $p = 0,0.01,\dots,0.99,1$, calculate the Betti numbers for the resulting complex and subtract the number of critical cells in the apparent gradients. Then we count how often this difference equals zero and divide it by the total number of generated complexes, i.e., we calculated the ratio of the apparent gradient being a perfect Morse matching for different values of $p$. For each $p$ we constructed $1000$ simplicial complexes to average over.

\begin{figure}[H]
%\centering%
\begin{subfigure}[c]{0.95\textwidth}
\begin{center}
% This file was created by tikzplotlib v0.9.2.
\begin{tikzpicture}

\definecolor{color0}{rgb}{0.12156862745098,0.466666666666667,0.705882352941177}

\begin{axis}[
tick align=outside,
tick pos=left,
xlabel = $p$,
ylabel = $\%$ of perfect gradients, 
x grid style={white!69.0196078431373!black},
xmin=-0.05, xmax=1.05,
xtick style={color=black},
y grid style={white!69.0196078431373!black},
ymin=-0.05, ymax=1.05,
ytick style={color=black}
]
\addplot [semithick, color0]
table {%
0 1
0.01 0.9795
0.02 0.9225
0.03 0.8375
0.04 0.729
0.05 0.633
0.06 0.511
0.07 0.43
0.08 0.323
0.09 0.247
0.1 0.1755
0.11 0.132
0.12 0.0865
0.13 0.0665
0.14 0.0435
0.15 0.0235
0.16 0.0175
0.17 0.0115
0.18 0.0075
0.19 0.0025
0.2 0.0035
0.21 0.001
0.22 0.0005
0.23 0.0005
0.24 0
0.25 0
0.26 0
0.27 0
0.28 0
0.29 0
0.3 0
0.31 0
0.32 0
0.33 0
0.34 0
0.35 0
0.36 0
0.37 0
0.38 0
0.39 0
0.4 0
0.41 0
0.42 0
0.43 0
0.44 0
0.45 0
0.46 0
0.47 0
0.48 0.0005
0.49 0
0.5 0
0.51 0
0.52 0
0.53 0.0005
0.54 0
0.55 0.001
0.56 0.0005
0.57 0.0015
0.58 0.001
0.59 0.002
0.6 0.0025
0.61 0.006
0.62 0.004
0.63 0.0055
0.64 0.008
0.65 0.006
0.66 0.0145
0.67 0.014
0.68 0.0125
0.69 0.0245
0.7 0.0255
0.71 0.025
0.72 0.031
0.73 0.0495
0.74 0.048
0.75 0.0585
0.76 0.0615
0.77 0.076
0.78 0.092
0.79 0.1045
0.8 0.1215
0.81 0.1325
0.82 0.1515
0.83 0.1785
0.84 0.204
0.85 0.228
0.86 0.2635
0.87 0.2905
0.88 0.3165
0.89 0.3325
0.9 0.3975
0.91 0.4215
0.92 0.493
0.93 0.53
0.94 0.594
0.95 0.6435
0.96 0.7105
0.97 0.773
0.98 0.8405
0.99 0.9155
1 1
};
\end{axis}

\end{tikzpicture}

\end{center}
\end{subfigure}
\caption{Probability of the apparent gradient being a perfect Morse matching for $Y_2(10,p)$ with different values of $p$.}
\label{fig:perfect_apparent}
\end{figure}

As we have previously argued for $p = 0$ the apparent gradient is perfect. In order for an apparent gradient to be not perfect, the number of critical cells in some dimension $l$ has to be larger than the Betti number $\beta_l$.

This can happen for example, if two triangles $[a,b,c]$ and $[b,c,d]$ have edge $[b,c]$ as a youngest facet but $[b,c]$ can only have one of them as the oldest cofacet. Without loss of generality it is $[a,b,c]$. This means that triangle $[b,c,d]$ is critical, i.e., the number of critical cells in dimension two is higher than $\beta_2$. Furthermore there also has to be some edge which can not be paired, hence the number of critical edges equals $\beta_1 + 1$. Up to a certain point it holds that the higher number of simplices in a complex drawn from $Y_2(10,p)$ causes situations to occur in which triangles can not be paired. This could explain the inital decline of the curve. 

If more and more simplices appear, we get closer to the structure of the full simplex in which the apparent gradient is a perfect Morse matching again, hence the increase after $p = 0.6$.

For the next figure we compute
\[
	d = \sum_{i=0}^2 c_i - \sum_{i=0}^2 \beta_i,
\] for each sampled complex. Here $c_i$ is the number of critical cells in dimension~$i$. We average these values over $1000$ sampled complexes for $Y_2(10,p)$ and $200$ sampled complexes for $Y_2(15,p)$ and $Y_2(20,p)$.

\begin{figure}[H]
%\centering%
\begin{subfigure}[c]{0.95\textwidth}
\begin{center}
% This file was created by tikzplotlib v0.9.2.
\begin{tikzpicture}

\definecolor{color0}{rgb}{0.12156862745098,0.466666666666667,0.705882352941177}

\definecolor{color1}{rgb}{0.976470588235294,0.450980392156863,0.0235294117647059}

\definecolor{color2}{rgb}{0.172549019607843,0.627450980392157,0.172549019607843}

\begin{axis}[
tick align=outside,
tick pos=left,
xlabel = $p$,
ylabel = Averaged $d$-value, 
x grid style={white!69.0196078431373!black},
xmin=-0.05, xmax=1.05,
xtick style={color=black},
y grid style={white!69.0196078431373!black},
%ymin=-0.954, ymax=20.034,
ymin=-0.954, ymax=120,
ytick style={color=black}
]
\addplot [semithick, color0]
table {%
0 0
0.01 0.042
0.02 0.165
0.03 0.364
0.04 0.661
0.05 0.973
0.06 1.417
0.07 1.843
0.08 2.426
0.09 3.007
0.1 3.752
0.11 4.361
0.12 5.252
0.13 5.913
0.14 6.795
0.15 7.717
0.16 8.612
0.17 9.508
0.18 10.421
0.19 11.115
0.2 12.136
0.21 13.261
0.22 14.325
0.23 14.945
0.24 15.778
0.25 16.516
0.26 17.246
0.27 17.808
0.28 18.305
0.29 18.625
0.3 18.719
0.31 19.023
0.32 19.08
0.33 19.029
0.34 18.858
0.35 18.935
0.36 18.641
0.37 18.673
0.38 18.373
0.39 18.021
0.4 17.762
0.41 17.279
0.42 16.928
0.43 16.368
0.44 16.164
0.45 15.79
0.46 15.101
0.47 14.849
0.48 14.319
0.49 14.09
0.5 13.664
0.51 13.237
0.52 12.881
0.53 12.352
0.54 11.867
0.55 11.66
0.56 10.956
0.57 10.786
0.58 10.348
0.59 10.173
0.6 9.696
0.61 9.39
0.62 8.938
0.63 8.65
0.64 8.271
0.65 8.208
0.66 7.671
0.67 7.398
0.68 7.091
0.69 6.714
0.7 6.513
0.71 6.151
0.72 5.937
0.73 5.681
0.74 5.373
0.75 5.175
0.76 4.866
0.77 4.579
0.78 4.43
0.79 4.1
0.8 3.897
0.81 3.679
0.82 3.414
0.83 3.168
0.84 2.984
0.85 2.727
0.86 2.519
0.87 2.352
0.88 2.186
0.89 2.028
0.9 1.759
0.91 1.626
0.92 1.363
0.93 1.242
0.94 1.022
0.95 0.867
0.96 0.674
0.97 0.502
0.98 0.349
0.99 0.176
1 0
};

\addplot [semithick, color1]
table {%
0 0
0.01 0.32
0.02 1.04
0.03 2.09
0.04 3.97
0.05 6.36
0.06 9.01
0.07 11.1
0.08 14.48
0.09 18.04
0.1 21.68
0.11 25.65
0.12 29.82
0.13 33.97
0.14 38.86
0.15 42.46
0.16 46.96
0.17 50.09
0.18 52.95
0.19 55.27
0.2 56.21
0.21 56.95
0.22 55.65
0.23 56.52
0.24 55.31
0.25 54.18
0.26 52.49
0.27 51.58
0.28 50.19
0.29 49.78
0.3 47.2
0.31 46.45
0.32 44.81
0.33 43.08
0.34 41.89
0.35 40.19
0.36 39.46
0.37 38.36
0.38 36.55
0.39 35.84
0.4 34.2
0.41 33.4
0.42 33.43
0.43 31.64
0.44 29.44
0.45 28.18
0.46 28.05
0.47 26.74
0.48 25.23
0.49 25.1
0.5 23.91
0.51 22.77
0.52 22.03
0.53 21.59
0.54 20.24
0.55 19.96
0.56 19.07
0.57 18.42
0.58 17.66
0.59 16.3
0.6 16.59
0.61 16.47
0.62 15.4
0.63 15.12
0.64 14.11
0.65 13.17
0.66 12.7
0.67 12.35
0.68 12.18
0.69 11.11
0.7 11.11
0.71 10.1
0.72 10.27
0.73 9.27
0.74 8.66
0.75 8.1
0.76 7.96
0.77 7.57
0.78 6.95
0.79 6.52
0.8 6.32
0.81 5.91
0.82 5.83
0.83 4.86
0.84 4.96
0.85 4.75
0.86 3.91
0.87 3.93
0.88 3.36
0.89 3.16
0.9 2.73
0.91 2.35
0.92 2.19
0.93 1.77
0.94 1.73
0.95 1.47
0.96 1.23
0.97 0.82
0.98 0.51
0.99 0.27
1 0
};

\addplot [semithick, color2]
table {%
0 0
0.02 3.82
0.04 13.74
0.06 29.44
0.08 49
0.1 71.58
0.12 92.88
0.14 111.12
0.16 116.02
0.18 109.94
0.2 105.66
0.22 98.82
0.24 91.02
0.26 83.58
0.28 79.86
0.3 71.14
0.32 67.34
0.34 63
0.36 56.4
0.38 53.44
0.4 49.08
0.42 45.36
0.44 42.56
0.46 39.12
0.48 34.58
0.5 33.5
0.52 32.12
0.54 29.24
0.56 28.22
0.58 24.92
0.6 23.76
0.62 20.74
0.64 20.38
0.66 17.78
0.68 16.78
0.7 15.4
0.72 13.7
0.74 12.58
0.76 10.66
0.78 10.48
0.8 8
0.82 7.88
0.84 6.56
0.86 6.14
0.88 3.98
0.9 3.88
0.92 3.06
0.94 2.06
0.96 1.2
0.98 0.62
1 0
};
\end{axis}


\end{tikzpicture}

\end{center}
\end{subfigure}
\caption{Average difference of summed Betti numbers and summed critical cells with resepct to the apparent gradient. $Y_2(10,p)$ in blue, $Y_2(15,p)$ in orange and $Y_2(20,p)$ in green.}
\label{fig:average_diff_betti_apparent}
\end{figure}

It is interesting to see, that the peak in averaged $d$-values is achieved earlier for larger $n$ and at low $p$-values in general. This implies that after a certain value of $p$ is reached there are more triangles that break up the previously discussed situations in which triangles can not get paired.

\section{Perfect Random Discrete Morse}
Consider the previously discussed random discrete Morse algorithm as specified by Lutz and Benedetti in \cite{lutzbenedetti}. Their construction of random discrete Morse functions is very different, in particular random, while the apparent gradient construction is deterministic. Yet both approaches are interesting to analyze in a similar manner. The following figure shows plots similar to the one from Figure \ref{fig:perfect_apparent}. This time for each $p = 0,0.01,\dots,0.99,1$, we draw a simplicial complex from $Y_2(15,p)$, $200$ times and for each one we calculated a random discrete Morse function $300$ times. Then we compute the percentage of how often the random discrete Morse function was perfect and again averaged this over the $200$ simplicial complexes drawn for some fixed~$p$. Figure \ref{fig:perfect_rdm_v_apparent} then shows this curve plotted against the percentage of perfect apparent gradients.


\begin{figure}[H]
%\centering%
\begin{subfigure}[c]{0.95\textwidth}
\begin{center}
% This file was created by tikzplotlib v0.9.2.
\begin{tikzpicture}

\definecolor{color0}{rgb}{0.12156862745098,0.466666666666667,0.705882352941177}
\definecolor{color1}{rgb}{0.976470588235294,0.450980392156863,0.0235294117647059}
\definecolor{color2}{rgb}{0.172549019607843,0.627450980392157,0.172549019607843}

\begin{axis}[
tick align=outside,
tick pos=left,
xlabel = $p$,
ylabel = $\%$ of perfect gradients,
x grid style={white!69.0196078431373!black},
xmin=-0.05, xmax=1.05,
xtick style={color=black},
y grid style={white!69.0196078431373!black},
ymin=-0.05, ymax=1.05,
ytick style={color=black}
]
\addplot [semithick, color2]
table {%
0 1
0.01 0.86
0.02 0.55
0.03 0.26
0.04 0.16
0.05 0.11
0.06 0.02
0.07 0
0.08 0
0.09 0
0.1 0
0.11 0
0.12 0
0.13 0
0.14 0
0.15 0
0.16 0
0.17 0
0.18 0
0.19 0
0.2 0
0.21 0
0.22 0
0.23 0
0.24 0
0.25 0
0.26 0
0.27 0
0.28 0
0.29 0
0.3 0
0.31 0
0.32 0
0.33 0
0.34 0
0.35 0
0.36 0
0.37 0
0.38 0
0.39 0
0.4 0
0.41 0
0.42 0
0.43 0
0.44 0
0.45 0
0.46 0
0.47 0
0.48 0
0.49 0
0.5 0
0.51 0
0.52 0
0.53 0
0.54 0
0.55 0
0.56 0
0.57 0
0.58 0
0.59 0
0.6 0
0.61 0
0.62 0
0.63 0
0.64 0
0.65 0
0.66 0
0.67 0
0.68 0
0.69 0
0.7 0
0.71 0.01
0.72 0
0.73 0
0.74 0
0.75 0
0.76 0.01
0.77 0
0.78 0.03
0.79 0.01
0.8 0.03
0.81 0.02
0.82 0.05
0.83 0.07
0.84 0.08
0.85 0.08
0.86 0.14
0.87 0.11
0.88 0.14
0.89 0.19
0.9 0.15
0.91 0.37
0.92 0.34
0.93 0.31
0.94 0.39
0.95 0.5
0.96 0.52
0.97 0.68
0.98 0.78
0.99 0.83
1 1
};

\addplot [semithick, color1]
table {%
0 0.96
0.01 0.96
0.02 0.93
0.03 0.93
0.04 0.94
0.05 0.94
0.06 0.97
0.07 0.93
0.08 0.96
0.09 0.97
0.1 0.92
0.11 0.96
0.12 0.93
0.13 0.95
0.14 0.92
0.15 0.93
0.16 0.95
0.17 0.91
0.18 0.78
0.19 0.67
0.2 0.55
0.21 0.58
0.22 0.49
0.23 0.58
0.24 0.51
0.25 0.5
0.26 0.59
0.27 0.38
0.28 0.57
0.29 0.47
0.3 0.33
0.31 0.46
0.32 0.47
0.33 0.41
0.34 0.47
0.35 0.46
0.36 0.52
0.37 0.49
0.38 0.46
0.39 0.54
0.4 0.45
0.41 0.38
0.42 0.45
0.43 0.49
0.44 0.48
0.45 0.43
0.46 0.41
0.47 0.45
0.48 0.45
0.49 0.5
0.5 0.47
0.51 0.49
0.52 0.47
0.53 0.44
0.54 0.51
0.55 0.44
0.56 0.42
0.57 0.44
0.58 0.48
0.59 0.43
0.6 0.46
0.61 0.45
0.62 0.46
0.63 0.46
0.64 0.42
0.65 0.42
0.66 0.46
0.67 0.48
0.68 0.48
0.69 0.45
0.7 0.46
0.71 0.46
0.72 0.48
0.73 0.53
0.74 0.47
0.75 0.55
0.76 0.41
0.77 0.54
0.78 0.45
0.79 0.51
0.8 0.47
0.81 0.47
0.82 0.37
0.83 0.43
0.84 0.45
0.85 0.5
0.86 0.37
0.87 0.5
0.88 0.56
0.89 0.47
0.9 0.48
0.91 0.41
0.92 0.4
0.93 0.44
0.94 0.44
0.95 0.45
0.96 0.45
0.97 0.46
0.98 0.45
0.99 0.48
1 0.39
};

\addplot [semithick, color0]
table {%
0 0.977919999999999
0.01 0.978279999999999
0.02 0.978919999999999
0.03 0.977939999999999
0.04 0.97596
0.05 0.977779999999999
0.06 0.977699999999999
0.07 0.977039999999999
0.08 0.97832
0.09 0.97604
0.1 0.97888
0.11 0.976419999999999
0.12 0.977859999999999
0.13 0.978259999999999
0.14 0.977419999999999
0.15 0.978939999999999
0.16 0.97788
0.17 0.97644
0.18 0.97676
0.19 0.97658
0.2 0.9783
0.21 0.977859999999999
0.22 0.97762
0.23 0.969359999999999
0.24 0.972519999999999
0.25 0.97028
0.26 0.967139999999999
0.27 0.95862
0.28 0.95432
0.29 0.953739999999999
0.3 0.9414
0.31 0.924959999999999
0.32 0.92172
0.33 0.918059999999999
0.34 0.92102
0.35 0.88968
0.36 0.897539999999999
0.37 0.89502
0.38 0.8868
0.39 0.87334
0.4 0.88752
0.41 0.87544
0.42 0.871800000000001
0.43 0.867079999999999
0.44 0.87568
0.45 0.871440000000001
0.46 0.86506
0.47 0.873139999999999
0.48 0.85838
0.49 0.87096
0.5 0.86572
0.51 0.85916
0.52 0.86772
0.53 0.86504
0.54 0.86674
0.55 0.86478
0.56 0.86816
0.57 0.86286
0.58 0.8643
0.59 0.8657
0.6 0.86518
0.61 0.860900000000001
0.62 0.86644
0.63 0.86578
0.64 0.865900000000001
0.65 0.86988
0.66 0.86608
0.67 0.864960000000001
0.68 0.86344
0.69 0.86448
0.7 0.8674
0.71 0.86586
0.72 0.86462
0.73 0.86252
0.74 0.862880000000001
0.75 0.865960000000001
0.76 0.86542
0.77 0.8654
0.78 0.8664
0.79 0.86518
0.8 0.866700000000001
0.81 0.863560000000001
0.82 0.86498
0.83 0.86728
0.84 0.86678
0.85 0.8641
0.86 0.861020000000001
0.87 0.864580000000001
0.88 0.865719999999999
0.89 0.86548
0.9 0.86548
0.91 0.866820000000001
0.92 0.86288
0.93 0.864700000000001
0.94 0.86376
0.95 0.867220000000001
0.96 0.86396
0.97 0.865320000000001
0.98 0.8626
0.99 0.86418
1 0.861
};
\end{axis}

\end{tikzpicture}

\end{center}
\end{subfigure}
\caption{How often do apparent gradients of  $Y_2(15,p)$ yield perfect Morse matchings (green) compared to how often the random discrete Morse function yields perfect Morse matchings for $Y_2(10,p)$ (blue) and $Y_2(15,p)$ (orange).}
\label{fig:perfect_rdm_v_apparent}
\end{figure}

Figure \ref{fig:perfect_rdm_v_apparent} allows two interesting observations. Firstly in terms of the apparent gradients, we see the same behaviour for $Y_2(15,p)$ as for $Y_2(10,p)$, although for $Y_2(15,p)$, the initial decline is steeper and the percentage of perfect apparent gradients also starts to rise again later. Indeed for any $p \notin \{0,1\}$ and $n$ large enough we expect there to be at least one configuration in the filtration where some simplex can not get paired with respect to apparent pairs. Hence the apparent gradient is not perfect. Therefore the larger the value for $n$ the lower the value of $p$ for which we might get a perfect apparent gradient. A similar argument explains the behavior for values of $p$ close to $1$. This means that for $n \rightarrow \infty$ we get perfect gradients if and only if $p \in \{0,1\}$.

Secondly we see that initially the random discrete Morse algorithm has a very high probability of finding a perfect Morse matching for $Y_2(10,p)$ and $Y_2(15,p)$. Then we get a  decline at around $p = 0.25$ and $p = 0.19$ respectively. Afterwards fewer of the found Morse matchings are perfect. It is surprising to see these different levels for different $n$ which seem to be quite stable over large ranges of values for $p$. 

In some future work we would like to find some explanation for this sudden jump in the percentage of perfect Morse matchings found by the random discrete Morse algorithm. The following figure could give a potential clue or starting point for further exploration. It shows a plot of the sum of Betti numbers of random $2$-complexes averaged over $50$ runs.

\begin{figure}[H]
%\centering%
\begin{subfigure}[c]{0.95\textwidth}
\begin{center}
% This file was created by tikzplotlib v0.9.2.
\begin{tikzpicture}

\definecolor{color0}{rgb}{0.12156862745098,0.466666666666667,0.705882352941177}

\begin{axis}[
tick align=outside,
tick pos=left,
title={Mean sum of Betti numbers},
xlabel = $p$,
ylabel = $\beta_0 + \beta_1 + \beta_2$,
x grid style={white!69.0196078431373!black},
xmin=-0.05, xmax=1.05,
xtick style={color=black},
y grid style={white!69.0196078431373!black},
ymin=1.301, ymax=382.319,
ytick style={color=black}
]
\addplot [semithick, color0]
table {%
0 92
0.01 87.5
0.02 82.72
0.03 78.3
0.04 74.06
0.05 68.54
0.06 64.86
0.07 61.26
0.08 55.44
0.09 51.8
0.1 49.62
0.11 43.64
0.12 38.6
0.13 33.86
0.14 31.9
0.15 25.56
0.16 23.14
0.17 21.64
0.18 21.16
0.19 19.1
0.2 18.62
0.21 19.06
0.22 22.6
0.23 23.5
0.24 26.3
0.25 30.24
0.26 33.16
0.27 38
0.28 39.34
0.29 47.06
0.3 48.66
0.31 53.54
0.32 56.08
0.33 60.5
0.34 65.18
0.35 69.06
0.36 74.4
0.37 81.02
0.38 85.66
0.39 88.38
0.4 93.26
0.41 95.72
0.42 103.42
0.43 107.72
0.44 111.1
0.45 114.06
0.46 119.14
0.47 121.06
0.48 127.1
0.49 132.44
0.5 137.02
0.51 139.78
0.52 149.02
0.53 150.94
0.54 156.3
0.55 159.94
0.56 166.2
0.57 169.74
0.58 171.42
0.59 175.76
0.6 183.06
0.61 185.46
0.62 193.62
0.63 196.64
0.64 199.34
0.65 205.56
0.66 210.92
0.67 214.86
0.68 219
0.69 225.62
0.7 230.1
0.71 233.82
0.72 236.18
0.73 243.04
0.74 246.4
0.75 252.88
0.76 254.3
0.77 256.94
0.78 264.16
0.79 270.44
0.8 273.94
0.81 277.48
0.82 280.46
0.83 289.3
0.84 293.36
0.85 297.48
0.86 300.86
0.87 306.66
0.88 311.12
0.89 315.08
0.9 319.7
0.91 324.74
0.92 329.04
0.93 332.04
0.94 337.78
0.95 341.66
0.96 347.28
0.97 350.24
0.98 355.56
0.99 360.44
1 365
};
\end{axis}

\end{tikzpicture}

\end{center}
\end{subfigure}
\caption{Sum of Betti numbers for $Y_2(15,p)$.}
\label{fig:summed_bettis}
\end{figure}

As we can see the sum of Betti numbers declines until $p$ reaches a value of about $0.2$ which is close to the value at which we see a jump in the percentage of perfect Morse matchings found by the random discrete Morse function. 

If in some practical situation one is interested in a perfect Morse matching on a random $k$-complex, in most cases searching for one via the random discrete Morse algorithm is the better way to go. For very small and very large values of $p$ however it might be a good idea to check the apparent gradient. 

\section{Alpha Complexes on Random Point Clouds}

Recall the definition of alpha complexes from Section \ref{sec:filtrations_from_point_clouds}. The set of points we construct the complex on can have a variety of origins. Maybe it is real world data of some kind or a handpicked set of points or, as in our case, some randomly generated set of points. By randomly generated we mean that each point is sampled via some random distribution. 

We generate point clouds using the \textbf{numpy.random} library for \textbf{Python3}, then we calculate the (inclusion-wise) largest possible Alpha complex on the point cloud and a simplexwise filtration of it using the \textbf{GUDHI} library. The exact construction is specified in \cite{gudhi_alpha}.

We chose the following distributions, since they also appear in real world situations. 

\subsection{Uniform Distribution}
In this case, we will sample from the $k$-dimensional unit cube.
For a point $p = (p_1, \dots, p_k)$ this means that each $p_i$ is drawn from the interval $[0,1]$ and that each value in the interval is equally likely. In other words the probability density we draw from equals one on the $k$-dimensional unit cube and zero elsewhere. Figure \ref{fig:uniform} gives an example of a two dimensional point cloud consisting of 100 points drawn like this. Subfigure (b) shows the simplicial complex $K_{\lfloor \frac{n}{2}\rfloor}$ of the filtration $F_* = \{K_1,\dots,K_n\}$, where $K_n$ is the Alpha complex, where $\epsilon > 0$ is so large, that further increasing its value does not change the resulting simplicial complex, which we will call the \textbf{maximal alpha complex}. 

Due to the discrete nature of floating point precision it is possible, although very unlikely, that several simplices appear at the same distance $\epsilon$. In this case we sort the simplices by dimension first and lexicographically second, hence we always get a simplexwise filtration.

\begin{figure}[H]
%\centering%
\begin{subfigure}[t]{0.49\textwidth}
\begin{center}
% This file was created by tikzplotlib v0.9.2.
\begin{tikzpicture}[thick,scale=0.8, every node/.style={transform shape}]

\definecolor{color0}{rgb}{0.12156862745098,0.466666666666667,0.705882352941177}

\begin{axis}[
tick align=outside,
tick pos=left,
x grid style={white!69.0196078431373!black},
xmin=-0.0472016021677426, xmax=1.04212434473219,
xtick style={color=black},
y grid style={white!69.0196078431373!black},
ymin=-0.00675939644529908, ymax=1.04406901760383,
ytick style={color=black}
]
\addplot [only marks, mark=*, draw=color0, fill=color0, colormap/viridis]
table{%
x                      y
0.0656955078449548 0.556158704850788
0.488517157678954 0.69370505416162
0.181862561225007 0.468128175467136
0.844533344061402 0.693545891548072
0.437740280960978 0.496309026394877
0.767231562363555 0.0860675211731177
0.161530883369563 0.149522611184683
0.788251177639878 0.474286327772315
0.766192297835678 0.733348483357132
0.649617451878118 0.367644291551499
0.0504928809316121 0.867642915553829
0.170496234193967 0.894845379322294
0.718729700185543 0.158201872935623
0.463819403422516 0.384045617799055
0.546308284855135 0.389105087054855
0.850219712772608 0.569999593186233
0.0428039347057998 0.18679851387532
0.573667306869477 0.74169251615805
0.830796451137699 0.637386850693855
0.09754863385918 0.506639045306481
0.176293594034698 0.368347919534268
0.558945574150558 0.363726295676814
0.787179566858828 0.738324537463096
0.732125010276887 0.905322560907993
0.440152466660368 0.0450691086687682
0.874505181504566 0.539834503981332
0.359650163755757 0.473142318396389
0.88183940202602 0.47882764457561
0.6190674997885 0.892085167037166
0.620007468347855 0.462139938538465
0.434312621954214 0.462123559134768
0.0558809818709056 0.815333816765134
0.460296317457924 0.174337725374721
0.937004669450955 0.501883644124649
0.129208296418927 0.60524715222521
0.621711178101921 0.602058371214988
0.0930339583508482 0.168174189826023
0.646497637212577 0.515829619327083
0.00231321360043601 0.973121363384185
0.269010530488123 0.736864398451098
0.67623642837757 0.272719320218006
0.48499005939712 0.742084656413199
0.468336897848343 0.183551404784195
0.741566295886979 0.701998568075081
0.226614592948441 0.0892462234244782
0.854492515665868 0.211588735873265
0.0103020564667987 0.580327888091732
0.70164552814407 0.687005996723676
0.306448355159162 0.901946766882589
0.6665798023091 0.60748307490518
0.649320898379969 0.850722397546172
0.429368424142197 0.804574630420432
0.576602731405145 0.594803970012622
0.48015110475138 0.0410055314660248
0.346814709956971 0.361390601145233
0.669995242191944 0.429101602778892
0.0308665950142099 0.872116233796655
0.0505858065112469 0.15467441279092
0.281620589189466 0.905552902924876
0.547524146840316 0.222137875364193
0.972105186204234 0.142213802563758
0.797951124554537 0.658900075698976
0.281347270295641 0.419788547388917
0.584518692663292 0.899839075148078
0.506434656037461 0.0734659500132817
0.686220186190104 0.623421790253092
0.934165564467276 0.962202483589049
0.131454879873151 0.996304089692503
0.885904610919406 0.765107958160786
0.28931761684864 0.338033790105762
0.876577875921493 0.866575669683953
0.193206840334259 0.951591205744689
0.925478745500447 0.9270343379018
0.726874021864532 0.115775075441666
0.782219352465157 0.558773137671638
0.868971829466657 0.412584348571824
0.977330218354481 0.397931727754565
0.714167369203424 0.645271412265081
0.213331794086431 0.444102592245728
0.279648133171411 0.956420011538135
0.0721688416845659 0.550630971981264
0.260963862668663 0.269311361195102
0.590350068616808 0.149312398274075
0.608963404612232 0.889896844048345
0.149963024059017 0.608943668591563
0.992609528964008 0.287200579008376
0.774734133265123 0.496627215016348
0.929472829761068 0.439095720490705
0.110434792844883 0.618239477228263
0.109298875857005 0.205648545579206
0.393767944012354 0.491990742021887
0.0783566045417764 0.369000302058699
0.659726956866604 0.0889856647168015
0.438741749568941 0.703747806517045
0.538058398480968 0.736545720828899
0.837906975806817 0.286819511659124
0.179003186619879 0.478469594552395
0.925475163513836 0.67809841245703
0.194055592803374 0.848205002543498
0.0677117661935651 0.351520919888751
};
\end{axis}

\end{tikzpicture}

\subcaption{Point cloud $S$}
\end{center}
\end{subfigure}
\begin{subfigure}[t]{0.49\textwidth}
\begin{center}
% This file was created by tikzplotlib v0.9.2.
\begin{tikzpicture}[thick,scale=0.8, every node/.style={transform shape}]

\definecolor{color0}{rgb}{0.12156862745098,0.466666666666667,0.705882352941177}
\definecolor{color1}{rgb}{1,0.498039215686275,0.0549019607843137}
\definecolor{color2}{rgb}{0.96078431372549,0.96078431372549,0.862745098039216}
\definecolor{color3}{rgb}{1,0.498039215686275,0.313725490196078}

\begin{axis}[
tick align=outside,
tick pos=left,
x grid style={white!69.0196078431373!black},
xmin=-0.0472016021677426, xmax=1.04212434473219,
xtick style={color=black},
y grid style={white!69.0196078431373!black},
ymin=-0.0067593964452991, ymax=1.04406901760383,
ytick style={color=black}
]
\addplot [only marks, mark=*, draw=color0, fill=color0, colormap/viridis]
table{%
x                      y
0.0656955078449548 0.556158704850788
0.488517157678954 0.69370505416162
0.181862561225007 0.468128175467136
0.844533344061402 0.693545891548072
0.437740280960978 0.496309026394877
0.767231562363555 0.0860675211731177
0.161530883369563 0.149522611184683
0.788251177639878 0.474286327772315
0.766192297835678 0.733348483357132
0.649617451878118 0.367644291551499
0.0504928809316121 0.867642915553829
0.170496234193967 0.894845379322294
0.718729700185543 0.158201872935623
0.463819403422516 0.384045617799055
0.546308284855135 0.389105087054855
0.850219712772608 0.569999593186233
0.0428039347057998 0.18679851387532
0.573667306869477 0.74169251615805
0.830796451137699 0.637386850693855
0.09754863385918 0.506639045306481
0.176293594034698 0.368347919534268
0.558945574150558 0.363726295676814
0.787179566858828 0.738324537463096
0.732125010276887 0.905322560907993
0.440152466660368 0.0450691086687682
0.874505181504566 0.539834503981332
0.359650163755757 0.473142318396389
0.88183940202602 0.47882764457561
0.6190674997885 0.892085167037166
0.620007468347855 0.462139938538465
0.434312621954214 0.462123559134768
0.0558809818709056 0.815333816765134
0.460296317457924 0.174337725374721
0.937004669450955 0.501883644124649
0.129208296418927 0.60524715222521
0.621711178101921 0.602058371214988
0.0930339583508482 0.168174189826023
0.646497637212577 0.515829619327083
0.00231321360043601 0.973121363384185
0.269010530488123 0.736864398451098
0.67623642837757 0.272719320218006
0.48499005939712 0.742084656413199
0.468336897848343 0.183551404784195
0.741566295886979 0.701998568075081
0.226614592948441 0.0892462234244782
0.854492515665868 0.211588735873265
0.0103020564667987 0.580327888091732
0.70164552814407 0.687005996723676
0.306448355159162 0.901946766882589
0.6665798023091 0.60748307490518
0.649320898379969 0.850722397546172
0.429368424142197 0.804574630420432
0.576602731405145 0.594803970012622
0.48015110475138 0.0410055314660248
0.346814709956971 0.361390601145233
0.669995242191944 0.429101602778892
0.0308665950142099 0.872116233796655
0.0505858065112469 0.15467441279092
0.281620589189466 0.905552902924876
0.547524146840316 0.222137875364193
0.972105186204234 0.142213802563758
0.797951124554537 0.658900075698976
0.281347270295641 0.419788547388917
0.584518692663292 0.899839075148078
0.506434656037461 0.0734659500132817
0.686220186190104 0.623421790253092
0.934165564467276 0.962202483589049
0.131454879873151 0.996304089692503
0.885904610919406 0.765107958160786
0.28931761684864 0.338033790105762
0.876577875921493 0.866575669683953
0.193206840334259 0.951591205744689
0.925478745500447 0.9270343379018
0.726874021864532 0.115775075441666
0.782219352465157 0.558773137671638
0.868971829466657 0.412584348571824
0.977330218354481 0.397931727754565
0.714167369203424 0.645271412265081
0.213331794086431 0.444102592245728
0.279648133171411 0.956420011538135
0.0721688416845659 0.550630971981264
0.260963862668663 0.269311361195102
0.590350068616808 0.149312398274075
0.608963404612232 0.889896844048345
0.149963024059017 0.608943668591563
0.992609528964008 0.287200579008376
0.774734133265123 0.496627215016348
0.929472829761068 0.439095720490705
0.110434792844883 0.618239477228263
0.109298875857005 0.205648545579206
0.393767944012354 0.491990742021887
0.0783566045417764 0.369000302058699
0.659726956866604 0.0889856647168015
0.438741749568941 0.703747806517045
0.538058398480968 0.736545720828899
0.837906975806817 0.286819511659124
0.179003186619879 0.478469594552395
0.925475163513836 0.67809841245703
0.194055592803374 0.848205002543498
0.0677117661935651 0.351520919888751
};
\path [draw=color2, fill=color2]
(axis cs:0.437740280960978,0.496309026394877)
--(axis cs:0.434312621954214,0.462123559134768)
--(axis cs:0.393767944012354,0.491990742021887)
--cycle;
\path [draw=color2, fill=color2]
(axis cs:0.0428039347057998,0.18679851387532)
--(axis cs:0.0930339583508482,0.168174189826023)
--(axis cs:0.0505858065112469,0.15467441279092)
--cycle;
\path [draw=color2, fill=color2]
(axis cs:0.129208296418927,0.60524715222521)
--(axis cs:0.149963024059017,0.608943668591563)
--(axis cs:0.110434792844883,0.618239477228263)
--cycle;
\path [draw=color2, fill=color2]
(axis cs:0.6190674997885,0.892085167037166)
--(axis cs:0.649320898379969,0.850722397546172)
--(axis cs:0.608963404612232,0.889896844048345)
--cycle;
\path [draw=color2, fill=color2]
(axis cs:0.844533344061402,0.693545891548072)
--(axis cs:0.830796451137699,0.637386850693855)
--(axis cs:0.797951124554537,0.658900075698976)
--cycle;
\path [draw=color2, fill=color2]
(axis cs:0.306448355159162,0.901946766882589)
--(axis cs:0.281620589189466,0.905552902924876)
--(axis cs:0.279648133171411,0.956420011538135)
--cycle;
\path [draw=color2, fill=color2]
(axis cs:0.488517157678954,0.69370505416162)
--(axis cs:0.48499005939712,0.742084656413199)
--(axis cs:0.438741749568941,0.703747806517045)
--cycle;
\path [draw=color2, fill=color2]
(axis cs:0.6190674997885,0.892085167037166)
--(axis cs:0.584518692663292,0.899839075148078)
--(axis cs:0.608963404612232,0.889896844048345)
--cycle;
\path [draw=color2, fill=color2]
(axis cs:0.741566295886979,0.701998568075081)
--(axis cs:0.70164552814407,0.687005996723676)
--(axis cs:0.714167369203424,0.645271412265081)
--cycle;
\path [draw=color2, fill=color2]
(axis cs:0.0504928809316121,0.867642915553829)
--(axis cs:0.0558809818709056,0.815333816765134)
--(axis cs:0.0308665950142099,0.872116233796655)
--cycle;
\path [draw=color2, fill=color2]
(axis cs:0.488517157678954,0.69370505416162)
--(axis cs:0.48499005939712,0.742084656413199)
--(axis cs:0.538058398480968,0.736545720828899)
--cycle;
\path [draw=color2, fill=color2]
(axis cs:0.0428039347057998,0.18679851387532)
--(axis cs:0.0930339583508482,0.168174189826023)
--(axis cs:0.109298875857005,0.205648545579206)
--cycle;
\path [draw=color2, fill=color2]
(axis cs:0.70164552814407,0.687005996723676)
--(axis cs:0.686220186190104,0.623421790253092)
--(axis cs:0.714167369203424,0.645271412265081)
--cycle;
\path [draw=color2, fill=color2]
(axis cs:0.88183940202602,0.47882764457561)
--(axis cs:0.937004669450955,0.501883644124649)
--(axis cs:0.929472829761068,0.439095720490705)
--cycle;
\path [draw=color2, fill=color2]
(axis cs:0.88183940202602,0.47882764457561)
--(axis cs:0.868971829466657,0.412584348571824)
--(axis cs:0.929472829761068,0.439095720490705)
--cycle;
\path [draw=color2, fill=color2]
(axis cs:0.874505181504566,0.539834503981332)
--(axis cs:0.88183940202602,0.47882764457561)
--(axis cs:0.937004669450955,0.501883644124649)
--cycle;
\path [draw=color2, fill=color2]
(axis cs:0.161530883369563,0.149522611184683)
--(axis cs:0.0930339583508482,0.168174189826023)
--(axis cs:0.109298875857005,0.205648545579206)
--cycle;
\path [draw=color2, fill=color2]
(axis cs:0.0656955078449548,0.556158704850788)
--(axis cs:0.0721688416845659,0.550630971981264)
--(axis cs:0.110434792844883,0.618239477228263)
--cycle;
\path [draw=color2, fill=color2]
(axis cs:0.129208296418927,0.60524715222521)
--(axis cs:0.0721688416845659,0.550630971981264)
--(axis cs:0.110434792844883,0.618239477228263)
--cycle;
\path [draw=color2, fill=color2]
(axis cs:0.181862561225007,0.468128175467136)
--(axis cs:0.213331794086431,0.444102592245728)
--(axis cs:0.179003186619879,0.478469594552395)
--cycle;
\path [draw=color2, fill=color2]
(axis cs:0.766192297835678,0.733348483357132)
--(axis cs:0.741566295886979,0.701998568075081)
--(axis cs:0.797951124554537,0.658900075698976)
--cycle;
\path [draw=color2, fill=color2]
(axis cs:0.766192297835678,0.733348483357132)
--(axis cs:0.787179566858828,0.738324537463096)
--(axis cs:0.797951124554537,0.658900075698976)
--cycle;
\path [draw=color2, fill=color2]
(axis cs:0.359650163755757,0.473142318396389)
--(axis cs:0.434312621954214,0.462123559134768)
--(axis cs:0.393767944012354,0.491990742021887)
--cycle;
\path [draw=color2, fill=color2]
(axis cs:0.844533344061402,0.693545891548072)
--(axis cs:0.787179566858828,0.738324537463096)
--(axis cs:0.797951124554537,0.658900075698976)
--cycle;
\path [draw=color2, fill=color2]
(axis cs:0.440152466660368,0.0450691086687682)
--(axis cs:0.48015110475138,0.0410055314660248)
--(axis cs:0.506434656037461,0.0734659500132817)
--cycle;
\path [draw=color2, fill=color2]
(axis cs:0.741566295886979,0.701998568075081)
--(axis cs:0.797951124554537,0.658900075698976)
--(axis cs:0.714167369203424,0.645271412265081)
--cycle;
\path [draw=color2, fill=color2]
(axis cs:0.620007468347855,0.462139938538465)
--(axis cs:0.646497637212577,0.515829619327083)
--(axis cs:0.669995242191944,0.429101602778892)
--cycle;
\path [draw=color2, fill=color2]
(axis cs:0.346814709956971,0.361390601145233)
--(axis cs:0.281347270295641,0.419788547388917)
--(axis cs:0.28931761684864,0.338033790105762)
--cycle;
\path [draw=color2, fill=color2]
(axis cs:0.718729700185543,0.158201872935623)
--(axis cs:0.726874021864532,0.115775075441666)
--(axis cs:0.659726956866604,0.0889856647168015)
--cycle;
\path [draw=color2, fill=color2]
(axis cs:0.850219712772608,0.569999593186233)
--(axis cs:0.830796451137699,0.637386850693855)
--(axis cs:0.782219352465157,0.558773137671638)
--cycle;
\path [draw=color2, fill=color2]
(axis cs:0.621711178101921,0.602058371214988)
--(axis cs:0.646497637212577,0.515829619327083)
--(axis cs:0.6665798023091,0.60748307490518)
--cycle;
\path [draw=color2, fill=color2]
(axis cs:0.281620589189466,0.905552902924876)
--(axis cs:0.193206840334259,0.951591205744689)
--(axis cs:0.279648133171411,0.956420011538135)
--cycle;
\path [draw=color2, fill=color2]
(axis cs:0.844533344061402,0.693545891548072)
--(axis cs:0.885904610919406,0.765107958160786)
--(axis cs:0.925475163513836,0.67809841245703)
--cycle;
\path [draw=color2, fill=color2]
(axis cs:0.48499005939712,0.742084656413199)
--(axis cs:0.429368424142197,0.804574630420432)
--(axis cs:0.438741749568941,0.703747806517045)
--cycle;
\path [draw=color2, fill=color2]
(axis cs:0.830796451137699,0.637386850693855)
--(axis cs:0.797951124554537,0.658900075698976)
--(axis cs:0.782219352465157,0.558773137671638)
--cycle;
\path [draw=color2, fill=color2]
(axis cs:0.181862561225007,0.468128175467136)
--(axis cs:0.176293594034698,0.368347919534268)
--(axis cs:0.213331794086431,0.444102592245728)
--cycle;
\path [draw=color2, fill=color2]
(axis cs:0.649617451878118,0.367644291551499)
--(axis cs:0.620007468347855,0.462139938538465)
--(axis cs:0.669995242191944,0.429101602778892)
--cycle;
\path [draw=color2, fill=color2]
(axis cs:0.844533344061402,0.693545891548072)
--(axis cs:0.830796451137699,0.637386850693855)
--(axis cs:0.925475163513836,0.67809841245703)
--cycle;
\path [draw=color2, fill=color2]
(axis cs:0.844533344061402,0.693545891548072)
--(axis cs:0.787179566858828,0.738324537463096)
--(axis cs:0.885904610919406,0.765107958160786)
--cycle;
\path [draw=color2, fill=color2]
(axis cs:0.788251177639878,0.474286327772315)
--(axis cs:0.88183940202602,0.47882764457561)
--(axis cs:0.868971829466657,0.412584348571824)
--cycle;
\path [draw=color2, fill=color2]
(axis cs:0.463819403422516,0.384045617799055)
--(axis cs:0.546308284855135,0.389105087054855)
--(axis cs:0.558945574150558,0.363726295676814)
--cycle;
\path [draw=color2, fill=color2]
(axis cs:0.621711178101921,0.602058371214988)
--(axis cs:0.646497637212577,0.515829619327083)
--(axis cs:0.576602731405145,0.594803970012622)
--cycle;
\path [draw=color2, fill=color2]
(axis cs:0.09754863385918,0.506639045306481)
--(axis cs:0.129208296418927,0.60524715222521)
--(axis cs:0.0721688416845659,0.550630971981264)
--cycle;
\path [draw=color2, fill=color2]
(axis cs:0.850219712772608,0.569999593186233)
--(axis cs:0.874505181504566,0.539834503981332)
--(axis cs:0.782219352465157,0.558773137671638)
--cycle;
\path [draw=color2, fill=color2]
(axis cs:0.788251177639878,0.474286327772315)
--(axis cs:0.874505181504566,0.539834503981332)
--(axis cs:0.88183940202602,0.47882764457561)
--cycle;
\path [draw=color2, fill=color2]
(axis cs:0.874505181504566,0.539834503981332)
--(axis cs:0.782219352465157,0.558773137671638)
--(axis cs:0.774734133265123,0.496627215016348)
--cycle;
\path [draw=color2, fill=color2]
(axis cs:0.788251177639878,0.474286327772315)
--(axis cs:0.874505181504566,0.539834503981332)
--(axis cs:0.774734133265123,0.496627215016348)
--cycle;
\path [draw=color2, fill=color2]
(axis cs:0.0656955078449548,0.556158704850788)
--(axis cs:0.0103020564667987,0.580327888091732)
--(axis cs:0.110434792844883,0.618239477228263)
--cycle;
\path [draw=color2, fill=color2]
(axis cs:0.170496234193967,0.894845379322294)
--(axis cs:0.131454879873151,0.996304089692503)
--(axis cs:0.193206840334259,0.951591205744689)
--cycle;
\path [draw=color2, fill=color2]
(axis cs:0.170496234193967,0.894845379322294)
--(axis cs:0.281620589189466,0.905552902924876)
--(axis cs:0.193206840334259,0.951591205744689)
--cycle;
\path [draw=color2, fill=color2]
(axis cs:0.170496234193967,0.894845379322294)
--(axis cs:0.281620589189466,0.905552902924876)
--(axis cs:0.194055592803374,0.848205002543498)
--cycle;
\path [draw=color2, fill=color2]
(axis cs:0.181862561225007,0.468128175467136)
--(axis cs:0.09754863385918,0.506639045306481)
--(axis cs:0.179003186619879,0.478469594552395)
--cycle;
\path [draw=color2, fill=color2]
(axis cs:0.732125010276887,0.905322560907993)
--(axis cs:0.6190674997885,0.892085167037166)
--(axis cs:0.649320898379969,0.850722397546172)
--cycle;
\path [draw=color2, fill=color2]
(axis cs:0.649617451878118,0.367644291551499)
--(axis cs:0.546308284855135,0.389105087054855)
--(axis cs:0.558945574150558,0.363726295676814)
--cycle;
\path [draw=color2, fill=color2]
(axis cs:0.686220186190104,0.623421790253092)
--(axis cs:0.782219352465157,0.558773137671638)
--(axis cs:0.714167369203424,0.645271412265081)
--cycle;
\path [draw=color2, fill=color2]
(axis cs:0.797951124554537,0.658900075698976)
--(axis cs:0.782219352465157,0.558773137671638)
--(axis cs:0.714167369203424,0.645271412265081)
--cycle;
\path [draw=color2, fill=color2]
(axis cs:0.359650163755757,0.473142318396389)
--(axis cs:0.346814709956971,0.361390601145233)
--(axis cs:0.281347270295641,0.419788547388917)
--cycle;
\path [draw=color2, fill=color2]
(axis cs:0.176293594034698,0.368347919534268)
--(axis cs:0.281347270295641,0.419788547388917)
--(axis cs:0.213331794086431,0.444102592245728)
--cycle;
\path [draw=color2, fill=color2]
(axis cs:0.649617451878118,0.367644291551499)
--(axis cs:0.546308284855135,0.389105087054855)
--(axis cs:0.620007468347855,0.462139938538465)
--cycle;
\path [draw=color2, fill=color2]
(axis cs:0.868971829466657,0.412584348571824)
--(axis cs:0.977330218354481,0.397931727754565)
--(axis cs:0.929472829761068,0.439095720490705)
--cycle;
\path [draw=color2, fill=color2]
(axis cs:0.176293594034698,0.368347919534268)
--(axis cs:0.281347270295641,0.419788547388917)
--(axis cs:0.28931761684864,0.338033790105762)
--cycle;
\path [draw=color2, fill=color2]
(axis cs:0.788251177639878,0.474286327772315)
--(axis cs:0.669995242191944,0.429101602778892)
--(axis cs:0.774734133265123,0.496627215016348)
--cycle;
\path [draw=color2, fill=color2]
(axis cs:0.767231562363555,0.0860675211731177)
--(axis cs:0.726874021864532,0.115775075441666)
--(axis cs:0.659726956866604,0.0889856647168015)
--cycle;
\path [draw=color2, fill=color2]
(axis cs:0.468336897848343,0.183551404784195)
--(axis cs:0.547524146840316,0.222137875364193)
--(axis cs:0.590350068616808,0.149312398274075)
--cycle;
\path [draw=color2, fill=color2]
(axis cs:0.621711178101921,0.602058371214988)
--(axis cs:0.6665798023091,0.60748307490518)
--(axis cs:0.686220186190104,0.623421790253092)
--cycle;
\path [draw=color2, fill=color2]
(axis cs:0.460296317457924,0.174337725374721)
--(axis cs:0.468336897848343,0.183551404784195)
--(axis cs:0.506434656037461,0.0734659500132817)
--cycle;
\path [draw=color2, fill=color2]
(axis cs:0.767231562363555,0.0860675211731177)
--(axis cs:0.718729700185543,0.158201872935623)
--(axis cs:0.726874021864532,0.115775075441666)
--cycle;
\path [draw=color2, fill=color2]
(axis cs:0.176293594034698,0.368347919534268)
--(axis cs:0.0783566045417764,0.369000302058699)
--(axis cs:0.0677117661935651,0.351520919888751)
--cycle;
\path [draw=color2, fill=color2]
(axis cs:0.718729700185543,0.158201872935623)
--(axis cs:0.590350068616808,0.149312398274075)
--(axis cs:0.659726956866604,0.0889856647168015)
--cycle;
\path [draw=color2, fill=color2]
(axis cs:0.09754863385918,0.506639045306481)
--(axis cs:0.129208296418927,0.60524715222521)
--(axis cs:0.149963024059017,0.608943668591563)
--cycle;
\path [draw=color2, fill=color2]
(axis cs:0.440152466660368,0.0450691086687682)
--(axis cs:0.460296317457924,0.174337725374721)
--(axis cs:0.506434656037461,0.0734659500132817)
--cycle;
\path [draw=color2, fill=color2]
(axis cs:0.6665798023091,0.60748307490518)
--(axis cs:0.686220186190104,0.623421790253092)
--(axis cs:0.782219352465157,0.558773137671638)
--cycle;
\path [draw=color2, fill=color2]
(axis cs:0.176293594034698,0.368347919534268)
--(axis cs:0.28931761684864,0.338033790105762)
--(axis cs:0.260963862668663,0.269311361195102)
--cycle;
\path [draw=color2, fill=color2]
(axis cs:0.0504928809316121,0.867642915553829)
--(axis cs:0.00231321360043601,0.973121363384185)
--(axis cs:0.0308665950142099,0.872116233796655)
--cycle;
\path [draw=color2, fill=color2]
(axis cs:0.850219712772608,0.569999593186233)
--(axis cs:0.830796451137699,0.637386850693855)
--(axis cs:0.925475163513836,0.67809841245703)
--cycle;
\path [draw=color2, fill=color2]
(axis cs:0.359650163755757,0.473142318396389)
--(axis cs:0.434312621954214,0.462123559134768)
--(axis cs:0.346814709956971,0.361390601145233)
--cycle;
\path [draw=color2, fill=color2]
(axis cs:0.937004669450955,0.501883644124649)
--(axis cs:0.977330218354481,0.397931727754565)
--(axis cs:0.929472829761068,0.439095720490705)
--cycle;
\path [draw=color2, fill=color2]
(axis cs:0.09754863385918,0.506639045306481)
--(axis cs:0.149963024059017,0.608943668591563)
--(axis cs:0.179003186619879,0.478469594552395)
--cycle;
\path [draw=color2, fill=color2]
(axis cs:0.463819403422516,0.384045617799055)
--(axis cs:0.434312621954214,0.462123559134768)
--(axis cs:0.346814709956971,0.361390601145233)
--cycle;
\path [draw=color2, fill=color2]
(axis cs:0.646497637212577,0.515829619327083)
--(axis cs:0.669995242191944,0.429101602778892)
--(axis cs:0.774734133265123,0.496627215016348)
--cycle;
\path [draw=color2, fill=color2]
(axis cs:0.621711178101921,0.602058371214988)
--(axis cs:0.70164552814407,0.687005996723676)
--(axis cs:0.686220186190104,0.623421790253092)
--cycle;
\path [draw=color2, fill=color2]
(axis cs:0.468336897848343,0.183551404784195)
--(axis cs:0.506434656037461,0.0734659500132817)
--(axis cs:0.590350068616808,0.149312398274075)
--cycle;
\path [draw=color2, fill=color2]
(axis cs:0.463819403422516,0.384045617799055)
--(axis cs:0.546308284855135,0.389105087054855)
--(axis cs:0.434312621954214,0.462123559134768)
--cycle;
\path [draw=color2, fill=color2]
(axis cs:0.0504928809316121,0.867642915553829)
--(axis cs:0.170496234193967,0.894845379322294)
--(axis cs:0.0558809818709056,0.815333816765134)
--cycle;
\path [draw=color2, fill=color2]
(axis cs:0.646497637212577,0.515829619327083)
--(axis cs:0.782219352465157,0.558773137671638)
--(axis cs:0.774734133265123,0.496627215016348)
--cycle;
\path [draw=color2, fill=color2]
(axis cs:0.170496234193967,0.894845379322294)
--(axis cs:0.0558809818709056,0.815333816765134)
--(axis cs:0.194055592803374,0.848205002543498)
--cycle;
\path [draw=color2, fill=color2]
(axis cs:0.181862561225007,0.468128175467136)
--(axis cs:0.176293594034698,0.368347919534268)
--(axis cs:0.0783566045417764,0.369000302058699)
--cycle;
\path [draw=color2, fill=color2]
(axis cs:0.646497637212577,0.515829619327083)
--(axis cs:0.6665798023091,0.60748307490518)
--(axis cs:0.782219352465157,0.558773137671638)
--cycle;
\path [draw=color2, fill=color2]
(axis cs:0.488517157678954,0.69370505416162)
--(axis cs:0.576602731405145,0.594803970012622)
--(axis cs:0.538058398480968,0.736545720828899)
--cycle;
\path [draw=color2, fill=color2]
(axis cs:0.573667306869477,0.74169251615805)
--(axis cs:0.576602731405145,0.594803970012622)
--(axis cs:0.538058398480968,0.736545720828899)
--cycle;
\path [draw=color2, fill=color2]
(axis cs:0.573667306869477,0.74169251615805)
--(axis cs:0.621711178101921,0.602058371214988)
--(axis cs:0.576602731405145,0.594803970012622)
--cycle;
\path [draw=color2, fill=color2]
(axis cs:0.181862561225007,0.468128175467136)
--(axis cs:0.09754863385918,0.506639045306481)
--(axis cs:0.0783566045417764,0.369000302058699)
--cycle;
\path [draw=color2, fill=color2]
(axis cs:0.620007468347855,0.462139938538465)
--(axis cs:0.646497637212577,0.515829619327083)
--(axis cs:0.576602731405145,0.594803970012622)
--cycle;
\path [draw=color2, fill=color2]
(axis cs:0.67623642837757,0.272719320218006)
--(axis cs:0.547524146840316,0.222137875364193)
--(axis cs:0.590350068616808,0.149312398274075)
--cycle;
\path [draw=color2, fill=color2]
(axis cs:0.649617451878118,0.367644291551499)
--(axis cs:0.558945574150558,0.363726295676814)
--(axis cs:0.67623642837757,0.272719320218006)
--cycle;
\path [draw=color2, fill=color2]
(axis cs:0.766192297835678,0.733348483357132)
--(axis cs:0.741566295886979,0.701998568075081)
--(axis cs:0.70164552814407,0.687005996723676)
--cycle;
\path [draw=color2, fill=color2]
(axis cs:0.850219712772608,0.569999593186233)
--(axis cs:0.874505181504566,0.539834503981332)
--(axis cs:0.925475163513836,0.67809841245703)
--cycle;
\path [draw=color2, fill=color2]
(axis cs:0.0504928809316121,0.867642915553829)
--(axis cs:0.170496234193967,0.894845379322294)
--(axis cs:0.131454879873151,0.996304089692503)
--cycle;
\path [draw=color2, fill=color2]
(axis cs:0.573667306869477,0.74169251615805)
--(axis cs:0.649320898379969,0.850722397546172)
--(axis cs:0.608963404612232,0.889896844048345)
--cycle;
\path [draw=color2, fill=color2]
(axis cs:0.506434656037461,0.0734659500132817)
--(axis cs:0.590350068616808,0.149312398274075)
--(axis cs:0.659726956866604,0.0889856647168015)
--cycle;
\path [draw=color2, fill=color2]
(axis cs:0.767231562363555,0.0860675211731177)
--(axis cs:0.718729700185543,0.158201872935623)
--(axis cs:0.854492515665868,0.211588735873265)
--cycle;
\path [draw=color2, fill=color2]
(axis cs:0.718729700185543,0.158201872935623)
--(axis cs:0.67623642837757,0.272719320218006)
--(axis cs:0.590350068616808,0.149312398274075)
--cycle;
\path [draw=color2, fill=color2]
(axis cs:0.0504928809316121,0.867642915553829)
--(axis cs:0.00231321360043601,0.973121363384185)
--(axis cs:0.131454879873151,0.996304089692503)
--cycle;
\path [draw=color2, fill=color2]
(axis cs:0.573667306869477,0.74169251615805)
--(axis cs:0.621711178101921,0.602058371214988)
--(axis cs:0.70164552814407,0.687005996723676)
--cycle;
\path [draw=color2, fill=color2]
(axis cs:0.787179566858828,0.738324537463096)
--(axis cs:0.885904610919406,0.765107958160786)
--(axis cs:0.876577875921493,0.866575669683953)
--cycle;
\path [draw=color2, fill=color2]
(axis cs:0.573667306869477,0.74169251615805)
--(axis cs:0.584518692663292,0.899839075148078)
--(axis cs:0.608963404612232,0.889896844048345)
--cycle;
\path [draw=color2, fill=color2]
(axis cs:0.854492515665868,0.211588735873265)
--(axis cs:0.992609528964008,0.287200579008376)
--(axis cs:0.837906975806817,0.286819511659124)
--cycle;
\path [draw=color2, fill=color2]
(axis cs:0.558945574150558,0.363726295676814)
--(axis cs:0.67623642837757,0.272719320218006)
--(axis cs:0.547524146840316,0.222137875364193)
--cycle;
\path [draw=color2, fill=color2]
(axis cs:0.161530883369563,0.149522611184683)
--(axis cs:0.260963862668663,0.269311361195102)
--(axis cs:0.109298875857005,0.205648545579206)
--cycle;
\path [draw=color2, fill=color2]
(axis cs:0.0428039347057998,0.18679851387532)
--(axis cs:0.109298875857005,0.205648545579206)
--(axis cs:0.0677117661935651,0.351520919888751)
--cycle;
\path [draw=color2, fill=color2]
(axis cs:0.269010530488123,0.736864398451098)
--(axis cs:0.281620589189466,0.905552902924876)
--(axis cs:0.194055592803374,0.848205002543498)
--cycle;
\path [draw=color2, fill=color2]
(axis cs:0.269010530488123,0.736864398451098)
--(axis cs:0.306448355159162,0.901946766882589)
--(axis cs:0.281620589189466,0.905552902924876)
--cycle;
\path [draw=color2, fill=color2]
(axis cs:0.854492515665868,0.211588735873265)
--(axis cs:0.972105186204234,0.142213802563758)
--(axis cs:0.992609528964008,0.287200579008376)
--cycle;
\path [draw=color2, fill=color2]
(axis cs:0.437740280960978,0.496309026394877)
--(axis cs:0.546308284855135,0.389105087054855)
--(axis cs:0.434312621954214,0.462123559134768)
--cycle;
\path [draw=color2, fill=color2]
(axis cs:0.573667306869477,0.74169251615805)
--(axis cs:0.584518692663292,0.899839075148078)
--(axis cs:0.538058398480968,0.736545720828899)
--cycle;
\path [draw=color2, fill=color2]
(axis cs:0.766192297835678,0.733348483357132)
--(axis cs:0.70164552814407,0.687005996723676)
--(axis cs:0.649320898379969,0.850722397546172)
--cycle;
\addplot [very thick, color3]
table {%
0.0656955078449548 0.556158704850788
0.0721688416845659 0.550630971981264
};
\addplot [very thick, color3]
table {%
0.6190674997885 0.892085167037166
0.608963404612232 0.889896844048345
};
\addplot [very thick, color3]
table {%
0.181862561225007 0.468128175467136
0.179003186619879 0.478469594552395
};
\addplot [very thick, color3]
table {%
0.460296317457924 0.174337725374721
0.468336897848343 0.183551404784195
};
\addplot [very thick, color3]
table {%
0.0504928809316121 0.867642915553829
0.0308665950142099 0.872116233796655
};
\addplot [very thick, color3]
table {%
0.0783566045417764 0.369000302058699
0.0677117661935651 0.351520919888751
};
\addplot [very thick, color3]
table {%
0.129208296418927 0.60524715222521
0.149963024059017 0.608943668591563
};
\addplot [very thick, color3]
table {%
0.766192297835678 0.733348483357132
0.787179566858828 0.738324537463096
};
\addplot [very thick, color3]
table {%
0.129208296418927 0.60524715222521
0.110434792844883 0.618239477228263
};
\addplot [very thick, color3]
table {%
0.306448355159162 0.901946766882589
0.281620589189466 0.905552902924876
};
\addplot [very thick, color3]
table {%
0.6665798023091 0.60748307490518
0.686220186190104 0.623421790253092
};
\addplot [very thick, color3]
table {%
0.788251177639878 0.474286327772315
0.774734133265123 0.496627215016348
};
\addplot [very thick, color3]
table {%
0.584518692663292 0.899839075148078
0.608963404612232 0.889896844048345
};
\addplot [very thick, color3]
table {%
0.546308284855135 0.389105087054855
0.558945574150558 0.363726295676814
};
\addplot [very thick, color3]
table {%
0.0428039347057998 0.18679851387532
0.0505858065112469 0.15467441279092
};
\addplot [very thick, color3]
table {%
0.437740280960978 0.496309026394877
0.434312621954214 0.462123559134768
};
\addplot [very thick, color3]
table {%
0.686220186190104 0.623421790253092
0.714167369203424 0.645271412265081
};
\addplot [very thick, color3]
table {%
0.573667306869477 0.74169251615805
0.538058398480968 0.736545720828899
};
\addplot [very thick, color3]
table {%
0.934165564467276 0.962202483589049
0.925478745500447 0.9270343379018
};
\addplot [very thick, color3]
table {%
0.850219712772608 0.569999593186233
0.874505181504566 0.539834503981332
};
\addplot [very thick, color3]
table {%
0.359650163755757 0.473142318396389
0.393767944012354 0.491990742021887
};
\addplot [very thick, color3]
table {%
0.830796451137699 0.637386850693855
0.797951124554537 0.658900075698976
};
\addplot [very thick, color3]
table {%
0.181862561225007 0.468128175467136
0.213331794086431 0.444102592245728
};
\addplot [very thick, color3]
table {%
0.766192297835678 0.733348483357132
0.741566295886979 0.701998568075081
};
\addplot [very thick, color3]
table {%
0.440152466660368 0.0450691086687682
0.48015110475138 0.0410055314660248
};
\addplot [very thick, color3]
table {%
0.0930339583508482 0.168174189826023
0.109298875857005 0.205648545579206
};
\addplot [very thick, color3]
table {%
0.48015110475138 0.0410055314660248
0.506434656037461 0.0734659500132817
};
\addplot [very thick, color3]
table {%
0.741566295886979 0.701998568075081
0.70164552814407 0.687005996723676
};
\addplot [very thick, color3]
table {%
0.718729700185543 0.158201872935623
0.726874021864532 0.115775075441666
};
\addplot [very thick, color3]
table {%
0.70164552814407 0.687005996723676
0.714167369203424 0.645271412265081
};
\addplot [very thick, color3]
table {%
0.437740280960978 0.496309026394877
0.393767944012354 0.491990742021887
};
\addplot [very thick, color3]
table {%
0.0930339583508482 0.168174189826023
0.0505858065112469 0.15467441279092
};
\addplot [very thick, color3]
table {%
0.621711178101921 0.602058371214988
0.6665798023091 0.60748307490518
};
\addplot [very thick, color3]
table {%
0.621711178101921 0.602058371214988
0.576602731405145 0.594803970012622
};
\addplot [very thick, color3]
table {%
0.488517157678954 0.69370505416162
0.48499005939712 0.742084656413199
};
\addplot [very thick, color3]
table {%
0.767231562363555 0.0860675211731177
0.726874021864532 0.115775075441666
};
\addplot [very thick, color3]
table {%
0.434312621954214 0.462123559134768
0.393767944012354 0.491990742021887
};
\addplot [very thick, color3]
table {%
0.488517157678954 0.69370505416162
0.438741749568941 0.703747806517045
};
\addplot [very thick, color3]
table {%
0.09754863385918 0.506639045306481
0.0721688416845659 0.550630971981264
};
\addplot [very thick, color3]
table {%
0.281620589189466 0.905552902924876
0.279648133171411 0.956420011538135
};
\addplot [very thick, color3]
table {%
0.6190674997885 0.892085167037166
0.649320898379969 0.850722397546172
};
\addplot [very thick, color3]
table {%
0.170496234193967 0.894845379322294
0.194055592803374 0.848205002543498
};
\addplot [very thick, color3]
table {%
0.0504928809316121 0.867642915553829
0.0558809818709056 0.815333816765134
};
\addplot [very thick, color3]
table {%
0.48499005939712 0.742084656413199
0.538058398480968 0.736545720828899
};
\addplot [very thick, color3]
table {%
0.0428039347057998 0.18679851387532
0.0930339583508482 0.168174189826023
};
\addplot [very thick, color3]
table {%
0.149963024059017 0.608943668591563
0.110434792844883 0.618239477228263
};
\addplot [very thick, color3]
table {%
0.844533344061402 0.693545891548072
0.830796451137699 0.637386850693855
};
\addplot [very thick, color3]
table {%
0.844533344061402 0.693545891548072
0.797951124554537 0.658900075698976
};
\addplot [very thick, color3]
table {%
0.88183940202602 0.47882764457561
0.937004669450955 0.501883644124649
};
\addplot [very thick, color3]
table {%
0.620007468347855 0.462139938538465
0.646497637212577 0.515829619327083
};
\addplot [very thick, color3]
table {%
0.620007468347855 0.462139938538465
0.669995242191944 0.429101602778892
};
\addplot [very thick, color3]
table {%
0.48499005939712 0.742084656413199
0.438741749568941 0.703747806517045
};
\addplot [very thick, color3]
table {%
0.0656955078449548 0.556158704850788
0.0103020564667987 0.580327888091732
};
\addplot [very thick, color3]
table {%
0.170496234193967 0.894845379322294
0.193206840334259 0.951591205744689
};
\addplot [very thick, color3]
table {%
0.874505181504566 0.539834503981332
0.88183940202602 0.47882764457561
};
\addplot [very thick, color3]
table {%
0.649320898379969 0.850722397546172
0.608963404612232 0.889896844048345
};
\addplot [very thick, color3]
table {%
0.306448355159162 0.901946766882589
0.279648133171411 0.956420011538135
};
\addplot [very thick, color3]
table {%
0.88183940202602 0.47882764457561
0.929472829761068 0.439095720490705
};
\addplot [very thick, color3]
table {%
0.346814709956971 0.361390601145233
0.28931761684864 0.338033790105762
};
\addplot [very thick, color3]
table {%
0.782219352465157 0.558773137671638
0.774734133265123 0.496627215016348
};
\addplot [very thick, color3]
table {%
0.6190674997885 0.892085167037166
0.584518692663292 0.899839075148078
};
\addplot [very thick, color3]
table {%
0.977330218354481 0.397931727754565
0.929472829761068 0.439095720490705
};
\addplot [very thick, color3]
table {%
0.741566295886979 0.701998568075081
0.714167369203424 0.645271412265081
};
\addplot [very thick, color3]
table {%
0.937004669450955 0.501883644124649
0.929472829761068 0.439095720490705
};
\addplot [very thick, color3]
table {%
0.649617451878118 0.367644291551499
0.669995242191944 0.429101602778892
};
\addplot [very thick, color3]
table {%
0.488517157678954 0.69370505416162
0.538058398480968 0.736545720828899
};
\addplot [very thick, color3]
table {%
0.0558809818709056 0.815333816765134
0.0308665950142099 0.872116233796655
};
\addplot [very thick, color3]
table {%
0.868971829466657 0.412584348571824
0.929472829761068 0.439095720490705
};
\addplot [very thick, color3]
table {%
0.88183940202602 0.47882764457561
0.868971829466657 0.412584348571824
};
\addplot [very thick, color3]
table {%
0.850219712772608 0.569999593186233
0.782219352465157 0.558773137671638
};
\addplot [very thick, color3]
table {%
0.0428039347057998 0.18679851387532
0.109298875857005 0.205648545579206
};
\addplot [very thick, color3]
table {%
0.850219712772608 0.569999593186233
0.830796451137699 0.637386850693855
};
\addplot [very thick, color3]
table {%
0.70164552814407 0.687005996723676
0.686220186190104 0.623421790253092
};
\addplot [very thick, color3]
table {%
0.741566295886979 0.701998568075081
0.797951124554537 0.658900075698976
};
\addplot [very thick, color3]
table {%
0.161530883369563 0.149522611184683
0.0930339583508482 0.168174189826023
};
\addplot [very thick, color3]
table {%
0.281347270295641 0.419788547388917
0.213331794086431 0.444102592245728
};
\addplot [very thick, color3]
table {%
0.726874021864532 0.115775075441666
0.659726956866604 0.0889856647168015
};
\addplot [very thick, color3]
table {%
0.844533344061402 0.693545891548072
0.787179566858828 0.738324537463096
};
\addplot [very thick, color3]
table {%
0.874505181504566 0.539834503981332
0.937004669450955 0.501883644124649
};
\addplot [very thick, color3]
table {%
0.28931761684864 0.338033790105762
0.260963862668663 0.269311361195102
};
\addplot [very thick, color3]
table {%
0.131454879873151 0.996304089692503
0.193206840334259 0.951591205744689
};
\addplot [very thick, color3]
table {%
0.0656955078449548 0.556158704850788
0.110434792844883 0.618239477228263
};
\addplot [very thick, color3]
table {%
0.161530883369563 0.149522611184683
0.109298875857005 0.205648545579206
};
\addplot [very thick, color3]
table {%
0.854492515665868 0.211588735873265
0.837906975806817 0.286819511659124
};
\addplot [very thick, color3]
table {%
0.876577875921493 0.866575669683953
0.925478745500447 0.9270343379018
};
\addplot [very thick, color3]
table {%
0.0721688416845659 0.550630971981264
0.110434792844883 0.618239477228263
};
\addplot [very thick, color3]
table {%
0.129208296418927 0.60524715222521
0.0721688416845659 0.550630971981264
};
\addplot [very thick, color3]
table {%
0.787179566858828 0.738324537463096
0.797951124554537 0.658900075698976
};
\addplot [very thick, color3]
table {%
0.213331794086431 0.444102592245728
0.179003186619879 0.478469594552395
};
\addplot [very thick, color3]
table {%
0.766192297835678 0.733348483357132
0.797951124554537 0.658900075698976
};
\addplot [very thick, color3]
table {%
0.281347270295641 0.419788547388917
0.28931761684864 0.338033790105762
};
\addplot [very thick, color3]
table {%
0.844533344061402 0.693545891548072
0.925475163513836 0.67809841245703
};
\addplot [very thick, color3]
table {%
0.463819403422516 0.384045617799055
0.546308284855135 0.389105087054855
};
\addplot [very thick, color3]
table {%
0.844533344061402 0.693545891548072
0.885904610919406 0.765107958160786
};
\addplot [very thick, color3]
table {%
0.359650163755757 0.473142318396389
0.434312621954214 0.462123559134768
};
\addplot [very thick, color3]
table {%
0.463819403422516 0.384045617799055
0.434312621954214 0.462123559134768
};
\addplot [very thick, color3]
table {%
0.48499005939712 0.742084656413199
0.429368424142197 0.804574630420432
};
\addplot [very thick, color3]
table {%
0.176293594034698 0.368347919534268
0.213331794086431 0.444102592245728
};
\addplot [very thick, color3]
table {%
0.547524146840316 0.222137875364193
0.590350068616808 0.149312398274075
};
\addplot [very thick, color3]
table {%
0.797951124554537 0.658900075698976
0.714167369203424 0.645271412265081
};
\addplot [very thick, color3]
table {%
0.440152466660368 0.0450691086687682
0.506434656037461 0.0734659500132817
};
\addplot [very thick, color3]
table {%
0.09754863385918 0.506639045306481
0.179003186619879 0.478469594552395
};
\addplot [very thick, color3]
table {%
0.193206840334259 0.951591205744689
0.279648133171411 0.956420011538135
};
\addplot [very thick, color3]
table {%
0.346814709956971 0.361390601145233
0.281347270295641 0.419788547388917
};
\addplot [very thick, color3]
table {%
0.468336897848343 0.183551404784195
0.547524146840316 0.222137875364193
};
\addplot [very thick, color3]
table {%
0.161530883369563 0.149522611184683
0.226614592948441 0.0892462234244782
};
\addplot [very thick, color3]
table {%
0.621711178101921 0.602058371214988
0.646497637212577 0.515829619327083
};
\addplot [very thick, color3]
table {%
0.646497637212577 0.515829619327083
0.669995242191944 0.429101602778892
};
\addplot [very thick, color3]
table {%
0.649617451878118 0.367644291551499
0.558945574150558 0.363726295676814
};
\addplot [very thick, color3]
table {%
0.590350068616808 0.149312398274075
0.659726956866604 0.0889856647168015
};
\addplot [very thick, color3]
table {%
0.830796451137699 0.637386850693855
0.782219352465157 0.558773137671638
};
\addplot [very thick, color3]
table {%
0.718729700185543 0.158201872935623
0.659726956866604 0.0889856647168015
};
\addplot [very thick, color3]
table {%
0.788251177639878 0.474286327772315
0.88183940202602 0.47882764457561
};
\addplot [very thick, color3]
table {%
0.646497637212577 0.515829619327083
0.6665798023091 0.60748307490518
};
\addplot [very thick, color3]
table {%
0.359650163755757 0.473142318396389
0.281347270295641 0.419788547388917
};
\addplot [very thick, color3]
table {%
0.885904610919406 0.765107958160786
0.925475163513836 0.67809841245703
};
\addplot [very thick, color3]
table {%
0.176293594034698 0.368347919534268
0.0783566045417764 0.369000302058699
};
\addplot [very thick, color3]
table {%
0.649617451878118 0.367644291551499
0.67623642837757 0.272719320218006
};
\addplot [very thick, color3]
table {%
0.732125010276887 0.905322560907993
0.649320898379969 0.850722397546172
};
\addplot [very thick, color3]
table {%
0.281620589189466 0.905552902924876
0.193206840334259 0.951591205744689
};
\addplot [very thick, color3]
table {%
0.429368424142197 0.804574630420432
0.438741749568941 0.703747806517045
};
\addplot [very thick, color3]
table {%
0.797951124554537 0.658900075698976
0.782219352465157 0.558773137671638
};
\addplot [very thick, color3]
table {%
0.788251177639878 0.474286327772315
0.868971829466657 0.412584348571824
};
\addplot [very thick, color3]
table {%
0.885904610919406 0.765107958160786
0.876577875921493 0.866575669683953
};
\addplot [very thick, color3]
table {%
0.181862561225007 0.468128175467136
0.176293594034698 0.368347919534268
};
\addplot [very thick, color3]
table {%
0.787179566858828 0.738324537463096
0.885904610919406 0.765107958160786
};
\addplot [very thick, color3]
table {%
0.649617451878118 0.367644291551499
0.620007468347855 0.462139938538465
};
\addplot [very thick, color3]
table {%
0.830796451137699 0.637386850693855
0.925475163513836 0.67809841245703
};
\addplot [very thick, color3]
table {%
0.546308284855135 0.389105087054855
0.620007468347855 0.462139938538465
};
\addplot [very thick, color3]
table {%
0.281620589189466 0.905552902924876
0.194055592803374 0.848205002543498
};
\addplot [very thick, color3]
table {%
0.00231321360043601 0.973121363384185
0.0308665950142099 0.872116233796655
};
\addplot [very thick, color3]
table {%
0.463819403422516 0.384045617799055
0.558945574150558 0.363726295676814
};
\addplot [very thick, color3]
table {%
0.646497637212577 0.515829619327083
0.576602731405145 0.594803970012622
};
\addplot [very thick, color3]
table {%
0.09754863385918 0.506639045306481
0.129208296418927 0.60524715222521
};
\addplot [very thick, color3]
table {%
0.874505181504566 0.539834503981332
0.782219352465157 0.558773137671638
};
\addplot [very thick, color3]
table {%
0.788251177639878 0.474286327772315
0.874505181504566 0.539834503981332
};
\addplot [very thick, color3]
table {%
0.874505181504566 0.539834503981332
0.774734133265123 0.496627215016348
};
\addplot [very thick, color3]
table {%
0.0103020564667987 0.580327888091732
0.110434792844883 0.618239477228263
};
\addplot [very thick, color3]
table {%
0.782219352465157 0.558773137671638
0.714167369203424 0.645271412265081
};
\addplot [very thick, color3]
table {%
0.460296317457924 0.174337725374721
0.506434656037461 0.0734659500132817
};
\addplot [very thick, color3]
table {%
0.170496234193967 0.894845379322294
0.281620589189466 0.905552902924876
};
\addplot [very thick, color3]
table {%
0.977330218354481 0.397931727754565
0.992609528964008 0.287200579008376
};
\addplot [very thick, color3]
table {%
0.170496234193967 0.894845379322294
0.131454879873151 0.996304089692503
};
\addplot [very thick, color3]
table {%
0.359650163755757 0.473142318396389
0.346814709956971 0.361390601145233
};
\addplot [very thick, color3]
table {%
0.181862561225007 0.468128175467136
0.09754863385918 0.506639045306481
};
\addplot [very thick, color3]
table {%
0.506434656037461 0.0734659500132817
0.590350068616808 0.149312398274075
};
\addplot [very thick, color3]
table {%
0.732125010276887 0.905322560907993
0.6190674997885 0.892085167037166
};
\addplot [very thick, color3]
table {%
0.649617451878118 0.367644291551499
0.546308284855135 0.389105087054855
};
\addplot [very thick, color3]
table {%
0.686220186190104 0.623421790253092
0.782219352465157 0.558773137671638
};
\addplot [very thick, color3]
table {%
0.176293594034698 0.368347919534268
0.28931761684864 0.338033790105762
};
\addplot [very thick, color3]
table {%
0.176293594034698 0.368347919534268
0.281347270295641 0.419788547388917
};
\addplot [very thick, color3]
table {%
0.463819403422516 0.384045617799055
0.346814709956971 0.361390601145233
};
\addplot [very thick, color3]
table {%
0.868971829466657 0.412584348571824
0.977330218354481 0.397931727754565
};
\addplot [very thick, color3]
table {%
0.718729700185543 0.158201872935623
0.67623642837757 0.272719320218006
};
\addplot [very thick, color3]
table {%
0.0504928809316121 0.867642915553829
0.170496234193967 0.894845379322294
};
\addplot [very thick, color3]
table {%
0.669995242191944 0.429101602778892
0.774734133265123 0.496627215016348
};
\addplot [very thick, color3]
table {%
0.788251177639878 0.474286327772315
0.669995242191944 0.429101602778892
};
\addplot [very thick, color3]
table {%
0.767231562363555 0.0860675211731177
0.659726956866604 0.0889856647168015
};
\addplot [very thick, color3]
table {%
0.468336897848343 0.183551404784195
0.590350068616808 0.149312398274075
};
\addplot [very thick, color3]
table {%
0.621711178101921 0.602058371214988
0.686220186190104 0.623421790253092
};
\addplot [very thick, color3]
table {%
0.468336897848343 0.183551404784195
0.506434656037461 0.0734659500132817
};
\addplot [very thick, color3]
table {%
0.767231562363555 0.0860675211731177
0.718729700185543 0.158201872935623
};
\addplot [very thick, color3]
table {%
0.176293594034698 0.368347919534268
0.0677117661935651 0.351520919888751
};
\addplot [very thick, color3]
table {%
0.718729700185543 0.158201872935623
0.590350068616808 0.149312398274075
};
\addplot [very thick, color3]
table {%
0.868971829466657 0.412584348571824
0.837906975806817 0.286819511659124
};
\addplot [very thick, color3]
table {%
0.646497637212577 0.515829619327083
0.774734133265123 0.496627215016348
};
\addplot [very thick, color3]
table {%
0.09754863385918 0.506639045306481
0.149963024059017 0.608943668591563
};
\addplot [very thick, color3]
table {%
0.176293594034698 0.368347919534268
0.260963862668663 0.269311361195102
};
\addplot [very thick, color3]
table {%
0.440152466660368 0.0450691086687682
0.460296317457924 0.174337725374721
};
\addplot [very thick, color3]
table {%
0.6665798023091 0.60748307490518
0.782219352465157 0.558773137671638
};
\addplot [very thick, color3]
table {%
0.00231321360043601 0.973121363384185
0.131454879873151 0.996304089692503
};
\addplot [very thick, color3]
table {%
0.0504928809316121 0.867642915553829
0.00231321360043601 0.973121363384185
};
\addplot [very thick, color3]
table {%
0.488517157678954 0.69370505416162
0.576602731405145 0.594803970012622
};
\addplot [very thick, color3]
table {%
0.573667306869477 0.74169251615805
0.649320898379969 0.850722397546172
};
\addplot [very thick, color3]
table {%
0.850219712772608 0.569999593186233
0.925475163513836 0.67809841245703
};
\addplot [very thick, color3]
table {%
0.434312621954214 0.462123559134768
0.346814709956971 0.361390601145233
};
\addplot [very thick, color3]
table {%
0.149963024059017 0.608943668591563
0.179003186619879 0.478469594552395
};
\addplot [very thick, color3]
table {%
0.269010530488123 0.736864398451098
0.194055592803374 0.848205002543498
};
\addplot [very thick, color3]
table {%
0.937004669450955 0.501883644124649
0.977330218354481 0.397931727754565
};
\addplot [very thick, color3]
table {%
0.854492515665868 0.211588735873265
0.972105186204234 0.142213802563758
};
\addplot [very thick, color3]
table {%
0.621711178101921 0.602058371214988
0.70164552814407 0.687005996723676
};
\addplot [very thick, color3]
table {%
0.67623642837757 0.272719320218006
0.547524146840316 0.222137875364193
};
\addplot [very thick, color3]
table {%
0.09754863385918 0.506639045306481
0.0783566045417764 0.369000302058699
};
\addplot [very thick, color3]
table {%
0.573667306869477 0.74169251615805
0.70164552814407 0.687005996723676
};
\addplot [very thick, color3]
table {%
0.546308284855135 0.389105087054855
0.434312621954214 0.462123559134768
};
\addplot [very thick, color3]
table {%
0.170496234193967 0.894845379322294
0.0558809818709056 0.815333816765134
};
\addplot [very thick, color3]
table {%
0.0558809818709056 0.815333816765134
0.194055592803374 0.848205002543498
};
\addplot [very thick, color3]
table {%
0.558945574150558 0.363726295676814
0.547524146840316 0.222137875364193
};
\addplot [very thick, color3]
table {%
0.646497637212577 0.515829619327083
0.782219352465157 0.558773137671638
};
\addplot [very thick, color3]
table {%
0.181862561225007 0.468128175467136
0.0783566045417764 0.369000302058699
};
\addplot [very thick, color3]
table {%
0.718729700185543 0.158201872935623
0.854492515665868 0.211588735873265
};
\addplot [very thick, color3]
table {%
0.972105186204234 0.142213802563758
0.992609528964008 0.287200579008376
};
\addplot [very thick, color3]
table {%
0.576602731405145 0.594803970012622
0.538058398480968 0.736545720828899
};
\addplot [very thick, color3]
table {%
0.573667306869477 0.74169251615805
0.576602731405145 0.594803970012622
};
\addplot [very thick, color3]
table {%
0.573667306869477 0.74169251615805
0.621711178101921 0.602058371214988
};
\addplot [very thick, color3]
table {%
0.732125010276887 0.905322560907993
0.876577875921493 0.866575669683953
};
\addplot [very thick, color3]
table {%
0.67623642837757 0.272719320218006
0.590350068616808 0.149312398274075
};
\addplot [very thick, color3]
table {%
0.620007468347855 0.462139938538465
0.576602731405145 0.594803970012622
};
\addplot [very thick, color3]
table {%
0.109298875857005 0.205648545579206
0.0677117661935651 0.351520919888751
};
\addplot [very thick, color3]
table {%
0.0504928809316121 0.867642915553829
0.131454879873151 0.996304089692503
};
\addplot [very thick, color3]
table {%
0.558945574150558 0.363726295676814
0.67623642837757 0.272719320218006
};
\addplot [very thick, color3]
table {%
0.767231562363555 0.0860675211731177
0.854492515665868 0.211588735873265
};
\addplot [very thick, color3]
table {%
0.766192297835678 0.733348483357132
0.70164552814407 0.687005996723676
};
\addplot [very thick, color3]
table {%
0.874505181504566 0.539834503981332
0.925475163513836 0.67809841245703
};
\addplot [very thick, color3]
table {%
0.573667306869477 0.74169251615805
0.608963404612232 0.889896844048345
};
\addplot [very thick, color3]
table {%
0.992609528964008 0.287200579008376
0.837906975806817 0.286819511659124
};
\addplot [very thick, color3]
table {%
0.506434656037461 0.0734659500132817
0.659726956866604 0.0889856647168015
};
\addplot [very thick, color3]
table {%
0.161530883369563 0.149522611184683
0.260963862668663 0.269311361195102
};
\addplot [very thick, color3]
table {%
0.306448355159162 0.901946766882589
0.429368424142197 0.804574630420432
};
\addplot [very thick, color3]
table {%
0.854492515665868 0.211588735873265
0.992609528964008 0.287200579008376
};
\addplot [very thick, color3]
table {%
0.787179566858828 0.738324537463096
0.876577875921493 0.866575669683953
};
\addplot [very thick, color3]
table {%
0.573667306869477 0.74169251615805
0.584518692663292 0.899839075148078
};
\addplot [very thick, color3]
table {%
0.67623642837757 0.272719320218006
0.837906975806817 0.286819511659124
};
\addplot [very thick, color3]
table {%
0.260963862668663 0.269311361195102
0.109298875857005 0.205648545579206
};
\addplot [very thick, color3]
table {%
0.766192297835678 0.733348483357132
0.649320898379969 0.850722397546172
};
\addplot [very thick, color3]
table {%
0.0428039347057998 0.18679851387532
0.0677117661935651 0.351520919888751
};
\addplot [very thick, color3]
table {%
0.269010530488123 0.736864398451098
0.281620589189466 0.905552902924876
};
\addplot [very thick, color3]
table {%
0.269010530488123 0.736864398451098
0.306448355159162 0.901946766882589
};
\addplot [very thick, color3]
table {%
0.437740280960978 0.496309026394877
0.576602731405145 0.594803970012622
};
\addplot [very thick, color3]
table {%
0.437740280960978 0.496309026394877
0.546308284855135 0.389105087054855
};
\addplot [very thick, color3]
table {%
0.70164552814407 0.687005996723676
0.649320898379969 0.850722397546172
};
\addplot [very thick, color3]
table {%
0.269010530488123 0.736864398451098
0.438741749568941 0.703747806517045
};
\addplot [very thick, color3]
table {%
0.584518692663292 0.899839075148078
0.538058398480968 0.736545720828899
};
\addplot [very thick, color3]
table {%
0.269010530488123 0.736864398451098
0.429368424142197 0.804574630420432
};
\addplot [only marks, mark=*, draw=color0, fill=color0, colormap/viridis]
table{%
x                      y
0.0656955078449548 0.556158704850788
0.488517157678954 0.69370505416162
0.181862561225007 0.468128175467136
0.844533344061402 0.693545891548072
0.437740280960978 0.496309026394877
0.767231562363555 0.0860675211731177
0.161530883369563 0.149522611184683
0.788251177639878 0.474286327772315
0.766192297835678 0.733348483357132
0.649617451878118 0.367644291551499
0.0504928809316121 0.867642915553829
0.170496234193967 0.894845379322294
0.718729700185543 0.158201872935623
0.463819403422516 0.384045617799055
0.546308284855135 0.389105087054855
0.850219712772608 0.569999593186233
0.0428039347057998 0.18679851387532
0.573667306869477 0.74169251615805
0.830796451137699 0.637386850693855
0.09754863385918 0.506639045306481
0.176293594034698 0.368347919534268
0.558945574150558 0.363726295676814
0.787179566858828 0.738324537463096
0.732125010276887 0.905322560907993
0.440152466660368 0.0450691086687682
0.874505181504566 0.539834503981332
0.359650163755757 0.473142318396389
0.88183940202602 0.47882764457561
0.6190674997885 0.892085167037166
0.620007468347855 0.462139938538465
0.434312621954214 0.462123559134768
0.0558809818709056 0.815333816765134
0.460296317457924 0.174337725374721
0.937004669450955 0.501883644124649
0.129208296418927 0.60524715222521
0.621711178101921 0.602058371214988
0.0930339583508482 0.168174189826023
0.646497637212577 0.515829619327083
0.00231321360043601 0.973121363384185
0.269010530488123 0.736864398451098
0.67623642837757 0.272719320218006
0.48499005939712 0.742084656413199
0.468336897848343 0.183551404784195
0.741566295886979 0.701998568075081
0.226614592948441 0.0892462234244782
0.854492515665868 0.211588735873265
0.0103020564667987 0.580327888091732
0.70164552814407 0.687005996723676
0.306448355159162 0.901946766882589
0.6665798023091 0.60748307490518
0.649320898379969 0.850722397546172
0.429368424142197 0.804574630420432
0.576602731405145 0.594803970012622
0.48015110475138 0.0410055314660248
0.346814709956971 0.361390601145233
0.669995242191944 0.429101602778892
0.0308665950142099 0.872116233796655
0.0505858065112469 0.15467441279092
0.281620589189466 0.905552902924876
0.547524146840316 0.222137875364193
0.972105186204234 0.142213802563758
0.797951124554537 0.658900075698976
0.281347270295641 0.419788547388917
0.584518692663292 0.899839075148078
0.506434656037461 0.0734659500132817
0.686220186190104 0.623421790253092
0.934165564467276 0.962202483589049
0.131454879873151 0.996304089692503
0.885904610919406 0.765107958160786
0.28931761684864 0.338033790105762
0.876577875921493 0.866575669683953
0.193206840334259 0.951591205744689
0.925478745500447 0.9270343379018
0.726874021864532 0.115775075441666
0.782219352465157 0.558773137671638
0.868971829466657 0.412584348571824
0.977330218354481 0.397931727754565
0.714167369203424 0.645271412265081
0.213331794086431 0.444102592245728
0.279648133171411 0.956420011538135
0.0721688416845659 0.550630971981264
0.260963862668663 0.269311361195102
0.590350068616808 0.149312398274075
0.608963404612232 0.889896844048345
0.149963024059017 0.608943668591563
0.992609528964008 0.287200579008376
0.774734133265123 0.496627215016348
0.929472829761068 0.439095720490705
0.110434792844883 0.618239477228263
0.109298875857005 0.205648545579206
0.393767944012354 0.491990742021887
0.0783566045417764 0.369000302058699
0.659726956866604 0.0889856647168015
0.438741749568941 0.703747806517045
0.538058398480968 0.736545720828899
0.837906975806817 0.286819511659124
0.179003186619879 0.478469594552395
0.925475163513836 0.67809841245703
0.194055592803374 0.848205002543498
0.0677117661935651 0.351520919888751
};
\end{axis}

\end{tikzpicture}

\subcaption{Intermediate step of the filtration of the largest possible Alpha complex on $S$.}
\end{center}
\end{subfigure}
\caption{Sampled uniform distribution and simplicial complex.}
\label{fig:uniform}
\end{figure}

The code that generated the point clouds and the visualizations in the previous and following figures can be found on \href{https://github.com/IvanSpirandelli/Masterarbeit}{[GitHub]}, see \cite{github}.

\subsection{Multivariate Gaussian Distribution}
In this case we will sample points from a multivariate Gaussian distribution. 
A $k$-dimensional random variable $X$ is multivariate normal distributed, if its density function is of the following form: \[
f_X(x) = \frac{\operatorname{exp}(-\frac{1}{2}(x-\mu)^T \Sigma^{-1}(x - \mu))}{\sqrt{(2 \Pi)^k |\Sigma|}},
\]
where $x\in \mathbb{R}^k$, $\Sigma$ is the covariance matrix, which we require to be positive definite, and $\mu$ is the mean vector. We write $X \sim \mathcal{N}(\mu,\Sigma)$.

Figure \ref{fig:multivariate} gives an example on 100 points and the median simplicial complex generated in the same manner as before.

\begin{figure}[H]
%\centering%
\begin{subfigure}[t]{0.49\textwidth}
\begin{center}
% This file was created by tikzplotlib v0.9.2.
\begin{tikzpicture}[thick,scale=0.8, every node/.style={transform shape}]

\definecolor{color0}{rgb}{0.12156862745098,0.466666666666667,0.705882352941177}

\begin{axis}[
tick align=outside,
tick pos=left,
x grid style={white!69.0196078431373!black},
xmin=-3.45180709372072, xmax=2.99361349512875,
xtick style={color=black},
y grid style={white!69.0196078431373!black},
ymin=-3.45180709372072, ymax=2.99361349512875,
%ymin=-2.1938158082693, ymax=2.17815875977889,
ytick style={color=black}
]
\addplot [only marks, mark=*, draw=color0, fill=color0, colormap/viridis]
table{%
x                      y
-2.34275812782514 -0.422337072515164
0.24722747298104 -1.61319873633128
-1.38019830256552 0.640505435159906
-1.51714816067073 0.258502056354496
-0.175822267271081 -0.0252651026286875
-1.07790812289209 -1.40003857215438
1.08243752843236 1.42960529549044
1.45446356041819 1.05019519757262
0.272266762130476 -0.356289816160233
0.885609873179297 -0.34308175412289
0.318497415524185 -0.772246256656776
-0.486464599849546 -0.321047644667546
0.713664525872475 0.45089573918883
-3.15883343059119 1.79910626932364
1.03249205013434 -0.276218486658694
0.724867796410222 1.0119644513881
-0.66403264904207 1.07732210290748
-0.107877676726452 -0.840289685265736
1.01272382397982 -1.86298022827367
1.13929172816044 -1.07553705644164
-1.3126448374852 1.17645213621684
1.16650560208541 -1.36988664811057
-0.200901763190504 0.83207623547168
-0.162766483689669 0.758482159427621
-0.835280645501797 -0.205178137113613
-0.645878092492392 0.244918441252124
0.80474946774176 -1.05858981481516
0.520329697755992 -1.55654582504081
0.648482472610915 -1.07166832936951
0.512282026122496 -1.05460220157653
-1.62753042956307 0.144763499946151
-1.85256401192557 -0.124092650272665
-0.038308979791596 -0.906263941779864
2.70063983199922 -0.145679091954431
0.790570154320513 -0.374829753224335
-0.517513317388311 -0.760009533697624
0.567908869640562 -0.574932508687353
0.661942961282725 0.682717899396752
-0.897657764611887 -0.509805000910038
-0.125549177171554 -0.678365273082073
-0.878227336337594 0.691452914917409
-0.683949025620437 -0.59776859652354
-0.848312798273908 -0.798926002882831
0.112985819436421 -0.690906617864546
0.225572130815729 -1.11820672366738
1.30701775361222 0.471310001618085
1.17257671373237 -0.686628400040247
1.65057712884217 0.943486007801425
1.27651703162676 -0.507892465272379
1.18397678448062 -0.00794043507333255
0.491156506328976 1.50289694168447
-1.03569391216883 -0.0587326120757943
0.075771071420608 -0.192872784109992
-0.282308197205086 -1.27740706587447
-1.00748672473045 0.638365038742785
-1.27749128471686 0.65534756920239
-0.124824503595767 1.17650662278542
0.106611314555847 1.01602134591746
-1.13247270174148 -0.0219117447622604
-0.043848450781368 0.192488329289336
-0.0438948279215397 0.885200349964678
-0.0457880275415337 0.755095216844757
-1.28516929949252 0.112243759124342
-0.440817927077724 -0.304467240653771
0.161305007531463 0.47151626644206
-0.151580682549982 -0.582783939415887
-0.447730695225613 -0.452165071633186
0.372995023166489 -0.615248964656559
-0.707200164270895 -0.110920852624309
0.57264119973551 0.222410149332283
-0.291802622064795 0.0837691031769381
-2.03628199559491 -1.99508969153983
-2.11509949221634 -1.20644336802718
0.633895330915757 0.988721140629871
1.18243868402131 -0.182835421102174
1.56571290677136 0.226015644404297
1.50701998276123 -0.439351332338839
0.413852404170514 1.97943264304943
-0.285361579892089 1.35847512096702
-0.804429884793085 1.63103253934514
-0.058591464870352 -0.250495361764648
0.435590639323754 1.40570053244992
-0.320788003316259 0.471261823633868
0.287974042518776 0.149003002147366
0.811316786821193 -1.42384003157491
-0.96405415610173 -1.6757287701837
-1.02698880542662 0.0878139020675124
-1.05068784812419 -0.196660891853283
-1.37362659980222 -0.192873683671634
-0.0392455500475513 1.09768115355305
0.575537860345507 0.803161640917396
0.929214222549169 0.651745790327805
-0.97525635438475 -0.546892506013557
-0.042750903342611 0.0797462803574467
-0.290049580989981 0.882474726828786
1.08385271681253 -0.717881298334075
1.98268177645895 -0.319409301778917
1.31068071341738 0.588970190069706
-0.98887418389988 -0.00418003306320407
-0.571790491417394 -0.0307990540877851
};
\end{axis}

\end{tikzpicture}

\subcaption{Point cloud $S$.}
\end{center}
\end{subfigure}
\begin{subfigure}[t]{0.49\textwidth}
\begin{center}
% This file was created by tikzplotlib v0.9.2.
\begin{tikzpicture}[thick,scale=0.8, every node/.style={transform shape}]

\definecolor{color0}{rgb}{0.12156862745098,0.466666666666667,0.705882352941177}
\definecolor{color1}{rgb}{1,0.498039215686275,0.0549019607843137}
\definecolor{color2}{rgb}{0.96078431372549,0.96078431372549,0.862745098039216}
\definecolor{color3}{rgb}{1,0.498039215686275,0.313725490196078}

\begin{axis}[
tick align=outside,
tick pos=left,
x grid style={white!69.0196078431373!black},
xmin=-3.45180709372072, xmax=2.99361349512875,
xtick style={color=black},
y grid style={white!69.0196078431373!black},
ymin=-3.45180709372072, ymax=2.99361349512875,
%ymin=-2.1938158082693, ymax=2.17815875977889,
ytick style={color=black}
]
\addplot [only marks, mark=*, draw=color0, fill=color0, colormap/viridis]
table{%
x                      y
-2.34275812782514 -0.422337072515164
0.24722747298104 -1.61319873633128
-1.38019830256552 0.640505435159906
-1.51714816067073 0.258502056354496
-0.175822267271081 -0.0252651026286875
-1.07790812289209 -1.40003857215438
1.08243752843236 1.42960529549044
1.45446356041819 1.05019519757262
0.272266762130476 -0.356289816160233
0.885609873179297 -0.34308175412289
0.318497415524185 -0.772246256656776
-0.486464599849546 -0.321047644667546
0.713664525872475 0.45089573918883
-3.15883343059119 1.79910626932364
1.03249205013434 -0.276218486658694
0.724867796410222 1.0119644513881
-0.66403264904207 1.07732210290748
-0.107877676726452 -0.840289685265736
1.01272382397982 -1.86298022827367
1.13929172816044 -1.07553705644164
-1.3126448374852 1.17645213621684
1.16650560208541 -1.36988664811057
-0.200901763190504 0.83207623547168
-0.162766483689669 0.758482159427621
-0.835280645501797 -0.205178137113613
-0.645878092492392 0.244918441252124
0.80474946774176 -1.05858981481516
0.520329697755992 -1.55654582504081
0.648482472610915 -1.07166832936951
0.512282026122496 -1.05460220157653
-1.62753042956307 0.144763499946151
-1.85256401192557 -0.124092650272665
-0.038308979791596 -0.906263941779864
2.70063983199922 -0.145679091954431
0.790570154320513 -0.374829753224335
-0.517513317388311 -0.760009533697624
0.567908869640562 -0.574932508687353
0.661942961282725 0.682717899396752
-0.897657764611887 -0.509805000910038
-0.125549177171554 -0.678365273082073
-0.878227336337594 0.691452914917409
-0.683949025620437 -0.59776859652354
-0.848312798273908 -0.798926002882831
0.112985819436421 -0.690906617864546
0.225572130815729 -1.11820672366738
1.30701775361222 0.471310001618085
1.17257671373237 -0.686628400040247
1.65057712884217 0.943486007801425
1.27651703162676 -0.507892465272379
1.18397678448062 -0.00794043507333255
0.491156506328976 1.50289694168447
-1.03569391216883 -0.0587326120757943
0.075771071420608 -0.192872784109992
-0.282308197205086 -1.27740706587447
-1.00748672473045 0.638365038742785
-1.27749128471686 0.65534756920239
-0.124824503595767 1.17650662278542
0.106611314555847 1.01602134591746
-1.13247270174148 -0.0219117447622604
-0.043848450781368 0.192488329289336
-0.0438948279215397 0.885200349964678
-0.0457880275415337 0.755095216844757
-1.28516929949252 0.112243759124342
-0.440817927077724 -0.304467240653771
0.161305007531463 0.47151626644206
-0.151580682549982 -0.582783939415887
-0.447730695225613 -0.452165071633186
0.372995023166489 -0.615248964656559
-0.707200164270895 -0.110920852624309
0.57264119973551 0.222410149332283
-0.291802622064795 0.0837691031769381
-2.03628199559491 -1.99508969153983
-2.11509949221634 -1.20644336802718
0.633895330915757 0.988721140629871
1.18243868402131 -0.182835421102174
1.56571290677136 0.226015644404297
1.50701998276123 -0.439351332338839
0.413852404170514 1.97943264304943
-0.285361579892089 1.35847512096702
-0.804429884793085 1.63103253934514
-0.058591464870352 -0.250495361764648
0.435590639323754 1.40570053244992
-0.320788003316259 0.471261823633868
0.287974042518776 0.149003002147366
0.811316786821193 -1.42384003157491
-0.96405415610173 -1.6757287701837
-1.02698880542662 0.0878139020675124
-1.05068784812419 -0.196660891853283
-1.37362659980222 -0.192873683671634
-0.0392455500475513 1.09768115355305
0.575537860345507 0.803161640917396
0.929214222549169 0.651745790327805
-0.97525635438475 -0.546892506013557
-0.042750903342611 0.0797462803574467
-0.290049580989981 0.882474726828786
1.08385271681253 -0.717881298334075
1.98268177645895 -0.319409301778917
1.31068071341738 0.588970190069706
-0.98887418389988 -0.00418003306320407
-0.571790491417394 -0.0307990540877851
};
\path [draw=color2, fill=color2]
(axis cs:-0.486464599849546,-0.321047644667546)
--(axis cs:-0.440817927077724,-0.304467240653771)
--(axis cs:-0.447730695225613,-0.452165071633186)
--cycle;
\path [draw=color2, fill=color2]
(axis cs:-1.03569391216883,-0.0587326120757943)
--(axis cs:-1.13247270174148,-0.0219117447622604)
--(axis cs:-0.98887418389988,-0.00418003306320407)
--cycle;
\path [draw=color2, fill=color2]
(axis cs:-1.13247270174148,-0.0219117447622604)
--(axis cs:-1.02698880542662,0.0878139020675124)
--(axis cs:-0.98887418389988,-0.00418003306320407)
--cycle;
\path [draw=color2, fill=color2]
(axis cs:-0.162766483689669,0.758482159427621)
--(axis cs:-0.0438948279215397,0.885200349964678)
--(axis cs:-0.0457880275415337,0.755095216844757)
--cycle;
\path [draw=color2, fill=color2]
(axis cs:-0.200901763190504,0.83207623547168)
--(axis cs:-0.162766483689669,0.758482159427621)
--(axis cs:-0.0438948279215397,0.885200349964678)
--cycle;
\path [draw=color2, fill=color2]
(axis cs:-1.03569391216883,-0.0587326120757943)
--(axis cs:-1.13247270174148,-0.0219117447622604)
--(axis cs:-1.05068784812419,-0.196660891853283)
--cycle;
\path [draw=color2, fill=color2]
(axis cs:0.106611314555847,1.01602134591746)
--(axis cs:-0.0438948279215397,0.885200349964678)
--(axis cs:-0.0392455500475513,1.09768115355305)
--cycle;
\path [draw=color2, fill=color2]
(axis cs:-0.835280645501797,-0.205178137113613)
--(axis cs:-1.03569391216883,-0.0587326120757943)
--(axis cs:-1.05068784812419,-0.196660891853283)
--cycle;
\path [draw=color2, fill=color2]
(axis cs:-0.175822267271081,-0.0252651026286875)
--(axis cs:-0.291802622064795,0.0837691031769381)
--(axis cs:-0.042750903342611,0.0797462803574467)
--cycle;
\path [draw=color2, fill=color2]
(axis cs:-0.835280645501797,-0.205178137113613)
--(axis cs:-1.03569391216883,-0.0587326120757943)
--(axis cs:-0.98887418389988,-0.00418003306320407)
--cycle;
\path [draw=color2, fill=color2]
(axis cs:-1.13247270174148,-0.0219117447622604)
--(axis cs:-1.28516929949252,0.112243759124342)
--(axis cs:-1.02698880542662,0.0878139020675124)
--cycle;
\path [draw=color2, fill=color2]
(axis cs:-0.107877676726452,-0.840289685265736)
--(axis cs:-0.038308979791596,-0.906263941779864)
--(axis cs:0.112985819436421,-0.690906617864546)
--cycle;
\path [draw=color2, fill=color2]
(axis cs:-0.107877676726452,-0.840289685265736)
--(axis cs:-0.125549177171554,-0.678365273082073)
--(axis cs:0.112985819436421,-0.690906617864546)
--cycle;
\path [draw=color2, fill=color2]
(axis cs:-0.043848450781368,0.192488329289336)
--(axis cs:-0.291802622064795,0.0837691031769381)
--(axis cs:-0.042750903342611,0.0797462803574467)
--cycle;
\path [draw=color2, fill=color2]
(axis cs:0.318497415524185,-0.772246256656776)
--(axis cs:0.112985819436421,-0.690906617864546)
--(axis cs:0.372995023166489,-0.615248964656559)
--cycle;
\path [draw=color2, fill=color2]
(axis cs:-0.897657764611887,-0.509805000910038)
--(axis cs:-0.848312798273908,-0.798926002882831)
--(axis cs:-0.97525635438475,-0.546892506013557)
--cycle;
\path [draw=color2, fill=color2]
(axis cs:-0.125549177171554,-0.678365273082073)
--(axis cs:0.112985819436421,-0.690906617864546)
--(axis cs:-0.151580682549982,-0.582783939415887)
--cycle;
\path [draw=color2, fill=color2]
(axis cs:-0.835280645501797,-0.205178137113613)
--(axis cs:-0.707200164270895,-0.110920852624309)
--(axis cs:-0.98887418389988,-0.00418003306320407)
--cycle;
\path [draw=color2, fill=color2]
(axis cs:0.724867796410222,1.0119644513881)
--(axis cs:0.633895330915757,0.988721140629871)
--(axis cs:0.575537860345507,0.803161640917396)
--cycle;
\path [draw=color2, fill=color2]
(axis cs:-0.175822267271081,-0.0252651026286875)
--(axis cs:0.075771071420608,-0.192872784109992)
--(axis cs:-0.058591464870352,-0.250495361764648)
--cycle;
\path [draw=color2, fill=color2]
(axis cs:-0.486464599849546,-0.321047644667546)
--(axis cs:-0.440817927077724,-0.304467240653771)
--(axis cs:-0.571790491417394,-0.0307990540877851)
--cycle;
\path [draw=color2, fill=color2]
(axis cs:-0.897657764611887,-0.509805000910038)
--(axis cs:-0.683949025620437,-0.59776859652354)
--(axis cs:-0.848312798273908,-0.798926002882831)
--cycle;
\path [draw=color2, fill=color2]
(axis cs:0.713664525872475,0.45089573918883)
--(axis cs:0.661942961282725,0.682717899396752)
--(axis cs:0.929214222549169,0.651745790327805)
--cycle;
\path [draw=color2, fill=color2]
(axis cs:-0.175822267271081,-0.0252651026286875)
--(axis cs:0.075771071420608,-0.192872784109992)
--(axis cs:-0.042750903342611,0.0797462803574467)
--cycle;
\path [draw=color2, fill=color2]
(axis cs:-0.486464599849546,-0.321047644667546)
--(axis cs:-0.707200164270895,-0.110920852624309)
--(axis cs:-0.571790491417394,-0.0307990540877851)
--cycle;
\path [draw=color2, fill=color2]
(axis cs:-0.200901763190504,0.83207623547168)
--(axis cs:-0.162766483689669,0.758482159427621)
--(axis cs:-0.290049580989981,0.882474726828786)
--cycle;
\path [draw=color2, fill=color2]
(axis cs:-0.517513317388311,-0.760009533697624)
--(axis cs:-0.683949025620437,-0.59776859652354)
--(axis cs:-0.447730695225613,-0.452165071633186)
--cycle;
\path [draw=color2, fill=color2]
(axis cs:-1.13247270174148,-0.0219117447622604)
--(axis cs:-1.28516929949252,0.112243759124342)
--(axis cs:-1.37362659980222,-0.192873683671634)
--cycle;
\path [draw=color2, fill=color2]
(axis cs:-1.13247270174148,-0.0219117447622604)
--(axis cs:-1.05068784812419,-0.196660891853283)
--(axis cs:-1.37362659980222,-0.192873683671634)
--cycle;
\path [draw=color2, fill=color2]
(axis cs:-0.200901763190504,0.83207623547168)
--(axis cs:-0.0438948279215397,0.885200349964678)
--(axis cs:-0.290049580989981,0.882474726828786)
--cycle;
\path [draw=color2, fill=color2]
(axis cs:-0.0438948279215397,0.885200349964678)
--(axis cs:-0.0392455500475513,1.09768115355305)
--(axis cs:-0.290049580989981,0.882474726828786)
--cycle;
\path [draw=color2, fill=color2]
(axis cs:-0.517513317388311,-0.760009533697624)
--(axis cs:-0.683949025620437,-0.59776859652354)
--(axis cs:-0.848312798273908,-0.798926002882831)
--cycle;
\path [draw=color2, fill=color2]
(axis cs:-0.124824503595767,1.17650662278542)
--(axis cs:-0.0392455500475513,1.09768115355305)
--(axis cs:-0.290049580989981,0.882474726828786)
--cycle;
\path [draw=color2, fill=color2]
(axis cs:-0.043848450781368,0.192488329289336)
--(axis cs:0.287974042518776,0.149003002147366)
--(axis cs:-0.042750903342611,0.0797462803574467)
--cycle;
\path [draw=color2, fill=color2]
(axis cs:1.03249205013434,-0.276218486658694)
--(axis cs:1.27651703162676,-0.507892465272379)
--(axis cs:1.18243868402131,-0.182835421102174)
--cycle;
\path [draw=color2, fill=color2]
(axis cs:-1.51714816067073,0.258502056354496)
--(axis cs:-1.62753042956307,0.144763499946151)
--(axis cs:-1.28516929949252,0.112243759124342)
--cycle;
\path [draw=color2, fill=color2]
(axis cs:-0.486464599849546,-0.321047644667546)
--(axis cs:-0.683949025620437,-0.59776859652354)
--(axis cs:-0.447730695225613,-0.452165071633186)
--cycle;
\path [draw=color2, fill=color2]
(axis cs:-0.835280645501797,-0.205178137113613)
--(axis cs:-0.897657764611887,-0.509805000910038)
--(axis cs:-1.05068784812419,-0.196660891853283)
--cycle;
\path [draw=color2, fill=color2]
(axis cs:0.724867796410222,1.0119644513881)
--(axis cs:0.661942961282725,0.682717899396752)
--(axis cs:0.575537860345507,0.803161640917396)
--cycle;
\path [draw=color2, fill=color2]
(axis cs:-0.897657764611887,-0.509805000910038)
--(axis cs:-1.05068784812419,-0.196660891853283)
--(axis cs:-0.97525635438475,-0.546892506013557)
--cycle;
\path [draw=color2, fill=color2]
(axis cs:1.03249205013434,-0.276218486658694)
--(axis cs:1.18397678448062,-0.00794043507333255)
--(axis cs:1.18243868402131,-0.182835421102174)
--cycle;
\path [draw=color2, fill=color2]
(axis cs:0.318497415524185,-0.772246256656776)
--(axis cs:0.567908869640562,-0.574932508687353)
--(axis cs:0.372995023166489,-0.615248964656559)
--cycle;
\path [draw=color2, fill=color2]
(axis cs:0.272266762130476,-0.356289816160233)
--(axis cs:0.112985819436421,-0.690906617864546)
--(axis cs:0.372995023166489,-0.615248964656559)
--cycle;
\path [draw=color2, fill=color2]
(axis cs:0.272266762130476,-0.356289816160233)
--(axis cs:0.567908869640562,-0.574932508687353)
--(axis cs:0.372995023166489,-0.615248964656559)
--cycle;
\path [draw=color2, fill=color2]
(axis cs:-0.486464599849546,-0.321047644667546)
--(axis cs:-0.835280645501797,-0.205178137113613)
--(axis cs:-0.707200164270895,-0.110920852624309)
--cycle;
\path [draw=color2, fill=color2]
(axis cs:-0.645878092492392,0.244918441252124)
--(axis cs:-0.707200164270895,-0.110920852624309)
--(axis cs:-0.571790491417394,-0.0307990540877851)
--cycle;
\path [draw=color2, fill=color2]
(axis cs:0.318497415524185,-0.772246256656776)
--(axis cs:0.512282026122496,-1.05460220157653)
--(axis cs:0.225572130815729,-1.11820672366738)
--cycle;
\path [draw=color2, fill=color2]
(axis cs:0.80474946774176,-1.05858981481516)
--(axis cs:0.648482472610915,-1.07166832936951)
--(axis cs:0.811316786821193,-1.42384003157491)
--cycle;
\path [draw=color2, fill=color2]
(axis cs:0.272266762130476,-0.356289816160233)
--(axis cs:0.075771071420608,-0.192872784109992)
--(axis cs:-0.058591464870352,-0.250495361764648)
--cycle;
\path [draw=color2, fill=color2]
(axis cs:0.318497415524185,-0.772246256656776)
--(axis cs:-0.038308979791596,-0.906263941779864)
--(axis cs:0.112985819436421,-0.690906617864546)
--cycle;
\path [draw=color2, fill=color2]
(axis cs:-0.645878092492392,0.244918441252124)
--(axis cs:-0.291802622064795,0.0837691031769381)
--(axis cs:-0.571790491417394,-0.0307990540877851)
--cycle;
\path [draw=color2, fill=color2]
(axis cs:-0.043848450781368,0.192488329289336)
--(axis cs:0.161305007531463,0.47151626644206)
--(axis cs:0.287974042518776,0.149003002147366)
--cycle;
\path [draw=color2, fill=color2]
(axis cs:1.13929172816044,-1.07553705644164)
--(axis cs:1.17257671373237,-0.686628400040247)
--(axis cs:1.08385271681253,-0.717881298334075)
--cycle;
\path [draw=color2, fill=color2]
(axis cs:0.106611314555847,1.01602134591746)
--(axis cs:-0.0438948279215397,0.885200349964678)
--(axis cs:-0.0457880275415337,0.755095216844757)
--cycle;
\path [draw=color2, fill=color2]
(axis cs:-0.517513317388311,-0.760009533697624)
--(axis cs:-0.125549177171554,-0.678365273082073)
--(axis cs:-0.151580682549982,-0.582783939415887)
--cycle;
\path [draw=color2, fill=color2]
(axis cs:-0.175822267271081,-0.0252651026286875)
--(axis cs:-0.440817927077724,-0.304467240653771)
--(axis cs:-0.058591464870352,-0.250495361764648)
--cycle;
\path [draw=color2, fill=color2]
(axis cs:0.075771071420608,-0.192872784109992)
--(axis cs:0.287974042518776,0.149003002147366)
--(axis cs:-0.042750903342611,0.0797462803574467)
--cycle;
\path [draw=color2, fill=color2]
(axis cs:1.27651703162676,-0.507892465272379)
--(axis cs:1.18243868402131,-0.182835421102174)
--(axis cs:1.50701998276123,-0.439351332338839)
--cycle;
\path [draw=color2, fill=color2]
(axis cs:-0.517513317388311,-0.760009533697624)
--(axis cs:-0.151580682549982,-0.582783939415887)
--(axis cs:-0.447730695225613,-0.452165071633186)
--cycle;
\path [draw=color2, fill=color2]
(axis cs:0.724867796410222,1.0119644513881)
--(axis cs:0.661942961282725,0.682717899396752)
--(axis cs:0.929214222549169,0.651745790327805)
--cycle;
\path [draw=color2, fill=color2]
(axis cs:-0.175822267271081,-0.0252651026286875)
--(axis cs:-0.440817927077724,-0.304467240653771)
--(axis cs:-0.291802622064795,0.0837691031769381)
--cycle;
\path [draw=color2, fill=color2]
(axis cs:0.318497415524185,-0.772246256656776)
--(axis cs:-0.038308979791596,-0.906263941779864)
--(axis cs:0.225572130815729,-1.11820672366738)
--cycle;
\path [draw=color2, fill=color2]
(axis cs:-0.440817927077724,-0.304467240653771)
--(axis cs:-0.291802622064795,0.0837691031769381)
--(axis cs:-0.571790491417394,-0.0307990540877851)
--cycle;
\path [draw=color2, fill=color2]
(axis cs:-0.043848450781368,0.192488329289336)
--(axis cs:-0.291802622064795,0.0837691031769381)
--(axis cs:-0.320788003316259,0.471261823633868)
--cycle;
\path [draw=color2, fill=color2]
(axis cs:-0.107877676726452,-0.840289685265736)
--(axis cs:-0.517513317388311,-0.760009533697624)
--(axis cs:-0.125549177171554,-0.678365273082073)
--cycle;
\path [draw=color2, fill=color2]
(axis cs:1.30701775361222,0.471310001618085)
--(axis cs:0.929214222549169,0.651745790327805)
--(axis cs:1.31068071341738,0.588970190069706)
--cycle;
\path [draw=color2, fill=color2]
(axis cs:-0.645878092492392,0.244918441252124)
--(axis cs:-1.02698880542662,0.0878139020675124)
--(axis cs:-0.98887418389988,-0.00418003306320407)
--cycle;
\path [draw=color2, fill=color2]
(axis cs:-0.162766483689669,0.758482159427621)
--(axis cs:-0.320788003316259,0.471261823633868)
--(axis cs:-0.290049580989981,0.882474726828786)
--cycle;
\path [draw=color2, fill=color2]
(axis cs:-0.835280645501797,-0.205178137113613)
--(axis cs:-0.897657764611887,-0.509805000910038)
--(axis cs:-0.683949025620437,-0.59776859652354)
--cycle;
\path [draw=color2, fill=color2]
(axis cs:-1.62753042956307,0.144763499946151)
--(axis cs:-1.28516929949252,0.112243759124342)
--(axis cs:-1.37362659980222,-0.192873683671634)
--cycle;
\path [draw=color2, fill=color2]
(axis cs:-0.440817927077724,-0.304467240653771)
--(axis cs:-0.151580682549982,-0.582783939415887)
--(axis cs:-0.447730695225613,-0.452165071633186)
--cycle;
\path [draw=color2, fill=color2]
(axis cs:-0.645878092492392,0.244918441252124)
--(axis cs:-0.707200164270895,-0.110920852624309)
--(axis cs:-0.98887418389988,-0.00418003306320407)
--cycle;
\path [draw=color2, fill=color2]
(axis cs:-0.440817927077724,-0.304467240653771)
--(axis cs:-0.151580682549982,-0.582783939415887)
--(axis cs:-0.058591464870352,-0.250495361764648)
--cycle;
\path [draw=color2, fill=color2]
(axis cs:1.17257671373237,-0.686628400040247)
--(axis cs:1.27651703162676,-0.507892465272379)
--(axis cs:1.08385271681253,-0.717881298334075)
--cycle;
\path [draw=color2, fill=color2]
(axis cs:-0.486464599849546,-0.321047644667546)
--(axis cs:-0.835280645501797,-0.205178137113613)
--(axis cs:-0.683949025620437,-0.59776859652354)
--cycle;
\path [draw=color2, fill=color2]
(axis cs:-0.162766483689669,0.758482159427621)
--(axis cs:-0.0457880275415337,0.755095216844757)
--(axis cs:-0.320788003316259,0.471261823633868)
--cycle;
\path [draw=color2, fill=color2]
(axis cs:1.03249205013434,-0.276218486658694)
--(axis cs:1.27651703162676,-0.507892465272379)
--(axis cs:1.08385271681253,-0.717881298334075)
--cycle;
\path [draw=color2, fill=color2]
(axis cs:0.885609873179297,-0.34308175412289)
--(axis cs:1.03249205013434,-0.276218486658694)
--(axis cs:1.08385271681253,-0.717881298334075)
--cycle;
\path [draw=color2, fill=color2]
(axis cs:1.13929172816044,-1.07553705644164)
--(axis cs:0.80474946774176,-1.05858981481516)
--(axis cs:1.08385271681253,-0.717881298334075)
--cycle;
\path [draw=color2, fill=color2]
(axis cs:-0.645878092492392,0.244918441252124)
--(axis cs:-0.291802622064795,0.0837691031769381)
--(axis cs:-0.320788003316259,0.471261823633868)
--cycle;
\path [draw=color2, fill=color2]
(axis cs:0.885609873179297,-0.34308175412289)
--(axis cs:0.790570154320513,-0.374829753224335)
--(axis cs:1.08385271681253,-0.717881298334075)
--cycle;
\path [draw=color2, fill=color2]
(axis cs:0.112985819436421,-0.690906617864546)
--(axis cs:-0.151580682549982,-0.582783939415887)
--(axis cs:-0.058591464870352,-0.250495361764648)
--cycle;
\path [draw=color2, fill=color2]
(axis cs:0.272266762130476,-0.356289816160233)
--(axis cs:0.112985819436421,-0.690906617864546)
--(axis cs:-0.058591464870352,-0.250495361764648)
--cycle;
\path [draw=color2, fill=color2]
(axis cs:-0.107877676726452,-0.840289685265736)
--(axis cs:-0.038308979791596,-0.906263941779864)
--(axis cs:-0.282308197205086,-1.27740706587447)
--cycle;
\path [draw=color2, fill=color2]
(axis cs:1.13929172816044,-1.07553705644164)
--(axis cs:1.16650560208541,-1.36988664811057)
--(axis cs:0.811316786821193,-1.42384003157491)
--cycle;
\path [draw=color2, fill=color2]
(axis cs:1.13929172816044,-1.07553705644164)
--(axis cs:0.80474946774176,-1.05858981481516)
--(axis cs:0.811316786821193,-1.42384003157491)
--cycle;
\path [draw=color2, fill=color2]
(axis cs:0.318497415524185,-0.772246256656776)
--(axis cs:0.512282026122496,-1.05460220157653)
--(axis cs:0.567908869640562,-0.574932508687353)
--cycle;
\path [draw=color2, fill=color2]
(axis cs:0.161305007531463,0.47151626644206)
--(axis cs:0.57264119973551,0.222410149332283)
--(axis cs:0.287974042518776,0.149003002147366)
--cycle;
\path [draw=color2, fill=color2]
(axis cs:-0.043848450781368,0.192488329289336)
--(axis cs:0.161305007531463,0.47151626644206)
--(axis cs:-0.320788003316259,0.471261823633868)
--cycle;
\path [draw=color2, fill=color2]
(axis cs:-0.0457880275415337,0.755095216844757)
--(axis cs:0.161305007531463,0.47151626644206)
--(axis cs:-0.320788003316259,0.471261823633868)
--cycle;
\path [draw=color2, fill=color2]
(axis cs:-1.62753042956307,0.144763499946151)
--(axis cs:-1.85256401192557,-0.124092650272665)
--(axis cs:-1.37362659980222,-0.192873683671634)
--cycle;
\path [draw=color2, fill=color2]
(axis cs:0.724867796410222,1.0119644513881)
--(axis cs:0.633895330915757,0.988721140629871)
--(axis cs:0.435590639323754,1.40570053244992)
--cycle;
\path [draw=color2, fill=color2]
(axis cs:1.45446356041819,1.05019519757262)
--(axis cs:1.65057712884217,0.943486007801425)
--(axis cs:1.31068071341738,0.588970190069706)
--cycle;
\path [draw=color2, fill=color2]
(axis cs:0.520329697755992,-1.55654582504081)
--(axis cs:0.648482472610915,-1.07166832936951)
--(axis cs:0.811316786821193,-1.42384003157491)
--cycle;
\path [draw=color2, fill=color2]
(axis cs:0.648482472610915,-1.07166832936951)
--(axis cs:0.512282026122496,-1.05460220157653)
--(axis cs:0.567908869640562,-0.574932508687353)
--cycle;
\path [draw=color2, fill=color2]
(axis cs:-0.124824503595767,1.17650662278542)
--(axis cs:-0.285361579892089,1.35847512096702)
--(axis cs:-0.290049580989981,0.882474726828786)
--cycle;
\path [draw=color2, fill=color2]
(axis cs:0.520329697755992,-1.55654582504081)
--(axis cs:0.648482472610915,-1.07166832936951)
--(axis cs:0.512282026122496,-1.05460220157653)
--cycle;
\path [draw=color2, fill=color2]
(axis cs:0.272266762130476,-0.356289816160233)
--(axis cs:0.075771071420608,-0.192872784109992)
--(axis cs:0.287974042518776,0.149003002147366)
--cycle;
\path [draw=color2, fill=color2]
(axis cs:1.30701775361222,0.471310001618085)
--(axis cs:1.18397678448062,-0.00794043507333255)
--(axis cs:1.56571290677136,0.226015644404297)
--cycle;
\path [draw=color2, fill=color2]
(axis cs:-1.38019830256552,0.640505435159906)
--(axis cs:-1.51714816067073,0.258502056354496)
--(axis cs:-1.27749128471686,0.65534756920239)
--cycle;
\path [draw=color2, fill=color2]
(axis cs:-0.66403264904207,1.07732210290748)
--(axis cs:-0.285361579892089,1.35847512096702)
--(axis cs:-0.290049580989981,0.882474726828786)
--cycle;
\path [draw=color2, fill=color2]
(axis cs:0.272266762130476,-0.356289816160233)
--(axis cs:0.790570154320513,-0.374829753224335)
--(axis cs:0.567908869640562,-0.574932508687353)
--cycle;
\path [draw=color2, fill=color2]
(axis cs:0.106611314555847,1.01602134591746)
--(axis cs:0.633895330915757,0.988721140629871)
--(axis cs:0.575537860345507,0.803161640917396)
--cycle;
\path [draw=color2, fill=color2]
(axis cs:-0.038308979791596,-0.906263941779864)
--(axis cs:0.225572130815729,-1.11820672366738)
--(axis cs:-0.282308197205086,-1.27740706587447)
--cycle;
\path [draw=color2, fill=color2]
(axis cs:0.24722747298104,-1.61319873633128)
--(axis cs:0.520329697755992,-1.55654582504081)
--(axis cs:0.225572130815729,-1.11820672366738)
--cycle;
\path [draw=color2, fill=color2]
(axis cs:0.790570154320513,-0.374829753224335)
--(axis cs:0.567908869640562,-0.574932508687353)
--(axis cs:1.08385271681253,-0.717881298334075)
--cycle;
\path [draw=color2, fill=color2]
(axis cs:-0.645878092492392,0.244918441252124)
--(axis cs:-0.878227336337594,0.691452914917409)
--(axis cs:-1.00748672473045,0.638365038742785)
--cycle;
\path [draw=color2, fill=color2]
(axis cs:1.01272382397982,-1.86298022827367)
--(axis cs:1.16650560208541,-1.36988664811057)
--(axis cs:0.811316786821193,-1.42384003157491)
--cycle;
\path [draw=color2, fill=color2]
(axis cs:0.520329697755992,-1.55654582504081)
--(axis cs:0.512282026122496,-1.05460220157653)
--(axis cs:0.225572130815729,-1.11820672366738)
--cycle;
\path [draw=color2, fill=color2]
(axis cs:0.80474946774176,-1.05858981481516)
--(axis cs:0.648482472610915,-1.07166832936951)
--(axis cs:0.567908869640562,-0.574932508687353)
--cycle;
\path [draw=color2, fill=color2]
(axis cs:-1.38019830256552,0.640505435159906)
--(axis cs:-1.3126448374852,1.17645213621684)
--(axis cs:-1.27749128471686,0.65534756920239)
--cycle;
\path [draw=color2, fill=color2]
(axis cs:-1.51714816067073,0.258502056354496)
--(axis cs:-1.27749128471686,0.65534756920239)
--(axis cs:-1.28516929949252,0.112243759124342)
--cycle;
\path [draw=color2, fill=color2]
(axis cs:0.661942961282725,0.682717899396752)
--(axis cs:0.161305007531463,0.47151626644206)
--(axis cs:0.575537860345507,0.803161640917396)
--cycle;
\path [draw=color2, fill=color2]
(axis cs:-1.05068784812419,-0.196660891853283)
--(axis cs:-1.37362659980222,-0.192873683671634)
--(axis cs:-0.97525635438475,-0.546892506013557)
--cycle;
\path [draw=color2, fill=color2]
(axis cs:0.713664525872475,0.45089573918883)
--(axis cs:0.161305007531463,0.47151626644206)
--(axis cs:0.57264119973551,0.222410149332283)
--cycle;
\path [draw=color2, fill=color2]
(axis cs:0.713664525872475,0.45089573918883)
--(axis cs:0.661942961282725,0.682717899396752)
--(axis cs:0.161305007531463,0.47151626644206)
--cycle;
\path [draw=color2, fill=color2]
(axis cs:0.106611314555847,1.01602134591746)
--(axis cs:0.435590639323754,1.40570053244992)
--(axis cs:-0.0392455500475513,1.09768115355305)
--cycle;
\path [draw=color2, fill=color2]
(axis cs:-0.107877676726452,-0.840289685265736)
--(axis cs:-0.517513317388311,-0.760009533697624)
--(axis cs:-0.282308197205086,-1.27740706587447)
--cycle;
\path [draw=color2, fill=color2]
(axis cs:0.106611314555847,1.01602134591746)
--(axis cs:0.633895330915757,0.988721140629871)
--(axis cs:0.435590639323754,1.40570053244992)
--cycle;
\path [draw=color2, fill=color2]
(axis cs:1.01272382397982,-1.86298022827367)
--(axis cs:0.520329697755992,-1.55654582504081)
--(axis cs:0.811316786821193,-1.42384003157491)
--cycle;
\addplot [very thick, color3]
table {%
-0.486464599849546 -0.321047644667546
-0.440817927077724 -0.304467240653771
};
\addplot [very thick, color3]
table {%
-1.03569391216883 -0.0587326120757943
-0.98887418389988 -0.00418003306320407
};
\addplot [very thick, color3]
table {%
-0.200901763190504 0.83207623547168
-0.162766483689669 0.758482159427621
};
\addplot [very thick, color3]
table {%
-0.897657764611887 -0.509805000910038
-0.97525635438475 -0.546892506013557
};
\addplot [very thick, color3]
table {%
0.724867796410222 1.0119644513881
0.633895330915757 0.988721140629871
};
\addplot [very thick, color3]
table {%
1.17257671373237 -0.686628400040247
1.08385271681253 -0.717881298334075
};
\addplot [very thick, color3]
table {%
-0.107877676726452 -0.840289685265736
-0.038308979791596 -0.906263941779864
};
\addplot [very thick, color3]
table {%
-0.125549177171554 -0.678365273082073
-0.151580682549982 -0.582783939415887
};
\addplot [very thick, color3]
table {%
-1.02698880542662 0.0878139020675124
-0.98887418389988 -0.00418003306320407
};
\addplot [very thick, color3]
table {%
0.885609873179297 -0.34308175412289
0.790570154320513 -0.374829753224335
};
\addplot [very thick, color3]
table {%
-0.200901763190504 0.83207623547168
-0.290049580989981 0.882474726828786
};
\addplot [very thick, color3]
table {%
-1.03569391216883 -0.0587326120757943
-1.13247270174148 -0.0219117447622604
};
\addplot [very thick, color3]
table {%
-1.38019830256552 0.640505435159906
-1.27749128471686 0.65534756920239
};
\addplot [very thick, color3]
table {%
0.491156506328976 1.50289694168447
0.435590639323754 1.40570053244992
};
\addplot [very thick, color3]
table {%
-0.043848450781368 0.192488329289336
-0.042750903342611 0.0797462803574467
};
\addplot [very thick, color3]
table {%
-0.124824503595767 1.17650662278542
-0.0392455500475513 1.09768115355305
};
\addplot [very thick, color3]
table {%
-0.162766483689669 0.758482159427621
-0.0457880275415337 0.755095216844757
};
\addplot [very thick, color3]
table {%
1.30701775361222 0.471310001618085
1.31068071341738 0.588970190069706
};
\addplot [very thick, color3]
table {%
-0.0438948279215397 0.885200349964678
-0.0457880275415337 0.755095216844757
};
\addplot [very thick, color3]
table {%
-0.486464599849546 -0.321047644667546
-0.447730695225613 -0.452165071633186
};
\addplot [very thick, color3]
table {%
0.648482472610915 -1.07166832936951
0.512282026122496 -1.05460220157653
};
\addplot [very thick, color3]
table {%
-1.03569391216883 -0.0587326120757943
-1.05068784812419 -0.196660891853283
};
\addplot [very thick, color3]
table {%
-0.878227336337594 0.691452914917409
-1.00748672473045 0.638365038742785
};
\addplot [very thick, color3]
table {%
0.075771071420608 -0.192872784109992
-0.058591464870352 -0.250495361764648
};
\addplot [very thick, color3]
table {%
-0.440817927077724 -0.304467240653771
-0.447730695225613 -0.452165071633186
};
\addplot [very thick, color3]
table {%
0.661942961282725 0.682717899396752
0.575537860345507 0.803161640917396
};
\addplot [very thick, color3]
table {%
-1.13247270174148 -0.0219117447622604
-1.02698880542662 0.0878139020675124
};
\addplot [very thick, color3]
table {%
-1.13247270174148 -0.0219117447622604
-0.98887418389988 -0.00418003306320407
};
\addplot [very thick, color3]
table {%
0.80474946774176 -1.05858981481516
0.648482472610915 -1.07166832936951
};
\addplot [very thick, color3]
table {%
-0.707200164270895 -0.110920852624309
-0.571790491417394 -0.0307990540877851
};
\addplot [very thick, color3]
table {%
-1.51714816067073 0.258502056354496
-1.62753042956307 0.144763499946151
};
\addplot [very thick, color3]
table {%
-0.835280645501797 -0.205178137113613
-0.707200164270895 -0.110920852624309
};
\addplot [very thick, color3]
table {%
-0.175822267271081 -0.0252651026286875
-0.291802622064795 0.0837691031769381
};
\addplot [very thick, color3]
table {%
0.885609873179297 -0.34308175412289
1.03249205013434 -0.276218486658694
};
\addplot [very thick, color3]
table {%
-0.107877676726452 -0.840289685265736
-0.125549177171554 -0.678365273082073
};
\addplot [very thick, color3]
table {%
-0.200901763190504 0.83207623547168
-0.0438948279215397 0.885200349964678
};
\addplot [very thick, color3]
table {%
0.318497415524185 -0.772246256656776
0.372995023166489 -0.615248964656559
};
\addplot [very thick, color3]
table {%
0.106611314555847 1.01602134591746
-0.0392455500475513 1.09768115355305
};
\addplot [very thick, color3]
table {%
-0.175822267271081 -0.0252651026286875
-0.042750903342611 0.0797462803574467
};
\addplot [very thick, color3]
table {%
-0.162766483689669 0.758482159427621
-0.0438948279215397 0.885200349964678
};
\addplot [very thick, color3]
table {%
1.18397678448062 -0.00794043507333255
1.18243868402131 -0.182835421102174
};
\addplot [very thick, color3]
table {%
1.03249205013434 -0.276218486658694
1.18243868402131 -0.182835421102174
};
\addplot [very thick, color3]
table {%
0.633895330915757 0.988721140629871
0.575537860345507 0.803161640917396
};
\addplot [very thick, color3]
table {%
0.567908869640562 -0.574932508687353
0.372995023166489 -0.615248964656559
};
\addplot [very thick, color3]
table {%
-1.13247270174148 -0.0219117447622604
-1.05068784812419 -0.196660891853283
};
\addplot [very thick, color3]
table {%
0.106611314555847 1.01602134591746
-0.0438948279215397 0.885200349964678
};
\addplot [very thick, color3]
table {%
-1.13247270174148 -0.0219117447622604
-1.28516929949252 0.112243759124342
};
\addplot [very thick, color3]
table {%
1.17257671373237 -0.686628400040247
1.27651703162676 -0.507892465272379
};
\addplot [very thick, color3]
table {%
-0.0438948279215397 0.885200349964678
-0.0392455500475513 1.09768115355305
};
\addplot [very thick, color3]
table {%
-0.835280645501797 -0.205178137113613
-1.05068784812419 -0.196660891853283
};
\addplot [very thick, color3]
table {%
0.318497415524185 -0.772246256656776
0.112985819436421 -0.690906617864546
};
\addplot [very thick, color3]
table {%
1.45446356041819 1.05019519757262
1.65057712884217 0.943486007801425
};
\addplot [very thick, color3]
table {%
-0.897657764611887 -0.509805000910038
-0.683949025620437 -0.59776859652354
};
\addplot [very thick, color3]
table {%
-0.517513317388311 -0.760009533697624
-0.683949025620437 -0.59776859652354
};
\addplot [very thick, color3]
table {%
0.713664525872475 0.45089573918883
0.661942961282725 0.682717899396752
};
\addplot [very thick, color3]
table {%
-0.125549177171554 -0.678365273082073
0.112985819436421 -0.690906617864546
};
\addplot [very thick, color3]
table {%
1.27651703162676 -0.507892465272379
1.50701998276123 -0.439351332338839
};
\addplot [very thick, color3]
table {%
-0.124824503595767 1.17650662278542
-0.285361579892089 1.35847512096702
};
\addplot [very thick, color3]
table {%
-0.835280645501797 -0.205178137113613
-1.03569391216883 -0.0587326120757943
};
\addplot [very thick, color3]
table {%
-0.291802622064795 0.0837691031769381
-0.042750903342611 0.0797462803574467
};
\addplot [very thick, color3]
table {%
-0.835280645501797 -0.205178137113613
-0.98887418389988 -0.00418003306320407
};
\addplot [very thick, color3]
table {%
-0.175822267271081 -0.0252651026286875
-0.058591464870352 -0.250495361764648
};
\addplot [very thick, color3]
table {%
0.272266762130476 -0.356289816160233
0.075771071420608 -0.192872784109992
};
\addplot [very thick, color3]
table {%
-1.28516929949252 0.112243759124342
-1.02698880542662 0.0878139020675124
};
\addplot [very thick, color3]
table {%
-0.683949025620437 -0.59776859652354
-0.848312798273908 -0.798926002882831
};
\addplot [very thick, color3]
table {%
-0.038308979791596 -0.906263941779864
0.112985819436421 -0.690906617864546
};
\addplot [very thick, color3]
table {%
-0.107877676726452 -0.840289685265736
0.112985819436421 -0.690906617864546
};
\addplot [very thick, color3]
table {%
0.713664525872475 0.45089573918883
0.57264119973551 0.222410149332283
};
\addplot [very thick, color3]
table {%
0.661942961282725 0.682717899396752
0.929214222549169 0.651745790327805
};
\addplot [very thick, color3]
table {%
-1.00748672473045 0.638365038742785
-1.27749128471686 0.65534756920239
};
\addplot [very thick, color3]
table {%
-0.043848450781368 0.192488329289336
-0.291802622064795 0.0837691031769381
};
\addplot [very thick, color3]
table {%
0.112985819436421 -0.690906617864546
0.372995023166489 -0.615248964656559
};
\addplot [very thick, color3]
table {%
-1.51714816067073 0.258502056354496
-1.28516929949252 0.112243759124342
};
\addplot [very thick, color3]
table {%
-0.683949025620437 -0.59776859652354
-0.447730695225613 -0.452165071633186
};
\addplot [very thick, color3]
table {%
0.272266762130476 -0.356289816160233
0.372995023166489 -0.615248964656559
};
\addplot [very thick, color3]
table {%
0.24722747298104 -1.61319873633128
0.520329697755992 -1.55654582504081
};
\addplot [very thick, color3]
table {%
-0.848312798273908 -0.798926002882831
-0.97525635438475 -0.546892506013557
};
\addplot [very thick, color3]
table {%
-0.645878092492392 0.244918441252124
-0.571790491417394 -0.0307990540877851
};
\addplot [very thick, color3]
table {%
-0.897657764611887 -0.509805000910038
-0.848312798273908 -0.798926002882831
};
\addplot [very thick, color3]
table {%
0.512282026122496 -1.05460220157653
0.225572130815729 -1.11820672366738
};
\addplot [very thick, color3]
table {%
0.57264119973551 0.222410149332283
0.287974042518776 0.149003002147366
};
\addplot [very thick, color3]
table {%
0.713664525872475 0.45089573918883
0.929214222549169 0.651745790327805
};
\addplot [very thick, color3]
table {%
1.13929172816044 -1.07553705644164
1.16650560208541 -1.36988664811057
};
\addplot [very thick, color3]
table {%
-1.13247270174148 -0.0219117447622604
-1.37362659980222 -0.192873683671634
};
\addplot [very thick, color3]
table {%
0.075771071420608 -0.192872784109992
-0.042750903342611 0.0797462803574467
};
\addplot [very thick, color3]
table {%
-1.07790812289209 -1.40003857215438
-0.96405415610173 -1.6757287701837
};
\addplot [very thick, color3]
table {%
0.790570154320513 -0.374829753224335
0.567908869640562 -0.574932508687353
};
\addplot [very thick, color3]
table {%
0.112985819436421 -0.690906617864546
-0.151580682549982 -0.582783939415887
};
\addplot [very thick, color3]
table {%
-0.707200164270895 -0.110920852624309
-0.98887418389988 -0.00418003306320407
};
\addplot [very thick, color3]
table {%
0.724867796410222 1.0119644513881
0.575537860345507 0.803161640917396
};
\addplot [very thick, color3]
table {%
-0.291802622064795 0.0837691031769381
-0.571790491417394 -0.0307990540877851
};
\addplot [very thick, color3]
table {%
-0.486464599849546 -0.321047644667546
-0.571790491417394 -0.0307990540877851
};
\addplot [very thick, color3]
table {%
-0.175822267271081 -0.0252651026286875
0.075771071420608 -0.192872784109992
};
\addplot [very thick, color3]
table {%
-0.440817927077724 -0.304467240653771
-0.571790491417394 -0.0307990540877851
};
\addplot [very thick, color3]
table {%
-0.486464599849546 -0.321047644667546
-0.707200164270895 -0.110920852624309
};
\addplot [very thick, color3]
table {%
-0.835280645501797 -0.205178137113613
-0.897657764611887 -0.509805000910038
};
\addplot [very thick, color3]
table {%
-0.517513317388311 -0.760009533697624
-0.447730695225613 -0.452165071633186
};
\addplot [very thick, color3]
table {%
-1.28516929949252 0.112243759124342
-1.37362659980222 -0.192873683671634
};
\addplot [very thick, color3]
table {%
0.520329697755992 -1.55654582504081
0.811316786821193 -1.42384003157491
};
\addplot [very thick, color3]
table {%
-1.05068784812419 -0.196660891853283
-1.37362659980222 -0.192873683671634
};
\addplot [very thick, color3]
table {%
-0.151580682549982 -0.582783939415887
-0.447730695225613 -0.452165071633186
};
\addplot [very thick, color3]
table {%
-0.162766483689669 0.758482159427621
-0.290049580989981 0.882474726828786
};
\addplot [very thick, color3]
table {%
-0.162766483689669 0.758482159427621
-0.320788003316259 0.471261823633868
};
\addplot [very thick, color3]
table {%
-0.0438948279215397 0.885200349964678
-0.290049580989981 0.882474726828786
};
\addplot [very thick, color3]
table {%
-0.0392455500475513 1.09768115355305
-0.290049580989981 0.882474726828786
};
\addplot [very thick, color3]
table {%
-0.517513317388311 -0.760009533697624
-0.848312798273908 -0.798926002882831
};
\addplot [very thick, color3]
table {%
-0.043848450781368 0.192488329289336
0.287974042518776 0.149003002147366
};
\addplot [very thick, color3]
table {%
1.13929172816044 -1.07553705644164
0.80474946774176 -1.05858981481516
};
\addplot [very thick, color3]
table {%
1.03249205013434 -0.276218486658694
1.27651703162676 -0.507892465272379
};
\addplot [very thick, color3]
table {%
-0.124824503595767 1.17650662278542
-0.290049580989981 0.882474726828786
};
\addplot [very thick, color3]
table {%
0.287974042518776 0.149003002147366
-0.042750903342611 0.0797462803574467
};
\addplot [very thick, color3]
table {%
1.27651703162676 -0.507892465272379
1.18243868402131 -0.182835421102174
};
\addplot [very thick, color3]
table {%
-0.038308979791596 -0.906263941779864
0.225572130815729 -1.11820672366738
};
\addplot [very thick, color3]
table {%
0.318497415524185 -0.772246256656776
0.512282026122496 -1.05460220157653
};
\addplot [very thick, color3]
table {%
-0.151580682549982 -0.582783939415887
-0.058591464870352 -0.250495361764648
};
\addplot [very thick, color3]
table {%
-0.043848450781368 0.192488329289336
0.161305007531463 0.47151626644206
};
\addplot [very thick, color3]
table {%
0.161305007531463 0.47151626644206
0.287974042518776 0.149003002147366
};
\addplot [very thick, color3]
table {%
-0.897657764611887 -0.509805000910038
-1.05068784812419 -0.196660891853283
};
\addplot [very thick, color3]
table {%
-1.62753042956307 0.144763499946151
-1.85256401192557 -0.124092650272665
};
\addplot [very thick, color3]
table {%
-0.0457880275415337 0.755095216844757
0.161305007531463 0.47151626644206
};
\addplot [very thick, color3]
table {%
-1.62753042956307 0.144763499946151
-1.28516929949252 0.112243759124342
};
\addplot [very thick, color3]
table {%
-0.486464599849546 -0.321047644667546
-0.683949025620437 -0.59776859652354
};
\addplot [very thick, color3]
table {%
0.724867796410222 1.0119644513881
0.661942961282725 0.682717899396752
};
\addplot [very thick, color3]
table {%
1.30701775361222 0.471310001618085
1.56571290677136 0.226015644404297
};
\addplot [very thick, color3]
table {%
0.318497415524185 -0.772246256656776
0.225572130815729 -1.11820672366738
};
\addplot [very thick, color3]
table {%
-1.05068784812419 -0.196660891853283
-0.97525635438475 -0.546892506013557
};
\addplot [very thick, color3]
table {%
1.16650560208541 -1.36988664811057
0.811316786821193 -1.42384003157491
};
\addplot [very thick, color3]
table {%
1.13929172816044 -1.07553705644164
1.08385271681253 -0.717881298334075
};
\addplot [very thick, color3]
table {%
1.03249205013434 -0.276218486658694
1.18397678448062 -0.00794043507333255
};
\addplot [very thick, color3]
table {%
0.80474946774176 -1.05858981481516
0.811316786821193 -1.42384003157491
};
\addplot [very thick, color3]
table {%
0.318497415524185 -0.772246256656776
0.567908869640562 -0.574932508687353
};
\addplot [very thick, color3]
table {%
0.272266762130476 -0.356289816160233
0.112985819436421 -0.690906617864546
};
\addplot [very thick, color3]
table {%
0.272266762130476 -0.356289816160233
0.567908869640562 -0.574932508687353
};
\addplot [very thick, color3]
table {%
-0.486464599849546 -0.321047644667546
-0.835280645501797 -0.205178137113613
};
\addplot [very thick, color3]
table {%
-0.645878092492392 0.244918441252124
-0.707200164270895 -0.110920852624309
};
\addplot [very thick, color3]
table {%
-0.175822267271081 -0.0252651026286875
-0.440817927077724 -0.304467240653771
};
\addplot [very thick, color3]
table {%
-0.440817927077724 -0.304467240653771
-0.058591464870352 -0.250495361764648
};
\addplot [very thick, color3]
table {%
0.929214222549169 0.651745790327805
1.31068071341738 0.588970190069706
};
\addplot [very thick, color3]
table {%
0.648482472610915 -1.07166832936951
0.811316786821193 -1.42384003157491
};
\addplot [very thick, color3]
table {%
-0.291802622064795 0.0837691031769381
-0.320788003316259 0.471261823633868
};
\addplot [very thick, color3]
table {%
-0.645878092492392 0.244918441252124
-0.291802622064795 0.0837691031769381
};
\addplot [very thick, color3]
table {%
0.272266762130476 -0.356289816160233
-0.058591464870352 -0.250495361764648
};
\addplot [very thick, color3]
table {%
0.318497415524185 -0.772246256656776
-0.038308979791596 -0.906263941779864
};
\addplot [very thick, color3]
table {%
-0.043848450781368 0.192488329289336
-0.320788003316259 0.471261823633868
};
\addplot [very thick, color3]
table {%
-0.645878092492392 0.244918441252124
-0.320788003316259 0.471261823633868
};
\addplot [very thick, color3]
table {%
1.13929172816044 -1.07553705644164
1.17257671373237 -0.686628400040247
};
\addplot [very thick, color3]
table {%
-0.517513317388311 -0.760009533697624
-0.125549177171554 -0.678365273082073
};
\addplot [very thick, color3]
table {%
0.075771071420608 -0.192872784109992
0.287974042518776 0.149003002147366
};
\addplot [very thick, color3]
table {%
0.106611314555847 1.01602134591746
-0.0457880275415337 0.755095216844757
};
\addplot [very thick, color3]
table {%
-1.38019830256552 0.640505435159906
-1.51714816067073 0.258502056354496
};
\addplot [very thick, color3]
table {%
-0.517513317388311 -0.760009533697624
-0.151580682549982 -0.582783939415887
};
\addplot [very thick, color3]
table {%
-0.645878092492392 0.244918441252124
-1.02698880542662 0.0878139020675124
};
\addplot [very thick, color3]
table {%
1.18243868402131 -0.182835421102174
1.50701998276123 -0.439351332338839
};
\addplot [very thick, color3]
table {%
0.724867796410222 1.0119644513881
0.929214222549169 0.651745790327805
};
\addplot [very thick, color3]
table {%
-0.440817927077724 -0.304467240653771
-0.291802622064795 0.0837691031769381
};
\addplot [very thick, color3]
table {%
-0.107877676726452 -0.840289685265736
-0.517513317388311 -0.760009533697624
};
\addplot [very thick, color3]
table {%
-0.66403264904207 1.07732210290748
-0.290049580989981 0.882474726828786
};
\addplot [very thick, color3]
table {%
1.30701775361222 0.471310001618085
0.929214222549169 0.651745790327805
};
\addplot [very thick, color3]
table {%
-1.62753042956307 0.144763499946151
-1.37362659980222 -0.192873683671634
};
\addplot [very thick, color3]
table {%
-0.645878092492392 0.244918441252124
-0.98887418389988 -0.00418003306320407
};
\addplot [very thick, color3]
table {%
0.885609873179297 -0.34308175412289
1.08385271681253 -0.717881298334075
};
\addplot [very thick, color3]
table {%
-0.320788003316259 0.471261823633868
-0.290049580989981 0.882474726828786
};
\addplot [very thick, color3]
table {%
-0.835280645501797 -0.205178137113613
-0.683949025620437 -0.59776859652354
};
\addplot [very thick, color3]
table {%
-0.440817927077724 -0.304467240653771
-0.151580682549982 -0.582783939415887
};
\addplot [very thick, color3]
table {%
1.27651703162676 -0.507892465272379
1.08385271681253 -0.717881298334075
};
\addplot [very thick, color3]
table {%
0.80474946774176 -1.05858981481516
1.08385271681253 -0.717881298334075
};
\addplot [very thick, color3]
table {%
-0.66403264904207 1.07732210290748
-0.878227336337594 0.691452914917409
};
\addplot [very thick, color3]
table {%
-0.038308979791596 -0.906263941779864
-0.282308197205086 -1.27740706587447
};
\addplot [very thick, color3]
table {%
-0.0457880275415337 0.755095216844757
-0.320788003316259 0.471261823633868
};
\addplot [very thick, color3]
table {%
1.03249205013434 -0.276218486658694
1.08385271681253 -0.717881298334075
};
\addplot [very thick, color3]
table {%
1.18397678448062 -0.00794043507333255
1.56571290677136 0.226015644404297
};
\addplot [very thick, color3]
table {%
0.790570154320513 -0.374829753224335
1.08385271681253 -0.717881298334075
};
\addplot [very thick, color3]
table {%
0.633895330915757 0.988721140629871
0.435590639323754 1.40570053244992
};
\addplot [very thick, color3]
table {%
-0.66403264904207 1.07732210290748
-0.285361579892089 1.35847512096702
};
\addplot [very thick, color3]
table {%
0.112985819436421 -0.690906617864546
-0.058591464870352 -0.250495361764648
};
\addplot [very thick, color3]
table {%
-0.107877676726452 -0.840289685265736
-0.282308197205086 -1.27740706587447
};
\addplot [very thick, color3]
table {%
1.13929172816044 -1.07553705644164
0.811316786821193 -1.42384003157491
};
\addplot [very thick, color3]
table {%
0.161305007531463 0.47151626644206
-0.320788003316259 0.471261823633868
};
\addplot [very thick, color3]
table {%
0.491156506328976 1.50289694168447
0.413852404170514 1.97943264304943
};
\addplot [very thick, color3]
table {%
1.45446356041819 1.05019519757262
1.31068071341738 0.588970190069706
};
\addplot [very thick, color3]
table {%
1.01272382397982 -1.86298022827367
0.811316786821193 -1.42384003157491
};
\addplot [very thick, color3]
table {%
-1.85256401192557 -0.124092650272665
-1.37362659980222 -0.192873683671634
};
\addplot [very thick, color3]
table {%
0.512282026122496 -1.05460220157653
0.567908869640562 -0.574932508687353
};
\addplot [very thick, color3]
table {%
0.161305007531463 0.47151626644206
0.57264119973551 0.222410149332283
};
\addplot [very thick, color3]
table {%
1.50701998276123 -0.439351332338839
1.98268177645895 -0.319409301778917
};
\addplot [very thick, color3]
table {%
1.65057712884217 0.943486007801425
1.31068071341738 0.588970190069706
};
\addplot [very thick, color3]
table {%
1.30701775361222 0.471310001618085
1.18397678448062 -0.00794043507333255
};
\addplot [very thick, color3]
table {%
0.24722747298104 -1.61319873633128
0.225572130815729 -1.11820672366738
};
\addplot [very thick, color3]
table {%
0.724867796410222 1.0119644513881
0.435590639323754 1.40570053244992
};
\addplot [very thick, color3]
table {%
0.520329697755992 -1.55654582504081
0.648482472610915 -1.07166832936951
};
\addplot [very thick, color3]
table {%
0.520329697755992 -1.55654582504081
0.512282026122496 -1.05460220157653
};
\addplot [very thick, color3]
table {%
0.648482472610915 -1.07166832936951
0.567908869640562 -0.574932508687353
};
\addplot [very thick, color3]
table {%
-0.645878092492392 0.244918441252124
-0.878227336337594 0.691452914917409
};
\addplot [very thick, color3]
table {%
-0.285361579892089 1.35847512096702
-0.290049580989981 0.882474726828786
};
\addplot [very thick, color3]
table {%
0.106611314555847 1.01602134591746
0.435590639323754 1.40570053244992
};
\addplot [very thick, color3]
table {%
0.272266762130476 -0.356289816160233
0.287974042518776 0.149003002147366
};
\addplot [very thick, color3]
table {%
0.106611314555847 1.01602134591746
0.575537860345507 0.803161640917396
};
\addplot [very thick, color3]
table {%
1.01272382397982 -1.86298022827367
1.16650560208541 -1.36988664811057
};
\addplot [very thick, color3]
table {%
-1.3126448374852 1.17645213621684
-1.27749128471686 0.65534756920239
};
\addplot [very thick, color3]
table {%
-1.51714816067073 0.258502056354496
-1.27749128471686 0.65534756920239
};
\addplot [very thick, color3]
table {%
0.106611314555847 1.01602134591746
0.633895330915757 0.988721140629871
};
\addplot [very thick, color3]
table {%
0.520329697755992 -1.55654582504081
0.225572130815729 -1.11820672366738
};
\addplot [very thick, color3]
table {%
0.272266762130476 -0.356289816160233
0.790570154320513 -0.374829753224335
};
\addplot [very thick, color3]
table {%
0.161305007531463 0.47151626644206
0.575537860345507 0.803161640917396
};
\addplot [very thick, color3]
table {%
1.08243752843236 1.42960529549044
1.45446356041819 1.05019519757262
};
\addplot [very thick, color3]
table {%
0.225572130815729 -1.11820672366738
-0.282308197205086 -1.27740706587447
};
\addplot [very thick, color3]
table {%
0.567908869640562 -0.574932508687353
1.08385271681253 -0.717881298334075
};
\addplot [very thick, color3]
table {%
-0.645878092492392 0.244918441252124
-1.00748672473045 0.638365038742785
};
\addplot [very thick, color3]
table {%
0.80474946774176 -1.05858981481516
0.567908869640562 -0.574932508687353
};
\addplot [very thick, color3]
table {%
-1.38019830256552 0.640505435159906
-1.3126448374852 1.17645213621684
};
\addplot [very thick, color3]
table {%
-1.27749128471686 0.65534756920239
-1.28516929949252 0.112243759124342
};
\addplot [very thick, color3]
table {%
0.661942961282725 0.682717899396752
0.161305007531463 0.47151626644206
};
\addplot [very thick, color3]
table {%
-1.37362659980222 -0.192873683671634
-0.97525635438475 -0.546892506013557
};
\addplot [very thick, color3]
table {%
1.08243752843236 1.42960529549044
0.724867796410222 1.0119644513881
};
\addplot [very thick, color3]
table {%
-1.00748672473045 0.638365038742785
-1.02698880542662 0.0878139020675124
};
\addplot [very thick, color3]
table {%
0.713664525872475 0.45089573918883
0.161305007531463 0.47151626644206
};
\addplot [very thick, color3]
table {%
-0.517513317388311 -0.760009533697624
-0.282308197205086 -1.27740706587447
};
\addplot [very thick, color3]
table {%
-0.66403264904207 1.07732210290748
-0.804429884793085 1.63103253934514
};
\addplot [very thick, color3]
table {%
-2.34275812782514 -0.422337072515164
-1.85256401192557 -0.124092650272665
};
\addplot [very thick, color3]
table {%
0.435590639323754 1.40570053244992
-0.0392455500475513 1.09768115355305
};
\addplot [very thick, color3]
table {%
1.01272382397982 -1.86298022827367
0.520329697755992 -1.55654582504081
};
\addplot [only marks, mark=*, draw=color0, fill=color0, colormap/viridis]
table{%
x                      y
-2.34275812782514 -0.422337072515164
0.24722747298104 -1.61319873633128
-1.38019830256552 0.640505435159906
-1.51714816067073 0.258502056354496
-0.175822267271081 -0.0252651026286875
-1.07790812289209 -1.40003857215438
1.08243752843236 1.42960529549044
1.45446356041819 1.05019519757262
0.272266762130476 -0.356289816160233
0.885609873179297 -0.34308175412289
0.318497415524185 -0.772246256656776
-0.486464599849546 -0.321047644667546
0.713664525872475 0.45089573918883
-3.15883343059119 1.79910626932364
1.03249205013434 -0.276218486658694
0.724867796410222 1.0119644513881
-0.66403264904207 1.07732210290748
-0.107877676726452 -0.840289685265736
1.01272382397982 -1.86298022827367
1.13929172816044 -1.07553705644164
-1.3126448374852 1.17645213621684
1.16650560208541 -1.36988664811057
-0.200901763190504 0.83207623547168
-0.162766483689669 0.758482159427621
-0.835280645501797 -0.205178137113613
-0.645878092492392 0.244918441252124
0.80474946774176 -1.05858981481516
0.520329697755992 -1.55654582504081
0.648482472610915 -1.07166832936951
0.512282026122496 -1.05460220157653
-1.62753042956307 0.144763499946151
-1.85256401192557 -0.124092650272665
-0.038308979791596 -0.906263941779864
2.70063983199922 -0.145679091954431
0.790570154320513 -0.374829753224335
-0.517513317388311 -0.760009533697624
0.567908869640562 -0.574932508687353
0.661942961282725 0.682717899396752
-0.897657764611887 -0.509805000910038
-0.125549177171554 -0.678365273082073
-0.878227336337594 0.691452914917409
-0.683949025620437 -0.59776859652354
-0.848312798273908 -0.798926002882831
0.112985819436421 -0.690906617864546
0.225572130815729 -1.11820672366738
1.30701775361222 0.471310001618085
1.17257671373237 -0.686628400040247
1.65057712884217 0.943486007801425
1.27651703162676 -0.507892465272379
1.18397678448062 -0.00794043507333255
0.491156506328976 1.50289694168447
-1.03569391216883 -0.0587326120757943
0.075771071420608 -0.192872784109992
-0.282308197205086 -1.27740706587447
-1.00748672473045 0.638365038742785
-1.27749128471686 0.65534756920239
-0.124824503595767 1.17650662278542
0.106611314555847 1.01602134591746
-1.13247270174148 -0.0219117447622604
-0.043848450781368 0.192488329289336
-0.0438948279215397 0.885200349964678
-0.0457880275415337 0.755095216844757
-1.28516929949252 0.112243759124342
-0.440817927077724 -0.304467240653771
0.161305007531463 0.47151626644206
-0.151580682549982 -0.582783939415887
-0.447730695225613 -0.452165071633186
0.372995023166489 -0.615248964656559
-0.707200164270895 -0.110920852624309
0.57264119973551 0.222410149332283
-0.291802622064795 0.0837691031769381
-2.03628199559491 -1.99508969153983
-2.11509949221634 -1.20644336802718
0.633895330915757 0.988721140629871
1.18243868402131 -0.182835421102174
1.56571290677136 0.226015644404297
1.50701998276123 -0.439351332338839
0.413852404170514 1.97943264304943
-0.285361579892089 1.35847512096702
-0.804429884793085 1.63103253934514
-0.058591464870352 -0.250495361764648
0.435590639323754 1.40570053244992
-0.320788003316259 0.471261823633868
0.287974042518776 0.149003002147366
0.811316786821193 -1.42384003157491
-0.96405415610173 -1.6757287701837
-1.02698880542662 0.0878139020675124
-1.05068784812419 -0.196660891853283
-1.37362659980222 -0.192873683671634
-0.0392455500475513 1.09768115355305
0.575537860345507 0.803161640917396
0.929214222549169 0.651745790327805
-0.97525635438475 -0.546892506013557
-0.042750903342611 0.0797462803574467
-0.290049580989981 0.882474726828786
1.08385271681253 -0.717881298334075
1.98268177645895 -0.319409301778917
1.31068071341738 0.588970190069706
-0.98887418389988 -0.00418003306320407
-0.571790491417394 -0.0307990540877851
};
\end{axis}

\end{tikzpicture}

\subcaption{Intermediate step of the filtration of the largest possible Alpha complex on $S$.}
\end{center}
\end{subfigure}
\caption{Sampled multivariate distribution.}
\label{fig:multivariate}
\end{figure}

\subsection{Gaussian Mixture}
Finally we will consider point clouds which are drawn from a mixture of $m$ Gaussian distributions. We do this by repeating the following steps: \begin{itemize}
    \item Choose a value $i= 1,\dots,m$ uniformly.
    \item Sample a point from $\mathcal{N}_i(\mu_i,\Sigma_i)$.
\end{itemize}
Figure \ref{fig:mixture} gives an example on 100 points and the median simplicial complex generated in the same manner as before.

\begin{figure}[H]
%\centering%
\begin{subfigure}[t]{0.49\textwidth}
\begin{center}

% This file was created by tikzplotlib v0.9.2.
\begin{tikzpicture}[thick,scale=0.8, every node/.style={transform shape}]

\definecolor{color0}{rgb}{0.12156862745098,0.466666666666667,0.705882352941177}

\begin{axis}[
tick align=outside,
tick pos=left,
x grid style={white!69.0196078431373!black},
%xmin=-6.05675528744101, xmax=5.27787573032565,
xmin=-6.05675528744101, xmax=6.80298415266092,
xtick style={color=black},
y grid style={white!69.0196078431373!black},
ymin=-4.05675528744101, ymax=8.80298415266092,
%ymin=-4.06578947588708, ymax=8.80298415266092,
ytick style={color=black}
]
\addplot [only marks, mark=*, draw=color0, fill=color0, colormap/viridis]
table{%
x                      y
-0.358111612392031 -1.03573022195866
2.54059011512305 5.40732724426877
2.64439303639391 5.33779192925331
2.37993615478195 5.08341926202954
-0.108138330543506 -0.272147725049101
-1.66612171369168 -1.16491671279916
1.82276797293495 1.66810423984943
-0.386954794949196 -0.887278921762747
0.297086897817065 0.995913395714418
4.20230143122709 5.04114381567126
4.18098479411033 5.20817168133169
-1.15979760167936 2.48579337246029
-1.21072940472115 2.32564645539021
-1.92310074281666 2.849737736476
-0.794077022732528 -3.09763838394167
-0.879810558070651 -3.48084522004399
0.897589460877768 1.97038397092996
-2.41240086456128 4.0225376563209
-2.14787238675959 4.06031111140356
0.185474121138129 0.813411089586105
0.208319009940496 0.379643070293184
-0.108331721452844 -0.315740368822207
0.915515455672217 -1.56223640998641
0.361160707852192 -1.12526391609712
-2.99962181068817 1.36366577935927
-1.86926540057446 4.31937963197559
1.97167127405905 5.37846226762404
3.38829769446094 6.04358633691308
3.17027424070748 6.50474613140325
2.86412145575591 2.58741981824194
3.57926152941062 6.35979132901094
3.91118505308693 6.02140162936865
0.734555168589413 0.618002625061071
-0.888552360386418 0.837344604908327
-2.93513629393346 2.25270029210883
-2.45909955124887 2.75444952657863
0.982944772711592 5.92494310211316
1.0722787109508 -0.143047009227478
3.76177050379184 5.73921743170339
3.30899718046626 5.68990456424791
3.15857161895561 5.66514809186076
3.36185274192307 5.77406588531824
3.85072378702374 5.2428792846038
4.11118344993143 6.74050825521424
4.63615766530576 6.47510482513503
-0.665776111307205 2.16134751297774
-0.317679888699185 -0.0629296458187575
1.19147659111797 3.25759452658187
-2.45255966188733 -0.649306539777947
-2.38827323869676 5.1019760098924
-3.80990947334099 3.54431506206447
-3.6945185892455 3.48456470113167
-3.09213735335708 3.76910734978884
-1.58108210183386 0.576205666538593
-0.0305854665203686 -0.424590954248704
-0.336311379908108 -0.997344391227889
1.35475062047888 4.49965906131902
1.8835208886715 4.40757117309586
2.66535600163303 3.77813231660804
2.78862863818822 3.11143766434187
2.85133778920804 3.36411834401323
-5.54154478663344 3.85984931952986
0.944278596087085 1.13182749047688
-0.874971110150818 1.52230867430528
-2.52364788757035 3.29543119127435
-2.21841271692549 3.26727594273737
2.03361333854756 -1.01225876863371
-2.89772819748342 3.67099747186657
-2.22758194752147 3.25979487107238
-2.15550016844155 3.42338679220924
3.44185983653917 4.45516210545079
2.51524615285251 4.37419916143041
2.17963557993397 4.78564637341144
2.81220545438285 4.70120393417241
2.10859389528023 4.18592294168646
2.81458938319616 4.32045257984774
3.29006165526776 4.6251028305993
0.218482926894079 -0.149901020185651
2.31841867483311 4.1696634344682
3.51218046217622 4.65210791346245
4.17446946920669 7.5111622752538
-0.430420348837371 3.45705996633102
3.98154238175133 3.96023539287441
-4.00531594259129 3.25737670503793
-4.69073919765948 3.38159404040119
-3.79987123609756 2.82295825942001
-3.53276051991429 2.84153490572953
-3.82369011056774 2.05838460399524
-3.40165671980728 2.17191730886362
-3.46174416837956 2.39161003248653
-1.81022369365007 3.81555936074182
-0.679764120459465 -0.736996202880417
-0.682644108116294 0.76958992767765
0.694353516552503 -0.704212714310419
1.27576879788079 -0.474284940941255
-1.8072024193766 -0.193590780546145
-1.7758920309332 6.10254964560158
-0.0361543454845905 0.819222700941846
3.6731850363502 8.21803989681783
4.76266522951807 3.69025802127189
};
\end{axis}

\end{tikzpicture}

\subcaption{Point cloud $S$.}
\end{center}
\end{subfigure}
\begin{subfigure}[t]{0.49\textwidth}
\begin{center}
% This file was created by tikzplotlib v0.9.2.
\begin{tikzpicture}[thick,scale=0.8, every node/.style={transform shape}]

\definecolor{color0}{rgb}{0.12156862745098,0.466666666666667,0.705882352941177}
\definecolor{color1}{rgb}{1,0.498039215686275,0.0549019607843137}
\definecolor{color2}{rgb}{0.96078431372549,0.96078431372549,0.862745098039216}
\definecolor{color3}{rgb}{1,0.498039215686275,0.313725490196078}

\begin{axis}[
tick align=outside,
tick pos=left,
x grid style={white!69.0196078431373!black},
%xmin=-6.05675528744101, xmax=5.27787573032565,
xmin=-6.05675528744101, xmax=6.80298415266092,
xtick style={color=black},
y grid style={white!69.0196078431373!black},
ymin=-4.05675528744101, ymax=8.80298415266092,
%ymin=-4.06578947588708, ymax=8.80298415266092,
ytick style={color=black}
]
\addplot [only marks, mark=*, draw=color0, fill=color0, colormap/viridis]
table{%
x                      y
-0.358111612392031 -1.03573022195866
2.54059011512305 5.40732724426877
2.64439303639391 5.33779192925331
2.37993615478195 5.08341926202954
-0.108138330543506 -0.272147725049101
-1.66612171369168 -1.16491671279916
1.82276797293495 1.66810423984943
-0.386954794949196 -0.887278921762747
0.297086897817065 0.995913395714418
4.20230143122709 5.04114381567126
4.18098479411033 5.20817168133169
-1.15979760167936 2.48579337246029
-1.21072940472115 2.32564645539021
-1.92310074281666 2.849737736476
-0.794077022732528 -3.09763838394167
-0.879810558070651 -3.48084522004399
0.897589460877768 1.97038397092996
-2.41240086456128 4.0225376563209
-2.14787238675959 4.06031111140356
0.185474121138129 0.813411089586105
0.208319009940496 0.379643070293184
-0.108331721452844 -0.315740368822207
0.915515455672217 -1.56223640998641
0.361160707852192 -1.12526391609712
-2.99962181068817 1.36366577935927
-1.86926540057446 4.31937963197559
1.97167127405905 5.37846226762404
3.38829769446094 6.04358633691308
3.17027424070748 6.50474613140325
2.86412145575591 2.58741981824194
3.57926152941062 6.35979132901094
3.91118505308693 6.02140162936865
0.734555168589413 0.618002625061071
-0.888552360386418 0.837344604908327
-2.93513629393346 2.25270029210883
-2.45909955124887 2.75444952657863
0.982944772711592 5.92494310211316
1.0722787109508 -0.143047009227478
3.76177050379184 5.73921743170339
3.30899718046626 5.68990456424791
3.15857161895561 5.66514809186076
3.36185274192307 5.77406588531824
3.85072378702374 5.2428792846038
4.11118344993143 6.74050825521424
4.63615766530576 6.47510482513503
-0.665776111307205 2.16134751297774
-0.317679888699185 -0.0629296458187575
1.19147659111797 3.25759452658187
-2.45255966188733 -0.649306539777947
-2.38827323869676 5.1019760098924
-3.80990947334099 3.54431506206447
-3.6945185892455 3.48456470113167
-3.09213735335708 3.76910734978884
-1.58108210183386 0.576205666538593
-0.0305854665203686 -0.424590954248704
-0.336311379908108 -0.997344391227889
1.35475062047888 4.49965906131902
1.8835208886715 4.40757117309586
2.66535600163303 3.77813231660804
2.78862863818822 3.11143766434187
2.85133778920804 3.36411834401323
-5.54154478663344 3.85984931952986
0.944278596087085 1.13182749047688
-0.874971110150818 1.52230867430528
-2.52364788757035 3.29543119127435
-2.21841271692549 3.26727594273737
2.03361333854756 -1.01225876863371
-2.89772819748342 3.67099747186657
-2.22758194752147 3.25979487107238
-2.15550016844155 3.42338679220924
3.44185983653917 4.45516210545079
2.51524615285251 4.37419916143041
2.17963557993397 4.78564637341144
2.81220545438285 4.70120393417241
2.10859389528023 4.18592294168646
2.81458938319616 4.32045257984774
3.29006165526776 4.6251028305993
0.218482926894079 -0.149901020185651
2.31841867483311 4.1696634344682
3.51218046217622 4.65210791346245
4.17446946920669 7.5111622752538
-0.430420348837371 3.45705996633102
3.98154238175133 3.96023539287441
-4.00531594259129 3.25737670503793
-4.69073919765948 3.38159404040119
-3.79987123609756 2.82295825942001
-3.53276051991429 2.84153490572953
-3.82369011056774 2.05838460399524
-3.40165671980728 2.17191730886362
-3.46174416837956 2.39161003248653
-1.81022369365007 3.81555936074182
-0.679764120459465 -0.736996202880417
-0.682644108116294 0.76958992767765
0.694353516552503 -0.704212714310419
1.27576879788079 -0.474284940941255
-1.8072024193766 -0.193590780546145
-1.7758920309332 6.10254964560158
-0.0361543454845905 0.819222700941846
3.6731850363502 8.21803989681783
4.76266522951807 3.69025802127189
};
\path [draw=color2, fill=color2]
(axis cs:-0.358111612392031,-1.03573022195866)
--(axis cs:-0.386954794949196,-0.887278921762747)
--(axis cs:-0.336311379908108,-0.997344391227889)
--cycle;
\path [draw=color2, fill=color2]
(axis cs:3.44185983653917,4.45516210545079)
--(axis cs:3.29006165526776,4.6251028305993)
--(axis cs:3.51218046217622,4.65210791346245)
--cycle;
\path [draw=color2, fill=color2]
(axis cs:-0.108138330543506,-0.272147725049101)
--(axis cs:-0.108331721452844,-0.315740368822207)
--(axis cs:-0.0305854665203686,-0.424590954248704)
--cycle;
\path [draw=color2, fill=color2]
(axis cs:3.30899718046626,5.68990456424791)
--(axis cs:3.15857161895561,5.66514809186076)
--(axis cs:3.36185274192307,5.77406588531824)
--cycle;
\path [draw=color2, fill=color2]
(axis cs:2.54059011512305,5.40732724426877)
--(axis cs:2.64439303639391,5.33779192925331)
--(axis cs:2.37993615478195,5.08341926202954)
--cycle;
\path [draw=color2, fill=color2]
(axis cs:-2.21841271692549,3.26727594273737)
--(axis cs:-2.22758194752147,3.25979487107238)
--(axis cs:-2.15550016844155,3.42338679220924)
--cycle;
\path [draw=color2, fill=color2]
(axis cs:-0.108138330543506,-0.272147725049101)
--(axis cs:-0.0305854665203686,-0.424590954248704)
--(axis cs:0.218482926894079,-0.149901020185651)
--cycle;
\path [draw=color2, fill=color2]
(axis cs:-3.80990947334099,3.54431506206447)
--(axis cs:-3.6945185892455,3.48456470113167)
--(axis cs:-4.00531594259129,3.25737670503793)
--cycle;
\path [draw=color2, fill=color2]
(axis cs:-2.52364788757035,3.29543119127435)
--(axis cs:-2.22758194752147,3.25979487107238)
--(axis cs:-2.15550016844155,3.42338679220924)
--cycle;
\path [draw=color2, fill=color2]
(axis cs:4.20230143122709,5.04114381567126)
--(axis cs:4.18098479411033,5.20817168133169)
--(axis cs:3.85072378702374,5.2428792846038)
--cycle;
\path [draw=color2, fill=color2]
(axis cs:0.297086897817065,0.995913395714418)
--(axis cs:0.185474121138129,0.813411089586105)
--(axis cs:-0.0361543454845905,0.819222700941846)
--cycle;
\path [draw=color2, fill=color2]
(axis cs:2.51524615285251,4.37419916143041)
--(axis cs:2.81220545438285,4.70120393417241)
--(axis cs:2.81458938319616,4.32045257984774)
--cycle;
\path [draw=color2, fill=color2]
(axis cs:-0.108138330543506,-0.272147725049101)
--(axis cs:-0.108331721452844,-0.315740368822207)
--(axis cs:-0.317679888699185,-0.0629296458187575)
--cycle;
\path [draw=color2, fill=color2]
(axis cs:3.38829769446094,6.04358633691308)
--(axis cs:3.76177050379184,5.73921743170339)
--(axis cs:3.36185274192307,5.77406588531824)
--cycle;
\path [draw=color2, fill=color2]
(axis cs:-3.82369011056774,2.05838460399524)
--(axis cs:-3.40165671980728,2.17191730886362)
--(axis cs:-3.46174416837956,2.39161003248653)
--cycle;
\path [draw=color2, fill=color2]
(axis cs:0.185474121138129,0.813411089586105)
--(axis cs:0.208319009940496,0.379643070293184)
--(axis cs:-0.0361543454845905,0.819222700941846)
--cycle;
\path [draw=color2, fill=color2]
(axis cs:3.76177050379184,5.73921743170339)
--(axis cs:3.30899718046626,5.68990456424791)
--(axis cs:3.36185274192307,5.77406588531824)
--cycle;
\path [draw=color2, fill=color2]
(axis cs:-1.92310074281666,2.849737736476)
--(axis cs:-2.21841271692549,3.26727594273737)
--(axis cs:-2.22758194752147,3.25979487107238)
--cycle;
\path [draw=color2, fill=color2]
(axis cs:-2.14787238675959,4.06031111140356)
--(axis cs:-1.86926540057446,4.31937963197559)
--(axis cs:-1.81022369365007,3.81555936074182)
--cycle;
\path [draw=color2, fill=color2]
(axis cs:3.38829769446094,6.04358633691308)
--(axis cs:3.17027424070748,6.50474613140325)
--(axis cs:3.57926152941062,6.35979132901094)
--cycle;
\path [draw=color2, fill=color2]
(axis cs:3.38829769446094,6.04358633691308)
--(axis cs:3.15857161895561,5.66514809186076)
--(axis cs:3.36185274192307,5.77406588531824)
--cycle;
\path [draw=color2, fill=color2]
(axis cs:3.38829769446094,6.04358633691308)
--(axis cs:3.91118505308693,6.02140162936865)
--(axis cs:3.76177050379184,5.73921743170339)
--cycle;
\path [draw=color2, fill=color2]
(axis cs:3.38829769446094,6.04358633691308)
--(axis cs:3.57926152941062,6.35979132901094)
--(axis cs:3.91118505308693,6.02140162936865)
--cycle;
\path [draw=color2, fill=color2]
(axis cs:-2.93513629393346,2.25270029210883)
--(axis cs:-3.40165671980728,2.17191730886362)
--(axis cs:-3.46174416837956,2.39161003248653)
--cycle;
\path [draw=color2, fill=color2]
(axis cs:-3.79987123609756,2.82295825942001)
--(axis cs:-3.53276051991429,2.84153490572953)
--(axis cs:-3.46174416837956,2.39161003248653)
--cycle;
\path [draw=color2, fill=color2]
(axis cs:-0.358111612392031,-1.03573022195866)
--(axis cs:-0.386954794949196,-0.887278921762747)
--(axis cs:-0.679764120459465,-0.736996202880417)
--cycle;
\path [draw=color2, fill=color2]
(axis cs:-2.45909955124887,2.75444952657863)
--(axis cs:-2.52364788757035,3.29543119127435)
--(axis cs:-2.22758194752147,3.25979487107238)
--cycle;
\path [draw=color2, fill=color2]
(axis cs:2.81220545438285,4.70120393417241)
--(axis cs:2.81458938319616,4.32045257984774)
--(axis cs:3.29006165526776,4.6251028305993)
--cycle;
\path [draw=color2, fill=color2]
(axis cs:2.54059011512305,5.40732724426877)
--(axis cs:2.37993615478195,5.08341926202954)
--(axis cs:1.97167127405905,5.37846226762404)
--cycle;
\path [draw=color2, fill=color2]
(axis cs:2.51524615285251,4.37419916143041)
--(axis cs:2.10859389528023,4.18592294168646)
--(axis cs:2.31841867483311,4.1696634344682)
--cycle;
\path [draw=color2, fill=color2]
(axis cs:0.297086897817065,0.995913395714418)
--(axis cs:0.185474121138129,0.813411089586105)
--(axis cs:0.734555168589413,0.618002625061071)
--cycle;
\path [draw=color2, fill=color2]
(axis cs:-1.15979760167936,2.48579337246029)
--(axis cs:-1.21072940472115,2.32564645539021)
--(axis cs:-0.665776111307205,2.16134751297774)
--cycle;
\path [draw=color2, fill=color2]
(axis cs:-0.108138330543506,-0.272147725049101)
--(axis cs:-0.317679888699185,-0.0629296458187575)
--(axis cs:0.218482926894079,-0.149901020185651)
--cycle;
\path [draw=color2, fill=color2]
(axis cs:1.8835208886715,4.40757117309586)
--(axis cs:2.17963557993397,4.78564637341144)
--(axis cs:2.10859389528023,4.18592294168646)
--cycle;
\path [draw=color2, fill=color2]
(axis cs:2.66535600163303,3.77813231660804)
--(axis cs:2.51524615285251,4.37419916143041)
--(axis cs:2.31841867483311,4.1696634344682)
--cycle;
\path [draw=color2, fill=color2]
(axis cs:2.66535600163303,3.77813231660804)
--(axis cs:2.51524615285251,4.37419916143041)
--(axis cs:2.81458938319616,4.32045257984774)
--cycle;
\path [draw=color2, fill=color2]
(axis cs:-1.92310074281666,2.849737736476)
--(axis cs:-2.45909955124887,2.75444952657863)
--(axis cs:-2.22758194752147,3.25979487107238)
--cycle;
\path [draw=color2, fill=color2]
(axis cs:2.51524615285251,4.37419916143041)
--(axis cs:2.17963557993397,4.78564637341144)
--(axis cs:2.10859389528023,4.18592294168646)
--cycle;
\path [draw=color2, fill=color2]
(axis cs:0.185474121138129,0.813411089586105)
--(axis cs:0.208319009940496,0.379643070293184)
--(axis cs:0.734555168589413,0.618002625061071)
--cycle;
\path [draw=color2, fill=color2]
(axis cs:2.37993615478195,5.08341926202954)
--(axis cs:1.97167127405905,5.37846226762404)
--(axis cs:2.17963557993397,4.78564637341144)
--cycle;
\path [draw=color2, fill=color2]
(axis cs:-2.14787238675959,4.06031111140356)
--(axis cs:-2.15550016844155,3.42338679220924)
--(axis cs:-1.81022369365007,3.81555936074182)
--cycle;
\path [draw=color2, fill=color2]
(axis cs:2.37993615478195,5.08341926202954)
--(axis cs:2.17963557993397,4.78564637341144)
--(axis cs:2.81220545438285,4.70120393417241)
--cycle;
\path [draw=color2, fill=color2]
(axis cs:2.51524615285251,4.37419916143041)
--(axis cs:2.17963557993397,4.78564637341144)
--(axis cs:2.81220545438285,4.70120393417241)
--cycle;
\path [draw=color2, fill=color2]
(axis cs:3.44185983653917,4.45516210545079)
--(axis cs:2.81458938319616,4.32045257984774)
--(axis cs:3.29006165526776,4.6251028305993)
--cycle;
\path [draw=color2, fill=color2]
(axis cs:-2.41240086456128,4.0225376563209)
--(axis cs:-2.14787238675959,4.06031111140356)
--(axis cs:-2.15550016844155,3.42338679220924)
--cycle;
\path [draw=color2, fill=color2]
(axis cs:2.64439303639391,5.33779192925331)
--(axis cs:2.37993615478195,5.08341926202954)
--(axis cs:2.81220545438285,4.70120393417241)
--cycle;
\path [draw=color2, fill=color2]
(axis cs:-0.386954794949196,-0.887278921762747)
--(axis cs:-0.108331721452844,-0.315740368822207)
--(axis cs:-0.0305854665203686,-0.424590954248704)
--cycle;
\path [draw=color2, fill=color2]
(axis cs:-4.00531594259129,3.25737670503793)
--(axis cs:-3.79987123609756,2.82295825942001)
--(axis cs:-3.53276051991429,2.84153490572953)
--cycle;
\path [draw=color2, fill=color2]
(axis cs:-3.6945185892455,3.48456470113167)
--(axis cs:-4.00531594259129,3.25737670503793)
--(axis cs:-3.53276051991429,2.84153490572953)
--cycle;
\path [draw=color2, fill=color2]
(axis cs:1.0722787109508,-0.143047009227478)
--(axis cs:0.694353516552503,-0.704212714310419)
--(axis cs:1.27576879788079,-0.474284940941255)
--cycle;
\path [draw=color2, fill=color2]
(axis cs:0.297086897817065,0.995913395714418)
--(axis cs:0.734555168589413,0.618002625061071)
--(axis cs:0.944278596087085,1.13182749047688)
--cycle;
\path [draw=color2, fill=color2]
(axis cs:0.208319009940496,0.379643070293184)
--(axis cs:-0.317679888699185,-0.0629296458187575)
--(axis cs:0.218482926894079,-0.149901020185651)
--cycle;
\path [draw=color2, fill=color2]
(axis cs:3.76177050379184,5.73921743170339)
--(axis cs:3.30899718046626,5.68990456424791)
--(axis cs:3.85072378702374,5.2428792846038)
--cycle;
\path [draw=color2, fill=color2]
(axis cs:4.18098479411033,5.20817168133169)
--(axis cs:3.76177050379184,5.73921743170339)
--(axis cs:3.85072378702374,5.2428792846038)
--cycle;
\path [draw=color2, fill=color2]
(axis cs:-0.386954794949196,-0.887278921762747)
--(axis cs:-0.108331721452844,-0.315740368822207)
--(axis cs:-0.679764120459465,-0.736996202880417)
--cycle;
\path [draw=color2, fill=color2]
(axis cs:2.54059011512305,5.40732724426877)
--(axis cs:2.64439303639391,5.33779192925331)
--(axis cs:3.15857161895561,5.66514809186076)
--cycle;
\path [draw=color2, fill=color2]
(axis cs:-0.386954794949196,-0.887278921762747)
--(axis cs:-0.0305854665203686,-0.424590954248704)
--(axis cs:-0.336311379908108,-0.997344391227889)
--cycle;
\path [draw=color2, fill=color2]
(axis cs:-1.92310074281666,2.849737736476)
--(axis cs:-2.21841271692549,3.26727594273737)
--(axis cs:-2.15550016844155,3.42338679220924)
--cycle;
\path [draw=color2, fill=color2]
(axis cs:-2.41240086456128,4.0225376563209)
--(axis cs:-2.52364788757035,3.29543119127435)
--(axis cs:-2.15550016844155,3.42338679220924)
--cycle;
\path [draw=color2, fill=color2]
(axis cs:-2.41240086456128,4.0225376563209)
--(axis cs:-2.52364788757035,3.29543119127435)
--(axis cs:-2.89772819748342,3.67099747186657)
--cycle;
\path [draw=color2, fill=color2]
(axis cs:3.57926152941062,6.35979132901094)
--(axis cs:3.91118505308693,6.02140162936865)
--(axis cs:4.11118344993143,6.74050825521424)
--cycle;
\path [draw=color2, fill=color2]
(axis cs:-0.108331721452844,-0.315740368822207)
--(axis cs:-0.317679888699185,-0.0629296458187575)
--(axis cs:-0.679764120459465,-0.736996202880417)
--cycle;
\path [draw=color2, fill=color2]
(axis cs:-3.79987123609756,2.82295825942001)
--(axis cs:-3.82369011056774,2.05838460399524)
--(axis cs:-3.46174416837956,2.39161003248653)
--cycle;
\path [draw=color2, fill=color2]
(axis cs:-0.358111612392031,-1.03573022195866)
--(axis cs:0.361160707852192,-1.12526391609712)
--(axis cs:-0.336311379908108,-0.997344391227889)
--cycle;
\path [draw=color2, fill=color2]
(axis cs:-0.0305854665203686,-0.424590954248704)
--(axis cs:0.218482926894079,-0.149901020185651)
--(axis cs:0.694353516552503,-0.704212714310419)
--cycle;
\path [draw=color2, fill=color2]
(axis cs:4.20230143122709,5.04114381567126)
--(axis cs:3.85072378702374,5.2428792846038)
--(axis cs:3.51218046217622,4.65210791346245)
--cycle;
\path [draw=color2, fill=color2]
(axis cs:-0.888552360386418,0.837344604908327)
--(axis cs:-0.874971110150818,1.52230867430528)
--(axis cs:-0.682644108116294,0.76958992767765)
--cycle;
\path [draw=color2, fill=color2]
(axis cs:-2.41240086456128,4.0225376563209)
--(axis cs:-3.09213735335708,3.76910734978884)
--(axis cs:-2.89772819748342,3.67099747186657)
--cycle;
\path [draw=color2, fill=color2]
(axis cs:0.361160707852192,-1.12526391609712)
--(axis cs:-0.0305854665203686,-0.424590954248704)
--(axis cs:0.694353516552503,-0.704212714310419)
--cycle;
\path [draw=color2, fill=color2]
(axis cs:0.361160707852192,-1.12526391609712)
--(axis cs:-0.0305854665203686,-0.424590954248704)
--(axis cs:-0.336311379908108,-0.997344391227889)
--cycle;
\path [draw=color2, fill=color2]
(axis cs:-1.21072940472115,2.32564645539021)
--(axis cs:-0.665776111307205,2.16134751297774)
--(axis cs:-0.874971110150818,1.52230867430528)
--cycle;
\path [draw=color2, fill=color2]
(axis cs:3.91118505308693,6.02140162936865)
--(axis cs:4.11118344993143,6.74050825521424)
--(axis cs:4.63615766530576,6.47510482513503)
--cycle;
\path [draw=color2, fill=color2]
(axis cs:1.0722787109508,-0.143047009227478)
--(axis cs:0.218482926894079,-0.149901020185651)
--(axis cs:0.694353516552503,-0.704212714310419)
--cycle;
\path [draw=color2, fill=color2]
(axis cs:0.915515455672217,-1.56223640998641)
--(axis cs:0.361160707852192,-1.12526391609712)
--(axis cs:0.694353516552503,-0.704212714310419)
--cycle;
\path [draw=color2, fill=color2]
(axis cs:-1.15979760167936,2.48579337246029)
--(axis cs:-1.21072940472115,2.32564645539021)
--(axis cs:-1.92310074281666,2.849737736476)
--cycle;
\path [draw=color2, fill=color2]
(axis cs:3.44185983653917,4.45516210545079)
--(axis cs:3.51218046217622,4.65210791346245)
--(axis cs:3.98154238175133,3.96023539287441)
--cycle;
\path [draw=color2, fill=color2]
(axis cs:-2.93513629393346,2.25270029210883)
--(axis cs:-3.53276051991429,2.84153490572953)
--(axis cs:-3.46174416837956,2.39161003248653)
--cycle;
\path [draw=color2, fill=color2]
(axis cs:-2.99962181068817,1.36366577935927)
--(axis cs:-2.93513629393346,2.25270029210883)
--(axis cs:-3.40165671980728,2.17191730886362)
--cycle;
\path [draw=color2, fill=color2]
(axis cs:4.18098479411033,5.20817168133169)
--(axis cs:3.91118505308693,6.02140162936865)
--(axis cs:3.76177050379184,5.73921743170339)
--cycle;
\path [draw=color2, fill=color2]
(axis cs:0.208319009940496,0.379643070293184)
--(axis cs:-0.317679888699185,-0.0629296458187575)
--(axis cs:-0.0361543454845905,0.819222700941846)
--cycle;
\path [draw=color2, fill=color2]
(axis cs:-3.80990947334099,3.54431506206447)
--(axis cs:-3.6945185892455,3.48456470113167)
--(axis cs:-3.09213735335708,3.76910734978884)
--cycle;
\path [draw=color2, fill=color2]
(axis cs:-3.80990947334099,3.54431506206447)
--(axis cs:-4.00531594259129,3.25737670503793)
--(axis cs:-4.69073919765948,3.38159404040119)
--cycle;
\path [draw=color2, fill=color2]
(axis cs:-0.317679888699185,-0.0629296458187575)
--(axis cs:-0.682644108116294,0.76958992767765)
--(axis cs:-0.0361543454845905,0.819222700941846)
--cycle;
\path [draw=color2, fill=color2]
(axis cs:1.35475062047888,4.49965906131902)
--(axis cs:1.8835208886715,4.40757117309586)
--(axis cs:2.17963557993397,4.78564637341144)
--cycle;
\path [draw=color2, fill=color2]
(axis cs:2.66535600163303,3.77813231660804)
--(axis cs:2.10859389528023,4.18592294168646)
--(axis cs:2.31841867483311,4.1696634344682)
--cycle;
\path [draw=color2, fill=color2]
(axis cs:3.38829769446094,6.04358633691308)
--(axis cs:3.17027424070748,6.50474613140325)
--(axis cs:3.15857161895561,5.66514809186076)
--cycle;
\path [draw=color2, fill=color2]
(axis cs:0.208319009940496,0.379643070293184)
--(axis cs:0.734555168589413,0.618002625061071)
--(axis cs:0.218482926894079,-0.149901020185651)
--cycle;
\path [draw=color2, fill=color2]
(axis cs:0.734555168589413,0.618002625061071)
--(axis cs:1.0722787109508,-0.143047009227478)
--(axis cs:0.218482926894079,-0.149901020185651)
--cycle;
\path [draw=color2, fill=color2]
(axis cs:-3.6945185892455,3.48456470113167)
--(axis cs:-3.09213735335708,3.76910734978884)
--(axis cs:-2.89772819748342,3.67099747186657)
--cycle;
\path [draw=color2, fill=color2]
(axis cs:3.85072378702374,5.2428792846038)
--(axis cs:3.29006165526776,4.6251028305993)
--(axis cs:3.51218046217622,4.65210791346245)
--cycle;
\path [draw=color2, fill=color2]
(axis cs:-3.6945185892455,3.48456470113167)
--(axis cs:-2.89772819748342,3.67099747186657)
--(axis cs:-3.53276051991429,2.84153490572953)
--cycle;
\path [draw=color2, fill=color2]
(axis cs:2.64439303639391,5.33779192925331)
--(axis cs:2.81220545438285,4.70120393417241)
--(axis cs:3.29006165526776,4.6251028305993)
--cycle;
\path [draw=color2, fill=color2]
(axis cs:-1.66612171369168,-1.16491671279916)
--(axis cs:-2.45255966188733,-0.649306539777947)
--(axis cs:-1.8072024193766,-0.193590780546145)
--cycle;
\path [draw=color2, fill=color2]
(axis cs:2.64439303639391,5.33779192925331)
--(axis cs:3.15857161895561,5.66514809186076)
--(axis cs:3.29006165526776,4.6251028305993)
--cycle;
\path [draw=color2, fill=color2]
(axis cs:3.30899718046626,5.68990456424791)
--(axis cs:3.15857161895561,5.66514809186076)
--(axis cs:3.29006165526776,4.6251028305993)
--cycle;
\path [draw=color2, fill=color2]
(axis cs:3.30899718046626,5.68990456424791)
--(axis cs:3.85072378702374,5.2428792846038)
--(axis cs:3.29006165526776,4.6251028305993)
--cycle;
\path [draw=color2, fill=color2]
(axis cs:1.97167127405905,5.37846226762404)
--(axis cs:1.35475062047888,4.49965906131902)
--(axis cs:2.17963557993397,4.78564637341144)
--cycle;
\path [draw=color2, fill=color2]
(axis cs:-2.93513629393346,2.25270029210883)
--(axis cs:-2.45909955124887,2.75444952657863)
--(axis cs:-3.53276051991429,2.84153490572953)
--cycle;
\path [draw=color2, fill=color2]
(axis cs:-2.14787238675959,4.06031111140356)
--(axis cs:-1.86926540057446,4.31937963197559)
--(axis cs:-2.38827323869676,5.1019760098924)
--cycle;
\path [draw=color2, fill=color2]
(axis cs:-2.41240086456128,4.0225376563209)
--(axis cs:-2.14787238675959,4.06031111140356)
--(axis cs:-2.38827323869676,5.1019760098924)
--cycle;
\path [draw=color2, fill=color2]
(axis cs:-1.92310074281666,2.849737736476)
--(axis cs:-2.15550016844155,3.42338679220924)
--(axis cs:-1.81022369365007,3.81555936074182)
--cycle;
\path [draw=color2, fill=color2]
(axis cs:2.66535600163303,3.77813231660804)
--(axis cs:2.78862863818822,3.11143766434187)
--(axis cs:2.85133778920804,3.36411834401323)
--cycle;
\path [draw=color2, fill=color2]
(axis cs:-2.99962181068817,1.36366577935927)
--(axis cs:-3.82369011056774,2.05838460399524)
--(axis cs:-3.40165671980728,2.17191730886362)
--cycle;
\path [draw=color2, fill=color2]
(axis cs:1.82276797293495,1.66810423984943)
--(axis cs:0.897589460877768,1.97038397092996)
--(axis cs:0.944278596087085,1.13182749047688)
--cycle;
\path [draw=color2, fill=color2]
(axis cs:4.20230143122709,5.04114381567126)
--(axis cs:3.51218046217622,4.65210791346245)
--(axis cs:3.98154238175133,3.96023539287441)
--cycle;
\path [draw=color2, fill=color2]
(axis cs:-0.874971110150818,1.52230867430528)
--(axis cs:-0.682644108116294,0.76958992767765)
--(axis cs:-0.0361543454845905,0.819222700941846)
--cycle;
\path [draw=color2, fill=color2]
(axis cs:-2.52364788757035,3.29543119127435)
--(axis cs:-2.89772819748342,3.67099747186657)
--(axis cs:-3.53276051991429,2.84153490572953)
--cycle;
\path [draw=color2, fill=color2]
(axis cs:-2.45909955124887,2.75444952657863)
--(axis cs:-2.52364788757035,3.29543119127435)
--(axis cs:-3.53276051991429,2.84153490572953)
--cycle;
\path [draw=color2, fill=color2]
(axis cs:0.915515455672217,-1.56223640998641)
--(axis cs:0.694353516552503,-0.704212714310419)
--(axis cs:1.27576879788079,-0.474284940941255)
--cycle;
\path [draw=color2, fill=color2]
(axis cs:0.297086897817065,0.995913395714418)
--(axis cs:0.897589460877768,1.97038397092996)
--(axis cs:0.944278596087085,1.13182749047688)
--cycle;
\path [draw=color2, fill=color2]
(axis cs:2.66535600163303,3.77813231660804)
--(axis cs:3.44185983653917,4.45516210545079)
--(axis cs:2.81458938319616,4.32045257984774)
--cycle;
\path [draw=color2, fill=color2]
(axis cs:3.17027424070748,6.50474613140325)
--(axis cs:3.57926152941062,6.35979132901094)
--(axis cs:4.11118344993143,6.74050825521424)
--cycle;
\path [draw=color2, fill=color2]
(axis cs:4.11118344993143,6.74050825521424)
--(axis cs:4.63615766530576,6.47510482513503)
--(axis cs:4.17446946920669,7.5111622752538)
--cycle;
\path [draw=color2, fill=color2]
(axis cs:-0.888552360386418,0.837344604908327)
--(axis cs:-1.58108210183386,0.576205666538593)
--(axis cs:-0.874971110150818,1.52230867430528)
--cycle;
\path [draw=color2, fill=color2]
(axis cs:-4.00531594259129,3.25737670503793)
--(axis cs:-4.69073919765948,3.38159404040119)
--(axis cs:-3.79987123609756,2.82295825942001)
--cycle;
\path [draw=color2, fill=color2]
(axis cs:2.66535600163303,3.77813231660804)
--(axis cs:2.85133778920804,3.36411834401323)
--(axis cs:3.44185983653917,4.45516210545079)
--cycle;
\path [draw=color2, fill=color2]
(axis cs:-1.66612171369168,-1.16491671279916)
--(axis cs:-0.679764120459465,-0.736996202880417)
--(axis cs:-1.8072024193766,-0.193590780546145)
--cycle;
\path [draw=color2, fill=color2]
(axis cs:0.915515455672217,-1.56223640998641)
--(axis cs:2.03361333854756,-1.01225876863371)
--(axis cs:1.27576879788079,-0.474284940941255)
--cycle;
\path [draw=color2, fill=color2]
(axis cs:2.85133778920804,3.36411834401323)
--(axis cs:3.44185983653917,4.45516210545079)
--(axis cs:3.98154238175133,3.96023539287441)
--cycle;
\path [draw=color2, fill=color2]
(axis cs:-1.15979760167936,2.48579337246029)
--(axis cs:-0.665776111307205,2.16134751297774)
--(axis cs:-0.430420348837371,3.45705996633102)
--cycle;
\path [draw=color2, fill=color2]
(axis cs:1.19147659111797,3.25759452658187)
--(axis cs:1.8835208886715,4.40757117309586)
--(axis cs:2.10859389528023,4.18592294168646)
--cycle;
\path [draw=color2, fill=color2]
(axis cs:1.19147659111797,3.25759452658187)
--(axis cs:1.35475062047888,4.49965906131902)
--(axis cs:1.8835208886715,4.40757117309586)
--cycle;
\path [draw=color2, fill=color2]
(axis cs:2.54059011512305,5.40732724426877)
--(axis cs:3.17027424070748,6.50474613140325)
--(axis cs:3.15857161895561,5.66514809186076)
--cycle;
\path [draw=color2, fill=color2]
(axis cs:4.18098479411033,5.20817168133169)
--(axis cs:3.91118505308693,6.02140162936865)
--(axis cs:4.63615766530576,6.47510482513503)
--cycle;
\path [draw=color2, fill=color2]
(axis cs:0.297086897817065,0.995913395714418)
--(axis cs:-0.874971110150818,1.52230867430528)
--(axis cs:-0.0361543454845905,0.819222700941846)
--cycle;
\path [draw=color2, fill=color2]
(axis cs:-0.888552360386418,0.837344604908327)
--(axis cs:-0.317679888699185,-0.0629296458187575)
--(axis cs:-0.682644108116294,0.76958992767765)
--cycle;
\path [draw=color2, fill=color2]
(axis cs:-0.358111612392031,-1.03573022195866)
--(axis cs:-1.66612171369168,-1.16491671279916)
--(axis cs:-0.679764120459465,-0.736996202880417)
--cycle;
\addplot [very thick, color3]
table {%
-2.21841271692549 3.26727594273737
-2.22758194752147 3.25979487107238
};
\addplot [very thick, color3]
table {%
-0.108138330543506 -0.272147725049101
-0.108331721452844 -0.315740368822207
};
\addplot [very thick, color3]
table {%
-0.358111612392031 -1.03573022195866
-0.336311379908108 -0.997344391227889
};
\addplot [very thick, color3]
table {%
3.30899718046626 5.68990456424791
3.36185274192307 5.77406588531824
};
\addplot [very thick, color3]
table {%
-0.386954794949196 -0.887278921762747
-0.336311379908108 -0.997344391227889
};
\addplot [very thick, color3]
table {%
2.54059011512305 5.40732724426877
2.64439303639391 5.33779192925331
};
\addplot [very thick, color3]
table {%
-3.80990947334099 3.54431506206447
-3.6945185892455 3.48456470113167
};
\addplot [very thick, color3]
table {%
-0.108331721452844 -0.315740368822207
-0.0305854665203686 -0.424590954248704
};
\addplot [very thick, color3]
table {%
3.30899718046626 5.68990456424791
3.15857161895561 5.66514809186076
};
\addplot [very thick, color3]
table {%
-1.15979760167936 2.48579337246029
-1.21072940472115 2.32564645539021
};
\addplot [very thick, color3]
table {%
-2.21841271692549 3.26727594273737
-2.15550016844155 3.42338679220924
};
\addplot [very thick, color3]
table {%
4.20230143122709 5.04114381567126
4.18098479411033 5.20817168133169
};
\addplot [very thick, color3]
table {%
-0.358111612392031 -1.03573022195866
-0.386954794949196 -0.887278921762747
};
\addplot [very thick, color3]
table {%
3.44185983653917 4.45516210545079
3.51218046217622 4.65210791346245
};
\addplot [very thick, color3]
table {%
2.10859389528023 4.18592294168646
2.31841867483311 4.1696634344682
};
\addplot [very thick, color3]
table {%
0.297086897817065 0.995913395714418
0.185474121138129 0.813411089586105
};
\addplot [very thick, color3]
table {%
-0.888552360386418 0.837344604908327
-0.682644108116294 0.76958992767765
};
\addplot [very thick, color3]
table {%
-3.09213735335708 3.76910734978884
-2.89772819748342 3.67099747186657
};
\addplot [very thick, color3]
table {%
0.185474121138129 0.813411089586105
-0.0361543454845905 0.819222700941846
};
\addplot [very thick, color3]
table {%
3.29006165526776 4.6251028305993
3.51218046217622 4.65210791346245
};
\addplot [very thick, color3]
table {%
-3.40165671980728 2.17191730886362
-3.46174416837956 2.39161003248653
};
\addplot [very thick, color3]
table {%
3.44185983653917 4.45516210545079
3.29006165526776 4.6251028305993
};
\addplot [very thick, color3]
table {%
2.78862863818822 3.11143766434187
2.85133778920804 3.36411834401323
};
\addplot [very thick, color3]
table {%
-2.41240086456128 4.0225376563209
-2.14787238675959 4.06031111140356
};
\addplot [very thick, color3]
table {%
-3.79987123609756 2.82295825942001
-3.53276051991429 2.84153490572953
};
\addplot [very thick, color3]
table {%
3.38829769446094 6.04358633691308
3.36185274192307 5.77406588531824
};
\addplot [very thick, color3]
table {%
2.51524615285251 4.37419916143041
2.31841867483311 4.1696634344682
};
\addplot [very thick, color3]
table {%
-0.108138330543506 -0.272147725049101
-0.0305854665203686 -0.424590954248704
};
\addplot [very thick, color3]
table {%
-0.108138330543506 -0.272147725049101
-0.317679888699185 -0.0629296458187575
};
\addplot [very thick, color3]
table {%
-2.52364788757035 3.29543119127435
-2.22758194752147 3.25979487107238
};
\addplot [very thick, color3]
table {%
2.51524615285251 4.37419916143041
2.81458938319616 4.32045257984774
};
\addplot [very thick, color3]
table {%
3.15857161895561 5.66514809186076
3.36185274192307 5.77406588531824
};
\addplot [very thick, color3]
table {%
1.8835208886715 4.40757117309586
2.10859389528023 4.18592294168646
};
\addplot [very thick, color3]
table {%
3.91118505308693 6.02140162936865
3.76177050379184 5.73921743170339
};
\addplot [very thick, color3]
table {%
-0.386954794949196 -0.887278921762747
-0.679764120459465 -0.736996202880417
};
\addplot [very thick, color3]
table {%
4.18098479411033 5.20817168133169
3.85072378702374 5.2428792846038
};
\addplot [very thick, color3]
table {%
-3.80990947334099 3.54431506206447
-4.00531594259129 3.25737670503793
};
\addplot [very thick, color3]
table {%
-0.108138330543506 -0.272147725049101
0.218482926894079 -0.149901020185651
};
\addplot [very thick, color3]
table {%
2.37993615478195 5.08341926202954
2.17963557993397 4.78564637341144
};
\addplot [very thick, color3]
table {%
2.54059011512305 5.40732724426877
2.37993615478195 5.08341926202954
};
\addplot [very thick, color3]
table {%
2.64439303639391 5.33779192925331
2.37993615478195 5.08341926202954
};
\addplot [very thick, color3]
table {%
3.38829769446094 6.04358633691308
3.57926152941062 6.35979132901094
};
\addplot [very thick, color3]
table {%
-2.22758194752147 3.25979487107238
-2.15550016844155 3.42338679220924
};
\addplot [very thick, color3]
table {%
-0.0305854665203686 -0.424590954248704
0.218482926894079 -0.149901020185651
};
\addplot [very thick, color3]
table {%
-2.14787238675959 4.06031111140356
-1.86926540057446 4.31937963197559
};
\addplot [very thick, color3]
table {%
2.81220545438285 4.70120393417241
2.81458938319616 4.32045257984774
};
\addplot [very thick, color3]
table {%
-3.6945185892455 3.48456470113167
-4.00531594259129 3.25737670503793
};
\addplot [very thick, color3]
table {%
1.0722787109508 -0.143047009227478
1.27576879788079 -0.474284940941255
};
\addplot [very thick, color3]
table {%
-0.794077022732528 -3.09763838394167
-0.879810558070651 -3.48084522004399
};
\addplot [very thick, color3]
table {%
3.76177050379184 5.73921743170339
3.36185274192307 5.77406588531824
};
\addplot [very thick, color3]
table {%
-2.52364788757035 3.29543119127435
-2.15550016844155 3.42338679220924
};
\addplot [very thick, color3]
table {%
4.20230143122709 5.04114381567126
3.85072378702374 5.2428792846038
};
\addplot [very thick, color3]
table {%
-2.14787238675959 4.06031111140356
-1.81022369365007 3.81555936074182
};
\addplot [very thick, color3]
table {%
3.17027424070748 6.50474613140325
3.57926152941062 6.35979132901094
};
\addplot [very thick, color3]
table {%
0.185474121138129 0.813411089586105
0.208319009940496 0.379643070293184
};
\addplot [very thick, color3]
table {%
0.297086897817065 0.995913395714418
-0.0361543454845905 0.819222700941846
};
\addplot [very thick, color3]
table {%
-3.82369011056774 2.05838460399524
-3.40165671980728 2.17191730886362
};
\addplot [very thick, color3]
table {%
2.51524615285251 4.37419916143041
2.81220545438285 4.70120393417241
};
\addplot [very thick, color3]
table {%
2.66535600163303 3.77813231660804
2.85133778920804 3.36411834401323
};
\addplot [very thick, color3]
table {%
-3.53276051991429 2.84153490572953
-3.46174416837956 2.39161003248653
};
\addplot [very thick, color3]
table {%
-0.108331721452844 -0.315740368822207
-0.317679888699185 -0.0629296458187575
};
\addplot [very thick, color3]
table {%
-2.93513629393346 2.25270029210883
-3.40165671980728 2.17191730886362
};
\addplot [very thick, color3]
table {%
3.57926152941062 6.35979132901094
3.91118505308693 6.02140162936865
};
\addplot [very thick, color3]
table {%
1.8835208886715 4.40757117309586
2.17963557993397 4.78564637341144
};
\addplot [very thick, color3]
table {%
-4.00531594259129 3.25737670503793
-3.79987123609756 2.82295825942001
};
\addplot [very thick, color3]
table {%
3.38829769446094 6.04358633691308
3.76177050379184 5.73921743170339
};
\addplot [very thick, color3]
table {%
2.81220545438285 4.70120393417241
3.29006165526776 4.6251028305993
};
\addplot [very thick, color3]
table {%
-3.82369011056774 2.05838460399524
-3.46174416837956 2.39161003248653
};
\addplot [very thick, color3]
table {%
2.37993615478195 5.08341926202954
1.97167127405905 5.37846226762404
};
\addplot [very thick, color3]
table {%
3.76177050379184 5.73921743170339
3.85072378702374 5.2428792846038
};
\addplot [very thick, color3]
table {%
0.208319009940496 0.379643070293184
-0.0361543454845905 0.819222700941846
};
\addplot [very thick, color3]
table {%
-1.86926540057446 4.31937963197559
-1.81022369365007 3.81555936074182
};
\addplot [very thick, color3]
table {%
3.38829769446094 6.04358633691308
3.17027424070748 6.50474613140325
};
\addplot [very thick, color3]
table {%
-1.92310074281666 2.849737736476
-2.22758194752147 3.25979487107238
};
\addplot [very thick, color3]
table {%
3.76177050379184 5.73921743170339
3.30899718046626 5.68990456424791
};
\addplot [very thick, color3]
table {%
-1.92310074281666 2.849737736476
-2.21841271692549 3.26727594273737
};
\addplot [very thick, color3]
table {%
-2.15550016844155 3.42338679220924
-1.81022369365007 3.81555936074182
};
\addplot [very thick, color3]
table {%
2.66535600163303 3.77813231660804
2.31841867483311 4.1696634344682
};
\addplot [very thick, color3]
table {%
3.38829769446094 6.04358633691308
3.91118505308693 6.02140162936865
};
\addplot [very thick, color3]
table {%
2.86412145575591 2.58741981824194
2.78862863818822 3.11143766434187
};
\addplot [very thick, color3]
table {%
0.208319009940496 0.379643070293184
0.218482926894079 -0.149901020185651
};
\addplot [very thick, color3]
table {%
-2.52364788757035 3.29543119127435
-2.89772819748342 3.67099747186657
};
\addplot [very thick, color3]
table {%
2.51524615285251 4.37419916143041
2.17963557993397 4.78564637341144
};
\addplot [very thick, color3]
table {%
3.38829769446094 6.04358633691308
3.15857161895561 5.66514809186076
};
\addplot [very thick, color3]
table {%
1.35475062047888 4.49965906131902
1.8835208886715 4.40757117309586
};
\addplot [very thick, color3]
table {%
0.361160707852192 -1.12526391609712
0.694353516552503 -0.704212714310419
};
\addplot [very thick, color3]
table {%
-1.92310074281666 2.849737736476
-2.45909955124887 2.75444952657863
};
\addplot [very thick, color3]
table {%
-2.45909955124887 2.75444952657863
-2.52364788757035 3.29543119127435
};
\addplot [very thick, color3]
table {%
-2.93513629393346 2.25270029210883
-3.46174416837956 2.39161003248653
};
\addplot [very thick, color3]
table {%
-3.79987123609756 2.82295825942001
-3.46174416837956 2.39161003248653
};
\addplot [very thick, color3]
table {%
0.734555168589413 0.618002625061071
0.944278596087085 1.13182749047688
};
\addplot [very thick, color3]
table {%
-2.45909955124887 2.75444952657863
-2.22758194752147 3.25979487107238
};
\addplot [very thick, color3]
table {%
-0.358111612392031 -1.03573022195866
-0.679764120459465 -0.736996202880417
};
\addplot [very thick, color3]
table {%
2.66535600163303 3.77813231660804
2.81458938319616 4.32045257984774
};
\addplot [very thick, color3]
table {%
2.81458938319616 4.32045257984774
3.29006165526776 4.6251028305993
};
\addplot [very thick, color3]
table {%
-1.21072940472115 2.32564645539021
-0.665776111307205 2.16134751297774
};
\addplot [very thick, color3]
table {%
2.54059011512305 5.40732724426877
1.97167127405905 5.37846226762404
};
\addplot [very thick, color3]
table {%
2.37993615478195 5.08341926202954
2.81220545438285 4.70120393417241
};
\addplot [very thick, color3]
table {%
0.208319009940496 0.379643070293184
0.734555168589413 0.618002625061071
};
\addplot [very thick, color3]
table {%
0.297086897817065 0.995913395714418
0.734555168589413 0.618002625061071
};
\addplot [very thick, color3]
table {%
2.51524615285251 4.37419916143041
2.10859389528023 4.18592294168646
};
\addplot [very thick, color3]
table {%
0.185474121138129 0.813411089586105
0.734555168589413 0.618002625061071
};
\addplot [very thick, color3]
table {%
-0.386954794949196 -0.887278921762747
-0.0305854665203686 -0.424590954248704
};
\addplot [very thick, color3]
table {%
4.11118344993143 6.74050825521424
4.63615766530576 6.47510482513503
};
\addplot [very thick, color3]
table {%
-1.15979760167936 2.48579337246029
-0.665776111307205 2.16134751297774
};
\addplot [very thick, color3]
table {%
-0.317679888699185 -0.0629296458187575
0.218482926894079 -0.149901020185651
};
\addplot [very thick, color3]
table {%
-2.41240086456128 4.0225376563209
-2.89772819748342 3.67099747186657
};
\addplot [very thick, color3]
table {%
2.17963557993397 4.78564637341144
2.10859389528023 4.18592294168646
};
\addplot [very thick, color3]
table {%
2.64439303639391 5.33779192925331
3.15857161895561 5.66514809186076
};
\addplot [very thick, color3]
table {%
2.66535600163303 3.77813231660804
2.51524615285251 4.37419916143041
};
\addplot [very thick, color3]
table {%
0.694353516552503 -0.704212714310419
1.27576879788079 -0.474284940941255
};
\addplot [very thick, color3]
table {%
1.97167127405905 5.37846226762404
2.17963557993397 4.78564637341144
};
\addplot [very thick, color3]
table {%
-2.14787238675959 4.06031111140356
-2.15550016844155 3.42338679220924
};
\addplot [very thick, color3]
table {%
2.17963557993397 4.78564637341144
2.81220545438285 4.70120393417241
};
\addplot [very thick, color3]
table {%
-0.682644108116294 0.76958992767765
-0.0361543454845905 0.819222700941846
};
\addplot [very thick, color3]
table {%
3.44185983653917 4.45516210545079
2.81458938319616 4.32045257984774
};
\addplot [very thick, color3]
table {%
-2.41240086456128 4.0225376563209
-2.15550016844155 3.42338679220924
};
\addplot [very thick, color3]
table {%
3.57926152941062 6.35979132901094
4.11118344993143 6.74050825521424
};
\addplot [very thick, color3]
table {%
2.64439303639391 5.33779192925331
2.81220545438285 4.70120393417241
};
\addplot [very thick, color3]
table {%
0.297086897817065 0.995913395714418
0.944278596087085 1.13182749047688
};
\addplot [very thick, color3]
table {%
-3.6945185892455 3.48456470113167
-3.53276051991429 2.84153490572953
};
\addplot [very thick, color3]
table {%
-0.386954794949196 -0.887278921762747
-0.108331721452844 -0.315740368822207
};
\addplot [very thick, color3]
table {%
-3.6945185892455 3.48456470113167
-3.09213735335708 3.76910734978884
};
\addplot [very thick, color3]
table {%
-0.665776111307205 2.16134751297774
-0.874971110150818 1.52230867430528
};
\addplot [very thick, color3]
table {%
-4.00531594259129 3.25737670503793
-3.53276051991429 2.84153490572953
};
\addplot [very thick, color3]
table {%
1.0722787109508 -0.143047009227478
0.694353516552503 -0.704212714310419
};
\addplot [very thick, color3]
table {%
3.85072378702374 5.2428792846038
3.51218046217622 4.65210791346245
};
\addplot [very thick, color3]
table {%
-0.888552360386418 0.837344604908327
-0.874971110150818 1.52230867430528
};
\addplot [very thick, color3]
table {%
0.208319009940496 0.379643070293184
-0.317679888699185 -0.0629296458187575
};
\addplot [very thick, color3]
table {%
-2.93513629393346 2.25270029210883
-2.45909955124887 2.75444952657863
};
\addplot [very thick, color3]
table {%
-4.00531594259129 3.25737670503793
-4.69073919765948 3.38159404040119
};
\addplot [very thick, color3]
table {%
3.30899718046626 5.68990456424791
3.85072378702374 5.2428792846038
};
\addplot [very thick, color3]
table {%
4.18098479411033 5.20817168133169
3.76177050379184 5.73921743170339
};
\addplot [very thick, color3]
table {%
0.915515455672217 -1.56223640998641
0.361160707852192 -1.12526391609712
};
\addplot [very thick, color3]
table {%
0.361160707852192 -1.12526391609712
-0.336311379908108 -0.997344391227889
};
\addplot [very thick, color3]
table {%
-0.108331721452844 -0.315740368822207
-0.679764120459465 -0.736996202880417
};
\addplot [very thick, color3]
table {%
0.218482926894079 -0.149901020185651
0.694353516552503 -0.704212714310419
};
\addplot [very thick, color3]
table {%
2.54059011512305 5.40732724426877
3.15857161895561 5.66514809186076
};
\addplot [very thick, color3]
table {%
3.44185983653917 4.45516210545079
3.98154238175133 3.96023539287441
};
\addplot [very thick, color3]
table {%
-0.0305854665203686 -0.424590954248704
-0.336311379908108 -0.997344391227889
};
\addplot [very thick, color3]
table {%
-2.41240086456128 4.0225376563209
-2.52364788757035 3.29543119127435
};
\addplot [very thick, color3]
table {%
-1.92310074281666 2.849737736476
-2.15550016844155 3.42338679220924
};
\addplot [very thick, color3]
table {%
-0.888552360386418 0.837344604908327
-1.58108210183386 0.576205666538593
};
\addplot [very thick, color3]
table {%
3.91118505308693 6.02140162936865
4.11118344993143 6.74050825521424
};
\addplot [very thick, color3]
table {%
-0.317679888699185 -0.0629296458187575
-0.679764120459465 -0.736996202880417
};
\addplot [very thick, color3]
table {%
-3.79987123609756 2.82295825942001
-3.82369011056774 2.05838460399524
};
\addplot [very thick, color3]
table {%
-0.358111612392031 -1.03573022195866
0.361160707852192 -1.12526391609712
};
\addplot [very thick, color3]
table {%
4.11118344993143 6.74050825521424
4.17446946920669 7.5111622752538
};
\addplot [very thick, color3]
table {%
-0.0305854665203686 -0.424590954248704
0.694353516552503 -0.704212714310419
};
\addplot [very thick, color3]
table {%
-2.45255966188733 -0.649306539777947
-1.8072024193766 -0.193590780546145
};
\addplot [very thick, color3]
table {%
4.20230143122709 5.04114381567126
3.51218046217622 4.65210791346245
};
\addplot [very thick, color3]
table {%
-1.58108210183386 0.576205666538593
-1.8072024193766 -0.193590780546145
};
\addplot [very thick, color3]
table {%
0.361160707852192 -1.12526391609712
-0.0305854665203686 -0.424590954248704
};
\addplot [very thick, color3]
table {%
-0.874971110150818 1.52230867430528
-0.682644108116294 0.76958992767765
};
\addplot [very thick, color3]
table {%
-2.41240086456128 4.0225376563209
-3.09213735335708 3.76910734978884
};
\addplot [very thick, color3]
table {%
3.98154238175133 3.96023539287441
4.76266522951807 3.69025802127189
};
\addplot [very thick, color3]
table {%
0.734555168589413 0.618002625061071
1.0722787109508 -0.143047009227478
};
\addplot [very thick, color3]
table {%
0.897589460877768 1.97038397092996
0.944278596087085 1.13182749047688
};
\addplot [very thick, color3]
table {%
-1.15979760167936 2.48579337246029
-1.92310074281666 2.849737736476
};
\addplot [very thick, color3]
table {%
1.0722787109508 -0.143047009227478
0.218482926894079 -0.149901020185651
};
\addplot [very thick, color3]
table {%
3.91118505308693 6.02140162936865
4.63615766530576 6.47510482513503
};
\addplot [very thick, color3]
table {%
4.17446946920669 7.5111622752538
3.6731850363502 8.21803989681783
};
\addplot [very thick, color3]
table {%
-1.21072940472115 2.32564645539021
-0.874971110150818 1.52230867430528
};
\addplot [very thick, color3]
table {%
0.915515455672217 -1.56223640998641
0.694353516552503 -0.704212714310419
};
\addplot [very thick, color3]
table {%
-2.99962181068817 1.36366577935927
-2.93513629393346 2.25270029210883
};
\addplot [very thick, color3]
table {%
-1.21072940472115 2.32564645539021
-1.92310074281666 2.849737736476
};
\addplot [very thick, color3]
table {%
-2.99962181068817 1.36366577935927
-3.40165671980728 2.17191730886362
};
\addplot [very thick, color3]
table {%
3.51218046217622 4.65210791346245
3.98154238175133 3.96023539287441
};
\addplot [very thick, color3]
table {%
-0.317679888699185 -0.0629296458187575
-0.682644108116294 0.76958992767765
};
\addplot [very thick, color3]
table {%
-2.93513629393346 2.25270029210883
-3.53276051991429 2.84153490572953
};
\addplot [very thick, color3]
table {%
2.03361333854756 -1.01225876863371
1.27576879788079 -0.474284940941255
};
\addplot [very thick, color3]
table {%
4.18098479411033 5.20817168133169
3.91118505308693 6.02140162936865
};
\addplot [very thick, color3]
table {%
-1.86926540057446 4.31937963197559
-2.38827323869676 5.1019760098924
};
\addplot [very thick, color3]
table {%
-1.66612171369168 -1.16491671279916
-2.45255966188733 -0.649306539777947
};
\addplot [very thick, color3]
table {%
-0.317679888699185 -0.0629296458187575
-0.0361543454845905 0.819222700941846
};
\addplot [very thick, color3]
table {%
-3.80990947334099 3.54431506206447
-3.09213735335708 3.76910734978884
};
\addplot [very thick, color3]
table {%
1.82276797293495 1.66810423984943
0.897589460877768 1.97038397092996
};
\addplot [very thick, color3]
table {%
-5.54154478663344 3.85984931952986
-4.69073919765948 3.38159404040119
};
\addplot [very thick, color3]
table {%
-3.80990947334099 3.54431506206447
-4.69073919765948 3.38159404040119
};
\addplot [very thick, color3]
table {%
-1.66612171369168 -1.16491671279916
-1.8072024193766 -0.193590780546145
};
\addplot [very thick, color3]
table {%
1.35475062047888 4.49965906131902
2.17963557993397 4.78564637341144
};
\addplot [very thick, color3]
table {%
2.66535600163303 3.77813231660804
2.10859389528023 4.18592294168646
};
\addplot [very thick, color3]
table {%
3.17027424070748 6.50474613140325
3.15857161895561 5.66514809186076
};
\addplot [very thick, color3]
table {%
0.734555168589413 0.618002625061071
0.218482926894079 -0.149901020185651
};
\addplot [very thick, color3]
table {%
1.82276797293495 1.66810423984943
0.944278596087085 1.13182749047688
};
\addplot [very thick, color3]
table {%
-3.6945185892455 3.48456470113167
-2.89772819748342 3.67099747186657
};
\addplot [very thick, color3]
table {%
3.85072378702374 5.2428792846038
3.29006165526776 4.6251028305993
};
\addplot [very thick, color3]
table {%
-2.89772819748342 3.67099747186657
-3.53276051991429 2.84153490572953
};
\addplot [very thick, color3]
table {%
3.15857161895561 5.66514809186076
3.29006165526776 4.6251028305993
};
\addplot [very thick, color3]
table {%
2.64439303639391 5.33779192925331
3.29006165526776 4.6251028305993
};
\addplot [very thick, color3]
table {%
3.30899718046626 5.68990456424791
3.29006165526776 4.6251028305993
};
\addplot [very thick, color3]
table {%
1.97167127405905 5.37846226762404
1.35475062047888 4.49965906131902
};
\addplot [very thick, color3]
table {%
-1.66612171369168 -1.16491671279916
-0.679764120459465 -0.736996202880417
};
\addplot [very thick, color3]
table {%
-2.45909955124887 2.75444952657863
-3.53276051991429 2.84153490572953
};
\addplot [very thick, color3]
table {%
-2.41240086456128 4.0225376563209
-2.38827323869676 5.1019760098924
};
\addplot [very thick, color3]
table {%
-2.14787238675959 4.06031111140356
-2.38827323869676 5.1019760098924
};
\addplot [very thick, color3]
table {%
-1.92310074281666 2.849737736476
-1.81022369365007 3.81555936074182
};
\addplot [very thick, color3]
table {%
2.66535600163303 3.77813231660804
2.78862863818822 3.11143766434187
};
\addplot [very thick, color3]
table {%
-2.99962181068817 1.36366577935927
-3.82369011056774 2.05838460399524
};
\addplot [very thick, color3]
table {%
4.20230143122709 5.04114381567126
3.98154238175133 3.96023539287441
};
\addplot [very thick, color3]
table {%
-2.52364788757035 3.29543119127435
-3.53276051991429 2.84153490572953
};
\addplot [very thick, color3]
table {%
-0.874971110150818 1.52230867430528
-0.0361543454845905 0.819222700941846
};
\addplot [very thick, color3]
table {%
1.97167127405905 5.37846226762404
0.982944772711592 5.92494310211316
};
\addplot [very thick, color3]
table {%
0.915515455672217 -1.56223640998641
1.27576879788079 -0.474284940941255
};
\addplot [very thick, color3]
table {%
0.297086897817065 0.995913395714418
0.897589460877768 1.97038397092996
};
\addplot [very thick, color3]
table {%
2.66535600163303 3.77813231660804
3.44185983653917 4.45516210545079
};
\addplot [very thick, color3]
table {%
-2.38827323869676 5.1019760098924
-1.7758920309332 6.10254964560158
};
\addplot [very thick, color3]
table {%
3.17027424070748 6.50474613140325
4.11118344993143 6.74050825521424
};
\addplot [very thick, color3]
table {%
-1.15979760167936 2.48579337246029
-0.430420348837371 3.45705996633102
};
\addplot [very thick, color3]
table {%
4.63615766530576 6.47510482513503
4.17446946920669 7.5111622752538
};
\addplot [very thick, color3]
table {%
0.915515455672217 -1.56223640998641
2.03361333854756 -1.01225876863371
};
\addplot [very thick, color3]
table {%
-0.679764120459465 -0.736996202880417
-1.8072024193766 -0.193590780546145
};
\addplot [very thick, color3]
table {%
1.19147659111797 3.25759452658187
1.35475062047888 4.49965906131902
};
\addplot [very thick, color3]
table {%
-1.58108210183386 0.576205666538593
-0.874971110150818 1.52230867430528
};
\addplot [very thick, color3]
table {%
2.85133778920804 3.36411834401323
3.98154238175133 3.96023539287441
};
\addplot [very thick, color3]
table {%
-4.69073919765948 3.38159404040119
-3.79987123609756 2.82295825942001
};
\addplot [very thick, color3]
table {%
2.85133778920804 3.36411834401323
3.44185983653917 4.45516210545079
};
\addplot [very thick, color3]
table {%
1.19147659111797 3.25759452658187
2.10859389528023 4.18592294168646
};
\addplot [very thick, color3]
table {%
-0.665776111307205 2.16134751297774
-0.430420348837371 3.45705996633102
};
\addplot [very thick, color3]
table {%
0.897589460877768 1.97038397092996
1.19147659111797 3.25759452658187
};
\addplot [very thick, color3]
table {%
1.19147659111797 3.25759452658187
1.8835208886715 4.40757117309586
};
\addplot [very thick, color3]
table {%
2.54059011512305 5.40732724426877
3.17027424070748 6.50474613140325
};
\addplot [very thick, color3]
table {%
4.18098479411033 5.20817168133169
4.63615766530576 6.47510482513503
};
\addplot [very thick, color3]
table {%
0.297086897817065 0.995913395714418
-0.874971110150818 1.52230867430528
};
\addplot [very thick, color3]
table {%
1.82276797293495 1.66810423984943
2.86412145575591 2.58741981824194
};
\addplot [very thick, color3]
table {%
-0.430420348837371 3.45705996633102
-1.81022369365007 3.81555936074182
};
\addplot [very thick, color3]
table {%
-0.888552360386418 0.837344604908327
-0.317679888699185 -0.0629296458187575
};
\addplot [very thick, color3]
table {%
-0.358111612392031 -1.03573022195866
-1.66612171369168 -1.16491671279916
};
\addplot [only marks, mark=*, draw=color0, fill=color0, colormap/viridis]
table{%
x                      y
-0.358111612392031 -1.03573022195866
2.54059011512305 5.40732724426877
2.64439303639391 5.33779192925331
2.37993615478195 5.08341926202954
-0.108138330543506 -0.272147725049101
-1.66612171369168 -1.16491671279916
1.82276797293495 1.66810423984943
-0.386954794949196 -0.887278921762747
0.297086897817065 0.995913395714418
4.20230143122709 5.04114381567126
4.18098479411033 5.20817168133169
-1.15979760167936 2.48579337246029
-1.21072940472115 2.32564645539021
-1.92310074281666 2.849737736476
-0.794077022732528 -3.09763838394167
-0.879810558070651 -3.48084522004399
0.897589460877768 1.97038397092996
-2.41240086456128 4.0225376563209
-2.14787238675959 4.06031111140356
0.185474121138129 0.813411089586105
0.208319009940496 0.379643070293184
-0.108331721452844 -0.315740368822207
0.915515455672217 -1.56223640998641
0.361160707852192 -1.12526391609712
-2.99962181068817 1.36366577935927
-1.86926540057446 4.31937963197559
1.97167127405905 5.37846226762404
3.38829769446094 6.04358633691308
3.17027424070748 6.50474613140325
2.86412145575591 2.58741981824194
3.57926152941062 6.35979132901094
3.91118505308693 6.02140162936865
0.734555168589413 0.618002625061071
-0.888552360386418 0.837344604908327
-2.93513629393346 2.25270029210883
-2.45909955124887 2.75444952657863
0.982944772711592 5.92494310211316
1.0722787109508 -0.143047009227478
3.76177050379184 5.73921743170339
3.30899718046626 5.68990456424791
3.15857161895561 5.66514809186076
3.36185274192307 5.77406588531824
3.85072378702374 5.2428792846038
4.11118344993143 6.74050825521424
4.63615766530576 6.47510482513503
-0.665776111307205 2.16134751297774
-0.317679888699185 -0.0629296458187575
1.19147659111797 3.25759452658187
-2.45255966188733 -0.649306539777947
-2.38827323869676 5.1019760098924
-3.80990947334099 3.54431506206447
-3.6945185892455 3.48456470113167
-3.09213735335708 3.76910734978884
-1.58108210183386 0.576205666538593
-0.0305854665203686 -0.424590954248704
-0.336311379908108 -0.997344391227889
1.35475062047888 4.49965906131902
1.8835208886715 4.40757117309586
2.66535600163303 3.77813231660804
2.78862863818822 3.11143766434187
2.85133778920804 3.36411834401323
-5.54154478663344 3.85984931952986
0.944278596087085 1.13182749047688
-0.874971110150818 1.52230867430528
-2.52364788757035 3.29543119127435
-2.21841271692549 3.26727594273737
2.03361333854756 -1.01225876863371
-2.89772819748342 3.67099747186657
-2.22758194752147 3.25979487107238
-2.15550016844155 3.42338679220924
3.44185983653917 4.45516210545079
2.51524615285251 4.37419916143041
2.17963557993397 4.78564637341144
2.81220545438285 4.70120393417241
2.10859389528023 4.18592294168646
2.81458938319616 4.32045257984774
3.29006165526776 4.6251028305993
0.218482926894079 -0.149901020185651
2.31841867483311 4.1696634344682
3.51218046217622 4.65210791346245
4.17446946920669 7.5111622752538
-0.430420348837371 3.45705996633102
3.98154238175133 3.96023539287441
-4.00531594259129 3.25737670503793
-4.69073919765948 3.38159404040119
-3.79987123609756 2.82295825942001
-3.53276051991429 2.84153490572953
-3.82369011056774 2.05838460399524
-3.40165671980728 2.17191730886362
-3.46174416837956 2.39161003248653
-1.81022369365007 3.81555936074182
-0.679764120459465 -0.736996202880417
-0.682644108116294 0.76958992767765
0.694353516552503 -0.704212714310419
1.27576879788079 -0.474284940941255
-1.8072024193766 -0.193590780546145
-1.7758920309332 6.10254964560158
-0.0361543454845905 0.819222700941846
3.6731850363502 8.21803989681783
4.76266522951807 3.69025802127189
};
\end{axis}

\end{tikzpicture}

\subcaption{Intermediate step of the filtration of the largest possible Alpha complex on $S$.}
\end{center}
\end{subfigure}
\caption{Sampled Gaussian mixture.}
\label{fig:mixture}
\end{figure}

There are a lot of variables here which can be tweaked. We could choose $i$ in a weighted manner or draw means and covariances from some distribution. In the following we will however restrict ourselves to handpicked means, identity matrices for covariance and uniform distribution when choosing the Gaussian to be drawn from.  

\section{Apparent Pairs on Random Alpha Complexes}
The goal of this section is to explore the relation between expensive persistent homology computations and apparent pairs from an experimental perspective.

The first question we will discuss is: how many pairs of our filtrations are part of an apparent pair? Or alternatively how many simplices of some filtration $F$ of an alpha complex are critical with respect to the apparent gradient $V$?

Each of our constructions has many different possible simplicial complexes and filtrations they can result in, given the same initial parameters. For example, two instances, of drawing $50$ points uniformly and then constructing an alpha complex on them, will most likely not yield the same simplicial complex and filtration. 

That is why we will look at the percentage of elements in our filtrations that are in apparent pairs. This means, given some filtration $F_*$ with $m$ elements and $r$ apparent pairs we calculate the percentage $q$ of elements in $F_*$ that are part of an apparent pair by: \[
    q \coloneqq \frac{200r}{m}.
\]

Since every maximal alpha complex is contractible, the $q$-value also gives a good indication of how close the apparent gradient is to the theoretically possible perfect Morse matching with a single critical vertex.

In our first experiment we draw $300$ two dimensional points uniformly $2000$ times. Then we construct the largest possible alpha complexes and the corresponding filtration on each of the point sets. Afterwards we calculate $q$ for each complex. Since we do this $2000$ times we are able to analyze how different $q$-values are distributed. To this end we fit a Gaussian curve to the results using \textbf{scipy.stats.norm.fit} from the Python library \textbf{scipy} which calculates the mean $\mu$ and the standard deviation $\sigma^2$ of the passed data. The Gaussian fits the data quite nicely and furthermore we have observed similar fits in all dimensions and for different sizes of point clouds, which indicates that the $q$-values are normally distributed.

The following plot shows the fitted Gaussian for the aforementioned setup and the calculated $q$ values as a histogram with $25$ bins. This means that several results are combined in the same bin, i.e., we have some information loss here, but it is a good way of visualizing the results. On the $x$-axis we have the values of $q$ and on the $y$-axis we have the values for the density function of the fitted Gaussian. The histogram is scaled such that the integral over it equals $1$. Just like that over the fitted Gaussian.

\begin{figure}[H]
%\centering%
\begin{subfigure}[c]{0.95\textwidth}
\begin{center}
% This file was created by tikzplotlib v0.9.2.
\begin{tikzpicture}

\definecolor{color0}{rgb}{0.12156862745098,0.466666666666667,0.705882352941177}

\begin{axis}[
tick align=outside,
tick pos=left,
title={Fit results: $\mu$ = 73.19,  $\sigma^2$ = 0.89},
xlabel = $q$,
ylabel = density,
x grid style={white!69.0196078431373!black},
xmin=69.8006981611337, xmax=76.8744622631038,
xtick style={color=black},
y grid style={white!69.0196078431373!black},
ymin=0, ymax=0.472848843800874,
ytick style={color=black}
]
\draw[draw=none,fill=color0,fill opacity=0.6] (axis cs:70.4145371947757,0) rectangle (axis cs:70.6483806361631,0.017105461569788);
\draw[draw=none,fill=color0,fill opacity=0.6] (axis cs:70.6483806361632,0) rectangle (axis cs:70.8822240775506,0.0128290961773418);
\draw[draw=none,fill=color0,fill opacity=0.6] (axis cs:70.8822240775506,0) rectangle (axis cs:71.116067518938,0.0256581923546821);
\draw[draw=none,fill=color0,fill opacity=0.6] (axis cs:71.116067518938,0) rectangle (axis cs:71.3499109603255,0.0256581923546836);
\draw[draw=none,fill=color0,fill opacity=0.6] (axis cs:71.3499109603255,0) rectangle (axis cs:71.5837544017129,0.0641454808867052);
\draw[draw=none,fill=color0,fill opacity=0.6] (axis cs:71.5837544017129,0) rectangle (axis cs:71.8175978431004,0.106909134811182);
\draw[draw=none,fill=color0,fill opacity=0.6] (axis cs:71.8175978431003,0) rectangle (axis cs:72.0514412844878,0.17105461569788);
\draw[draw=none,fill=color0,fill opacity=0.6] (axis cs:72.0514412844878,0) rectangle (axis cs:72.2852847258752,0.237338279280824);
\draw[draw=none,fill=color0,fill opacity=0.6] (axis cs:72.2852847258752,0) rectangle (axis cs:72.5191281672627,0.265134654331715);
\draw[draw=none,fill=color0,fill opacity=0.6] (axis cs:72.5191281672627,0) rectangle (axis cs:72.7529716086501,0.414807443067385);
\draw[draw=none,fill=color0,fill opacity=0.6] (axis cs:72.7529716086501,0) rectangle (axis cs:72.9868150500376,0.397701981497572);
\draw[draw=none,fill=color0,fill opacity=0.6] (axis cs:72.9868150500376,0) rectangle (axis cs:73.220658491425,0.438327452725845);
\draw[draw=none,fill=color0,fill opacity=0.6] (axis cs:73.220658491425,0) rectangle (axis cs:73.4545019328124,0.449018366206936);
\draw[draw=none,fill=color0,fill opacity=0.6] (axis cs:73.4545019328125,0) rectangle (axis cs:73.6883453741999,0.436189270029622);
\draw[draw=none,fill=color0,fill opacity=0.6] (axis cs:73.6883453741999,0) rectangle (axis cs:73.9221888155873,0.352800144876878);
\draw[draw=none,fill=color0,fill opacity=0.6] (axis cs:73.9221888155873,0) rectangle (axis cs:74.1560322569748,0.260858288939284);
\draw[draw=none,fill=color0,fill opacity=0.6] (axis cs:74.1560322569748,0) rectangle (axis cs:74.3898756983622,0.213818269622351);
\draw[draw=none,fill=color0,fill opacity=0.6] (axis cs:74.3898756983622,0) rectangle (axis cs:74.6237191397497,0.179607346482785);
\draw[draw=none,fill=color0,fill opacity=0.6] (axis cs:74.6237191397497,0) rectangle (axis cs:74.8575625811371,0.0833891251527167);
\draw[draw=none,fill=color0,fill opacity=0.6] (axis cs:74.8575625811371,0) rectangle (axis cs:75.0914060225245,0.0577309327980382);
\draw[draw=none,fill=color0,fill opacity=0.6] (axis cs:75.0914060225245,0) rectangle (axis cs:75.325249463912,0.0342109231395761);
\draw[draw=none,fill=color0,fill opacity=0.6] (axis cs:75.325249463912,0) rectangle (axis cs:75.5590929052994,0.0171054615697891);
\draw[draw=none,fill=color0,fill opacity=0.6] (axis cs:75.5590929052994,0) rectangle (axis cs:75.7929363466869,0.00641454808867052);
\draw[draw=none,fill=color0,fill opacity=0.6] (axis cs:75.7929363466869,0) rectangle (axis cs:76.0267797880743,0.00427636539244701);
\draw[draw=none,fill=color0,fill opacity=0.6] (axis cs:76.0267797880743,0) rectangle (axis cs:76.2606232294618,0.00427636539244727);
\addplot [thick, black]
table {%
70.1222328930414 0.00111592147405805
70.1286700248113 0.00114433777213781
70.1351071565812 0.00117341571714936
70.1415442883512 0.00120316901217809
70.1479814201211 0.00123361159980918
70.154418551891 0.00126475766521881
70.1608556836609 0.00129662163927844
70.1672928154309 0.00132921820167227
70.1737299472008 0.00136256228402616
70.1801670789707 0.00139666907304857
70.1866042107406 0.00143155401368184
70.1930413425106 0.00146723281226385
70.1994784742805 0.00150372143969809
70.2059156060504 0.00154103613463286
70.2123527378203 0.00157919340664766
70.2187898695903 0.00161821003944673
70.2252270013602 0.00165810309405776
70.2316641331301 0.00169888991203635
70.2381012649 0.0017405881186741
70.24453839667 0.00178321562621047
70.2509755284399 0.0018267906370459
70.2574126602098 0.00187133164695702
70.2638497919797 0.00191685744831217
70.2702869237497 0.00196338713328521
70.2767240555196 0.00201094009706857
70.2831611872895 0.00205953604108271
70.2895983190594 0.00210919497618234
70.2960354508294 0.00215993722585633
70.3024725825993 0.00221178342942252
70.3089097143692 0.00226475454521439
70.3153468461391 0.00231887185375977
70.321783977909 0.00237415696094857
70.328221109679 0.0024306318011905
70.3346582414489 0.00248831864055967
70.3410953732188 0.00254724007992635
70.3475325049887 0.00260741905807226
70.3539696367587 0.00266887885479089
70.3604067685286 0.00273164309396919
70.3668439002985 0.00279573574665098
70.3732810320684 0.00286118113407825
70.3797181638384 0.00292800393071192
70.3861552956083 0.00299622916722804
70.3925924273782 0.00306588223349
70.3990295591481 0.00313698888149247
70.4054666909181 0.00320957522827858
70.411903822688 0.0032836677588274
70.4183409544579 0.00335929332890862
70.4247780862278 0.00343647916790572
70.4312152179978 0.00351525288160358
70.4376523497677 0.00359564245494075
70.4440894815376 0.00367767625472181
70.4505266133075 0.00376138303229168
70.4569637450775 0.00384679192616733
70.4634008768474 0.00393393246462713
70.4698380086173 0.00402283456825312
70.4762751403872 0.00411352855242817
70.4827122721572 0.00420604512978303
70.4891494039271 0.00430041541259379
70.495586535697 0.0043966709151246
70.5020236674669 0.00449484355591772
70.5084607992369 0.00459496566002583
70.5148979310068 0.00469706996118693
70.5213350627767 0.00480118960393648
70.5277721945466 0.00490735814565893
70.5342093263166 0.00501560955857454
70.5406464580865 0.00512597823165743
70.5470835898564 0.00523849897248709
70.5535207216263 0.00535320700902751
70.5599578533963 0.00547013799133476
70.5663949851662 0.00558932799318678
70.5728321169361 0.00571081351363833
70.579269248706 0.00583463147849479
70.585706380476 0.00596081924170579
70.5921435122459 0.00608941458667213
70.5985806440158 0.0062204557274691
70.6050177757857 0.00635398130997969
70.6114549075557 0.0064900304129387
70.6178920393256 0.00662864254888092
70.6243291710955 0.00676985766499678
70.6307663028654 0.00691371614388856
70.6372034346354 0.00706025880422849
70.6436405664053 0.00720952690131126
70.6500776981752 0.00736156212750466
70.6565148299451 0.00751640661259332
70.6629519617151 0.0076741029240098
70.669389093485 0.00783469406695735
70.6758262252549 0.00799822348441635
70.6822633570248 0.00816473505703621
70.6887004887948 0.00833427310290476
70.6951376205647 0.00850688237719966
70.7015747523346 0.00868260807171388
70.7080118841045 0.00886149581425712
70.7144490158745 0.00904359166792475
70.7208861476444 0.00922894213023936
70.7273232794143 0.00941759413215653
70.7337604111842 0.00960959503693704
70.7401975429541 0.00980499263887649
70.7466346747241 0.0100038351618983
70.753071806494 0.010206171258001
70.7595089382639 0.0104120500055622
70.7659460700338 0.0106215209074906
70.7723832018038 0.0108346338892311
70.7788203335737 0.0110514392966154
70.7852574653436 0.0112719878935595
70.7916945971135 0.0114963308595985
70.7981317288835 0.0117245197872666
70.8045688606534 0.0119566066793141
70.8110059924233 0.0121926439457553
70.8174431241932 0.0124326844007548
70.8238802559632 0.0126767812593407
70.8303173877331 0.0129249881339501
70.836754519503 0.0131773590307948
70.8431916512729 0.0134339483460562
70.8496287830429 0.0136948108618995
70.8560659148128 0.013960001742309
70.8625030465827 0.0142295765287366
70.8689401783526 0.0145035911355699
70.8753773101226 0.0147821018454103
70.8818144418925 0.0150651653041659
70.8882515736624 0.0153528385159458
70.8946887054323 0.0156451788377688
70.9011258372023 0.0159422439740718
70.9075629689722 0.0162440919710256
70.9140001007421 0.0165507812106449
70.920437232512 0.0168623704047024
70.926874364282 0.0171789185884409
70.9333114960519 0.017500485114074
70.9397486278218 0.0178271296440871
70.9461857595917 0.0181589121443261
70.9526228913617 0.0184958928768802
70.9590600231316 0.0188381323927463
70.9654971549015 0.0191856915242866
70.9719342866714 0.019538631377468
70.9783714184414 0.019897013323889
70.9848085502113 0.0202608989925818
70.9912456819812 0.0206303502616025
70.9976828137511 0.0210054292493967
71.0041199455211 0.0213861983059479
71.010557077291 0.0217727200036949
71.0169942090609 0.0221650571282338
71.0234313408308 0.0225632726687888
71.0298684726008 0.0229674298084635
71.0363056043707 0.0233775919142549
71.0427427361406 0.0237938225268484
71.0491798679105 0.0242161853501832
71.0556169996805 0.0246447442407807
71.0620541314504 0.0250795631968509
71.0684912632203 0.0255207063471636
71.0749283949902 0.0259682379396923
71.0813655267602 0.0264222223300186
71.0878026585301 0.0268827239695117
71.0942397903 0.0273498073932721
71.1006769220699 0.0278235372078463
71.1071140538399 0.0283039780787001
71.1135511856098 0.0287911947174682
71.1199883173797 0.0292852518689654
71.1264254491496 0.029786214297971
71.1328625809196 0.0302941467757703
71.1392997126895 0.0308091140664743
71.1457368444594 0.0313311809131016
71.1521739762293 0.0318604120234349
71.1586111079992 0.0323968720556363
71.1650482397692 0.0329406256036435
71.1714853715391 0.0334917371823295
71.177922503309 0.0340502712124422
71.1843596350789 0.034616292005303
71.1907967668489 0.0351898637472912
71.1972338986188 0.0357710504841029
71.2036710303887 0.0363599161047748
71.2101081621586 0.0369565243254978
71.2165452939286 0.0375609386732029
71.2229824256985 0.0381732224689345
71.2294195574684 0.0387934388109935
71.2358566892383 0.039421650557878
71.2422938210083 0.0400579203110013
71.2487309527782 0.0407023103972069
71.2551680845481 0.0413548828510601
71.261605216318 0.0420156993969454
71.268042348088 0.0426848214309509
71.2744794798579 0.0433623100025589
71.2809166116278 0.0440482257961217
71.2873537433977 0.0447426291121546
71.2937908751677 0.0454455798484266
71.3002280069376 0.0461571374808678
71.3066651387075 0.0468773610442736
71.3131022704774 0.0476063091128376
71.3195394022474 0.0483440397805027
71.3259765340173 0.0490906106411198
71.3324136657872 0.0498460787684473
71.3388507975571 0.0506105006959699
71.3452879293271 0.0513839323965601
71.351725061097 0.0521664292619596
71.3581621928669 0.0529580460821181
71.3645993246368 0.0537588370243662
71.3710364564068 0.0545688556124486
71.3774735881767 0.0553881547053928
71.3839107199466 0.0562167864762535
71.3903478517165 0.0570548023907089
71.3967849834865 0.0579022531855351
71.4032221152564 0.0587591888469355
71.4096592470263 0.0596256585887644
71.4160963787962 0.0605017108306229
71.4225335105662 0.0613873931758538
71.4289706423361 0.0622827523894102
71.435407774106 0.0631878343756416
71.4418449058759 0.0641026841559732
71.4482820376459 0.0650273458465067
71.4547191694158 0.0659618626355161
71.4611563011857 0.0669062767608843
71.4675934329556 0.0678606294874674
71.4740305647256 0.0688249610843733
71.4804676964955 0.0697993108022025
71.4869048282654 0.0707837168502226
71.4933419600353 0.0717782163735084
71.4997790918053 0.0727828454300205
71.5062162235752 0.0737976389676727
71.5126533553451 0.0748226308013579
71.519090487115 0.07585785358997
71.525527618885 0.0769033388133892
71.5319647506549 0.0779591167494848
71.5384018824248 0.0790252164511059
71.5448390141947 0.080101665723096
71.5512761459646 0.0811884910993002
71.5577132777346 0.08228571781962
71.5641504095045 0.083393369807086
71.5705875412744 0.084511469644985
71.5770246730443 0.0856400385540092
71.5834618048143 0.086779096369484
71.5898989365842 0.0879286615186582
71.5963360683541 0.0890887509980436
71.602773200124 0.0902593803508594
71.609210331894 0.091440563644549
71.6156474636639 0.0926323134484087
71.6220845954338 0.0938346408112951
71.6285217272037 0.095047555239471
71.6349588589737 0.0962710646745578
71.6413959907436 0.0975051754716342
71.6478331225135 0.0987498923774461
71.6542702542834 0.100005218508792
71.6607073860534 0.101271155331048
71.6671445178233 0.102547702636874
71.6735816495932 0.103834858525069
71.6800187813631 0.105132619379632
71.6864559131331 0.106440979849004
71.692893044903 0.107759932825525
71.6993301766729 0.109089469425073
71.7057673084428 0.110429578966946
71.7122044402128 0.111780248953979
71.7186415719827 0.113141465052864
71.7250787037526 0.11451321107475
71.7315158355225 0.115895468956084
71.7379529672925 0.117288218739731
71.7443900990624 0.118691438556335
71.7508272308323 0.120105104606007
71.7572643626022 0.12152919114027
71.7637014943722 0.122963670444341
71.7701386261421 0.124408512819678
71.776575757912 0.125863686566894
71.7830128896819 0.127329157968972
71.7894500214519 0.128804891274838
71.7958871532218 0.130290848683257
71.8023242849917 0.131786990327114
71.8087614167616 0.13329327425804
71.8151985485316 0.13480965643144
71.8216356803015 0.136336090691866
71.8280728120714 0.137872528758822
71.8345099438413 0.139418920212952
71.8409470756113 0.140975212482651
71.8473842073812 0.14254135083108
71.8538213391511 0.144117278343621
71.860258470921 0.145702935915789
71.866695602691 0.147298262241546
71.8731327344609 0.148903193802114
71.8795698662308 0.150517664855229
71.8860069980007 0.152141607424885
71.8924441297707 0.153774951291532
71.8988812615406 0.155417623982793
71.9053183933105 0.157069550764656
71.9117555250804 0.158730654633203
71.9181926568504 0.160400856306804
71.9246297886203 0.162080074218881
71.9310669203902 0.163768224511171
71.9375040521601 0.165465221027556
71.9439411839301 0.167170975308395
71.9503783157 0.16888539658545
71.9568154474699 0.170608391777345
71.9632525792398 0.172339865485623
71.9696897110097 0.17407971999133
71.9761268427797 0.17582785525222
71.9825639745496 0.177584168900546
71.9890011063195 0.179348556241412
71.9954382380894 0.181120910251765
72.0018753698594 0.182901121579969
72.0083125016293 0.18468907854602
72.0147496333992 0.186484667142337
72.0211867651691 0.18828777103522
72.0276238969391 0.190098271566916
72.034061028709 0.191916047758345
72.0404981604789 0.193740976312426
72.0469352922488 0.195572931618092
72.0533724240188 0.197411785754937
72.0598095557887 0.19925740849853
72.0662466875586 0.201109667326368
72.0726838193285 0.202968427424521
72.0791209510985 0.20483355169493
72.0855580828684 0.206704900763395
72.0919952146383 0.208582332988212
72.0984323464082 0.210465704469512
72.1048694781782 0.212354869059299
72.1113066099481 0.214249678372124
72.117743741718 0.216149981796499
72.1241808734879 0.218055626506979
72.1306180052579 0.219966457476956
72.1370551370278 0.221882317492113
72.1434922687977 0.223803047164619
72.1499294005676 0.225728484948
72.1563665323376 0.227658467152728
72.1628036641075 0.229592827962484
72.1692407958774 0.231531399451153
72.1756779276473 0.23347401160051
72.1821150594173 0.235420492318627
72.1885521911872 0.237370667458946
72.1949893229571 0.239324360840097
72.201426454727 0.241281394266399
72.207863586497 0.243241587549083
72.2143007182669 0.245204758528177
72.2207378500368 0.247170723095143
72.2271749818067 0.24913929521617
72.2336121135767 0.251110286956212
72.2400492453466 0.25308350850366
72.2464863771165 0.255058768195758
72.2529235088864 0.257035872544706
72.2593606406564 0.259014626264411
72.2657977724263 0.26099483229797
72.2722349041962 0.262976291845801
72.2786720359661 0.264958804394488
72.2851091677361 0.266942167746248
72.291546299506 0.268926178049112
72.2979834312759 0.27091062982775
72.3044205630458 0.272895316014978
72.3108576948158 0.274880027983876
72.3172948265857 0.276864555580601
72.3237319583556 0.278848687157818
72.3301690901255 0.280832209608794
72.3366062218955 0.282814908402077
72.3430433536654 0.284796567616855
72.3494804854353 0.286776969978892
72.3559176172052 0.288755896897114
72.3623547489752 0.290733128500749
72.3687918807451 0.292708443677103
72.375229012515 0.29468162010993
72.3816661442849 0.296652434318334
72.3881032760548 0.298620661696297
72.3945404078248 0.300586076552732
72.4009775395947 0.302548452152132
72.4074146713646 0.304507560755703
72.4138518031345 0.306463173663092
72.4202889349045 0.308415061254601
72.4267260666744 0.310362993033956
72.4331631984443 0.312306737671527
72.4396003302142 0.314246063048091
72.4460374619842 0.316180736299056
72.4524745937541 0.31811052385918
72.458911725524 0.320035191507707
72.4653488572939 0.321954504414006
72.4717859890639 0.323868227183609
72.4782231208338 0.325776123904716
72.4846602526037 0.327677958195058
72.4910973843736 0.329573493249211
72.4975345161436 0.331462491886297
72.5039716479135 0.333344716598017
72.5104087796834 0.335219929597097
72.5168459114533 0.337087892866058
72.5232830432233 0.338948368206342
72.5297201749932 0.340801117287728
72.5361573067631 0.34264590169809
72.542594438533 0.344482482993432
72.549031570303 0.346310622748224
72.5554687020729 0.348130082605962
72.5619058338428 0.349940624330027
72.5683429656127 0.351742009854751
72.5747800973827 0.353534001336728
72.5812172291526 0.355316361206297
72.5876543609225 0.357088852219247
72.5940914926924 0.358851237508687
72.6005286244624 0.360603280637089
72.6069657562323 0.36234474564844
72.6134028880022 0.36407539712056
72.6198400197721 0.365795000217518
72.6262771515421 0.367503320742152
72.632714283312 0.369200125188646
72.6391514150819 0.370885180795199
72.6455885468518 0.37255825559674
72.6520256786218 0.374219118477643
72.6584628103917 0.375867539224493
72.6648999421616 0.377503288578828
72.6713370739315 0.379126138289886
72.6777742057015 0.380735861167274
72.6842113374714 0.382332231133628
72.6906484692413 0.383915023277175
72.6970856010112 0.385484013904236
72.7035227327812 0.387038980591586
72.7099598645511 0.388579702238728
72.716396996321 0.390105959120014
72.7228341280909 0.391617532936624
72.7292712598609 0.393114206868346
72.7357083916308 0.394595765625195
72.7421455234007 0.396061995498811
72.7485826551706 0.39751268441365
72.7550197869406 0.398947621977902
72.7614569187105 0.400366599534184
72.7678940504804 0.401769410209955
72.7743311822503 0.403155848967613
72.7807683140202 0.404525712654314
72.7872054457902 0.405878800051448
72.7936425775601 0.407214911923792
72.80007970933 0.408533851068272
72.8065168410999 0.409835422362378
72.8129539728699 0.411119432812177
72.8193911046398 0.412385691599926
72.8258282364097 0.413634010131233
72.8322653681796 0.414864202081812
72.8387024999496 0.416076083443754
72.8451396317195 0.417269472571348
72.8515767634894 0.418444190226381
72.8580138952593 0.419600059622962
72.8644510270293 0.420736906471813
72.8708881587992 0.421854559024027
72.8773252905691 0.422952848114262
72.883762422339 0.424031607203381
72.890199554109 0.42509067242052
72.8966366858789 0.426129882604523
72.9030738176488 0.427149079344805
72.9095109494187 0.42814810702156
72.9159480811887 0.429126812845356
72.9223852129586 0.430085046896044
72.9288223447285 0.431022662161022
72.9352594764984 0.431939514572808
72.9416966082684 0.432835463045927
72.9481337400383 0.433710369513067
72.9545708718082 0.434564098960541
72.9610080035781 0.435396519462994
72.9674451353481 0.436207502217377
72.973882267118 0.436996921576152
72.9803193988879 0.437764655079728
72.9867565306578 0.438510583488118
72.9931936624278 0.439234590811808
72.9996307941977 0.439936564341795
73.0060679259676 0.44061639467884
73.0125050577375 0.441273975761873
73.0189421895075 0.441909204895575
73.0253793212774 0.442521982777097
73.0318164530473 0.443112213521934
73.0382535848172 0.443679804688931
73.0446907165872 0.444224667304409
73.0511278483571 0.444746715885409
73.057564980127 0.445245868462043
73.0640021118969 0.445722046598947
73.0704392436669 0.446175175415815
73.0768763754368 0.446605183607033
73.0833135072067 0.44701200346037
73.0897506389766 0.447395570874755
73.0961877707466 0.447755825377109
73.1026249025165 0.448092710138231
73.1090620342864 0.448406171987743
73.1154991660563 0.448696161428075
73.1219362978263 0.448962632647491
73.1283734295962 0.449205543532149
73.1348105613661 0.449424855677198
73.141247693136 0.449620534396893
73.147684824906 0.449792548733741
73.1541219566759 0.449940871466662
73.1605590884458 0.450065479118171
73.1669962202157 0.450166351960567
73.1734333519857 0.450243474021142
73.1798704837556 0.450296833086394
73.1863076155255 0.450326420705252
73.1927447472954 0.450332232191309
73.1991818790653 0.450314266624052
73.2056190108353 0.450272526849114
73.2120561426052 0.450207019477516
73.2184932743751 0.450117754883924
73.224930406145 0.450004747203911
73.231367537915 0.449868014330229
73.2378046696849 0.449707577908079
73.2442418014548 0.449523463329412
73.2506789332247 0.44931569972622
73.2571160649947 0.449084319962862
73.2635531967646 0.448829360627394
73.2699903285345 0.448550862021933
73.2764274603044 0.448248868152031
73.2828645920744 0.447923426715094
73.2893017238443 0.447574589087822
73.2957388556142 0.447202410312693
73.3021759873841 0.446806949083489
73.3086131191541 0.446388267729868
73.315050250924 0.44594643220099
73.3214873826939 0.44548151204821
73.3279245144638 0.444993580406824
73.3343616462338 0.444482713976898
73.3407987780037 0.443948993003175
73.3472359097736 0.443392501254068
73.3536730415435 0.442813325999751
73.3601101733135 0.442211557989342
73.3665473050834 0.441587291427217
73.3729844368533 0.440940623948424
73.3794215686232 0.440271656593243
73.3858587003932 0.439580493780866
73.3922958321631 0.438867243282245
73.398732963933 0.438132016192084
73.4051700957029 0.437374926900003
73.4116072274729 0.436596093060877
73.4180443592428 0.435795635564374
73.4244814910127 0.434973678503685
73.4309186227826 0.43413034914346
73.4373557545526 0.433265777886989
73.4437928863225 0.432380098242589
73.4502300180924 0.431473446789266
73.4566671498623 0.430545963141607
73.4631042816323 0.429597789913978
73.4695414134022 0.428629072683978
73.4759785451721 0.42763995995521
73.482415676942 0.426630603119342
73.488852808712 0.425601156417528
73.4952899404819 0.424551776901129
73.5017270722518 0.423482624391815
73.5081642040217 0.422393861441012
73.5146013357917 0.421285653288757
73.5210384675616 0.420158167821919
73.5274755993315 0.419011575531859
73.5339127311014 0.417846049471482
73.5403498628714 0.416661765211772
73.5467869946413 0.41545890079775
73.5532241264112 0.414237636703904
73.5596612581811 0.412998155789135
73.5660983899511 0.411740643251176
73.572535521721 0.410465286580543
73.5789726534909 0.40917227551401
73.5854097852608 0.407861801987656
73.5918469170308 0.406534060089459
73.5982840488007 0.405189246011482
73.6047211805706 0.403827558001649
73.6111583123405 0.402449196315158
73.6175954441104 0.40105436316551
73.6240325758804 0.399643262675191
73.6304697076503 0.398216100826012
73.6369068394202 0.396773085409165
73.6433439711901 0.395314425974936
73.6497811029601 0.39384033378217
73.65621823473 0.392351021747437
73.6626553664999 0.390846704393994
73.6690924982698 0.389327597800472
73.6755296300398 0.387793919549366
73.6819667618097 0.386245888675338
73.6884038935796 0.384683725613326
73.6948410253495 0.383107652146491
73.7012781571195 0.381517891354015
73.7077152888894 0.37991466755879
73.7141524206593 0.37829820627497
73.7205895524292 0.376668734155444
73.7270266841992 0.375026478939214
73.7334638159691 0.373371669398743
73.739900947739 0.371704535287226
73.7463380795089 0.370025307285858
73.7527752112789 0.368334216951064
73.7592123430488 0.366631496661774
73.7656494748187 0.36491737956669
73.7720866065886 0.363192099531605
73.7785237383586 0.361455891086764
73.7849608701285 0.35970898937433
73.7913980018984 0.357951630095905
73.7978351336683 0.356184049460171
73.8042722654383 0.354406484130641
73.8107093972082 0.352619171173574
73.8171465289781 0.350822348006014
73.823583660748 0.349016252343995
73.830020792518 0.347201122150965
73.8364579242879 0.345377195586371
73.8428950560578 0.343544710954483
73.8493321878277 0.341703906653418
73.8557693195977 0.339855021124457
73.8622064513676 0.337998292801572
73.8686435831375 0.336133960061257
73.8750807149074 0.33426226117262
73.8815178466774 0.332383434247818
73.8879549784473 0.33049771719277
73.8943921102172 0.328605347658219
73.9008292419871 0.326706562991116
73.9072663737571 0.324801600186407
73.913703505527 0.322890695839137
73.9201406372969 0.320974086096975
73.9265777690668 0.319052006613103
73.9330149008368 0.317124692499551
73.9394520326067 0.315192378280928
73.9458891643766 0.313255297848573
73.9523262961465 0.311313684415194
73.9587634279165 0.309367770469925
73.9652005596864 0.307417787733867
73.9716376914563 0.305463967116088
73.9780748232262 0.303506538670159
73.9845119549962 0.301545731551139
73.9909490867661 0.299581773973103
73.997386218536 0.297614893167161
74.0038233503059 0.295645315340057
74.0102604820759 0.293673265633257
74.0166976138458 0.29169896808262
74.0231347456157 0.289722645578597
74.0295718773856 0.287744519827044
74.0360090091555 0.285764811310569
74.0424461409255 0.283783739250483
74.0488832726954 0.281801521569326
74.0553204044653 0.279818374854033
74.0617575362352 0.277834514319666
74.0681946680052 0.275850153773768
74.0746317997751 0.273865505581368
74.081068931545 0.271880780630585
74.0875060633149 0.269896188298878
74.0939431950849 0.267911936419921
74.1003803268548 0.265928231251167
74.1068174586247 0.263945277442017
74.1132545903946 0.261963278002676
74.1196917221646 0.259982434273644
74.1261288539345 0.258002945895919
74.1325659857044 0.256025010781828
74.1390031174743 0.25404882508656
74.1454402492443 0.252074583180348
74.1518773810142 0.250102477621389
74.1583145127841 0.248132699129391
74.164751644554 0.246165436559851
74.171188776324 0.244200876878991
74.1776259080939 0.24223920513943
74.1840630398638 0.24028060445652
74.1905001716337 0.238325255985391
74.1969373034037 0.236373338898687
74.2033744351736 0.234425030365033
74.2098115669435 0.232480505528177
74.2162486987134 0.230539937486829
74.2226858304834 0.228603497275242
74.2291229622533 0.226671353844462
74.2355600940232 0.224743674044299
74.2419972257931 0.222820622605985
74.2484343575631 0.220902362125571
74.254871489333 0.218989053047992
74.2613086211029 0.217080853651851
74.2677457528728 0.215177920034888
74.2741828846428 0.21328040610018
74.2806200164127 0.211388463543008
74.2870571481826 0.209502241838436
74.2934942799525 0.20762188822956
74.2999314117225 0.205747547716493
74.3063685434924 0.203879363046001
74.3128056752623 0.202017474701814
74.3192428070322 0.200162020895676
74.3256799388022 0.19831313755901
74.3321170705721 0.196470958335299
74.338554202342 0.194635614573124
74.3449913341119 0.192807235319859
74.3514284658819 0.190985947316059
74.3578655976518 0.189171874990486
74.3643027294217 0.187365140455772
74.3707398611916 0.18556586350478
74.3771769929616 0.183774161607567
74.3836141247315 0.181990149909001
74.3900512565014 0.180213941227004
74.3964883882713 0.178445646051449
74.4029255200413 0.176685372543645
74.4093626518112 0.174933226536465
74.4157997835811 0.173189311535055
74.422236915351 0.171453728718194
74.4286740471209 0.169726576940204
74.4351111788909 0.16800795273348
74.4415483106608 0.166297950311585
74.4479854424307 0.164596661572948
74.4544225742006 0.162904176105117
74.4608597059706 0.161220581189554
74.4672968377405 0.159545961807042
74.4737339695104 0.157880400643585
74.4801711012803 0.156223978096885
74.4866082330503 0.154576772283322
74.4930453648202 0.152938859045497
74.4994824965901 0.151310311960266
74.50591962836 0.149691202347291
74.51235676013 0.148081599278085
74.5187938918999 0.146481569585575
74.5252310236698 0.144891177874125
74.5316681554397 0.143310486530048
74.5381052872097 0.141739555732573
74.5445424189796 0.140178443465304
74.5509795507495 0.138627205528091
74.5574166825194 0.137085895549375
74.5638538142894 0.135554564998934
74.5702909460593 0.134033263201098
74.5767280778292 0.132522037348342
74.5831652095991 0.131020932515307
74.5896023413691 0.129529991673199
74.596039473139 0.128049255704609
74.6024766049089 0.126578763418689
74.6089137366788 0.125118551566693
74.6153508684488 0.12366865485791
74.6217880002187 0.122229105975923
74.6282251319886 0.120799935595225
74.6346622637585 0.119381172398148
74.6410993955285 0.117972843092161
74.6475365272984 0.116574972427438
74.6539736590683 0.115187583214759
74.6604107908382 0.11381069634369
74.6668479226082 0.112444330801078
74.6732850543781 0.111088503689801
74.679722186148 0.109743230247795
74.6861593179179 0.108408523867337
74.6925964496879 0.107084396114596
74.6990335814578 0.10577085674941
74.7054707132277 0.104467913745289
74.7119078449976 0.103175573309668
74.7183449767676 0.101893839904354
74.7247821085375 0.10062271626618
74.7312192403074 0.0993622034278666
74.7376563720773 0.0981123007390494
74.7440935038473 0.0968730058875184
74.7505306356172 0.0956443149205981
74.7569677673871 0.09442622226669
74.763404899157 0.0932187207569917
74.769842030927 0.0920218016473311
74.7762791626969 0.0908354546401517
74.7827162944668 0.0896596679066084
74.7891534262367 0.0884944281088054
74.7955905580067 0.0873397204221179
74.8020276897766 0.0861955285576294
74.8084648215465 0.0850618347846448
74.8149019533164 0.0839386199533137
74.8213390850864 0.0828258635173031
74.8277762168563 0.0817235435565525
74.8342133486262 0.0806316368000736
74.8406504803961 0.0795501186488258
74.847087612166 0.0784789631986132
74.853524743936 0.0774181432630169
74.8599618757059 0.0763676303963773
74.8663990074758 0.0753273949167726
74.8728361392457 0.074297405929024
74.8792732710157 0.0732776313476934
74.8857104027856 0.072268037920101
74.8921475345555 0.0712685912493114
74.8985846663254 0.0702792558171175
74.9050217980954 0.069299995006987
74.9114589298653 0.0683307711270015
74.9178960616352 0.0673715454327361
74.9243331934051 0.0664222781501093
74.9307703251751 0.0654829284981691
74.937207456945 0.064553454711844
74.9436445887149 0.0636338140646096
74.9500817204848 0.062723962891099
74.9565188522548 0.0618238566096255
74.9629559840247 0.0609334497446431
74.9693931157946 0.0600526959491013
74.9758302475645 0.0591815480267174
74.9822673793345 0.0583199579541389
74.9887045111044 0.0574678769030213
74.9951416428743 0.0566252552619779
75.0015787746442 0.0557920426584157
75.0080159064142 0.0549681879802677
75.0144530381841 0.0541536393975805
75.020890169954 0.0533483443839842
75.0273273017239 0.0525522497380152
75.0337644334939 0.0517653016043174
75.0402015652638 0.0509874454946822
75.0466386970337 0.0502186263089486
75.0530758288036 0.0494587883557414
75.0595129605736 0.0487078753730684
75.0659500923435 0.047965830548739
75.0723872241134 0.0472325965406278
75.0788243558833 0.0465081154967591
75.0852614876533 0.0457923290752343
75.0916986194232 0.0450851784639683
75.0981357511931 0.0443866044002456
75.104572882963 0.0436965471901081
75.111010014733 0.0430149467275384
75.1174471465029 0.0423417425134631
75.1238842782728 0.0416768736745619
75.1303214100427 0.0410202789818738
75.1367585418127 0.0403718968692194
75.1431956735826 0.0397316654514086
75.1496328053525 0.0390995225422449
75.1560699371224 0.0384754056723355
75.1625070688924 0.0378592521066796
75.1689442006623 0.0372509988620538
75.1753813324322 0.0366505827241758
75.1818184642021 0.0360579402646682
75.1882555959721 0.0354730078577905
75.194692727742 0.0348957216969616
75.2011298595119 0.0343260178110538
75.2075669912818 0.0337638320804786
75.2140041230518 0.0332091002530351
75.2204412548217 0.0326617579595431
75.2268783865916 0.0321217407292422
75.2333155183615 0.0315889840049763
75.2397526501314 0.0310634231581391
75.2461897819014 0.0305449935033914
75.2526269136713 0.0300336303131586
75.2590640454412 0.0295292688318867
75.2655011772111 0.0290318442900739
75.2719383089811 0.0285412919180637
75.278375440751 0.0280575469596179
75.2848125725209 0.0275805446852448
75.2912497042908 0.0271102204053026
75.2976868360608 0.0266465094828633
75.3041239678307 0.0261893473463534
75.3105610996006 0.0257386695019525
75.3169982313705 0.0252944115457644
75.3234353631405 0.02485650917575
75.3298724949104 0.0244248982034375
75.3363096266803 0.0239995145653906
75.3427467584502 0.0235802943344517
75.3491838902202 0.0231671737307468
75.3556210219901 0.0227600891324712
75.36205815376 0.0223589770864338
75.3684952855299 0.0219637743183794
75.3749324172999 0.0215744177430765
75.3813695490698 0.0211908444741864
75.3878066808397 0.0208129918338977
75.3942438126096 0.0204407973623356
75.4006809443796 0.0200741988267538
75.4071180761495 0.0197131342304947
75.4135552079194 0.0193575418217324
75.4199923396893 0.0190073601019895
75.4264294714593 0.0186625278344427
75.4328666032292 0.0183229840520012
75.4393037349991 0.0179886680651737
75.445740866769 0.0176595194697152
75.452177998539 0.0173354781540665
75.4586151303089 0.017016484306575
75.4650522620788 0.0167024784225081
75.4714893938487 0.016393401310854
75.4779265256187 0.0160891941009219
75.4843636573886 0.0157897982487293
75.4908007891585 0.0154951555431874
75.4972379209284 0.0152052081120887
75.5036750526984 0.014919898427889
75.5101121844683 0.0146391693132936
75.5165493162382 0.0143629639466479
75.5229864480081 0.0140912258671306
75.5294235797781 0.01382389897976
75.535860711548 0.0135609275602057
75.5422978433179 0.0133022562594118
75.5487349750878 0.0130478301080387
75.5551721068578 0.0127975945207166
75.5616092386277 0.0125514953001199
75.5680463703976 0.0123094786408586
75.5744835021675 0.0120714911331987
75.5809206339375 0.0118374797666017
75.5873577657074 0.0116073919330944
75.5937948974773 0.0113811754304655
75.6002320292472 0.0111587784652991
75.6066691610172 0.010940149655837
75.6131062927871 0.0107252380346809
75.619543424557 0.0105139930513302
75.6259805563269 0.0103063645745658
75.6324176880968 0.0101023028946737
75.6388548198668 0.0099017587255135
75.6452919516367 0.00970468320644086
75.6517290834066 0.0095110279040748
75.6581662151765 0.00932074481392163
75.6646033469465 0.00913378636185164
75.6710404787164 0.008950105405438
75.6774776104863 0.00876965523515187
75.6839147422562 0.00859238957542289
75.6903518740262 0.00841826258556249
75.6967890057961 0.00824722886055885
75.703226137566 0.00807924343173801
75.7096632693359 0.00791426176729988
75.7161004011059 0.00775223977272702
75.7225375328758 0.00759313379107435
75.7289746646457 0.00743690060313509
75.7354117964156 0.00728349742749108
75.7418489281856 0.00713288192044556
75.7482860599555 0.00698501217584637
75.7547231917254 0.00683984672479516
75.7611603234953 0.00669734453525042
75.7675974552653 0.0065574650115226
75.7740345870352 0.00642016799366898
75.7804717188051 0.00628541375678449
75.786908850575 0.0061531630101938
75.793345982345 0.00602337689654948
75.7997831141149 0.00589601699083315
75.8062202458848 0.0057710452992663
75.8126573776547 0.00564842425812998
75.8190945094247 0.00552811673249979
75.8255316411946 0.00541008601489301
75.8319687729645 0.00529429582383449
75.8384059047344 0.00518071030234039
75.8448430365044 0.00506929401632589
75.8512801682743 0.00496001195293416
75.8577173000442 0.00485282951879267
75.8641544318141 0.00474771253819606
75.8705915635841 0.00464462725122156
75.877028695354 0.00454354031177421
75.8834658271239 0.00444441878556789
75.8899029588938 0.00434723014804128
75.8963400906638 0.00425194228221453
75.9027772224337 0.00415852347648408
75.9092143542036 0.00406694242236124
75.9156514859735 0.00397716821215393
75.9220886177435 0.00388917033659678
75.9285257495134 0.00380291868242762
75.9349628812833 0.00371838352991428
75.9414000130532 0.00363553555033502
75.9478371448232 0.00355434580341107
75.9542742765931 0.00347478573469592
75.960711408363 0.00339682717292112
75.9671485401329 0.00332044232730313
75.9735856719029 0.00324560378480951
75.9800228036728 0.00317228450738888
75.9864599354427 0.00310045782916441
75.9928970672126 0.00303009745359514
75.9993341989826 0.00296117745060351
76.0057713307525 0.00289367225367321
76.0122084625224 0.00282755665691731
76.0186455942923 0.00276280581212044
76.0250827260622 0.00269939522575382
76.0315198578322 0.00263730075596601
76.0379569896021 0.00257649860955223
76.044394121372 0.00251696533890081
76.0508312531419 0.0024586778389207
76.0572683849119 0.00240161334394973
76.0637055166818 0.00234574942464705
76.0701426484517 0.00229106398486862
76.0765797802216 0.00223753525852914
76.0830169119916 0.00218514180645019
76.0894540437615 0.00213386251319783
76.0958911755314 0.00208367658390852
76.1023283073013 0.00203456354110654
76.1087654390713 0.00198650322151267
76.1152025708412 0.00193947577284718
76.1216397026111 0.001893461650626
76.128076834381 0.00184844161495304
76.134513966151 0.00180439672730845
76.1409510979209 0.00176130834733554
76.1473882296908 0.00171915812962541
76.1538253614607 0.001677928020502
76.1602624932307 0.00163760025480723
76.1666996250006 0.00159815735268893
76.1731367567705 0.00155958211639058
76.1795738885404 0.00152185762704478
76.1860110203104 0.00148496724147208
76.1924481520803 0.00144889458898432
76.1988852838502 0.00141362356819497
76.2053224156201 0.001379138343836
76.2117595473901 0.00134542334358369
76.21819667916 0.00131246325489232
76.2246338109299 0.00128024302183803
76.2310709426998 0.00124874784197243
76.2375080744698 0.00121796316318814
76.2439452062397 0.00118787468059524
76.2503823380096 0.0011584683334107
76.2568194697795 0.00112973030186051
76.2632566015495 0.00110164700409618
76.2696937333194 0.00107420509312509
76.2761308650893 0.00104739145375606
76.2825679968592 0.00102119319956032
76.2890051286292 0.000995597669849079
76.2954422603991 0.000970592426667228
76.301879392169 0.00094616525180461
76.3083165239389 0.000922304143824658
76.3147536557089 0.000898997315111775
76.3211907874788 0.000876233188936876
76.3276279192487 0.000854000396542061
76.3340650510186 0.000832287774245277
76.3405021827886 0.00081108436056442
76.3469393145585 0.000790379393362068
76.3533764463284 0.000770162307010669
76.3598135780983 0.000750422729579246
76.3662507098683 0.000731150480041066
76.3726878416382 0.000712335565503328
76.3791249734081 0.000693968178458637
76.385562105178 0.000676038694059251
76.391999236948 0.000658537667413523
76.3984363687179 0.000641455830905478
76.4048735004878 0.000624784091537294
76.4113106322577 0.000608513528295518
76.4177477640276 0.000592635389540516
76.4241848957976 0.000577141090419735
76.4306220275675 0.000562022210305232
76.4370591593374 0.00054727049025502
76.4434962911073 0.000532877830498911
76.4499334228773 0.00051883628794865
76.4563705546472 0.000505138073732943
76.4628076864171 0.000491775550756892
76.469244818187 0.000478741231286455
76.475681949957 0.00046602777455766
76.4821190817269 0.00045362798441112
76.4885562134968 0.000441534806951361
76.4949933452667 0.00042974132823148
76.5014304770367 0.000418240771962853
76.5078676088066 0.000407026497250371
76.5143047405765 0.000396091996352696
76.5207418723464 0.000385430892467998
76.5271790041164 0.00037503693754486
76.5336161358863 0.000364904010118771
76.5400532676562 0.000355026113173684
76.5464903994261 0.000345397372029049
76.5529275311961 0.000336012032251984
};
\end{axis}

\end{tikzpicture}

\end{center}
\end{subfigure}
\caption{2000 alpha complexes on 300 points drawn from a uniform distribution.}
\label{fig:300fit}
\end{figure}

What we can take from the plot and the generated data is that we have an average $q$-value of $73.19\%$ and values ranging from  $70\%$ to $76\%$. Meaning that more than two thirds of simplices are part of an apparent pair in all filtrations we constructed, while on average almost three quarters are part of an apparent pair.

The following figure shows a plot where we map the size of the two dimensional point clouds to the average mean $q$-values. We observe an almost monotone downtrend, which indicates that the percentage of simplices in apparent pairs decreases with increasing length of the filtrations. In the following two figures the scattered points are the actually calculated values and the line segments between them are linearly interpolated.

\begin{figure}[H]
%\centering%
\begin{subfigure}[c]{0.95\textwidth}
\begin{center}
% This file was created by tikzplotlib v0.9.2.
\begin{tikzpicture}

\definecolor{color0}{rgb}{0.12156862745098,0.466666666666667,0.705882352941177}
\definecolor{color1}{rgb}{0.976470588235294,0.450980392156863,0.0235294117647059}

\begin{axis}[
tick align=outside,
tick pos=left,
xlabel = $m$,
ylabel = average $q$,
x grid style={white!69.0196078431373!black},
xmin=2.5, xmax=1047.5,
xtick style={color=black},
y grid style={white!69.0196078431373!black},
%´ymin=72, ymax=91,
ymin=72.5198618350407, ymax=74.3989386078973,
ytick style={color=black},
scatter/classes={a={mark=o,draw=black}}
]
\addplot[scatter, scatter src=explicit symbolic, semithick, color0]
table {%
50 74.3135260273129
100 73.9007987162947
150 73.548228008839
200 73.3330687122862
250 73.256039993344
300 73.1981969846523
350 73.0861031936403
400 73.0127258791942
450 72.9523915686429
500 72.9466183308799
550 72.873379839116
600 72.8254219384188
650 72.7987235551989
700 72.7696569674668
750 72.7625372033231
800 72.6942015413158
850 72.6658903806182
900 72.6621935833012
950 72.6374432276665
1000 72.6052744156251
};

\end{axis}

\end{tikzpicture}

\end{center}
\end{subfigure}
\caption{Means of 2000 averaged alpha complexes on differently sized points clouds drawn from a uniform distribution.}
\label{fig:2d_means}
\end{figure}

Since the $q$-values are well fitted by normal distributions, we generated a similar plot for the standard deviations.

\begin{figure}[H]
%\centering%
\begin{subfigure}[c]{0.95\textwidth}
\begin{center}
% This file was created by tikzplotlib v0.9.2.
\begin{tikzpicture}

\definecolor{color0}{rgb}{0.12156862745098,0.466666666666667,0.705882352941177}
\definecolor{color1}{rgb}{0.976470588235294,0.450980392156863,0.0235294117647059}

\begin{axis}[
tick align=outside,
tick pos=left,
ylabel = $\sigma^2$,
xlabel = $m$,
ylabel style={rotate=-90},
x grid style={white!69.0196078431373!black},
xmin=2.5, xmax=1047.5,
xtick style={color=black},
y grid style={white!69.0196078431373!black},
ymin=0.404619119795701, ymax=2.33384993531293,
ytick style={color=black},
scatter/classes={a={mark=o,draw=black}}
]
\addplot[scatter, scatter src=explicit symbolic, semithick, color0]
table {%
50 2.24615762551669
100 1.5792435037505
150 1.27389257927507
200 1.1147961093539
250 1.00194489752104
300 0.883268619119549
350 0.814502452409831
400 0.776380080599139
450 0.745367391163501
500 0.681014417056458
550 0.676891814240733
600 0.656527982762773
650 0.619120834767529
700 0.59424703343731
750 0.564172814777307
800 0.533426280089837
850 0.543871439397632
900 0.509264003967767
950 0.517621741132293
1000 0.492311429591938
};
\end{axis}

\end{tikzpicture}

\end{center}
\end{subfigure}
\caption{Standard deviations of 2000 averaged alpha complexes on differently sized points clouds drawn from a uniform distribution.}
\label{fig:2d_stds}
\end{figure}

As we can see, the larger the complex, the less variance we have with respect to the $q$-values. This could be explained by some local configurations containing lots of apparent pairs while others are containing only a few. The larger the complex, the more likely that we have these evenly distributed, while in a smaller complex the probability for extreme cases is higher. 

Overall this means the larger the complex the fewer of its elements are part of an apparent pair percentage wise, however larger complexes are more similar to one another in that regard than smaller complexes.

We have also calculated an average $q$-value for two-dimensional alpha complexes on 5000 points and got a mean of $72.20$ and a standard deviation of $0.20$. It would be interesting to explore if and to what value these values converge. Or if they are bounded somehow in a random setting. 

As we have previously seen, it is theoretically possible to have a filtration of any length with a single apparent pair between edges and triangles. 

One might expect a similar looking curve when increasing the dimension of the points instead of their number, since an alpha complex constructed on $100$ points in dimension two is expected to be much smaller than one in dimension three. The mean value of the $q$-values however increases substantially.

\begin{figure}[H]
%\centering%
\begin{subfigure}[c]{0.95\textwidth}
\begin{center}
% This file was created by tikzplotlib v0.9.2.
\begin{tikzpicture}

\definecolor{color0}{rgb}{0.12156862745098,0.466666666666667,0.705882352941177}

\begin{axis}[
tick align=outside,
tick pos=left,
xlabel = dimension,
ylabel = average $q$,
ylabel style={rotate=-90},
x grid style={white!69.0196078431373!black},
xmin=1.8, xmax=6.2,
xtick style={color=black},
y grid style={white!69.0196078431373!black},
ymin=72.9311706171997, ymax=100.719766845615,
ytick style={color=black},
scatter/classes={%
    a={mark=o,draw=color0}}
]
\addplot [scatter,only marks,%
    scatter src=explicit symbolic,color0]
table {%
2 74.1942886275822
3 90.1118873880867
4 95.401135865881
5 98.4740519194136
6 99.4566488352329
};
\end{axis}

\end{tikzpicture}

\end{center}
\end{subfigure}
\caption{Means of alpha complexes on 30 points in varying dimensions, averaged over 50 runs.}
\label{fig:means_by-dimension}
\end{figure}

While in dimension two, on average, about three quarters of all simplices are part of an apparent pair on an alpha complex on thirty points, in dimension six we have an average $q$-value of $99.46 \%$. A partial explanation is that with increasing dimension more simplices can be paired with lower or higher dimensional simplices and the number of maximal faces of the complex decreases. 

Note that in the last experiment we averaged the values over $50$ runs instead of $2000$. This is partially caused by the fact that simplicial complexes grow quite fast in higher dimensions and that running time was a limiting factor when performing the experiments. In the remainder of this chapter we will see values averaged over different amounts of tries due to this.

We carried out the same experiments for point clouds sampled from a multivariate Gaussian and a Gaussian mixture model. For the multivariate Gaussian we have taken the point $(0,0)$ as mean and the identity matrix as covariance matrix. For the mixture model we chose $(0,0)$, $(1,1)$ and $(0,3)$ as mean values and each covariance matrix was again the identity matrix. The values for the Gaussian mixture were picked to get some overlap between the areas of high density but not too much, since if they are too close we expect a similar set of points to that of a single multivariate distribution and if they are too far away they locally would also have the same structure as a point cloud drawn from a single multivariate distribution. 

We observe the same general behavior for all three distributions. However there are noteable differences in the values of the means. Figure \ref{fig:2d_all_compared} shows the development of the means for all three generated data sets in one diagram. The continuous line represents the uniform distribution, the dotted line corresponds to the multivariate Gaussian and the dashed line corresponds to the Gaussian mixture model.

\begin{figure}[H]
%\centering%

\begin{subfigure}[c]{0.99\textwidth}
\begin{center}
% This file was created by tikzplotlib v0.9.2.
\begin{tikzpicture}

\definecolor{color0}{rgb}{0.12156862745098,0.466666666666667,0.705882352941177}

\begin{axis}[
tick align=outside,
tick pos=left,
xlabel = $m$,
ylabel = average $q$,
ylabel style={rotate=-90},
x grid style={white!69.0196078431373!black},
xmin=2.5, xmax=1047.5,
xtick style={color=black},
y grid style={white!69.0196078431373!black},
ymin=72.1408722544235, ymax=74.9646784830289,
ytick style={color=black}
]
\addplot [thick, color0, dotted]
table {%
50 74.751221902748
100 73.7920620385357
150 73.4803739655178
200 73.244056612421
250 73.0385644533163
300 72.9036301595062
350 72.8136863565804
400 72.7229801131946
450 72.638716960172
500 72.5917237632603
550 72.5491378420652
600 72.5096471359426
650 72.4730934536537
700 72.4249485291089
750 72.378279089354
800 72.3896536253948
850 72.3218743554731
900 72.3368743405545
950 72.2845377486291
1000 72.3004042654642
};

\addplot [thick, color0, dashed]
table {%
50 74.7440091823553
100 73.7186939751261
150 73.3647881875886
200 73.1389162360907
250 72.9682849309851
300 72.8459369302215
350 72.7103796794705
400 72.6327957832213
450 72.5969537461766
500 72.5518642799392
550 72.4602006772107
600 72.461873858482
650 72.4035463050674
700 72.3909778284513
750 72.3387470840265
800 72.294478801029
850 72.2975470698247
900 72.2937838331733
950 72.2676937029656
1000 72.2349609142709
};
\addplot[thick, color0]
table {%
50 74.3135260273129
100 73.9007987162947
150 73.548228008839
200 73.3330687122862
250 73.256039993344
300 73.1981969846523
350 73.0861031936403
400 73.0127258791942
450 72.9523915686429
500 72.9466183308799
550 72.873379839116
600 72.8254219384188
650 72.7987235551989
700 72.7696569674668
750 72.7625372033231
800 72.6942015413158
850 72.6658903806182
900 72.6621935833012
950 72.6374432276665
1000 72.6052744156251
};
\end{axis}

\end{tikzpicture}


\end{center}
\end{subfigure}


%\begin{subfigure}[c]{0.99\textwidth}
%\begin{center}
%% This file was created by tikzplotlib v0.9.2.
\begin{tikzpicture}

\definecolor{color0}{rgb}{0.12156862745098,0.466666666666667,0.705882352941177}

\begin{axis}[
tick align=outside,
tick pos=left,
x grid style={white!69.0196078431373!black},
xmin=2.5, xmax=1047.5,
xtick style={color=black},
y grid style={white!69.0196078431373!black},
ymin=0.395394258663213, ymax=2.41598322003903,
ytick style={color=black}
]
\addplot [thick, black, dashed]
table {%
50 2.32413826724922
100 1.6017798725875
150 1.27569876899754
200 1.11569841878204
250 1.03198710689561
300 0.909575719051176
350 0.85673532804327
400 0.768062421883222
450 0.757084132537467
500 0.703841068123042
550 0.655567344924125
600 0.628660144146404
650 0.609964523254271
700 0.55626807435996
750 0.571403742096503
800 0.559237227591519
850 0.537978856444726
900 0.530597459789274
950 0.508673050702853
1000 0.487239211453023
};

\addplot [thick, black]
table {%
50 2.24615762551669
100 1.5792435037505
150 1.27389257927507
200 1.1147961093539
250 1.00194489752104
300 0.883268619119549
350 0.814502452409831
400 0.776380080599139
450 0.745367391163501
500 0.681014417056458
550 0.676891814240733
600 0.656527982762773
650 0.619120834767529
700 0.59424703343731
750 0.564172814777307
800 0.533426280089837
850 0.543871439397632
900 0.509264003967767
950 0.517621741132293
1000 0.492311429591938
};

\addplot [thick, black, dotted]
table {%
50 2.2861808651118
100 1.64986784602092
150 1.28683882516498
200 1.13016921748758
250 1.00585682814682
300 0.897678965646122
350 0.871678397927158
400 0.776296030264333
450 0.755134146237627
500 0.67702370736897
550 0.64349690377966
600 0.661206342404475
650 0.601008063114883
700 0.588765385643078
750 0.556921114871381
800 0.555686528665616
850 0.525900473767235
900 0.533350195248551
950 0.523497215225528
1000 0.505036296302693
};

\end{axis}

\end{tikzpicture}

%\subcaption{Standard deviation of $q$-values compared between differently drawn point clouds.}
%\end{center}
%\end{subfigure}

\caption{Mean $q$-values compared between differently drawn point clouds.}
\label{fig:2d_all_compared}
\end{figure}

Let us consider the difference we see between the means of the two samples from Gaussian distributions and the uniform distribution. Note that the difference in number of simplices is quite small for all cases we considered. For example the average number of elements in filtrations on $1000$ points is $5957.4$ in case of the uniform distribution and $5971.16$ in case of the Gaussian mixture. Therefore the gap between the curves is not explicable by a difference in number of elements of the filtrations. 

A possible explanation for the differing developments of means is that point sets with one or several clusters and some outliers yield bad local structures with respect to simplices being a part of an apparent pair. Although the difference between the single multivariate Gaussian and the Gaussian mixture model is quite small it is nevertheless consistent for different sizes of point clouds. What exactly this potentially \enquote{bad} structure is remains to be uncovered.

The similar curves in standard deviation strengthen the intuition, that these mainly depend on the size of the complex and that \enquote{good} and \enquote{bad} local configurations appear more evenly distributed. 

%One practical observation that has been made by several authors and people concerned with persistent homology is that although the running time in theory is cubic we only see slightly super linear running times in practice. Perhaps the canonical persistence pairs defined by the apparent pairs might be useful to explore this phenomenon. In the next section we will look at apparent pairs and the running time of the standard reduction scheme.

\section{Apparent Pairs in Standard Reduction}

A natural question is if there is some kind of connection between the value $q$ for some filtration $F$ of simplicial complex $K$ and the number of additions $\operatorname{add}(B)$ needed to reduce the boundary matrix $B$ of $F$. We can guess from what we discussed in Chapter \ref{ch:connection} that only considering the number of apparent pairs does not suffice to explain the number of additions. Nevertheless we checked if there is some linear correlation between the value of $q$ and $\frac{\operatorname{add}(B)}{m}$, where $m$ is the number of elements in $K$. We calculated the correlation coefficient via \textbf{numpy.corrcoef} of the Python library \textbf{numpy}. The following figure shows a scatter plot of $100$ alpha complexes on $100$ points. On the $x$-axis we have the values of $q$ and on the $y$-axis we have $\frac{\operatorname{add}(B)}{m}$. Above the plot we display the calculated linear correlation coefficient.

\begin{figure}[H]
%\centering%
\begin{subfigure}[c]{0.95\textwidth}
\begin{center}
% This file was created by tikzplotlib v0.9.2.
\begin{tikzpicture}

\definecolor{color0}{rgb}{0.0235294117647059,0.603921568627451,0.952941176470588}

\begin{axis}[
tick align=outside,
tick pos=left,
title={Linear correlation: 0.069},
ylabel style={rotate=-90},
xlabel = $q$,
ylabel = $\frac{\operatorname{add}(B)}{m}$,
x grid style={white!69.0196078431373!black},
xmin=88.0233754034327, xmax=92.0615058308513,
xtick style={color=black},
y grid style={white!69.0196078431373!black},
ymin=17.9477669720933, ymax=29.8578650389586,
ytick style={color=black}
]
\addplot [only marks, mark=*, draw=color0, fill=color0, colormap/viridis]
table{%
x                      y
90.2169101372289 20.3315626383355
91.0737386804657 20.257007330746
90.5027932960894 22.532015470563
89.5980107749689 21.596353087443
89.2811020232458 23.1696082651743
90.5611135276207 21.2235754675946
90.3902643726395 24.3646663869073
91.0184787279759 20.1267726686721
91.1903738719381 24.0077352814783
88.9760348583878 21.1015250544662
89.8488120950324 21.8159827213823
90.1175446234218 21.1841532433609
90.0384779820436 22.7195382642155
90.1859057501081 24.9165585819282
88.8132709485325 23.4942577626542
90.1359053046909 23.2341078474353
90.0650759219089 22.3952277657267
90.0128589798543 23.8011144449207
90.5528950805398 19.9821506312582
91.8779544477868 21.8972926514826
90.2281532501076 22.3723633232889
90.1559020044543 21.1576837416481
91.7056571671629 21.9179072735006
90.1185770750988 23.6012296881862
88.3253170091823 23.0126803672934
90.3032891926527 23.052541648868
90.7234798047048 24.1056369285397
91.3210773834972 23.7892261650278
89.7624190064795 26.5680345572354
89.6849374190764 24.0690548122572
89.5611418832552 18.4891350660418
89.3801473775466 21.0840918942349
90.8627284317892 22.0246493837654
90.5291240551356 25.2867941307248
90.0482244629548 23.8386672512056
90.0042176296921 21.7794179671025
91.2046686119216 22.0754481033764
90.4168457241083 24.6923076923077
89.7838066977533 23.3738872403561
89.3212278426286 20.6519671422395
90.8459869848156 22.6911062906725
90.8229211546747 20.6363636363636
89.7941305300044 25.7700394218134
90.8554572271386 21.7623261694058
90.3115663679044 22.2765685019206
90.2028485110056 23.9430297798878
89.1294932871373 21.8289302728454
89.7888841016803 22.8866867729427
90.0302114803625 22.9741044454035
90.2386117136659 20.939262472885
89.5859473023839 22.4642409033877
90.2131361461505 21.7516311439756
89.9368421052632 23.9031578947368
90.0650759219089 21.0056399132321
89.7245299519021 23.0008745080892
90.0043271311121 23.5430549545651
89.4894894894895 21.1149721149721
89.471346068414 19.1372723234118
89.1692040017399 22.6267942583732
89.2468437091859 23.4453635176317
90.6699547883272 26.4188244965064
89.9608865710561 22.787049109083
89.5444685466377 21.4008676789588
88.8030888030888 23.2818532818533
89.8728627794827 22.7176676896098
90.8776480760917 23.0051880674449
89.9615548910722 20.6505766766339
89.5303748384317 21.8147350280052
88.2069267864971 22.0306882946076
89.7692642577275 27.6129734436221
89.6760259179266 23.985313174946
89.7516930022573 29.2523702031603
89.2934547030776 20.3892501083658
89.1147818720881 21.8936891147819
89.2936472678809 24.8551754775655
90.3308981521272 22.50021486893
89.8030634573304 20.6516411378556
88.2226980728051 18.9700214132762
90.0461990760185 23.3137337253255
90.3337667967057 21.9826614651062
89.6522112494633 20.7050236152855
89.6098202542744 19.7601928978518
89.755679382769 22.0938705529361
90.00429000429 21.6572286572287
90.1943844492441 20.7568034557235
88.9083735203858 22.2656729504603
90.2407002188184 22.3242888402626
91.0216718266254 22.687748783724
89.925208974923 24.0950285965684
91.1370514483355 24.63121487246
89.0553474823138 21.4419475655431
89.2655367231638 23.6327683615819
89.5080539834567 23.4614714845451
90.8089792460822 20.3689114781872
90.2150065818341 19.3242650285213
90.7025662599916 22.3618005889777
90.020366598778 29.3164969450102
89.4852617685878 19.3677958644963
90.1408450704225 20.8813486982501
91.1752044769694 23.1510977184675
};
\end{axis}

\end{tikzpicture}

\end{center}
\end{subfigure}
\caption{Value of $q$ related to the number of additions per column in the standard reduction scheme for some filtration $F$ with $m$ elements.}
\label{fig:correlation_3d}
\end{figure}

Unfortunately but not unsurprisingly there is no or only a very small linear correlation between the number of additions divided by the number of elements in the complex and $q$. This means that the complicatedness of persistent homology computations is not explicable by the number of critical cells of the apparent gradient. \\ 

In the following we will try to illustrate how much of the work during reduction is done on apparent pairs. To be more precise, this means how much work is done on the tails of apparent pairs since the columns corresponding to the heads of the apparent pairs are reduced from the beginning. See the proof of Lemma \ref{lem:app_is_pers}. Note that we are not counting the number of column additions as in Definition \ref{def:col_red_steps} but the total number of addition operations, i.e. the number of additions of non-zero elements during the addition of columns.

As a preprocessing step for reduction we can set all tails of apparent pairs to zero. This will save us all additions needed to reduce the respective columns. It comes at the cost of finding the apparent pairs first however. Furthermore there are other more potent preprocessing steps that can be done. 

For example for each simplex $\sigma$ of a filtration one can just set the column of the youngest facet to zero, since it creates a cycle with the other facets of the simplex that has to be closed by simplex $\sigma$ at the latest. This procedure also implicitly sets the columns corresponding to tails of apparent pairs to zero.

In Section \ref{sec:twist} we already discussed the so called twisted reduction scheme which starts in the highest dimension and works its way down. Preprocessing with apparent pairs yields no improvement for this procedure, as every column corresponding to the tail of an apparent pair will just be set to zero without doing any calculations when its head is processed. Nevertheless the analysis we do in this section still tells us how much of the improvements of the twisted reduction scheme over the standard reduction scheme stem from columns corresponding to tails of apparent pairs. Recall that we refer to the process of setting columns to zero when we know that they correspond to a simplex creating a homology class as \textbf{clearing}. See Section \ref{sec:twist}.

We have already seen that in the setting of alpha complexes on random point clouds of a fixed size that the percentage of simplices that are part of an apparent pair increases with dimension. This means that we would expect higher speedups for the standard reduction scheme, see Algorithm \ref{algo:column_reduction_algorithm}, in higher dimensions. We will try to measure the speedups in the following way. 

For given filtration $F$, let $SRT$ (\textbf{standard reduction time}) denote the time it takes to construct and reduce some boundary matrix via the standard reduction scheme. Let $ART$ (\textbf{apparent reduction time}) denote the time it takes to find all apparent pairs, clear columns in the boundary matrix corresponding to tails of apparent pairs, and do the standard reduction. We will consider \[
s \coloneqq \frac{SRT}{ART},
\] i.e., the factor by which the overall reduction time decreases when we do clearing in the preprocessing via apparent pairs. If $s > 1$, we save time, if $s<1$ we loose time, i.e., the cost for finding the apparent pairs and clearing the columns is higher than the gain compared to the standard reduction. 

The implementations of the algorithms can be found on \href{https://github.com/IvanSpirandelli/Masterarbeit/blob/master/Algorithms/column_algo/column_algorithm.py}{[GitHub]}, see \cite{github}.

We will average each setup of parameters over hundreds of runs to make sure the values we are seeing are no anomaly caused by some external factor.

All experiments were done on an \enquote{Intel Core i7-3770 CPU \@ 3.40GHz~x~8} processor. 

Note that the implemented algorithms are not optimized for speed since the primary interest was not to develop a fast persistent homology computation but to analyse and compare examples with respect to additions in the boundary matrices. 

Hence, the consideration of how many additions we save by clearing the apparent columns is more meaningful. For a given filtration $F$, let $SRA$ (\textbf{standard reduction additions}), denote the number of additions in the standard reduction process and let $ARA$ (\textbf{apparent reduction additions}) denote the number of additions on the boundary matrix which was cleared by setting tails of apparent pairs to zero. Then we define
\[
a \coloneqq 100 (1- \frac{ARA}{SRA}),
\]
i.e., the percentage of additions that are not necessary when reducing the cleared boundary matrix. Or in other words: A high $a$-value implies that the clearing yields a big improvement while a low value implies that even after clearing we still have lots of additions to do. The following table shows the average $s$ and $a$ values for alpha complexes on $30$ points in dimensions two and three drawn from the previously mentioned distributions. We averaged the values over $200$ runs.



 \begin{table}[H]
     \begin{center}
     \begin{tabular}{|c|c|c|c|}
     \hline
     & \multicolumn{3}{|c|}{Execution Time Speedup} \\ \cline{1-4}
     Sample distribution & Uniform & Multivariate Normal & Gaussian 		Mixture\\ \hline
     dimension two & 1.96 & 2.09 & 2.13\\ \hline
     dimension three & 3.02 & 3.37 & 3.67\\ \hline

     \end{tabular}
     
     \caption{$s$-value for alpha complexes on randomly generated point clouds.}
     \label{tab:time_STA_TWI}
     \end{center}
 \end{table}




 \begin{table}[H]
     \begin{center}
     \begin{tabular}{|c|c|c|c|}
     \hline
     & \multicolumn{3}{|c|}{Percentage of Saved Additions} \\ \cline{1-4}
     Sample distribution & Uniform & Multivariate Normal & Gaussian 		Mixture\\ \hline
     dimension two & 66.64 & 69.90 & 70.98 \\ \hline
     dimension three & 68.98 & 73.64 & 76.75 \\ \hline

     \end{tabular}
     
     \caption{$a$-value for alpha complexes on randomly generated point clouds.}
     \label{tab:add_STA_TWI}
     \end{center}
 \end{table}

Looking at the tables we see that there is substantial time and additions saved in all cases, while the speedup is generally higher on point clouds sampled from Gaussian distribution or a Gaussian mixture. 

When reevaluating Figure \ref{fig:2d_all_compared} we might come to the conclusion that this stems from the fact that for alpha complexes on few points the $q$-value is higher for the Gaussian distribution and mixture than for the uniform distribution. However, as we will see later on, this is not the case.

Note that for the Gaussian mixture model in dimension three, we chose the points $(0,0,0)$, $(1,1,1)$, and $(0,3,0.5)$ as means and the identity matrix as covariance matrix in all cases. 

When looking at the percentages of saved additions and speedups in running time in dimension three in more detail, an interesting pattern emerges. Consider the following plot in which we count how often a percentage of saved additions occurs for the multivariate Gaussian distribution in three dimensions. 


\begin{figure}[H]
%\centering%
\begin{subfigure}[c]{0.95\textwidth}
\begin{center}
% This file was created by tikzplotlib v0.9.2.
\begin{tikzpicture}

\definecolor{color0}{rgb}{0.12156862745098,0.466666666666667,0.705882352941177}

\begin{axis}[
tick align=outside,
tick pos=left,
title={mean = 73.64\%},
xlabel = $a$,
ylabel = density,
x grid style={white!69.0196078431373!black},
xmin=-4.29979344454833, xmax=102.750450824271,
xtick style={color=black},
y grid style={white!69.0196078431373!black},
ymin=0, ymax=0.151050566119165,
ytick style={color=black},
yticklabel style={
        /pgf/number format/fixed,
        /pgf/number format/precision=5
},
scaled y ticks=false
]
\draw[draw=none,fill=color0,fill opacity=0.6] (axis cs:0.566126749488902,0) rectangle (axis cs:1.53931078829635,0.0205510974311789);
\draw[draw=none,fill=color0,fill opacity=0.6] (axis cs:1.53931078829635,0) rectangle (axis cs:2.5124948271038,0.00513777435779472);
\draw[draw=none,fill=color0,fill opacity=0.6] (axis cs:2.5124948271038,0) rectangle (axis cs:3.48567886591124,0.0102755487155894);
\draw[draw=none,fill=color0,fill opacity=0.6] (axis cs:3.48567886591124,0) rectangle (axis cs:4.45886290471869,0.0154133230733842);
\draw[draw=none,fill=color0,fill opacity=0.6] (axis cs:4.45886290471869,0) rectangle (axis cs:5.43204694352614,0.00513777435779472);
\draw[draw=none,fill=color0,fill opacity=0.6] (axis cs:5.43204694352614,0) rectangle (axis cs:6.40523098233358,0.0205510974311789);
\draw[draw=none,fill=color0,fill opacity=0.6] (axis cs:6.40523098233358,0) rectangle (axis cs:7.37841502114103,0.00513777435779472);
\draw[draw=none,fill=color0,fill opacity=0.6] (axis cs:7.37841502114103,0) rectangle (axis cs:8.35159905994848,0.0154133230733842);
\draw[draw=none,fill=color0,fill opacity=0.6] (axis cs:8.35159905994848,0) rectangle (axis cs:9.32478309875592,0.0102755487155894);
\draw[draw=none,fill=color0,fill opacity=0.6] (axis cs:9.32478309875592,0) rectangle (axis cs:10.2979671375634,0.0154133230733841);
\draw[draw=none,fill=color0,fill opacity=0.6] (axis cs:10.2979671375634,0) rectangle (axis cs:11.2711511763708,0.00513777435779471);
\draw[draw=none,fill=color0,fill opacity=0.6] (axis cs:11.2711511763708,0) rectangle (axis cs:12.2443352151783,0.00513777435779472);
\draw[draw=none,fill=color0,fill opacity=0.6] (axis cs:12.2443352151783,0) rectangle (axis cs:13.2175192539857,0.0154133230733842);
\draw[draw=none,fill=color0,fill opacity=0.6] (axis cs:13.2175192539857,0) rectangle (axis cs:14.1907032927932,0.0102755487155894);
\draw[draw=none,fill=color0,fill opacity=0.6] (axis cs:14.1907032927932,0) rectangle (axis cs:15.1638873316006,0);
\draw[draw=none,fill=color0,fill opacity=0.6] (axis cs:15.1638873316006,0) rectangle (axis cs:16.1370713704081,0.00513777435779472);
\draw[draw=none,fill=color0,fill opacity=0.6] (axis cs:16.1370713704081,0) rectangle (axis cs:17.1102554092155,0.00513777435779472);
\draw[draw=none,fill=color0,fill opacity=0.6] (axis cs:17.1102554092155,0) rectangle (axis cs:18.0834394480229,0.00513777435779472);
\draw[draw=none,fill=color0,fill opacity=0.6] (axis cs:18.0834394480229,0) rectangle (axis cs:19.0566234868304,0.0051377743577947);
\draw[draw=none,fill=color0,fill opacity=0.6] (axis cs:19.0566234868304,0) rectangle (axis cs:20.0298075256378,0);
\draw[draw=none,fill=color0,fill opacity=0.6] (axis cs:20.0298075256378,0) rectangle (axis cs:21.0029915644453,0.0205510974311789);
\draw[draw=none,fill=color0,fill opacity=0.6] (axis cs:21.0029915644453,0) rectangle (axis cs:21.9761756032527,0.0051377743577947);
\draw[draw=none,fill=color0,fill opacity=0.6] (axis cs:21.9761756032527,0) rectangle (axis cs:22.9493596420602,0.00513777435779472);
\draw[draw=none,fill=color0,fill opacity=0.6] (axis cs:22.9493596420602,0) rectangle (axis cs:23.9225436808676,0.00513777435779472);
\draw[draw=none,fill=color0,fill opacity=0.6] (axis cs:23.9225436808676,0) rectangle (axis cs:24.8957277196751,0);
\draw[draw=none,fill=color0,fill opacity=0.6] (axis cs:24.8957277196751,0) rectangle (axis cs:25.8689117584825,0);
\draw[draw=none,fill=color0,fill opacity=0.6] (axis cs:25.8689117584825,0) rectangle (axis cs:26.84209579729,0.0154133230733841);
\draw[draw=none,fill=color0,fill opacity=0.6] (axis cs:26.84209579729,0) rectangle (axis cs:27.8152798360974,0.00513777435779472);
\draw[draw=none,fill=color0,fill opacity=0.6] (axis cs:27.8152798360974,0) rectangle (axis cs:28.7884638749049,0);
\draw[draw=none,fill=color0,fill opacity=0.6] (axis cs:28.7884638749049,0) rectangle (axis cs:29.7616479137123,0);
\draw[draw=none,fill=color0,fill opacity=0.6] (axis cs:29.7616479137123,0) rectangle (axis cs:30.7348319525198,0);
\draw[draw=none,fill=color0,fill opacity=0.6] (axis cs:30.7348319525198,0) rectangle (axis cs:31.7080159913272,0);
\draw[draw=none,fill=color0,fill opacity=0.6] (axis cs:31.7080159913272,0) rectangle (axis cs:32.6812000301346,0.00513777435779472);
\draw[draw=none,fill=color0,fill opacity=0.6] (axis cs:32.6812000301346,0) rectangle (axis cs:33.6543840689421,0);
\draw[draw=none,fill=color0,fill opacity=0.6] (axis cs:33.6543840689421,0) rectangle (axis cs:34.6275681077495,0);
\draw[draw=none,fill=color0,fill opacity=0.6] (axis cs:34.6275681077495,0) rectangle (axis cs:35.600752146557,0);
\draw[draw=none,fill=color0,fill opacity=0.6] (axis cs:35.600752146557,0) rectangle (axis cs:36.5739361853644,0);
\draw[draw=none,fill=color0,fill opacity=0.6] (axis cs:36.5739361853644,0) rectangle (axis cs:37.5471202241719,0.0102755487155894);
\draw[draw=none,fill=color0,fill opacity=0.6] (axis cs:37.5471202241719,0) rectangle (axis cs:38.5203042629793,0);
\draw[draw=none,fill=color0,fill opacity=0.6] (axis cs:38.5203042629793,0) rectangle (axis cs:39.4934883017868,0);
\draw[draw=none,fill=color0,fill opacity=0.6] (axis cs:39.4934883017868,0) rectangle (axis cs:40.4666723405942,0);
\draw[draw=none,fill=color0,fill opacity=0.6] (axis cs:40.4666723405942,0) rectangle (axis cs:41.4398563794017,0);
\draw[draw=none,fill=color0,fill opacity=0.6] (axis cs:41.4398563794017,0) rectangle (axis cs:42.4130404182091,0.00513777435779472);
\draw[draw=none,fill=color0,fill opacity=0.6] (axis cs:42.4130404182091,0) rectangle (axis cs:43.3862244570166,0);
\draw[draw=none,fill=color0,fill opacity=0.6] (axis cs:43.3862244570166,0) rectangle (axis cs:44.359408495824,0);
\draw[draw=none,fill=color0,fill opacity=0.6] (axis cs:44.359408495824,0) rectangle (axis cs:45.3325925346315,0);
\draw[draw=none,fill=color0,fill opacity=0.6] (axis cs:45.3325925346315,0) rectangle (axis cs:46.3057765734389,0);
\draw[draw=none,fill=color0,fill opacity=0.6] (axis cs:46.3057765734389,0) rectangle (axis cs:47.2789606122464,0);
\draw[draw=none,fill=color0,fill opacity=0.6] (axis cs:47.2789606122464,0) rectangle (axis cs:48.2521446510538,0);
\draw[draw=none,fill=color0,fill opacity=0.6] (axis cs:48.2521446510538,0) rectangle (axis cs:49.2253286898612,0);
\draw[draw=none,fill=color0,fill opacity=0.6] (axis cs:49.2253286898612,0) rectangle (axis cs:50.1985127286687,0);
\draw[draw=none,fill=color0,fill opacity=0.6] (axis cs:50.1985127286687,0) rectangle (axis cs:51.1716967674761,0);
\draw[draw=none,fill=color0,fill opacity=0.6] (axis cs:51.1716967674761,0) rectangle (axis cs:52.1448808062836,0);
\draw[draw=none,fill=color0,fill opacity=0.6] (axis cs:52.1448808062836,0) rectangle (axis cs:53.118064845091,0);
\draw[draw=none,fill=color0,fill opacity=0.6] (axis cs:53.118064845091,0) rectangle (axis cs:54.0912488838985,0);
\draw[draw=none,fill=color0,fill opacity=0.6] (axis cs:54.0912488838985,0) rectangle (axis cs:55.0644329227059,0);
\draw[draw=none,fill=color0,fill opacity=0.6] (axis cs:55.0644329227059,0) rectangle (axis cs:56.0376169615134,0);
\draw[draw=none,fill=color0,fill opacity=0.6] (axis cs:56.0376169615134,0) rectangle (axis cs:57.0108010003208,0);
\draw[draw=none,fill=color0,fill opacity=0.6] (axis cs:57.0108010003208,0) rectangle (axis cs:57.9839850391283,0);
\draw[draw=none,fill=color0,fill opacity=0.6] (axis cs:57.9839850391283,0) rectangle (axis cs:58.9571690779357,0);
\draw[draw=none,fill=color0,fill opacity=0.6] (axis cs:58.9571690779357,0) rectangle (axis cs:59.9303531167432,0);
\draw[draw=none,fill=color0,fill opacity=0.6] (axis cs:59.9303531167432,0) rectangle (axis cs:60.9035371555506,0);
\draw[draw=none,fill=color0,fill opacity=0.6] (axis cs:60.9035371555506,0) rectangle (axis cs:61.8767211943581,0);
\draw[draw=none,fill=color0,fill opacity=0.6] (axis cs:61.8767211943581,0) rectangle (axis cs:62.8499052331655,0);
\draw[draw=none,fill=color0,fill opacity=0.6] (axis cs:62.8499052331655,0) rectangle (axis cs:63.8230892719729,0);
\draw[draw=none,fill=color0,fill opacity=0.6] (axis cs:63.823089271973,0) rectangle (axis cs:64.7962733107804,0);
\draw[draw=none,fill=color0,fill opacity=0.6] (axis cs:64.7962733107804,0) rectangle (axis cs:65.7694573495878,0);
\draw[draw=none,fill=color0,fill opacity=0.6] (axis cs:65.7694573495878,0) rectangle (axis cs:66.7426413883953,0);
\draw[draw=none,fill=color0,fill opacity=0.6] (axis cs:66.7426413883953,0) rectangle (axis cs:67.7158254272027,0);
\draw[draw=none,fill=color0,fill opacity=0.6] (axis cs:67.7158254272027,0) rectangle (axis cs:68.6890094660102,0);
\draw[draw=none,fill=color0,fill opacity=0.6] (axis cs:68.6890094660102,0) rectangle (axis cs:69.6621935048176,0);
\draw[draw=none,fill=color0,fill opacity=0.6] (axis cs:69.6621935048176,0) rectangle (axis cs:70.6353775436251,0);
\draw[draw=none,fill=color0,fill opacity=0.6] (axis cs:70.6353775436251,0) rectangle (axis cs:71.6085615824325,0);
\draw[draw=none,fill=color0,fill opacity=0.6] (axis cs:71.6085615824325,0) rectangle (axis cs:72.58174562124,0);
\draw[draw=none,fill=color0,fill opacity=0.6] (axis cs:72.58174562124,0) rectangle (axis cs:73.5549296600474,0);
\draw[draw=none,fill=color0,fill opacity=0.6] (axis cs:73.5549296600474,0) rectangle (axis cs:74.5281136988549,0);
\draw[draw=none,fill=color0,fill opacity=0.6] (axis cs:74.5281136988549,0) rectangle (axis cs:75.5012977376623,0);
\draw[draw=none,fill=color0,fill opacity=0.6] (axis cs:75.5012977376623,0) rectangle (axis cs:76.4744817764697,0);
\draw[draw=none,fill=color0,fill opacity=0.6] (axis cs:76.4744817764698,0) rectangle (axis cs:77.4476658152772,0);
\draw[draw=none,fill=color0,fill opacity=0.6] (axis cs:77.4476658152772,0) rectangle (axis cs:78.4208498540846,0);
\draw[draw=none,fill=color0,fill opacity=0.6] (axis cs:78.4208498540847,0) rectangle (axis cs:79.3940338928921,0);
\draw[draw=none,fill=color0,fill opacity=0.6] (axis cs:79.3940338928921,0) rectangle (axis cs:80.3672179316995,0);
\draw[draw=none,fill=color0,fill opacity=0.6] (axis cs:80.3672179316995,0) rectangle (axis cs:81.340401970507,0);
\draw[draw=none,fill=color0,fill opacity=0.6] (axis cs:81.340401970507,0) rectangle (axis cs:82.3135860093144,0);
\draw[draw=none,fill=color0,fill opacity=0.6] (axis cs:82.3135860093144,0) rectangle (axis cs:83.2867700481219,0);
\draw[draw=none,fill=color0,fill opacity=0.6] (axis cs:83.2867700481219,0) rectangle (axis cs:84.2599540869293,0.00513777435779472);
\draw[draw=none,fill=color0,fill opacity=0.6] (axis cs:84.2599540869293,0) rectangle (axis cs:85.2331381257368,0.0102755487155894);
\draw[draw=none,fill=color0,fill opacity=0.6] (axis cs:85.2331381257368,0) rectangle (axis cs:86.2063221645442,0.00513777435779465);
\draw[draw=none,fill=color0,fill opacity=0.6] (axis cs:86.2063221645442,0) rectangle (axis cs:87.1795062033517,0);
\draw[draw=none,fill=color0,fill opacity=0.6] (axis cs:87.1795062033517,0) rectangle (axis cs:88.1526902421591,0.0205510974311789);
\draw[draw=none,fill=color0,fill opacity=0.6] (axis cs:88.1526902421591,0) rectangle (axis cs:89.1258742809666,0.00513777435779472);
\draw[draw=none,fill=color0,fill opacity=0.6] (axis cs:89.1258742809666,0) rectangle (axis cs:90.099058319774,0.0308266461467683);
\draw[draw=none,fill=color0,fill opacity=0.6] (axis cs:90.099058319774,0) rectangle (axis cs:91.0722423585814,0.0565155179357419);
\draw[draw=none,fill=color0,fill opacity=0.6] (axis cs:91.0722423585815,0) rectangle (axis cs:92.0454263973889,0.0565155179357419);
\draw[draw=none,fill=color0,fill opacity=0.6] (axis cs:92.0454263973889,0) rectangle (axis cs:93.0186104361963,0.0667910666513314);
\draw[draw=none,fill=color0,fill opacity=0.6] (axis cs:93.0186104361964,0) rectangle (axis cs:93.9917944750038,0.143857682018252);
\draw[draw=none,fill=color0,fill opacity=0.6] (axis cs:93.9917944750038,0) rectangle (axis cs:94.9649785138112,0.092479938440305);
\draw[draw=none,fill=color0,fill opacity=0.6] (axis cs:94.9649785138112,0) rectangle (axis cs:95.9381625526187,0.118168810229279);
\draw[draw=none,fill=color0,fill opacity=0.6] (axis cs:95.9381625526187,0) rectangle (axis cs:96.9113465914261,0.0976177127980997);
\draw[draw=none,fill=color0,fill opacity=0.6] (axis cs:96.9113465914261,0) rectangle (axis cs:97.8845306302336,0.0616532922935367);
\end{axis}

\end{tikzpicture}

\end{center}
\end{subfigure}
\caption{Occurences of percentages of saved additions in three dimensions.}
\label{fig:saved_additions_3d}
\end{figure}

As we can see in Figure \ref{fig:saved_additions_3d}, while we have a mean percentage of saved additions of about $73\%$ there is actually a majority of cases with $a$-values of more than $90\%$. On the other hand there are a lot of cases with $a$-values below $20\%$. This is remarkable or even surprising, since it looks like our examples more or less fall into two categories in three dimensions. Namely one, where almost all the work is done on the tails of apparent pairs, and one where almost none of the work is done on the tails of apparent pairs. However this behaviour does not occur in two dimensions as Figure \ref{fig:saved_additions_2d} illustrates. Note in particular that the values on the $x$-axis only go from $55$ to $90$ and not from $0$ to $100$ as in \ref{fig:saved_additions_3d}.

\begin{figure}[H]
%\centering%
\begin{subfigure}[c]{0.95\textwidth}
\begin{center}
% This file was created by tikzplotlib v0.9.2.
\begin{tikzpicture}

\definecolor{color0}{rgb}{0.12156862745098,0.466666666666667,0.705882352941177}

\begin{axis}[
tick align=outside,
tick pos=left,
title={mean = 69.90\%},
x grid style={white!69.0196078431373!black},
xmin=55.3829134003992, xmax=88.7546598014462,
xtick style={color=black},
yticklabel style={
        /pgf/number format/fixed,
        /pgf/number format/precision=5
},
xlabel = $a$,
ylabel = density,
scaled y ticks=false,
y grid style={white!69.0196078431373!black},
ymin=0, ymax=0.155745520103709,
ytick style={color=black}
]
\draw[draw=none,fill=color0,fill opacity=0.6] (axis cs:56.8998109640832,0) rectangle (axis cs:57.20319047682,0.0164810074183815);
\draw[draw=none,fill=color0,fill opacity=0.6] (axis cs:57.20319047682,0) rectangle (axis cs:57.5065699895567,0.0164810074183815);
\draw[draw=none,fill=color0,fill opacity=0.6] (axis cs:57.5065699895568,0) rectangle (axis cs:57.8099495022935,0.0164810074183815);
\draw[draw=none,fill=color0,fill opacity=0.6] (axis cs:57.8099495022935,0) rectangle (axis cs:58.1133290150303,0);
\draw[draw=none,fill=color0,fill opacity=0.6] (axis cs:58.1133290150303,0) rectangle (axis cs:58.4167085277671,0);
\draw[draw=none,fill=color0,fill opacity=0.6] (axis cs:58.4167085277671,0) rectangle (axis cs:58.7200880405039,0);
\draw[draw=none,fill=color0,fill opacity=0.6] (axis cs:58.7200880405039,0) rectangle (axis cs:59.0234675532407,0.0164810074183815);
\draw[draw=none,fill=color0,fill opacity=0.6] (axis cs:59.0234675532407,0) rectangle (axis cs:59.3268470659775,0);
\draw[draw=none,fill=color0,fill opacity=0.6] (axis cs:59.3268470659775,0) rectangle (axis cs:59.6302265787143,0.0164810074183815);
\draw[draw=none,fill=color0,fill opacity=0.6] (axis cs:59.6302265787143,0) rectangle (axis cs:59.9336060914511,0);
\draw[draw=none,fill=color0,fill opacity=0.6] (axis cs:59.9336060914511,0) rectangle (axis cs:60.2369856041879,0.0164810074183815);
\draw[draw=none,fill=color0,fill opacity=0.6] (axis cs:60.2369856041879,0) rectangle (axis cs:60.5403651169247,0.0164810074183815);
\draw[draw=none,fill=color0,fill opacity=0.6] (axis cs:60.5403651169247,0) rectangle (axis cs:60.8437446296615,0.0164810074183815);
\draw[draw=none,fill=color0,fill opacity=0.6] (axis cs:60.8437446296614,0) rectangle (axis cs:61.1471241423982,0.0164810074183815);
\draw[draw=none,fill=color0,fill opacity=0.6] (axis cs:61.1471241423982,0) rectangle (axis cs:61.450503655135,0.032962014836763);
\draw[draw=none,fill=color0,fill opacity=0.6] (axis cs:61.450503655135,0) rectangle (axis cs:61.7538831678718,0);
\draw[draw=none,fill=color0,fill opacity=0.6] (axis cs:61.7538831678718,0) rectangle (axis cs:62.0572626806086,0.0494430222551434);
\draw[draw=none,fill=color0,fill opacity=0.6] (axis cs:62.0572626806086,0) rectangle (axis cs:62.3606421933454,0);
\draw[draw=none,fill=color0,fill opacity=0.6] (axis cs:62.3606421933454,0) rectangle (axis cs:62.6640217060822,0.0164810074183815);
\draw[draw=none,fill=color0,fill opacity=0.6] (axis cs:62.6640217060822,0) rectangle (axis cs:62.967401218819,0.0164810074183815);
\draw[draw=none,fill=color0,fill opacity=0.6] (axis cs:62.967401218819,0) rectangle (axis cs:63.2707807315558,0.0164810074183815);
\draw[draw=none,fill=color0,fill opacity=0.6] (axis cs:63.2707807315558,0) rectangle (axis cs:63.5741602442926,0.0164810074183811);
\draw[draw=none,fill=color0,fill opacity=0.6] (axis cs:63.5741602442926,0) rectangle (axis cs:63.8775397570294,0.0164810074183815);
\draw[draw=none,fill=color0,fill opacity=0.6] (axis cs:63.8775397570294,0) rectangle (axis cs:64.1809192697662,0.0494430222551446);
\draw[draw=none,fill=color0,fill opacity=0.6] (axis cs:64.1809192697662,0) rectangle (axis cs:64.4842987825029,0.115367051928673);
\draw[draw=none,fill=color0,fill opacity=0.6] (axis cs:64.4842987825029,0) rectangle (axis cs:64.7876782952397,0.0329620148367623);
\draw[draw=none,fill=color0,fill opacity=0.6] (axis cs:64.7876782952397,0) rectangle (axis cs:65.0910578079765,0.0659240296735246);
\draw[draw=none,fill=color0,fill opacity=0.6] (axis cs:65.0910578079765,0) rectangle (axis cs:65.3944373207133,0.0659240296735276);
\draw[draw=none,fill=color0,fill opacity=0.6] (axis cs:65.3944373207133,0) rectangle (axis cs:65.6978168334501,0.0824050370919057);
\draw[draw=none,fill=color0,fill opacity=0.6] (axis cs:65.6978168334501,0) rectangle (axis cs:66.0011963461869,0.0659240296735276);
\draw[draw=none,fill=color0,fill opacity=0.6] (axis cs:66.0011963461869,0) rectangle (axis cs:66.3045758589237,0.0659240296735246);
\draw[draw=none,fill=color0,fill opacity=0.6] (axis cs:66.3045758589237,0) rectangle (axis cs:66.6079553716605,0.0329620148367638);
\draw[draw=none,fill=color0,fill opacity=0.6] (axis cs:66.6079553716605,0) rectangle (axis cs:66.9113348843973,0.0988860445102868);
\draw[draw=none,fill=color0,fill opacity=0.6] (axis cs:66.9113348843973,0) rectangle (axis cs:67.2147143971341,0.0494430222551434);
\draw[draw=none,fill=color0,fill opacity=0.6] (axis cs:67.2147143971341,0) rectangle (axis cs:67.5180939098708,0.148329066765437);
\draw[draw=none,fill=color0,fill opacity=0.6] (axis cs:67.5180939098708,0) rectangle (axis cs:67.8214734226076,0.0329620148367623);
\draw[draw=none,fill=color0,fill opacity=0.6] (axis cs:67.8214734226076,0) rectangle (axis cs:68.1248529353444,0.0824050370919095);
\draw[draw=none,fill=color0,fill opacity=0.6] (axis cs:68.1248529353444,0) rectangle (axis cs:68.4282324480812,0.0988860445102868);
\draw[draw=none,fill=color0,fill opacity=0.6] (axis cs:68.4282324480812,0) rectangle (axis cs:68.731611960818,0.0494430222551434);
\draw[draw=none,fill=color0,fill opacity=0.6] (axis cs:68.731611960818,0) rectangle (axis cs:69.0349914735548,0.0164810074183819);
\draw[draw=none,fill=color0,fill opacity=0.6] (axis cs:69.0349914735548,0) rectangle (axis cs:69.3383709862916,0.0988860445102868);
\draw[draw=none,fill=color0,fill opacity=0.6] (axis cs:69.3383709862916,0) rectangle (axis cs:69.6417504990284,0.115367051928673);
\draw[draw=none,fill=color0,fill opacity=0.6] (axis cs:69.6417504990284,0) rectangle (axis cs:69.9451300117652,0.0494430222551434);
\draw[draw=none,fill=color0,fill opacity=0.6] (axis cs:69.9451300117652,0) rectangle (axis cs:70.248509524502,0.0659240296735246);
\draw[draw=none,fill=color0,fill opacity=0.6] (axis cs:70.248509524502,0) rectangle (axis cs:70.5518890372388,0.0988860445102915);
\draw[draw=none,fill=color0,fill opacity=0.6] (axis cs:70.5518890372388,0) rectangle (axis cs:70.8552685499755,0.0329620148367623);
\draw[draw=none,fill=color0,fill opacity=0.6] (axis cs:70.8552685499755,0) rectangle (axis cs:71.1586480627123,0.0494430222551457);
\draw[draw=none,fill=color0,fill opacity=0.6] (axis cs:71.1586480627123,0) rectangle (axis cs:71.4620275754491,0.0988860445102868);
\draw[draw=none,fill=color0,fill opacity=0.6] (axis cs:71.4620275754491,0) rectangle (axis cs:71.7654070881859,0.0494430222551457);
\draw[draw=none,fill=color0,fill opacity=0.6] (axis cs:71.7654070881859,0) rectangle (axis cs:72.0687866009227,0.0824050370919057);
\draw[draw=none,fill=color0,fill opacity=0.6] (axis cs:72.0687866009227,0) rectangle (axis cs:72.3721661136595,0.0329620148367638);
\draw[draw=none,fill=color0,fill opacity=0.6] (axis cs:72.3721661136595,0) rectangle (axis cs:72.6755456263963,0.0824050370919057);
\draw[draw=none,fill=color0,fill opacity=0.6] (axis cs:72.6755456263963,0) rectangle (axis cs:72.9789251391331,0.0659240296735246);
\draw[draw=none,fill=color0,fill opacity=0.6] (axis cs:72.9789251391331,0) rectangle (axis cs:73.2823046518699,0.0494430222551457);
\draw[draw=none,fill=color0,fill opacity=0.6] (axis cs:73.2823046518699,0) rectangle (axis cs:73.5856841646067,0.0494430222551434);
\draw[draw=none,fill=color0,fill opacity=0.6] (axis cs:73.5856841646066,0) rectangle (axis cs:73.8890636773434,0.0659240296735276);
\draw[draw=none,fill=color0,fill opacity=0.6] (axis cs:73.8890636773434,0) rectangle (axis cs:74.1924431900802,0.0659240296735246);
\draw[draw=none,fill=color0,fill opacity=0.6] (axis cs:74.1924431900802,0) rectangle (axis cs:74.495822702817,0.0329620148367623);
\draw[draw=none,fill=color0,fill opacity=0.6] (axis cs:74.495822702817,0) rectangle (axis cs:74.7992022155538,0.0494430222551457);
\draw[draw=none,fill=color0,fill opacity=0.6] (axis cs:74.7992022155538,0) rectangle (axis cs:75.1025817282906,0.0329620148367623);
\draw[draw=none,fill=color0,fill opacity=0.6] (axis cs:75.1025817282906,0) rectangle (axis cs:75.4059612410274,0.0329620148367638);
\draw[draw=none,fill=color0,fill opacity=0.6] (axis cs:75.4059612410274,0) rectangle (axis cs:75.7093407537642,0.0659240296735246);
\draw[draw=none,fill=color0,fill opacity=0.6] (axis cs:75.7093407537642,0) rectangle (axis cs:76.012720266501,0.0329620148367623);
\draw[draw=none,fill=color0,fill opacity=0.6] (axis cs:76.012720266501,0) rectangle (axis cs:76.3160997792378,0);
\draw[draw=none,fill=color0,fill opacity=0.6] (axis cs:76.3160997792378,0) rectangle (axis cs:76.6194792919745,0.0494430222551457);
\draw[draw=none,fill=color0,fill opacity=0.6] (axis cs:76.6194792919746,0) rectangle (axis cs:76.9228588047114,0.0329620148367623);
\draw[draw=none,fill=color0,fill opacity=0.6] (axis cs:76.9228588047114,0) rectangle (axis cs:77.2262383174481,0.0329620148367623);
\draw[draw=none,fill=color0,fill opacity=0.6] (axis cs:77.2262383174481,0) rectangle (axis cs:77.5296178301849,0);
\draw[draw=none,fill=color0,fill opacity=0.6] (axis cs:77.5296178301849,0) rectangle (axis cs:77.8329973429217,0.0494430222551434);
\draw[draw=none,fill=color0,fill opacity=0.6] (axis cs:77.8329973429217,0) rectangle (axis cs:78.1363768556585,0.0494430222551457);
\draw[draw=none,fill=color0,fill opacity=0.6] (axis cs:78.1363768556585,0) rectangle (axis cs:78.4397563683953,0.0494430222551434);
\draw[draw=none,fill=color0,fill opacity=0.6] (axis cs:78.4397563683953,0) rectangle (axis cs:78.7431358811321,0);
\draw[draw=none,fill=color0,fill opacity=0.6] (axis cs:78.7431358811321,0) rectangle (axis cs:79.0465153938689,0.0329620148367638);
\draw[draw=none,fill=color0,fill opacity=0.6] (axis cs:79.0465153938689,0) rectangle (axis cs:79.3498949066057,0);
\draw[draw=none,fill=color0,fill opacity=0.6] (axis cs:79.3498949066057,0) rectangle (axis cs:79.6532744193425,0.0164810074183819);
\draw[draw=none,fill=color0,fill opacity=0.6] (axis cs:79.6532744193425,0) rectangle (axis cs:79.9566539320793,0.0164810074183811);
\draw[draw=none,fill=color0,fill opacity=0.6] (axis cs:79.9566539320793,0) rectangle (axis cs:80.2600334448161,0.0329620148367623);
\draw[draw=none,fill=color0,fill opacity=0.6] (axis cs:80.2600334448161,0) rectangle (axis cs:80.5634129575528,0.0329620148367638);
\draw[draw=none,fill=color0,fill opacity=0.6] (axis cs:80.5634129575528,0) rectangle (axis cs:80.8667924702896,0);
\draw[draw=none,fill=color0,fill opacity=0.6] (axis cs:80.8667924702896,0) rectangle (axis cs:81.1701719830264,0);
\draw[draw=none,fill=color0,fill opacity=0.6] (axis cs:81.1701719830264,0) rectangle (axis cs:81.4735514957632,0);
\draw[draw=none,fill=color0,fill opacity=0.6] (axis cs:81.4735514957632,0) rectangle (axis cs:81.7769310085,0);
\draw[draw=none,fill=color0,fill opacity=0.6] (axis cs:81.7769310085,0) rectangle (axis cs:82.0803105212368,0.0164810074183819);
\draw[draw=none,fill=color0,fill opacity=0.6] (axis cs:82.0803105212368,0) rectangle (axis cs:82.3836900339736,0);
\draw[draw=none,fill=color0,fill opacity=0.6] (axis cs:82.3836900339736,0) rectangle (axis cs:82.6870695467104,0);
\draw[draw=none,fill=color0,fill opacity=0.6] (axis cs:82.6870695467104,0) rectangle (axis cs:82.9904490594472,0);
\draw[draw=none,fill=color0,fill opacity=0.6] (axis cs:82.9904490594472,0) rectangle (axis cs:83.293828572184,0);
\draw[draw=none,fill=color0,fill opacity=0.6] (axis cs:83.293828572184,0) rectangle (axis cs:83.5972080849208,0);
\draw[draw=none,fill=color0,fill opacity=0.6] (axis cs:83.5972080849207,0) rectangle (axis cs:83.9005875976575,0);
\draw[draw=none,fill=color0,fill opacity=0.6] (axis cs:83.9005875976575,0) rectangle (axis cs:84.2039671103943,0);
\draw[draw=none,fill=color0,fill opacity=0.6] (axis cs:84.2039671103943,0) rectangle (axis cs:84.5073466231311,0);
\draw[draw=none,fill=color0,fill opacity=0.6] (axis cs:84.5073466231311,0) rectangle (axis cs:84.8107261358679,0);
\draw[draw=none,fill=color0,fill opacity=0.6] (axis cs:84.8107261358679,0) rectangle (axis cs:85.1141056486047,0);
\draw[draw=none,fill=color0,fill opacity=0.6] (axis cs:85.1141056486047,0) rectangle (axis cs:85.4174851613415,0);
\draw[draw=none,fill=color0,fill opacity=0.6] (axis cs:85.4174851613415,0) rectangle (axis cs:85.7208646740783,0);
\draw[draw=none,fill=color0,fill opacity=0.6] (axis cs:85.7208646740783,0) rectangle (axis cs:86.0242441868151,0);
\draw[draw=none,fill=color0,fill opacity=0.6] (axis cs:86.0242441868151,0) rectangle (axis cs:86.3276236995519,0);
\draw[draw=none,fill=color0,fill opacity=0.6] (axis cs:86.3276236995519,0) rectangle (axis cs:86.6310032122887,0);
\draw[draw=none,fill=color0,fill opacity=0.6] (axis cs:86.6310032122886,0) rectangle (axis cs:86.9343827250254,0);
\draw[draw=none,fill=color0,fill opacity=0.6] (axis cs:86.9343827250254,0) rectangle (axis cs:87.2377622377622,0.0164810074183811);
\end{axis}

\end{tikzpicture}

\end{center}
\end{subfigure}
\caption{Occurences of percentages of saved additions in two dimensions.}
\label{fig:saved_additions_2d}
\end{figure}

Similar behavior showed for point clouds drawn from the uniform distribution and the Gaussian mixture. However, when increasing the number of points in three dimensions the distribution of $a$-values changes. Consider the following figure.

\begin{figure}[H]
%\centering%
\begin{subfigure}[c]{0.95\textwidth}
\begin{center}
% This file was created by tikzplotlib v0.9.2.
\begin{tikzpicture}

\definecolor{color0}{rgb}{0.12156862745098,0.466666666666667,0.705882352941177}

\begin{axis}[
tick align=outside,
tick pos=left,
title={mean = 4.45\%},
xlabel = $a$,
ylabel = density,
x grid style={white!69.0196078431373!black},
xmin=-4.53430785783378, xmax=95.5162073032124,
xtick style={color=black},
y grid style={white!69.0196078431373!black},
yticklabel style={
        /pgf/number format/fixed,
        /pgf/number format/precision=5
},
scaled y ticks=false,
ymin=0, ymax=0.277060042673249,
ytick style={color=black}
]
\draw[draw=none,fill=color0,fill opacity=0.6] (axis cs:0.0134428313046868,0) rectangle (axis cs:0.922992969132379,0.263866707307856);
\draw[draw=none,fill=color0,fill opacity=0.6] (axis cs:0.922992969132379,0) rectangle (axis cs:1.83254310696007,0.230883368894374);
\draw[draw=none,fill=color0,fill opacity=0.6] (axis cs:1.83254310696007,0) rectangle (axis cs:2.74209324478776,0.156121135157148);
\draw[draw=none,fill=color0,fill opacity=0.6] (axis cs:2.74209324478776,0) rectangle (axis cs:3.65164338261546,0.11654112906097);
\draw[draw=none,fill=color0,fill opacity=0.6] (axis cs:3.65164338261546,0) rectangle (axis cs:4.56119352044315,0.0769611229647914);
\draw[draw=none,fill=color0,fill opacity=0.6] (axis cs:4.56119352044315,0) rectangle (axis cs:5.47074365827084,0.0461766737788748);
\draw[draw=none,fill=color0,fill opacity=0.6] (axis cs:5.47074365827084,0) rectangle (axis cs:6.38029379609854,0.0483755630064403);
\draw[draw=none,fill=color0,fill opacity=0.6] (axis cs:6.38029379609853,0) rectangle (axis cs:7.28984393392623,0.0197900030480892);
\draw[draw=none,fill=color0,fill opacity=0.6] (axis cs:7.28984393392623,0) rectangle (axis cs:8.19939407175392,0.0197900030480892);
\draw[draw=none,fill=color0,fill opacity=0.6] (axis cs:8.19939407175392,0) rectangle (axis cs:9.10894420958161,0.0175911138205238);
\draw[draw=none,fill=color0,fill opacity=0.6] (axis cs:9.10894420958161,0) rectangle (axis cs:10.0184943474093,0.0153922245929583);
\draw[draw=none,fill=color0,fill opacity=0.6] (axis cs:10.0184943474093,0) rectangle (axis cs:10.928044485237,0.00439777845513093);
\draw[draw=none,fill=color0,fill opacity=0.6] (axis cs:10.928044485237,0) rectangle (axis cs:11.8375946230647,0.00879555691026186);
\draw[draw=none,fill=color0,fill opacity=0.6] (axis cs:11.8375946230647,0) rectangle (axis cs:12.7471447608924,0.00219888922756547);
\draw[draw=none,fill=color0,fill opacity=0.6] (axis cs:12.7471447608924,0) rectangle (axis cs:13.6566948987201,0.00659666768269641);
\draw[draw=none,fill=color0,fill opacity=0.6] (axis cs:13.6566948987201,0) rectangle (axis cs:14.5662450365478,0.00879555691026186);
\draw[draw=none,fill=color0,fill opacity=0.6] (axis cs:14.5662450365478,0) rectangle (axis cs:15.4757951743755,0.00439777845513093);
\draw[draw=none,fill=color0,fill opacity=0.6] (axis cs:15.4757951743755,0) rectangle (axis cs:16.3853453122032,0);
\draw[draw=none,fill=color0,fill opacity=0.6] (axis cs:16.3853453122032,0) rectangle (axis cs:17.2948954500308,0.00439777845513095);
\draw[draw=none,fill=color0,fill opacity=0.6] (axis cs:17.2948954500308,0) rectangle (axis cs:18.2044455878585,0.00439777845513093);
\draw[draw=none,fill=color0,fill opacity=0.6] (axis cs:18.2044455878585,0) rectangle (axis cs:19.1139957256862,0);
\draw[draw=none,fill=color0,fill opacity=0.6] (axis cs:19.1139957256862,0) rectangle (axis cs:20.0235458635139,0);
\draw[draw=none,fill=color0,fill opacity=0.6] (axis cs:20.0235458635139,0) rectangle (axis cs:20.9330960013416,0.00439777845513093);
\draw[draw=none,fill=color0,fill opacity=0.6] (axis cs:20.9330960013416,0) rectangle (axis cs:21.8426461391693,0);
\draw[draw=none,fill=color0,fill opacity=0.6] (axis cs:21.8426461391693,0) rectangle (axis cs:22.752196276997,0.0065966676826964);
\draw[draw=none,fill=color0,fill opacity=0.6] (axis cs:22.752196276997,0) rectangle (axis cs:23.6617464148247,0.00439777845513093);
\draw[draw=none,fill=color0,fill opacity=0.6] (axis cs:23.6617464148247,0) rectangle (axis cs:24.5712965526524,0.00219888922756547);
\draw[draw=none,fill=color0,fill opacity=0.6] (axis cs:24.5712965526524,0) rectangle (axis cs:25.4808466904801,0);
\draw[draw=none,fill=color0,fill opacity=0.6] (axis cs:25.4808466904801,0) rectangle (axis cs:26.3903968283078,0.00219888922756547);
\draw[draw=none,fill=color0,fill opacity=0.6] (axis cs:26.3903968283078,0) rectangle (axis cs:27.2999469661355,0.00219888922756547);
\draw[draw=none,fill=color0,fill opacity=0.6] (axis cs:27.2999469661355,0) rectangle (axis cs:28.2094971039632,0.00219888922756547);
\draw[draw=none,fill=color0,fill opacity=0.6] (axis cs:28.2094971039632,0) rectangle (axis cs:29.1190472417909,0);
\draw[draw=none,fill=color0,fill opacity=0.6] (axis cs:29.1190472417909,0) rectangle (axis cs:30.0285973796185,0);
\draw[draw=none,fill=color0,fill opacity=0.6] (axis cs:30.0285973796185,0) rectangle (axis cs:30.9381475174462,0);
\draw[draw=none,fill=color0,fill opacity=0.6] (axis cs:30.9381475174462,0) rectangle (axis cs:31.8476976552739,0);
\draw[draw=none,fill=color0,fill opacity=0.6] (axis cs:31.8476976552739,0) rectangle (axis cs:32.7572477931016,0);
\draw[draw=none,fill=color0,fill opacity=0.6] (axis cs:32.7572477931016,0) rectangle (axis cs:33.6667979309293,0);
\draw[draw=none,fill=color0,fill opacity=0.6] (axis cs:33.6667979309293,0) rectangle (axis cs:34.576348068757,0);
\draw[draw=none,fill=color0,fill opacity=0.6] (axis cs:34.576348068757,0) rectangle (axis cs:35.4858982065847,0);
\draw[draw=none,fill=color0,fill opacity=0.6] (axis cs:35.4858982065847,0) rectangle (axis cs:36.3954483444124,0);
\draw[draw=none,fill=color0,fill opacity=0.6] (axis cs:36.3954483444124,0) rectangle (axis cs:37.3049984822401,0.00439777845513093);
\draw[draw=none,fill=color0,fill opacity=0.6] (axis cs:37.3049984822401,0) rectangle (axis cs:38.2145486200678,0.00219888922756547);
\draw[draw=none,fill=color0,fill opacity=0.6] (axis cs:38.2145486200678,0) rectangle (axis cs:39.1240987578955,0);
\draw[draw=none,fill=color0,fill opacity=0.6] (axis cs:39.1240987578955,0) rectangle (axis cs:40.0336488957232,0);
\draw[draw=none,fill=color0,fill opacity=0.6] (axis cs:40.0336488957232,0) rectangle (axis cs:40.9431990335509,0.00219888922756547);
\draw[draw=none,fill=color0,fill opacity=0.6] (axis cs:40.9431990335509,0) rectangle (axis cs:41.8527491713785,0);
\draw[draw=none,fill=color0,fill opacity=0.6] (axis cs:41.8527491713785,0) rectangle (axis cs:42.7622993092062,0);
\draw[draw=none,fill=color0,fill opacity=0.6] (axis cs:42.7622993092062,0) rectangle (axis cs:43.6718494470339,0.00219888922756547);
\draw[draw=none,fill=color0,fill opacity=0.6] (axis cs:43.6718494470339,0) rectangle (axis cs:44.5813995848616,0);
\draw[draw=none,fill=color0,fill opacity=0.6] (axis cs:44.5813995848616,0) rectangle (axis cs:45.4909497226893,0);
\draw[draw=none,fill=color0,fill opacity=0.6] (axis cs:45.4909497226893,0) rectangle (axis cs:46.400499860517,0);
\draw[draw=none,fill=color0,fill opacity=0.6] (axis cs:46.400499860517,0) rectangle (axis cs:47.3100499983447,0);
\draw[draw=none,fill=color0,fill opacity=0.6] (axis cs:47.3100499983447,0) rectangle (axis cs:48.2196001361724,0);
\draw[draw=none,fill=color0,fill opacity=0.6] (axis cs:48.2196001361724,0) rectangle (axis cs:49.1291502740001,0);
\draw[draw=none,fill=color0,fill opacity=0.6] (axis cs:49.1291502740001,0) rectangle (axis cs:50.0387004118278,0);
\draw[draw=none,fill=color0,fill opacity=0.6] (axis cs:50.0387004118278,0) rectangle (axis cs:50.9482505496555,0);
\draw[draw=none,fill=color0,fill opacity=0.6] (axis cs:50.9482505496555,0) rectangle (axis cs:51.8578006874832,0);
\draw[draw=none,fill=color0,fill opacity=0.6] (axis cs:51.8578006874832,0) rectangle (axis cs:52.7673508253109,0);
\draw[draw=none,fill=color0,fill opacity=0.6] (axis cs:52.7673508253109,0) rectangle (axis cs:53.6769009631386,0);
\draw[draw=none,fill=color0,fill opacity=0.6] (axis cs:53.6769009631386,0) rectangle (axis cs:54.5864511009662,0);
\draw[draw=none,fill=color0,fill opacity=0.6] (axis cs:54.5864511009662,0) rectangle (axis cs:55.4960012387939,0.00219888922756547);
\draw[draw=none,fill=color0,fill opacity=0.6] (axis cs:55.4960012387939,0) rectangle (axis cs:56.4055513766216,0);
\draw[draw=none,fill=color0,fill opacity=0.6] (axis cs:56.4055513766216,0) rectangle (axis cs:57.3151015144493,0);
\draw[draw=none,fill=color0,fill opacity=0.6] (axis cs:57.3151015144493,0) rectangle (axis cs:58.224651652277,0);
\draw[draw=none,fill=color0,fill opacity=0.6] (axis cs:58.224651652277,0) rectangle (axis cs:59.1342017901047,0);
\draw[draw=none,fill=color0,fill opacity=0.6] (axis cs:59.1342017901047,0) rectangle (axis cs:60.0437519279324,0);
\draw[draw=none,fill=color0,fill opacity=0.6] (axis cs:60.0437519279324,0) rectangle (axis cs:60.9533020657601,0);
\draw[draw=none,fill=color0,fill opacity=0.6] (axis cs:60.9533020657601,0) rectangle (axis cs:61.8628522035878,0);
\draw[draw=none,fill=color0,fill opacity=0.6] (axis cs:61.8628522035878,0) rectangle (axis cs:62.7724023414155,0);
\draw[draw=none,fill=color0,fill opacity=0.6] (axis cs:62.7724023414155,0) rectangle (axis cs:63.6819524792432,0);
\draw[draw=none,fill=color0,fill opacity=0.6] (axis cs:63.6819524792432,0) rectangle (axis cs:64.5915026170709,0);
\draw[draw=none,fill=color0,fill opacity=0.6] (axis cs:64.5915026170709,0) rectangle (axis cs:65.5010527548986,0);
\draw[draw=none,fill=color0,fill opacity=0.6] (axis cs:65.5010527548986,0) rectangle (axis cs:66.4106028927262,0);
\draw[draw=none,fill=color0,fill opacity=0.6] (axis cs:66.4106028927262,0) rectangle (axis cs:67.3201530305539,0);
\draw[draw=none,fill=color0,fill opacity=0.6] (axis cs:67.3201530305539,0) rectangle (axis cs:68.2297031683816,0);
\draw[draw=none,fill=color0,fill opacity=0.6] (axis cs:68.2297031683816,0) rectangle (axis cs:69.1392533062093,0);
\draw[draw=none,fill=color0,fill opacity=0.6] (axis cs:69.1392533062093,0) rectangle (axis cs:70.048803444037,0);
\draw[draw=none,fill=color0,fill opacity=0.6] (axis cs:70.048803444037,0) rectangle (axis cs:70.9583535818647,0);
\draw[draw=none,fill=color0,fill opacity=0.6] (axis cs:70.9583535818647,0) rectangle (axis cs:71.8679037196924,0);
\draw[draw=none,fill=color0,fill opacity=0.6] (axis cs:71.8679037196924,0) rectangle (axis cs:72.7774538575201,0);
\draw[draw=none,fill=color0,fill opacity=0.6] (axis cs:72.7774538575201,0) rectangle (axis cs:73.6870039953478,0);
\draw[draw=none,fill=color0,fill opacity=0.6] (axis cs:73.6870039953478,0) rectangle (axis cs:74.5965541331755,0);
\draw[draw=none,fill=color0,fill opacity=0.6] (axis cs:74.5965541331755,0) rectangle (axis cs:75.5061042710032,0);
\draw[draw=none,fill=color0,fill opacity=0.6] (axis cs:75.5061042710032,0) rectangle (axis cs:76.4156544088309,0);
\draw[draw=none,fill=color0,fill opacity=0.6] (axis cs:76.4156544088309,0) rectangle (axis cs:77.3252045466586,0);
\draw[draw=none,fill=color0,fill opacity=0.6] (axis cs:77.3252045466586,0) rectangle (axis cs:78.2347546844863,0);
\draw[draw=none,fill=color0,fill opacity=0.6] (axis cs:78.2347546844863,0) rectangle (axis cs:79.144304822314,0);
\draw[draw=none,fill=color0,fill opacity=0.6] (axis cs:79.144304822314,0) rectangle (axis cs:80.0538549601416,0);
\draw[draw=none,fill=color0,fill opacity=0.6] (axis cs:80.0538549601416,0) rectangle (axis cs:80.9634050979693,0);
\draw[draw=none,fill=color0,fill opacity=0.6] (axis cs:80.9634050979693,0) rectangle (axis cs:81.872955235797,0);
\draw[draw=none,fill=color0,fill opacity=0.6] (axis cs:81.872955235797,0) rectangle (axis cs:82.7825053736247,0);
\draw[draw=none,fill=color0,fill opacity=0.6] (axis cs:82.7825053736247,0) rectangle (axis cs:83.6920555114524,0);
\draw[draw=none,fill=color0,fill opacity=0.6] (axis cs:83.6920555114524,0) rectangle (axis cs:84.6016056492801,0);
\draw[draw=none,fill=color0,fill opacity=0.6] (axis cs:84.6016056492801,0) rectangle (axis cs:85.5111557871078,0);
\draw[draw=none,fill=color0,fill opacity=0.6] (axis cs:85.5111557871078,0) rectangle (axis cs:86.4207059249355,0);
\draw[draw=none,fill=color0,fill opacity=0.6] (axis cs:86.4207059249355,0) rectangle (axis cs:87.3302560627632,0);
\draw[draw=none,fill=color0,fill opacity=0.6] (axis cs:87.3302560627632,0) rectangle (axis cs:88.2398062005909,0.00219888922756547);
\draw[draw=none,fill=color0,fill opacity=0.6] (axis cs:88.2398062005909,0) rectangle (axis cs:89.1493563384186,0);
\draw[draw=none,fill=color0,fill opacity=0.6] (axis cs:89.1493563384186,0) rectangle (axis cs:90.0589064762463,0);
\draw[draw=none,fill=color0,fill opacity=0.6] (axis cs:90.0589064762463,0) rectangle (axis cs:90.968456614074,0.00439777845513093);
\end{axis}

\end{tikzpicture}

\end{center}
\end{subfigure}
\caption{Occurences of percentages of saved additions in three dimensions on point clouds of $200$ points.}
\label{fig:saved_additions_3d_large_complex}
\end{figure}

Figure \ref{fig:saved_additions_3d_large_complex} in the vast majority of cases, there are very few additions saved. We still see some rare cases around the $90\%$ mark and some values in between $10\%$ and $60\%$, but we end up with a mean for saved additions of $4.45\%$ which is substantially lower than the $73.64\%$ we saw in Figure~\ref{fig:saved_additions_3d}. Note that all constructed complexes had elements that were part of apparent pairs between $89.5\%$ and $91\%$. Indeed some more detailed analysis in which we split the filtrations into two classes, namely those with $a$-values above and below $50\%$ revealed no other readily accessible differences. The values of the means of the size of the filtrations, the percentage of apparent pairs, and the number of additions needed in the standard reduction without clearing were within $1\%$ of each other for all examples generated on a fixed number of points. 

The following table has the same columns and rows as Table \ref{tab:add_STA_TWI} but this time we are looking at the averaged values of alpha complexes on $300$ points.

 \begin{table}[H]
     \begin{center}
     \begin{tabular}{|c|c|c|c|}
     \hline
     & \multicolumn{3}{|c|}{Percentage of Saved Additions} \\ \cline{1-4}
     Sample distribution & Uniform & Multivariate Normal & Gaussian 		Mixture\\ \hline
     dimension two & 57.53 & 60.63 & 60.91 \\ \hline
     dimension three & 1.58 & 2.21 & 3.73 \\ \hline

     \end{tabular}
     
     \caption{$a$-value for alpha complexes on $300$ points.}
     \label{tab:add_STA_TWI_300}
     \end{center}
 \end{table}
 
As we can see, the percentage of saved additions in dimension two is still significant, while in dimension three it is very small. Assuming that this trend continues as complexes get larger, implies that preprocessing or clearing schemes that implicitly contain the apparent pairs of some filtration generate their speedup outside of them, at least in three dimensions on alpha complexes. This might be of particular interest with respect to the widely used twisted variant of the standard reduction algorithm.

Another interesting thing to note is that again we see bigger improvements for the Gaussian distribution and mixture. But we also know that the average $q$ value in filtrations on $300$ points is lower than for the uniform case. This implies that the percentage of additions saved does not depend on the percentage of elements in apparent pairs but on some structural differences between the filtrations generated on differently sampled point clouds. Furthermore, there is some difference between points drawn from a single or multiple multivariate distributions. This might be within the margin of error for our experiments but might also be caused by structural differences between the two. Further experiments are required to shed further light on this.

\section{Conclusions}
The experiments we considered in this section are initial observations on the relation between apparent gradients, Betti numbers and complicated persistent homology computations that could give an indication in which direction a closer look might yield results.
 
We have seen that the majority of elements in the filtrations we generated by constructing an alpha complex on a random point cloud is part of an apparent pair. Previously, we showed in Chapter \ref{ch:connection} how the reduction of critical cells with respect to an apparent gradient has a lower bound dependent on $V$-paths defined by the apparent pairs. Perhaps it is possible to find descriptions of filtrations based on their apparent gradients that already give us a good indication on whether the filtration will result in expensive or cheap persistent homology computations. A desired result would be a better understanding of why persistent homology computations in practice show linear or slightly super linear growth although the worst case bound is in $\mathcal{O}(n^3)$ \cite{pershom}. 

The behavior of the random discrete Morse function as seen in Figure~\ref{fig:perfect_rdm_v_apparent} is intriguing.

Finally, the partitioning into two classes for three dimensional alpha complexes on few points, as seen in Figure \ref{fig:saved_additions_3d}, might be an interesting starting point for some further analysis. 



%\newpage
%%\newpage\null\thispagestyle{empty}\newpage
%\chapter{Final Remarks}
\label{ch:final}
 
Wow dat was nice


\bibliographystyle{halpha}
\bibliography{references.bib}
\end{document}
