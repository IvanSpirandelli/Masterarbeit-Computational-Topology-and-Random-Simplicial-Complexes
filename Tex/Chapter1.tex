\chapter{Foundations}

This first section is committed to the introduction of the mathematical concepts we will need throughout this work. The goal is not an in depth build up, but rather a concise collection of basic definitions and theorems.

\section{Simplicial Complexes} \label{sec:Simplicial Complexes}
In this section we will introduce some notions of discrete geometry. For the following definitions let $S = \{s_1,...,s_m\} \subseteq \mathbb{R}^n$ be a finite set of points.

\begin{defi}[Affine combination]
Let $\lambda_i \in \mathbb{R}$, $i = 1,...,m$ be scalars, which satisfy $\sum\limits_{n=1}^m \lambda_i = 1$. Then:
\begin{center}
$\sum\limits_{n=1}^m \lambda_i s_i$
\end{center}
is called an  \textit{\textbf{affine combination}} of the points $s_i$. \cite[Definition 2.11]{Polyhedral+and+Algebraic+Methods}
\end{defi}

\begin{defi}[Affine hull, affinely independent]
The set of all affine combinations of points in $S$ is called the  \textit{\textbf{affine hull}}. We say that the points in $S$ are \textit{\textbf{affinely independent}}
if they generate a subspace of dimension $m-1$. \cite[Definition 2.11]{Polyhedral+and+Algebraic+Methods}
\end{defi}


An alternative definition is given in \cite[III.1, paragraph 3]{Computational+Topology}. The points in $S$ are called  \textit{\textbf{affinely independent}}
if any two affine combinations $x = \sum\limits_{n=1}^m \lambda_i s_i$ and $y = \sum\limits_{n=1}^m \mu_i s_i$ are the same, iff $\mu_i = \lambda_i$ for $i = 1,...,m$.


\begin{defi}[Convex combination]
Let $S = \{s_1,...,s_m\} \subseteq \mathbb{R}^n$, $1 \leq m \in \mathbb{N}$. Let $0 \leq \lambda_i \in \mathbb{R}$, $i = 1,...,m$ be scalars, which satisfy $\sum\limits_{n=1}^m \lambda_i = 1$. Then the sum:
\begin{center}
$\sum\limits_{n=1}^m \lambda_i s_i$
\end{center}
is called a  \textit{\textbf{convex combination}} of the points $s_i$. \cite[Definition 2.12]{Polyhedral+and+Algebraic+Methods}
\end{defi}

In other words, a convex combination is an affine combination, in which the $\lambda_i \geq 0$. Now we can analogously define the convex hull.

\begin{defi}[Convex hull]
The set of all convex combinations of points in $S$ is the \textbf{\textit{convex hull}} of $S$. 
\end{defi} 

We are particularly interested in the following special class of convex hulls.

\begin{defi}[k-Simplex]
A \textit{\textbf{k-Simplex}} is the convex hull of $k+1$ affinely independent points. 
\end{defi}

This means a 0-Simplex is a point, a 1-Simplex is a line, a 2-Simplex is a triangle and so on and so forth. See the following figure for an example of a 2-Simplex and a 3-Simplex.


\begin{figure}[H]
\captionsetup[subfigure]{justification=centering}
\begin{subfigure}[t]{0.5\textwidth}
\begin{center}
\begin{tikzpicture}
\node[above= 2pt of {(0.75,1)}, outer sep=2pt] {a};
\node[below = 2pt of {(0,-0.5))}, outer sep=2pt] {b};
\node[below = 2pt of {(1.5,-0.5)}, outer sep=2pt] {c};
\node[above = 2pt of {(1,2,2)}, outer sep=2pt] {};
\fill [opacity = 0.2] (0,-0.5)--(1.5,-0.5)--(0.75,1)--(0,-0.5);
\draw [-,very thick] (0,-0.5)--(1.5,-0.5);
\draw [-,very thick] (1.5,-0.5)--(0.75,1);
\draw [-,very thick] (0.75,1)--(0,-0.5);
\foreach \Point in {(0,-0.5),(1.5,-0.5),(0.75,1)}{
\node[black] at \Point {\textbullet};
}
\end{tikzpicture}
\end{center}
\subcaption{A 2-simplex or triangle}
\end{subfigure}
\begin{subfigure}[t]{0.5\textwidth}
\begin{center}
\begin{tikzpicture}
\node[below= 2pt of {(2,0,0.25)}, outer sep=2pt] {};
\node[below = 2pt of {(0.25,0,1)}, outer sep=2pt] {};
\node[below right= 2pt of {(2,0,2)}, outer sep=2pt] {};
\node[above = 2pt of {(1,2,2)}, outer sep=2pt] {};
\fill [opacity = 0.2] (0.25,0,1)--(2,0,2)--(1,2,2)--(0.25,0,1);
\fill [opacity = 0.2] (2,0,0.25)--(2,0,2)--(1,2,2)--(2,0,0.25);
\draw [dashed, thin] (2,0,0.25) -- (0.25,0,1);
\draw [-,very thick] (2,0,2) -- (0.25,0,1);
\draw [-,very thick] (2,0,2) -- (2,0,0.25);
\draw [-,very thick] (2,0,2) -- (1,2,2);
\draw [-,very thick] (2,0,0.25) -- (1,2,2);
\draw [-,very thick] (0.25,0,1) -- (1,2,2);
\foreach \Point in {(2,0,0.25), (0.25,0,1), (2,0,2), (1,2,2)}{
\node[black] at \Point {\textbullet};
}
\end{tikzpicture}
\end{center}
\subcaption{A 3-Simplex or tetrahedron}
\end{subfigure}
\label{Figure 1.1}
\caption{An example of a 2- and a 3-Simplex}
\end{figure}

The following definitions follow \cite[III.1]{Computational+Topology}.

\begin{defi}[(proper) face/coface]
    Consider a k-Simplex $\sigma$, which is the convex hull of points $S = \{s_0, ... , s_{k+1}\}$. The convex hull of a non-empty subset $U$ of $S$ is called a \textbf{face} of $\sigma$.
    A face $\theta$ is called \textbf{proper}, if $|U| < k+1$. If $|U| = k$, we say that $\tau = \operatorname{conv}(U)$ is a \textbf{coface} of $\sigma$.
\end{defi}

A popular notation is to write $\tau \leq \sigma$ if $\tau$ is a face of $\sigma$ and $\tau < \sigma$ if it is proper.

\begin{defi}[boundary and interior]
    Consider a Simplex $\sigma$ as before. The \textbf{boundary} of $\sigma$, which we will denote by $\operatorname{bd}(\sigma)$ is the union of all proper faces of $\sigma$. Its \textbf{interior}
    is: $\operatorname{int}(\sigma) \coloneqq \sigma\setminus\operatorname{bd}(\sigma)$
\end{defi}

A notation we will use in this work, is to write $[a,b,c]$ when talking about the simplex that has $a,b$ and $c$ for vertices.\\
Now Consider Figure 1.1 (a) \ref{Figure 1.1} , for an illustration.
The triangle $[a,b,c]$ has the gray area as interior and the Union of the edges $[a,b],[a,c],[b,c]$ and the vertices $[a],[b],[c]$
as its boundary. The proper faces of the triangle $[a,b,c]$  are the edges in its boundary and the proper faces of the edges are the respective vertices. \\
We will now proceed to define simplicial complexes.

\begin{defi}[Simplicial complex, subcomplex, k-skeleton]
    Let $K$ be a finite collection of Simplices. $K$ is a \textbf{simplicial complex} if the following two conditions hold:
    \begin{enumerate}
        \item $\sigma \in K \operatorname{and} \tau < \sigma \implies \tau \in K$
        \item $\sigma_0, \sigma_1 \in K \implies \sigma_0 \cap \sigma_1 \in K \operatorname{or} \sigma_0 \cap \sigma_1 = \emptyset$
    \end{enumerate}
    'The \textbf{dimension} of $K$ is the maximum dimension of any of its simplices.' \cite[p. 63, paragraph 4]{Computational+Topology}.
    A subset $L \subset K$ is called a \textbf{subcomplex} of $K$ if it itself a simplicial complex.
    The \textbf{k-skeleton} of $K$ is the set of all simplices of dimension k or less.
\end{defi}

By the way we defined simplices, it is clear, that such a simplicial complex is a subset of $\mathbb{R}^n$, where $n$ is the dimension of the simplicial complex.
Each vertex has coordinates and each $k$-simplex is the convex hull of $k+1$ vertices. This is why these are also referred to as \textbf{geometric} simplicial complexes.\\
The following Figure \ref{Figure 1.2} gives an example of a simplicial complex in two dimensions and of a
set of simplices, that is not a simplicial complex.

\begin{figure}[H]
\captionsetup[subfigure]{justification=centering}
\begin{subfigure}[t]{0.5\textwidth}
\begin{center}
\begin{tikzpicture}
\node[below= 2pt of {(0,0)}, outer sep=2pt] {a};
\node[below = 2pt of {(2,0))}, outer sep=2pt] {b};
\node[below right = 2pt of {(2,2)}, outer sep=2pt] {d};
\node[left = 2pt of {(0,2)}, outer sep=2pt] {c};
\node[right = 2pt of {(2,4)}, outer sep=2pt] {f};
\node[right = 2pt of {(4,2)}, outer sep=2pt] {e};

\fill [opacity = 0.2] (0,0)--(2,0)--(2,2)--(0,2)--(0,0);
\draw [-,very thick] (0,0)--(2,0);
\draw [-,very thick] (2,0)--(0,2);
\draw [-,very thick] (0,2)--(0,0);
\draw [-,very thick] (0,2)--(2,2);
\draw [-,very thick] (2,2)--(4,2);
\draw [-,very thick] (2,2)--(2,0);
\draw [-,very thick] (2,4)--(2,2);
\foreach \Point in {(0,0),(2,0),(0,2),(2,2),(4,2),(2,4)}{
\node[black] at \Point {\textbullet};
}
\end{tikzpicture}
\end{center}
\subcaption{A simplicial complex.}
\end{subfigure}
\begin{subfigure}[t]{0.5\textwidth}
\begin{center}
\begin{tikzpicture}
\node[below= 2pt of {(0,0)}, outer sep=2pt] {a};
\node[left = 2pt of {(0,2))}, outer sep=2pt] {b};
\node[right = 2pt of {(2,4)}, outer sep=2pt] {c};
\node[below = 2pt of {(3,0)}, outer sep=2pt] {d};
\node[right = 2pt of {(3,2)}, outer sep=2pt] {e};
\node[left = 2pt of {(1,4)}, outer sep=2pt] {f};

\fill [opacity = 0.2] (0,0)--(0,2)--(2,4)--(0,0);
\fill [opacity = 0.2] (3,0)--(3,2)--(1,4)--(3,0);
\draw [-,very thick] (0,0)--(0,2);
\draw [-,very thick] (2,4)--(0,2);
\draw [-,very thick] (0,0)--(2,4);
\draw [-,very thick] (3,0)--(3,2);
\draw [-,very thick] (1,4)--(3,2);
\draw [-,very thick] (3,0)--(1,4);

\foreach \Point in {(0,0),(0,2),(2,4),(3,0),(3,2),(1,4)}{
\node[black] at \Point {\textbullet};
}
\end{tikzpicture}
\end{center}
\subcaption{Not a simplicial complex. }
\end{subfigure}
\label{Figure 1.2}
\caption{Sets of simplices that do and do not fullfill Definition 1.1.8}
\end{figure}

Now when we look at the simplices of \ref{Figure 1.2} (a) and list them. We get
\begin{center}
    \begin{tabular}{lc}
        Vertices: & $[a], [b], [c], [d], [e], [f]$ \\
        Edges: & $[a,b],[a,c],[b,c],[b,d],[c,d],[d,e],[d,f]$ \\
        Triangles: & $[a,b,c], [b,c,d]$
    \end{tabular}
\end{center}

What is encoded here is just combinatorial data telling us, that some triangle $[a,b,c]$ contains edges $[a,b],[a,c],[b,c]$ and vertices $[a,b,c]$.\\
This is what we call an abstract simplicial complex.
\begin{defi}[Abstract simplicial complex]
    Let $A$ be a finite collection of sets. $A$ is called an \textbf{abstract simplicial complex} iff:
    \begin{enumerate}
        \item $\alpha \in A$ and $\beta \subset \alpha \Rightarrow \beta \in A$
    \end{enumerate}
\end{defi}

\begin{rem}
    In this work, when we say, simplicial complex we refer to a \textbf{geometric} simplicial complex.
    Let $A$ be an abstract simplicial complex. As for simplicial complexes we call the sets in $A$ its simplices. We will also write them in square brackets and the
    \textbf{dimension} of $A$ is the maximum cardinality of one of its simplices.
\end{rem}

\begin{obs}
    When comparing the definition of the abstract simplicial complex to the definition of the (geometric) simplicial complex, we can see, that the second
    condition is automatically fullfilled for abstract simplicial complexes.
\end{obs}

Consider the following set: \\
\begin{center}
    $A = \{[a],[b],[c],[d],[e],[f],[a,b],[a,c],[b,c],[b,d],[c,d],[d,e],[d,f],[a,b,c],[b,c,d]\}$.
\end{center}
This is the same set of vertices edges and triangles as in the tabular above and it is easy to verify, that this is an abstract simplicial complex.
The simplicial complex in Figure 1.2 (a) is a so called \textbf{geometric realization} of the abstract simplicial complex.
This realization is obviously not unique. If we perturb any vertex slightly enough, we will get a different simplicial complex,
but the abstract simplicial complex defined by the incidence relations stays the same. \\
So every simplicial complex induces an abstract simplicial complex and geometric realizations of abstract simplicial complexes in $\mathbb{R}^n$ are not unique.
Does every abstract simplicial complex have a geometric realization? Yes! \cite[p. 64]{Computational+Topology}

\begin{thm}[Geometric Realization Theorem]
    Every $d$-dimensional abstract simplicial complex has a geometric realization in $\mathbb{R}^{2d+1}$
\end{thm}


