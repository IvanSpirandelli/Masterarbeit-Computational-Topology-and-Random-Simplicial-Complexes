\chapter{Foundations}

This first section is committed to the introduction of the mathematical concepts we will need throughout this work. The goal is not an in depth build up, but rather a concise collection of basic definitions and theorems.

\section{Simplicial Complexes} \label{sec:Simplicial Complexes}
In this section we will introduce some notions of discrete geometry. For the following definitions let $S = \{s_1,...,s_m\} \subseteq \mathbb{R}^n$ be a finite set of points.

\begin{defi}[Affine combination]
Let $\lambda_i \in \mathbb{R}$, $i = 1,...,m$ be scalars, which satisfy $\sum\limits_{n=1}^m \lambda_i = 1$. Then:
\begin{center}
$\sum\limits_{n=1}^m \lambda_i s_i$
\end{center}
is called an  \textit{\textbf{affine combination}} of the points $s_i$. \cite[Definition 2.11]{Polyhedral+and+Algebraic+Methods}
\end{defi}

\begin{defi}[Affine hull, affinely independent]
The set of all affine combinations of points in $S$ is called the  \textit{\textbf{affine hull}}. We say that the points in $S$ are \textit{\textbf{affinely independent}} if they generate a subspace of dimension $m-1$. \cite[Definition 2.11]{Polyhedral+and+Algebraic+Methods}
\end{defi}


An alternative definition is given in \cite[III.1, paragraph 3]{Computational+Topology}. The points in $S$ are called  \textit{\textbf{affinely independent}} if any two affine combinations $x = \sum\limits_{n=1}^m \lambda_i s_i$ and $y = \sum\limits_{n=1}^m \mu_i s_i$ are the same, iff $\mu_i = \lambda_i$ for $i = 1,...,m$.


\begin{defi}[Convex combination]
Let $S = \{s_1,...,s_m\} \subseteq \mathbb{R}^n$, $1 \leq m \in \mathbb{N}$. Let $0 \leq \lambda_i \in \mathbb{R}$, $i = 1,...,m$ be scalars, which satisfy $\sum\limits_{n=1}^m \lambda_i = 1$. Then the sum:
\begin{center}
$\sum\limits_{n=1}^m \lambda_i s_i$
\end{center}
is called a  \textit{\textbf{convex combination}} of the points $s_i$. \cite[Definition 2.12]{Polyhedral+and+Algebraic+Methods}
\end{defi}

In other words, a convex combination is an affine combination, in which the $\lambda_i \geq 0$. Now we can analogously define the convex hull.

\begin{defi}[Convex hull]
The set of all convex combinations of points in $S$ is the \textbf{\textit{convex hull}} of $S$. 
\end{defi} 

We are particularly interested in the following special class of convex hulls.

\begin{defi}[k-Simplex]
A \textit{\textbf{k-Simplex}} is the convex hull of $k+1$ affinely independent points. 
\end{defi}